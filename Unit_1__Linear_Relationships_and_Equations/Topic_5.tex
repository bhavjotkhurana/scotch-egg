\documentclass[14pt]{extarticle}

% ---------- PACKAGES ----------
\usepackage{amsmath, amssymb}
\usepackage{geometry}
\usepackage{setspace}
\usepackage{titlesec}
\usepackage{xcolor}
\usepackage{helvet}
\usepackage{enumitem}

% ---------- PAGE SETUP ----------
\geometry{margin=1in}
\setstretch{1.5}
\setlength{\parindent}{0pt}
\setlength{\parskip}{0.75em}
\setlist{itemsep=0.75\baselineskip, topsep=0.5\baselineskip}
\titleformat{\section}{\normalfont\Large\bfseries}{\thesection}{1em}{}
\titleformat{\subsection}{\normalfont\large\bfseries}{\thesubsection}{1em}{}
\renewcommand{\familydefault}{\sfdefault}
\renewcommand{\emph}[1]{\textbf{#1}}

% ---------- DOCUMENT ----------
\begin{document}
\raggedright
\pagenumbering{gobble}

\begin{center}
    \LARGE \textbf{Unit 1: Linear Relationships and Equations} \\[6pt]
    \Large \textbf{Topic 5: Slope and Rate of Change}
\end{center}

\vspace{1em}

\section*{Concept Summary}

The \textbf{slope} of a line describes how steep the line is — it measures how much \(y\) changes when \(x\) increases by 1.  
In other words, slope represents the \textbf{rate of change} between two quantities.

\[
\text{Slope } (m) = \frac{\text{change in } y}{\text{change in } x} = \frac{y_2 - y_1}{x_2 - x_1}
\]

If you imagine moving along a line:
- The numerator \((y_2 - y_1)\) represents the “rise” (vertical change).  
- The denominator \((x_2 - x_1)\) represents the “run” (horizontal change).  

\textbf{Slope as Rate:}  
On the SAT, slope often represents a real-world rate such as:
\[
\text{miles per hour}, \quad \text{dollars per item}, \quad \text{points per game}, \quad \text{etc.}
\]

\subsection*{Types of Slope}
\[
\begin{aligned}
m > 0 &\Rightarrow \text{line rises left to right (positive slope)} \\
m < 0 &\Rightarrow \text{line falls left to right (negative slope)} \\
m = 0 &\Rightarrow \text{horizontal line} \\
\text{Undefined slope} &\Rightarrow \text{vertical line}
\end{aligned}
\]

\subsection*{Slope Relationships}
\begin{itemize}
    \item \textbf{Parallel lines} have the same slope.
    \item \textbf{Perpendicular lines} have slopes that are negative reciprocals:  
    \[
    m_1 \cdot m_2 = -1
    \]
\end{itemize}

\section*{Core Skills}
\begin{itemize}
    \item Find slope from two points, a graph, or an equation.
    \item Interpret slope as a rate of change in context.
    \item Recognize slope patterns for parallel and perpendicular lines.
    \item Identify slope units in real-world models.
\end{itemize}

\section*{Example 1: Finding Slope from Two Points}

Find the slope of the line passing through the points \((2, 5)\) and \((6, 13)\).

\textbf{Step 1: Label the points.}  
\((x_1, y_1) = (2, 5)\), \((x_2, y_2) = (6, 13)\)

\textbf{Step 2: Use the slope formula.}
\[
m = \frac{y_2 - y_1}{x_2 - x_1} = \frac{13 - 5}{6 - 2} = \frac{8}{4} = 2
\]

\textbf{Final Answer:} \(\boxed{m = 2}\)

\textbf{Interpretation:} For every 1 unit increase in \(x\), \(y\) increases by 2 units.

\section*{Example 2: Slope as a Rate of Change}

A car travels 120 miles in 3 hours. What is its average rate of change in miles per hour?

\textbf{Step 1: Identify the change in distance and change in time.}  
\[
\text{Change in distance} = 120 \text{ miles}, \quad \text{Change in time} = 3 \text{ hours}
\]

\textbf{Step 2: Compute the rate.}
\[
m = \frac{\text{change in distance}}{\text{change in time}} = \frac{120}{3} = 40
\]

\textbf{Final Answer:} \(\boxed{40 \text{ miles per hour}}\)

\textbf{Interpretation:} The slope of the line on a distance–time graph would be 40, showing a steady rate of travel.

\section*{Key Takeaways}
\begin{itemize}
    \item Slope measures the rate of change between two variables.
    \item Use \(m = \dfrac{y_2 - y_1}{x_2 - x_1}\) to find slope from points.
    \item Parallel lines have equal slopes; perpendicular lines have negative reciprocal slopes.
    \item In word problems, slope often represents a real-world rate such as “cost per item” or “distance per time.”
\end{itemize}

\newpage

% ============================================================
% QUESTIONS — TOPIC 5: SLOPE AND RATE OF CHANGE
% ============================================================

\section*{Practice Questions: Slope and Rate of Change}

\subsection*{Part A: Finding Slope from Points}
\begin{enumerate}
    \item Find the slope of the line through \((2, 3)\) and \((6, 11)\).
    \item Find the slope of the line through \((5, 7)\) and \((9, 15)\).
    \item Find the slope of the line through \((-3, 4)\) and \((5, 0)\).
    \item Find the slope of the line through \((0, 8)\) and \((3, -4)\).
    \item Find the slope of the line through \((-2, -3)\) and \((2, 5)\).
\end{enumerate}

\subsection*{Part B: Identifying Slope from an Equation}
\begin{enumerate}
    \setcounter{enumi}{5}
    \item Find the slope of \(y = 3x + 7\).
    \item Find the slope of \(y = -2x + 4\).
    \item Find the slope of \(2x + 5y = 10\).
    \item Find the slope of \(4x - y = 12\).
    \item Find the slope of \(3y + 6x = 9\).
\end{enumerate}

\subsection*{Part C: Parallel and Perpendicular Lines}
\begin{enumerate}
    \setcounter{enumi}{10}
    \item Find the slope of a line parallel to \(y = \tfrac{1}{2}x - 3\).
    \item Find the slope of a line perpendicular to \(y = -\tfrac{2}{3}x + 4\).
    \item Determine whether the lines \(y = 2x + 5\) and \(y = -\tfrac{1}{2}x - 3\) are perpendicular.
    \item Determine whether the lines \(3x - 2y = 6\) and \(y = \tfrac{3}{2}x + 1\) are parallel, perpendicular, or neither.
    \item The line through \((1, 4)\) and \((3, 8)\) — find the slope of any line perpendicular to it.
\end{enumerate}

\subsection*{Part D: Rate of Change in Context}
\begin{enumerate}
    \setcounter{enumi}{15}
    \item A runner covers 400 meters in 50 seconds. Find the rate of change (meters per second).
    \item The price of gas increased from \$3.20 to \$4.00 over 4 months. Find the average rate of change per month.
    \item The temperature dropped from 68°F to 50°F over 3 hours. Find the rate of change per hour.
    \item A business earns \$500 for 20 units sold and \$650 for 30 units sold. Find the rate of change in dollars per unit.
    \item A car travels 150 miles in 3 hours. What is its average speed?
\end{enumerate}

\subsection*{Part E: SAT-Style Applications}
\begin{enumerate}
    \setcounter{enumi}{20}
    \item The equation \(y = 25x + 100\) models the total cost \(y\) (in dollars) for renting a car for \(x\) days.  
    What does the number 25 represent in this context?
    \item The line passing through \((0, 50)\) and \((10, 80)\) represents a company’s revenue over time.  
    Find the rate of change and interpret its meaning.
    \item A student’s test score increased from 70 to 85 over 3 exams.  
    Find the rate of change per exam.
    \item The graph of a line has slope \(-4\).  
    What does this tell you about the relationship between \(x\) and \(y\)?
    \item Two lines have slopes \(m_1 = 3\) and \(m_2 = -\tfrac{1}{3}\).  
    Are the lines parallel, perpendicular, or neither?
\end{enumerate}

\newpage

% ============================================================
% SOLUTIONS — TOPIC 5: SLOPE AND RATE OF CHANGE
% ============================================================

\section*{Answer Key and Solutions: Slope and Rate of Change}

\subsection*{Part A Solutions: Finding Slope from Points}
\begin{enumerate}
    \item \((2,3)\) to \((6,11)\): \(m=\dfrac{11-3}{6-2}=\dfrac{8}{4}=\boxed{2}\)
    \item \((5,7)\) to \((9,15)\): \(m=\dfrac{15-7}{9-5}=\dfrac{8}{4}=\boxed{2}\)
    \item \((-3,4)\) to \((5,0)\): \(m=\dfrac{0-4}{5-(-3)}=\dfrac{-4}{8}=\boxed{-\tfrac{1}{2}}\)
    \item \((0,8)\) to \((3,-4)\): \(m=\dfrac{-4-8}{3-0}=\dfrac{-12}{3}=\boxed{-4}\)
    \item \((-2,-3)\) to \((2,5)\): \(m=\dfrac{5-(-3)}{2-(-2)}=\dfrac{8}{4}=\boxed{2}\)
\end{enumerate}

\subsection*{Part B Solutions: Identifying Slope from an Equation}
\begin{enumerate}
    \setcounter{enumi}{5}
    \item \(y=3x+7\Rightarrow m=\boxed{3}\)
    \item \(y=-2x+4\Rightarrow m=\boxed{-2}\)
    \item \(2x+5y=10\Rightarrow 5y=-2x+10\Rightarrow y=-\tfrac{2}{5}x+2 \Rightarrow m=\boxed{-\tfrac{2}{5}}\)
    \item \(4x-y=12\Rightarrow -y=-4x+12\Rightarrow y=4x-12 \Rightarrow m=\boxed{4}\)
    \item \(3y+6x=9\Rightarrow 3y=-6x+9\Rightarrow y=-2x+3 \Rightarrow m=\boxed{-2}\)
\end{enumerate}

\subsection*{Part C Solutions: Parallel and Perpendicular Lines}
\begin{enumerate}
    \setcounter{enumi}{10}
    \item Parallel line to \(y=\tfrac{1}{2}x-3\): \(m=\boxed{\tfrac{1}{2}}\)
    \item Perpendicular to \(y=-\tfrac{2}{3}x+4\): \(m=\boxed{\tfrac{3}{2}}\) \quad (negative reciprocal)
    \item Slopes \(2\) and \(-\tfrac{1}{2}\): \(2\cdot(-\tfrac{1}{2})=-1\Rightarrow\) \boxed{Perpendicular}
    \item \(3x-2y=6 \Rightarrow y=\tfrac{3}{2}x-3\) and \(y=\tfrac{3}{2}x+1\): same slope \(\tfrac{3}{2}\Rightarrow\) \boxed{Parallel}
    \item Slope through \((1,4)\) and \((3,8)\): \(m=\dfrac{8-4}{3-1}=\dfrac{4}{2}=2\). Perpendicular slope \(=\boxed{-\tfrac{1}{2}}\).
\end{enumerate}

\subsection*{Part D Solutions: Rate of Change in Context}
\begin{enumerate}
    \setcounter{enumi}{15}
    \item \(\dfrac{400\text{ m}}{50\text{ s}}=\boxed{8\ \text{m/s}}\)
    \item \(\dfrac{4.00-3.20}{4}=\dfrac{0.80}{4}=\boxed{\$0.20\ \text{per month}}\)
    \item \(\dfrac{50-68}{3}=\dfrac{-18}{3}=\boxed{-6^\circ\text{F per hour}}\)
    \item \(\dfrac{650-500}{30-20}=\dfrac{150}{10}=\boxed{\$15\ \text{per unit}}\)
    \item \(\dfrac{150\text{ miles}}{3\text{ h}}=\boxed{50\ \text{mph}}\)
\end{enumerate}

\subsection*{Part E Solutions: SAT-Style Applications}
\begin{enumerate}
    \setcounter{enumi}{20}
    \item In \(y=25x+100\), the slope \(25\) is the \boxed{\text{cost per day}}.
    \item Slope \(\dfrac{80-50}{10-0}=\dfrac{30}{10}=\boxed{3}\). Interpretation: revenue increases by \(\$3\) per unit of time.
    \item \(\dfrac{85-70}{3}=\dfrac{15}{3}=\boxed{5\ \text{points per exam}}\)
    \item Slope \(-4\): as \(x\) increases by 1, \(y\) \boxed{decreases by 4}. Negative linear relationship.
    \item \(m_1=3,\ m_2=-\tfrac{1}{3}\Rightarrow 3\cdot(-\tfrac{1}{3})=-1\Rightarrow\boxed{\text{Perpendicular}}\)
\end{enumerate}




\end{document}
