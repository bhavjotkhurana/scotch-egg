\documentclass[14pt]{extarticle}

% ---------- PACKAGES ----------
\usepackage{amsmath, amssymb}
\usepackage{geometry}
\usepackage{setspace}
\usepackage{titlesec}
\usepackage{xcolor}
\usepackage{helvet}
\usepackage{enumitem}

% ---------- PAGE SETUP ----------
\geometry{margin=1in}
\setstretch{1.5}
\setlength{\parindent}{0pt}
\setlength{\parskip}{0.75em}
\setlist{itemsep=0.75\baselineskip, topsep=0.5\baselineskip}
\titleformat{\section}{\normalfont\Large\bfseries}{\thesection}{1em}{}
\titleformat{\subsection}{\normalfont\large\bfseries}{\thesubsection}{1em}{}
\renewcommand{\familydefault}{\sfdefault}
\renewcommand{\emph}[1]{\textbf{#1}}

% ---------- DOCUMENT ----------
\begin{document}
\raggedright
\pagenumbering{gobble}

\begin{center}
    \LARGE \textbf{Unit 1: Linear Relationships and Equations} \\[6pt]
    \Large \textbf{Topic 7: Graphing and Interpreting Linear Models}
\end{center}

\vspace{1em}

\section*{Concept Summary}

Every linear equation represents a straight line on the coordinate plane.  
Understanding how the numbers in an equation affect its graph is key to solving many SAT questions.

\subsection*{Slope–Intercept Form and Graphing}
For a line written as
\[
y = mx + b
\]
\begin{itemize}
    \item \(m\) is the \textbf{slope}, showing how steep the line is.
    \item \(b\) is the \textbf{\(y\)-intercept}, showing where the line crosses the \(y\)-axis.
\end{itemize}

To graph:
1. Plot the \(y\)-intercept \((0, b)\).
2. Use the slope to find another point:  
   rise = change in \(y\), run = change in \(x\).
3. Connect the points with a straight line.

\textbf{Example:}  
For \(y = 2x + 1\):
- \(m = 2\): rise 2, run 1  
- \(b = 1\): start at \((0, 1)\)

\subsection*{Standard Form and Intercepts}
A line in standard form
\[
Ax + By = C
\]
can be graphed quickly using intercepts:
\[
\text{\(x\)-intercept: set } y = 0, \quad \text{\(y\)-intercept: set } x = 0.
\]
\textbf{Example:}  
For \(2x + 3y = 12\):
\[
x\text{-int: } (6, 0), \quad y\text{-int: } (0, 4)
\]

\subsection*{Interpreting Linear Models}
Linear equations often model real-world relationships, such as cost, distance, or temperature over time.

\begin{itemize}
    \item The \textbf{slope} represents the \textbf{rate of change} (how much one quantity changes per unit of another).
    \item The \textbf{\(y\)-intercept} represents the \textbf{starting value} when \(x = 0\).
\end{itemize}

\textbf{Example:}  
A taxi fare is given by \(C = 2.5x + 4\), where \(C\) is cost and \(x\) is miles.
\[
\text{slope } 2.5 \Rightarrow \text{cost per mile}, \quad \text{intercept } 4 \Rightarrow \text{base fee}.
\]

\subsection*{Intersections of Lines}
The intersection point of two lines represents the \textbf{solution to both equations}.  
It can be found by solving the system of equations simultaneously.

\textbf{Example:}
\[
\begin{cases}
y = 2x + 3 \\
y = -x + 9
\end{cases}
\]
Set equal:
\[
2x + 3 = -x + 9 \Rightarrow 3x = 6 \Rightarrow x = 2, \quad y = 7
\]
\[
\boxed{(2, 7)}
\]
Interpretation: both models have the same value at \(x = 2\).

\section*{Key Takeaways}
\begin{itemize}
    \item In \(y = mx + b\), \(m\) controls steepness and \(b\) controls vertical position.
    \item The intersection of two lines represents the solution to a system.
    \item On the SAT, slope often represents a rate (like cost per mile or speed).
    \item Understand how changes in \(m\) and \(b\) shift the graph:
    \begin{itemize}
        \item Increasing \(m\): steeper line
        \item Increasing \(b\): line shifts upward
    \end{itemize}
\end{itemize}

\newpage

% ============================================================
% QUESTIONS — TOPIC 7: GRAPHING AND INTERPRETING LINEAR MODELS
% ============================================================

\section*{Practice Questions: Graphing and Interpreting Linear Models}

\subsection*{Part A: Slope and Intercept Identification}
\begin{enumerate}
    \item Identify the slope and \(y\)-intercept of \(y = 3x - 4\).
    \item Identify the slope and \(y\)-intercept of \(y = -\tfrac{1}{2}x + 6\).
    \item Identify the slope and \(y\)-intercept of \(y = 5 - 2x\).
    \item Write the slope and \(y\)-intercept of \(2y = 8x + 10\).
    \item Which of the following lines is steeper: \(y = 4x + 1\) or \(y = 2x + 3\)?
\end{enumerate}

\subsection*{Part B: Finding and Using Intercepts}
\begin{enumerate}
    \setcounter{enumi}{5}
    \item Find the \(x\)- and \(y\)-intercepts of \(2x + y = 8\).
    \item Find the \(x\)- and \(y\)-intercepts of \(3x + 6y = 12\).
    \item Find the \(x\)-intercept of \(4x - 2y = 12\).
    \item Find the \(y\)-intercept of \(5x + 10y = 20\).
    \item Determine whether the line \(4x + 2y = 8\) passes through the point \((0,4)\).
\end{enumerate}

\subsection*{Part C: Effects of Slope and Intercept Changes}
\begin{enumerate}
    \setcounter{enumi}{10}
    \item Compare the graphs of \(y = 2x + 3\) and \(y = 2x - 1\).  
          How are they different?
    \item Compare the graphs of \(y = 3x + 2\) and \(y = -3x + 2\).  
          How are they different?
    \item How does increasing \(m\) in \(y = mx + b\) affect the steepness of the line?
    \item How does increasing \(b\) in \(y = mx + b\) affect the graph’s position?
    \item The line \(y = 4x + 2\) is shifted downward by 5 units.  
          Write the new equation.
\end{enumerate}

\subsection*{Part D: Intersections and Systems}
\begin{enumerate}
    \setcounter{enumi}{15}
    \item Find the intersection of \(y = x + 3\) and \(y = -x + 7\).
    \item Find the intersection of \(y = 2x + 4\) and \(y = -x + 1\).
    \item The lines \(y = 3x - 2\) and \(y = 3x + 5\) — do they intersect? Explain.
    \item The lines \(y = -2x + 8\) and \(2x + y = 4\) — find their intersection point.
    \item Find the \(x\)-value where \(y = -x + 10\) and \(y = 2x + 1\) have the same output.
\end{enumerate}

\subsection*{Part E: SAT-Style Applications}
\begin{enumerate}
    \setcounter{enumi}{20}
    \item A cell phone plan costs \$40 per month plus \$10 per gigabyte of data.  
          Write the equation for the total cost \(C\) in terms of data \(g\).
    \item A car’s value decreases from \$20,000 to \$14,000 over 3 years.  
          Write a linear equation for value \(V\) as a function of years \(t\).
    \item The temperature is 50°F at 6 AM and 62°F at noon.  
          Write the linear model for temperature \(T\) as a function of time \(t\) (hours after 6 AM).
    \item Two water tanks fill at different rates.  
          Tank A: \(y = 4x + 10\); Tank B: \(y = 6x + 2\).  
          Find when both have the same amount of water.
    \item A line passes through the origin with slope \(-\tfrac{3}{4}\).  
          Write its equation and describe the direction of the line.
\end{enumerate}

\newpage

% ============================================================
% SOLUTIONS — TOPIC 7: GRAPHING AND INTERPRETING LINEAR MODELS
% ============================================================

\section*{Answer Key and Solutions: Graphing and Interpreting Linear Models}

\subsection*{Part A Solutions: Slope and Intercept Identification}
\begin{enumerate}
    \item \(y=3x-4:\ m=\boxed{3},\ b=\boxed{-4}\).
    \item \(y=-\tfrac12 x+6:\ m=\boxed{-\tfrac12},\ b=\boxed{6}\).
    \item \(y=5-2x=-2x+5:\ m=\boxed{-2},\ b=\boxed{5}\).
    \item \(2y=8x+10\Rightarrow y=4x+5:\ m=\boxed{4},\ b=\boxed{5}\).
    \item Compare \(|m|\): \(4\) vs \(2\). Steeper: \(\boxed{y=4x+1}\).
\end{enumerate}

\subsection*{Part B Solutions: Finding and Using Intercepts}
\begin{enumerate}
    \setcounter{enumi}{5}
    \item \(2x+y=8:\ x\text{-int }(4,0),\ y\text{-int }(0,8)\).
    \item \(3x+6y=12:\ x\text{-int }(4,0),\ y\text{-int }(0,2)\).
    \item \(4x-2y=12:\) set \(y=0\Rightarrow 4x=12\Rightarrow x=3\Rightarrow \boxed{(3,0)}\).
    \item \(5x+10y=20:\) set \(x=0\Rightarrow 10y=20\Rightarrow y=2\Rightarrow \boxed{(0,2)}\).
    \item \(4(0)+2(4)=8\) is true \(\Rightarrow\) \boxed{Yes, it passes through }(0,4).
\end{enumerate}

\subsection*{Part C Solutions: Effects of Slope and Intercept Changes}
\begin{enumerate}
    \setcounter{enumi}{10}
    \item Same slope \(m=2\Rightarrow\) parallel; \(y\)-intercepts differ by \(4\). \(y=2x-1\) is \(\boxed{4}\) units below \(y=2x+3\).
    \item Same \(y\)-intercept \(2\); slopes \(3\) and \(-3\): one rises, one falls; \(\boxed{\text{not perpendicular}}\) (product \(-9\)).
    \item Increasing \(|m|\) makes the line \(\boxed{\text{steeper}}\).
    \item Increasing \(b\) shifts the line \(\boxed{\text{up}}\) (vertical translation).
    \item Down by \(5:\ y=4x+2-5=\boxed{y=4x-3}\).
\end{enumerate}

\subsection*{Part D Solutions: Intersections and Systems}
\begin{enumerate}
    \setcounter{enumi}{15}
    \item \(x+3=-x+7\Rightarrow 2x=4\Rightarrow x=2,\ y=5\Rightarrow \boxed{(2,5)}\).
    \item \(2x+4=-x+1\Rightarrow 3x=-3\Rightarrow x=-1,\ y=2\Rightarrow \boxed{(-1,2)}\).
    \item Same slope \(m=3\), different intercepts \(\Rightarrow\) \boxed{parallel, no intersection}.
    \item \(y=-2x+8\) and \(y=4-2x\): setting equal gives \(8=4\) (false) \(\Rightarrow\) \boxed{no solution (parallel)}.
    \item \(-x+10=2x+1\Rightarrow 3x=9\Rightarrow \boxed{x=3}\).
\end{enumerate}

\subsection*{Part E Solutions: SAT-Style Applications}
\begin{enumerate}
    \setcounter{enumi}{20}
    \item Cost model: \(\boxed{C=10g+40}\) (slope = \$10/GB, intercept = \$40).
    \item Slope \(\frac{14000-20000}{3}=-2000\); initial \(20000\Rightarrow \boxed{V=-2000t+20000}\).
    \item Points \((0,50)\) and \((6,62)\): \(m=\frac{62-50}{6}=2\Rightarrow \boxed{T=2t+50}\).
    \item Solve \(4x+10=6x+2\Rightarrow -2x=-8\Rightarrow x=4\). Amount \(y=4(4)+10=26\Rightarrow \boxed{x=4,\ y=26}\).
    \item Through origin, slope \(-\tfrac34\Rightarrow \boxed{y=-\tfrac34 x}\). The line \(\boxed{\text{decreases}}:\) down \(3\) for right \(4\).
\end{enumerate}


\end{document}
