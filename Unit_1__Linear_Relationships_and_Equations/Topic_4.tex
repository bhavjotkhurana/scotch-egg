\documentclass[14pt]{extarticle}

% ---------- PACKAGES ----------
\usepackage{amsmath, amssymb}
\usepackage{geometry}
\usepackage{setspace}
\usepackage{titlesec}
\usepackage{xcolor}
\usepackage{helvet}
\usepackage{enumitem}

% ---------- PAGE SETUP ----------
\geometry{margin=1in}
\setstretch{1.5}
\setlength{\parindent}{0pt}
\setlength{\parskip}{0.75em}
\setlist{itemsep=0.75\baselineskip, topsep=0.5\baselineskip}
\titleformat{\section}{\normalfont\Large\bfseries}{\thesection}{1em}{}
\titleformat{\subsection}{\normalfont\large\bfseries}{\thesubsection}{1em}{}
\renewcommand{\familydefault}{\sfdefault}
\renewcommand{\emph}[1]{\textbf{#1}}

% ---------- DOCUMENT ----------
\begin{document}
\raggedright
\pagenumbering{gobble}

\begin{center}
    \LARGE \textbf{Unit 1: Linear Relationships and Equations} \\[6pt]
    \Large \textbf{Topic 4: Linear Inequalities (Including Compound and Absolute Value)}
\end{center}

\vspace{1em}

\section*{Concept Summary}

An \textbf{inequality} shows a relationship where two expressions are not necessarily equal.  
Instead, one side is greater or less than the other.

\[
\begin{aligned}
a < b &\quad \text{means } a \text{ is less than } b \\
a > b &\quad \text{means } a \text{ is greater than } b \\
a \leq b &\quad \text{means } a \text{ is less than or equal to } b \\
a \geq b &\quad \text{means } a \text{ is greater than or equal to } b
\end{aligned}
\]

Solving inequalities is very similar to solving equations — the same operations can be performed on both sides.  
However, there is one key rule to remember:

\[
\textbf{When you multiply or divide by a negative number, you must flip the inequality sign.}
\]

\section*{Compound Inequalities}

Sometimes two inequalities are joined by the words \textbf{and} or \textbf{or}.

\begin{itemize}
    \item “\textbf{and}” means both conditions must be true — the overlap of the solution sets.
    \item “\textbf{or}” means either condition can be true — the combined solution set.
\end{itemize}

Example of an “and” compound inequality:
\[
-2 < x \leq 5
\]
This means \(x\) is greater than \(-2\) \emph{and} less than or equal to \(5\).

\section*{Absolute Value Equations and Inequalities}

The \textbf{absolute value} of a number represents its distance from 0 on the number line.  
Distance is always positive.

\[
|x| = 
\begin{cases}
x, & x \ge 0 \\
-x, & x < 0
\end{cases}
\]

\begin{itemize}
    \item If \(|x| = a\), then \(x = a\) or \(x = -a\).
    \item If \(|x| < a\), then \(-a < x < a\).
    \item If \(|x| > a\), then \(x < -a\) or \(x > a\).
\end{itemize}

\section*{Core Skills}
\begin{itemize}
    \item Apply the same operations to both sides of an inequality.
    \item Reverse the inequality symbol when multiplying or dividing by a negative.
    \item Solve and represent compound inequalities.
    \item Split absolute value inequalities into two separate linear inequalities.
\end{itemize}

\section*{Example 1: Solving a Simple Inequality}

Solve for \(x\):
\[
2x - 5 < 7
\]

\textbf{Step 1: Add 5 to both sides.}
\[
2x < 12
\]

\textbf{Step 2: Divide by 2.}
\[
x < 6
\]

\textbf{Final Answer:} \(\boxed{x < 6}\)

\textbf{Graph:} Shade all numbers less than 6 on the number line (open circle at 6).

\section*{Example 2: Absolute Value Inequality}

Solve for \(x\):
\[
|x - 3| \le 5
\]

\textbf{Step 1: Write as a compound inequality.}
\[
-5 \le x - 3 \le 5
\]

\textbf{Step 2: Add 3 to all sides.}
\[
-2 \le x \le 8
\]

\textbf{Final Answer:} \(\boxed{-2 \le x \le 8}\)

\textbf{Interpretation:} All values of \(x\) that are within 5 units of 3 satisfy the inequality.

\section*{Key Takeaways}
\begin{itemize}
    \item Solving inequalities follows the same steps as solving equations, except when multiplying or dividing by a negative number - always flip the sign.
    \item Compound inequalities combine solution sets with “and” or “or”.
    \item Absolute value inequalities describe distances from a central point on the number line.
\end{itemize}
\newpage

% ============================================================
% QUESTIONS — TOPIC 4: LINEAR INEQUALITIES (INCLUDING COMPOUND AND ABSOLUTE VALUE)
% ============================================================

\section*{Practice Questions: Linear Inequalities (Including Compound and Absolute Value)}

\subsection*{Part A: Basic Inequalities}
\begin{enumerate}
    \item Solve for \(x\): \(x + 5 < 9\)
    \item Solve for \(x\): \(2x - 3 \ge 7\)
    \item Solve for \(x\): \(5x + 4 < 19\)
    \item Solve for \(x\): \(4x - 8 > 0\)
    \item Solve for \(x\): \(-3x + 2 \le 11\)
\end{enumerate}

\subsection*{Part B: Inequalities with Negative Coefficients}
\begin{enumerate}
    \setcounter{enumi}{5}
    \item Solve for \(x\): \(-2x + 5 > 9\)
    \item Solve for \(x\): \(-4x - 8 < 0\)
    \item Solve for \(x\): \(-3x + 7 \le 1\)
    \item Solve for \(x\): \(-6x > 12\)
    \item Solve for \(x\): \(-5x + 4 \ge 9\)
\end{enumerate}

\subsection*{Part C: Compound Inequalities}
\begin{enumerate}
    \setcounter{enumi}{10}
    \item Solve for \(x\): \(2 < x + 5 \le 9\)
    \item Solve for \(x\): \(-3 \le 2x - 1 < 5\)
    \item Solve for \(x\): \(x - 4 > 2\) or \(x + 1 < 0\)
    \item Solve for \(x\): \(3x + 2 \le 8\) and \(x - 1 > 0\)
    \item Solve for \(x\): \(x + 2 > 6\) or \(x - 3 < -2\)
\end{enumerate}

\subsection*{Part D: Absolute Value Equations and Inequalities}
\begin{enumerate}
    \setcounter{enumi}{15}
    \item Solve for \(x\): \(|x| = 4\)
    \item Solve for \(x\): \(|x - 3| = 7\)
    \item Solve for \(x\): \(|x + 2| < 5\)
    \item Solve for \(x\): \(|2x - 4| \le 6\)
    \item Solve for \(x\): \(|3x + 1| > 7\)
\end{enumerate}

\subsection*{Part E: SAT-Style Word and Application Problems}
\begin{enumerate}
    \setcounter{enumi}{20}
    \item A phone plan costs \$20 per month plus \$0.10 per text message.  
    If a customer’s bill must stay under \$50, write and solve an inequality for the number of messages \(t\).
    \item The inequality \(5x + 20 \le 70\) represents a budget constraint. What is the greatest possible value of \(x\)?
    \item The temperature \(T\) (in °F) must stay within 8 degrees of 72°F. Write an absolute value inequality that represents this situation.
    \item The length \(L\) of a rod must be within 0.5 cm of 10 cm. Write and solve an inequality for \(L\).
    \item The profit \(P\) from selling \(x\) products is given by \(P = 15x - 120\).  
    The company wants at least \$60 profit. Write and solve an inequality for \(x\).
\end{enumerate}

\newpage

% ============================================================
% SOLUTIONS — TOPIC 4: LINEAR INEQUALITIES (INCLUDING COMPOUND AND ABSOLUTE VALUE)
% ============================================================

\section*{Answer Key and Solutions: Linear Inequalities (Including Compound and Absolute Value)}

\subsection*{Part A Solutions: Basic Inequalities}
\begin{enumerate}
    \item \(x + 5 < 9 \Rightarrow x < \boxed{4}\)
    \item \(2x - 3 \ge 7 \Rightarrow 2x \ge 10 \Rightarrow x \ge \boxed{5}\)
    \item \(5x + 4 < 19 \Rightarrow 5x < 15 \Rightarrow x < \boxed{3}\)
    \item \(4x - 8 > 0 \Rightarrow 4x > 8 \Rightarrow x > \boxed{2}\)
    \item \(-3x + 2 \le 11 \Rightarrow -3x \le 9 \Rightarrow x \ge \boxed{-3}\) \quad (flip sign when dividing by \(-3\))
\end{enumerate}

\subsection*{Part B Solutions: Inequalities with Negative Coefficients}
\begin{enumerate}
    \setcounter{enumi}{5}
    \item \(-2x + 5 > 9 \Rightarrow -2x > 4 \Rightarrow x < \boxed{-2}\)
    \item \(-4x - 8 < 0 \Rightarrow -4x < 8 \Rightarrow x > \boxed{-2}\)
    \item \(-3x + 7 \le 1 \Rightarrow -3x \le -6 \Rightarrow x \ge \boxed{2}\)
    \item \(-6x > 12 \Rightarrow x < \boxed{-2}\)
    \item \(-5x + 4 \ge 9 \Rightarrow -5x \ge 5 \Rightarrow x \le \boxed{-1}\)
\end{enumerate}

\subsection*{Part C Solutions: Compound Inequalities}
\begin{enumerate}
    \setcounter{enumi}{10}
    \item \(2 < x + 5 \le 9 \Rightarrow -3 < x \le \boxed{4}\)
    \item \(-3 \le 2x - 1 < 5 \Rightarrow -2 \le 2x < 6 \Rightarrow -1 \le x < \boxed{3}\)
    \item \(x - 4 > 2\) or \(x + 1 < 0 \Rightarrow x > 6\) or \(x < \boxed{-1}\)
    \item \(3x + 2 \le 8\) and \(x - 1 > 0 \Rightarrow x \le 2\) and \(x > 1\). Combine: \( \boxed{1 < x \le 2}\)
    \item \(x + 2 > 6\) or \(x - 3 < -2 \Rightarrow x > 4\) or \(x < \boxed{1}\)
\end{enumerate}

\subsection*{Part D Solutions: Absolute Value Equations and Inequalities}
\begin{enumerate}
    \setcounter{enumi}{15}
    \item \(|x| = 4 \Rightarrow x = \boxed{4} \text{ or } x = \boxed{-4}\)
    \item \(|x - 3| = 7 \Rightarrow x - 3 = 7 \text{ or } x - 3 = -7 \Rightarrow x = \boxed{10} \text{ or } x = \boxed{-4}\)
    \item \(|x + 2| < 5 \Rightarrow -5 < x + 2 < 5 \Rightarrow \boxed{-7 < x < 3}\)
    \item \(|2x - 4| \le 6 \Rightarrow -6 \le 2x - 4 \le 6 \Rightarrow -2 \le 2x \le 10 \Rightarrow \boxed{-1 \le x \le 5}\)
    \item \(|3x + 1| > 7 \Rightarrow 3x + 1 > 7 \text{ or } 3x + 1 < -7 \Rightarrow x > 2 \text{ or } x < \boxed{-\tfrac{8}{3}}\)
\end{enumerate}

\subsection*{Part E Solutions: SAT-Style Word and Application Problems}
\begin{enumerate}
    \setcounter{enumi}{20}
    \item Cost model: \(20 + 0.10t < 50 \Rightarrow 0.10t < 30 \Rightarrow t < \boxed{300}\). For whole texts, \(t \le 299\).
    \item \(5x + 20 \le 70 \Rightarrow 5x \le 50 \Rightarrow x \le \boxed{10}\). Greatest possible value is \(10\).
    \item Temperatures within 8 of 72: \(\boxed{|T - 72| \le 8}\).
    \item Length within 0.5 of 10: \(|L - 10| \le 0.5 \Rightarrow \boxed{9.5 \le L \le 10.5}\).
    \item Profit at least 60: \(15x - 120 \ge 60 \Rightarrow 15x \ge 180 \Rightarrow x \ge \boxed{12}\).
\end{enumerate}


\end{document}
