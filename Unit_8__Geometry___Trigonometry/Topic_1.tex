\documentclass[12pt]{article}

\usepackage{amsmath, amssymb}
\usepackage{geometry}
\usepackage{setspace}
\usepackage{titlesec}
\usepackage{lmodern}
\usepackage{xcolor}
\usepackage{enumitem}

\geometry{margin=1in}
\setstretch{1.2}
\titleformat{\section}{\normalfont\Large\bfseries}{\thesection}{1em}{}
\titleformat{\subsection}{\normalfont\large\bfseries}{\thesubsection}{1em}{}
\pagenumbering{gobble}

\begin{document}

\begin{center}
    \LARGE \textbf{Unit 8: Geometry and Trigonometry} \\[6pt]
    \Large \textbf{Topic 1: Lines and Angles — Parallel, Perpendicular, and Transversals}
\end{center}

\vspace{1em}

\section*{Concept Summary}

Lines and angles are foundational to all geometry problems on the SAT.  
A \textbf{line} extends infinitely in both directions, while a \textbf{ray} has one endpoint and extends in one direction.  
When two lines meet, they form \textbf{angles} — measured in degrees.

A \textbf{transversal} is a line that cuts across two or more other lines.  
If those lines are parallel, several angle relationships form:

\[
\begin{array}{c|l}
\textbf{Angle Pair} & \textbf{Relationship} \\ \hline
\text{Corresponding Angles} & \text{Equal (congruent)} \\
\text{Alternate Interior Angles} & \text{Equal (congruent)} \\
\text{Alternate Exterior Angles} & \text{Equal (congruent)} \\
\text{Consecutive Interior Angles} & \text{Supplementary (sum = 180°)}
\end{array}
\]

\textbf{Perpendicular lines} intersect to form right angles (90°).  
If two lines are perpendicular, their slopes multiply to \(-1\):  
\[
m_1 \cdot m_2 = -1.
\]

\textbf{Parallel lines} never meet and have the same slope:  
\[
m_1 = m_2.
\]

(Insert diagram note: two parallel lines cut by a transversal showing corresponding, alternate interior, and alternate exterior angles.)

\section*{Core Skills}
\begin{itemize}
  \item Identify angle pairs formed by a transversal.
  \item Use properties of parallel and perpendicular lines to find unknown angles.
  \item Apply the sum of angles on a line (180°) and at a point (360°).
  \item Relate slopes of parallel and perpendicular lines in coordinate geometry.
  \item Use logical reasoning to find missing measures in angle diagrams.
\end{itemize}

\section*{Example 1: Parallel Lines and Transversal}

Lines \(l_1\) and \(l_2\) are parallel, and transversal \(t\) crosses them.  
If one acute angle measures \(65^\circ,\) find its alternate interior angle.

Alternate interior angles are congruent:
\[
\boxed{65^\circ.}
\]

\section*{Example 2: Supplementary Angles}

When two parallel lines are cut by a transversal, a pair of consecutive interior angles are supplementary.

If one angle is \(110^\circ,\)
\[
180 - 110 = 70.
\]
\(\boxed{70^\circ.}\)

\section*{Example 3: Perpendicular Lines}

Two lines intersect perpendicularly. One angle measures \(x + 15^\circ,\) and the adjacent angle measures \(3x - 45^\circ.\)

Set their sum equal to \(90^\circ:\)
\[
(x + 15) + (3x - 45) = 90 \Rightarrow 4x - 30 = 90 \Rightarrow 4x = 120 \Rightarrow x = 30.
\]

\(\boxed{x = 30}\)

\section*{Example 4: Slopes of Parallel and Perpendicular Lines}

If a line has equation \(y = 2x + 3,\)
\begin{itemize}
  \item A parallel line has slope \(m = 2.\)
  \item A perpendicular line has slope \(m = -\frac{1}{2}.\)
\end{itemize}

\(\boxed{\text{Parallel: } y = 2x + b, \quad \text{Perpendicular: } y = -\tfrac{1}{2}x + b.}\)

\section*{Example 5: Angle Relationships at a Point}

At a point, four angles form. Two adjacent ones are \(x\) and \(2x + 30.\)  
Since angles around a point sum to 360°:
\[
2(x + 2x + 30) = 360 \Rightarrow 6x + 60 = 360 \Rightarrow 6x = 300 \Rightarrow x = 50.
\]

\(\boxed{x = 50^\circ}\)

\section*{Example 6: SAT-Style Interpretation}

In the coordinate plane, line \(k\) has slope \(3.\)  
A line perpendicular to \(k\) passes through \((1, 4)\).  
Find its equation.

Slope of perpendicular line = \(-\frac{1}{3}\).  
Point-slope form:
\[
y - 4 = -\tfrac{1}{3}(x - 1) \Rightarrow y = -\tfrac{1}{3}x + \tfrac{1}{3} + 4 = -\tfrac{1}{3}x + \tfrac{13}{3}.
\]

\(\boxed{y = -\tfrac{1}{3}x + \tfrac{13}{3}}\)

\section*{Key Takeaways}
\begin{itemize}
  \item Parallel lines have equal slopes; perpendicular lines’ slopes multiply to \(-1.\)
  \item Corresponding, alternate interior, and alternate exterior angles are congruent for parallel lines.
  \item Consecutive interior angles are supplementary.
  \item Angles on a straight line sum to 180°; angles around a point sum to 360°.
  \item Always mark known relationships on diagrams before solving.
\end{itemize}

\newpage

% ============================================================
% QUESTIONS — UNIT 8, TOPIC 1: LINES AND ANGLES (PARALLEL, PERPENDICULAR, TRANSVERSALS)
% ============================================================

\section*{Practice Questions: Lines and Angles — Parallel, Perpendicular, and Transversals}

\subsection*{Part A: Angle Relationships (Parallel Lines and Transversals)}
(Diagram needed: two parallel lines cut by a transversal.)

\begin{enumerate}
  \item Two parallel lines are cut by a transversal. One angle measures \(65^\circ.\) Find the measure of each of the other seven angles.
  \item If two lines are parallel and a transversal forms one angle of \(120^\circ,\) what is the measure of the alternate interior angle?
  \item Lines \(l_1\) and \(l_2\) are parallel. If a consecutive interior angle is \(x + 20\) and its partner is \(2x - 10,\) find \(x.\)
  \item When a transversal cuts parallel lines, one corresponding angle measures \(3x + 15,\) and its alternate exterior angle measures \(6x - 30.\) Find \(x.\)
  \item A transversal cuts two parallel lines. If one acute angle is \(55^\circ,\) what is the measure of each obtuse angle?
\end{enumerate}

\subsection*{Part B: Angles on a Line and Around a Point}
(Diagram needed: intersecting lines forming vertical angles.)

\begin{enumerate}
  \setcounter{enumi}{5}
  \item Two supplementary angles have measures \(x\) and \(2x - 30.\) Find \(x.\)
  \item Two vertical angles are given as \(3x - 20\) and \(x + 10.\) Find the measure of each angle.
  \item Angles on a straight line are \(4x\) and \(2x + 12.\) Find \(x.\)
  \item Four angles around a point have measures \(x, 2x, 3x,\) and \(4x.\) Find each angle measure.
  \item One angle is twice another, and together they form a right angle. Find both angles.
\end{enumerate}

\subsection*{Part C: Parallel and Perpendicular Slopes (Coordinate Geometry)}
(Diagram will make it better: two lines on a coordinate plane showing slopes.)

\begin{enumerate}
  \setcounter{enumi}{10}
  \item Find the slope of a line parallel to \(y = 3x + 5.\)
  \item Find the slope of a line perpendicular to \(y = -\frac{2}{3}x + 4.\)
  \item Write the equation of a line parallel to \(y = 4x - 7\) that passes through \((0, 2).\)
  \item Write the equation of a line perpendicular to \(y = -\tfrac{1}{2}x + 3\) that passes through \((4, 1).\)
  \item Determine if the lines \(y = 2x + 3\) and \(2y = 4x - 5\) are parallel, perpendicular, or neither.
\end{enumerate}

\subsection*{Part D: Mixed Reasoning (Angles + Slope Logic)}
(Diagram will make it better: intersecting transversals with labeled angles.)

\begin{enumerate}
  \setcounter{enumi}{15}
  \item Lines \(l_1\) and \(l_2\) are parallel. A transversal makes an angle of \(70^\circ\) with \(l_1.\) What is the alternate exterior angle with \(l_2?\)
  \item Two perpendicular lines intersect. One angle is labeled \(2x + 10.\) Find \(x.\)
  \item A line has slope \(m = 3.\) What is the slope of any line perpendicular to it?
  \item In a diagram, one angle of a triangle formed by two intersecting lines and a transversal is \(40^\circ,\) and another is \(70^\circ.\) Find the third angle.
  \item The line \(k\) passes through \((2, 3)\) and has slope 4. Find the equation of a line perpendicular to \(k\) passing through the origin.
\end{enumerate}

\subsection*{Part E: SAT-Style Applications}
(Diagram needed: typical SAT parallel-line transversal setup.)

\begin{enumerate}
  \setcounter{enumi}{20}
  \item In the figure, lines \(a\) and \(b\) are parallel, and a transversal \(t\) intersects them. If one angle measures \(x + 25\) and the alternate interior angle measures \(3x - 15,\) find \(x.\)
  \item The lines \(y = 2x + 1\) and \(y = 2x - 4\) are parallel. What is the distance between them conceptually dependent on?
  \item A line perpendicular to \(y = \tfrac{1}{2}x - 6\) passes through \((2, 3)\). Find its equation.
  \item If two lines are perpendicular, what is true about the product of their slopes?
  \item Lines \(l_1\) and \(l_2\) are parallel. A transversal cuts them creating an angle of \(110^\circ.\) What is the measure of the angle supplementary to its corresponding angle?
\end{enumerate}

\newpage

% ============================================================
% SOLUTIONS — UNIT 8, TOPIC 1: LINES AND ANGLES (PARALLEL, PERPENDICULAR, TRANSVERSALS)
% ============================================================

\section*{Answer Key and Solutions: Lines and Angles}

\subsection*{Part A Solutions: Angle Relationships (Parallel Lines and Transversals)}
(Diagram needed)

\begin{enumerate}
  \item One angle is \(65^\circ.\)

  In a parallel-line transversal setup:
  \begin{itemize}
    \item All acute angles are equal.
    \item All obtuse angles are equal.
    \item Any acute + any obtuse = 180°.
  \end{itemize}

  So all acute angles are \(65^\circ.\)  
  All obtuse angles are \(180 - 65 = 115^\circ.\)

  \(\boxed{\text{Angles are }65^\circ \text{ or }115^\circ}\)

  \item If one angle is \(120^\circ,\) then its alternate interior angle is congruent.

  \(\boxed{120^\circ}\)

  \item Consecutive interior angles are supplementary:
  \[
  (x + 20) + (2x - 10) = 180
  \Rightarrow 3x + 10 = 180
  \Rightarrow 3x = 170
  \Rightarrow x = \frac{170}{3}.
  \]

  \(\boxed{x = \tfrac{170}{3}}\)

  \item For parallel lines, corresponding angles = alternate exterior angles.  
  So:
  \[
  3x + 15 = 6x - 30
  \Rightarrow 45 = 3x
  \Rightarrow x = 15.
  \]

  \(\boxed{x = 15}\)

  \item If an acute angle is \(55^\circ,\) each obtuse angle is its supplement:
  \[
  180 - 55 = 125^\circ.
  \]

  \(\boxed{\text{Obtuse angles are }125^\circ}\)
\end{enumerate}

\subsection*{Part B Solutions: Angles on a Line and Around a Point}
(Diagram needed)

\begin{enumerate}
  \setcounter{enumi}{5}
  \item Supplementary angles sum to 180°:
  \[
  x + (2x - 30) = 180
  \Rightarrow 3x - 30 = 180
  \Rightarrow 3x = 210
  \Rightarrow x = 70.
  \]

  \(\boxed{x = 70}\)

  The angles are \(70^\circ\) and \(110^\circ.\)

  \item Vertical angles are equal:
  \[
  3x - 20 = x + 10
  \Rightarrow 2x = 30
  \Rightarrow x = 15.
  \]

  Each angle:
  \[
  3x - 20 = 3(15) - 20 = 45 - 20 = 25^\circ.
  \]

  \(\boxed{25^\circ}\)

  \item Angles on a straight line sum to 180°:
  \[
  4x + (2x + 12) = 180
  \Rightarrow 6x + 12 = 180
  \Rightarrow 6x = 168
  \Rightarrow x = 28.
  \]

  \(\boxed{x = 28}\)

  \item Angles around a point sum to 360°:
  \[
  x + 2x + 3x + 4x = 360
  \Rightarrow 10x = 360
  \Rightarrow x = 36.
  \]

  Then:
  \[
  x = 36^\circ,\;
  2x = 72^\circ,\;
  3x = 108^\circ,\;
  4x = 144^\circ.
  \]

  \(\boxed{36^\circ,\;72^\circ,\;108^\circ,\;144^\circ}\)

  \item Let the smaller angle be \(a.\) The larger is \(2a.\)  
  They form a right angle, so:
  \[
  a + 2a = 90
  \Rightarrow 3a = 90
  \Rightarrow a = 30.
  \]

  So the angles are \(30^\circ\) and \(60^\circ.\)

  \(\boxed{30^\circ \text{ and } 60^\circ}\)
\end{enumerate}

\subsection*{Part C Solutions: Parallel and Perpendicular Slopes (Coordinate Geometry)}
(Diagram will make it better)

\begin{enumerate}
  \setcounter{enumi}{10}
  \item Line parallel to \(y = 3x + 5\) has the same slope.

  \(\boxed{m = 3}\)

  \item Perpendicular slope is the negative reciprocal.  
  Given slope is \(-\tfrac{2}{3}.\)

  Negative reciprocal of \(-\tfrac{2}{3}\) is \(\tfrac{3}{2}.\)

  \(\boxed{m = \tfrac{3}{2}}\)

  \item A line parallel to \(y = 4x - 7\) has slope 4.  
  Use point \((0,2)\). Point-slope or slope-intercept:

  Since it passes through \((0,2)\), \(b = 2\). So:
  \[
  y = 4x + 2.
  \]

  \(\boxed{y = 4x + 2}\)

  \item Line perpendicular to \(y = -\tfrac{1}{2}x + 3\):

  Given slope is \(-\tfrac{1}{2}.\)  
  Perpendicular slope is 2 (negative reciprocal).

  Through \((4,1)\):
  \[
  y - 1 = 2(x - 4)
  \Rightarrow y - 1 = 2x - 8
  \Rightarrow y = 2x - 7.
  \]

  \(\boxed{y = 2x - 7}\)

  \item Compare \(y = 2x + 3\) and \(2y = 4x - 5.\)

  Rewrite the second:
  \[
  2y = 4x - 5 \Rightarrow y = 2x - \frac{5}{2}.
  \]

  Both have slope 2. Same slope, different intercepts.

  \(\boxed{\text{Parallel}}\)
\end{enumerate}

\subsection*{Part D Solutions: Mixed Reasoning}
(Diagram will make it better)

\begin{enumerate}
  \setcounter{enumi}{15}
  \item For parallel lines, alternate exterior angles are congruent.  
  Given \(70^\circ,\) the alternate exterior angle is also \(70^\circ.\)

  \(\boxed{70^\circ}\)

  \item Two perpendicular lines meet at 90°.  
  One angle is \(2x + 10.\) Set equal to \(90^\circ:\)
  \[
  2x + 10 = 90
  \Rightarrow 2x = 80
  \Rightarrow x = 40.
  \]

  \(\boxed{x = 40}\)

  \item If a line has slope 3, any perpendicular line has slope \(-\tfrac{1}{3}.\)

  \(\boxed{-\tfrac{1}{3}}\)

  \item Angles in a triangle sum to 180°.  
  Given two are \(40^\circ\) and \(70^\circ:\)
  \[
  40 + 70 + \text{(third angle)} = 180
  \Rightarrow \text{third angle} = 70^\circ.
  \]

  \(\boxed{70^\circ}\)

  \item Line \(k\): slope \(4\).  
  Perpendicular slope: \(-\tfrac{1}{4}.\)  
  We want line through origin \((0,0)\) with slope \(-\tfrac{1}{4}.\)

  Equation:
  \[
  y = -\tfrac{1}{4}x.
  \]

  \(\boxed{y = -\tfrac{1}{4}x}\)
\end{enumerate}

\subsection*{Part E Solutions: SAT-Style Applications}
(Diagram needed)

\begin{enumerate}
  \setcounter{enumi}{20}
  \item Alternate interior angles are equal if lines are parallel:
  \[
  x + 25 = 3x - 15
  \Rightarrow 25 + 15 = 3x - x
  \Rightarrow 40 = 2x
  \Rightarrow x = 20.
  \]

  \(\boxed{x = 20}\)

  \item The distance between two parallel lines of the form \(y = mx + b\) depends on how far apart the intercepts are, measured \textit{perpendicular} to the lines.

  Concept: depends on vertical shift (difference in intercepts) when slopes match.

  \(\boxed{\text{It depends on how far apart the two lines are along a perpendicular}}\)

  \item Line perpendicular to \(y = \tfrac{1}{2}x - 6\) has slope \(-2\) (negative reciprocal).  
  Through \((2,3)\):

  \[
  y - 3 = -2(x - 2)
  \Rightarrow y - 3 = -2x + 4
  \Rightarrow y = -2x + 7.
  \]

  \(\boxed{y = -2x + 7}\)

  \item For perpendicular lines:
  \[
  m_1 \cdot m_2 = -1.
  \]

  \(\boxed{\text{Product of slopes is } -1}\)

  \item Angle given is \(110^\circ.\) The corresponding angle is also \(110^\circ\) (corresponding angles match).  
  Supplementary to that angle:
  \[
  180 - 110 = 70^\circ.
  \]

  \(\boxed{70^\circ}\)
\end{enumerate}


\end{document}
