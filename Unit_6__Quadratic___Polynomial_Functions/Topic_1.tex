\documentclass[12pt]{article}

\usepackage{amsmath, amssymb}
\usepackage{geometry}
\usepackage{setspace}
\usepackage{titlesec}
\usepackage{lmodern}
\usepackage{xcolor}
\usepackage{enumitem}

\geometry{margin=1in}
\setstretch{1.2}
\titleformat{\section}{\normalfont\Large\bfseries}{\thesection}{1em}{}
\titleformat{\subsection}{\normalfont\large\bfseries}{\thesubsection}{1em}{}
\pagenumbering{gobble}

\begin{document}

\begin{center}
    \LARGE \textbf{Unit 6: Quadratic and Polynomial Functions} \\[6pt]
    \Large \textbf{Topic 1: Forms of Quadratic Equations}
\end{center}

\vspace{1em}

\section*{Concept Summary}

A \textbf{quadratic equation} is any equation that can be written in the form:
\[
ax^2 + bx + c = 0, \quad a \ne 0.
\]
Quadratics describe parabolic relationships and appear throughout the SAT in both algebraic and graphical contexts.

There are three main forms of quadratic equations, each revealing different information:

1. \textbf{Standard Form:} \(y = ax^2 + bx + c\)  
   — easiest for identifying \(a, b, c\) and using the quadratic formula.

2. \textbf{Factored Form:} \(y = a(x - r_1)(x - r_2)\)  
   — shows the \textbf{zeros} (x-intercepts) directly: \(x = r_1, r_2.\)

3. \textbf{Vertex Form:} \(y = a(x - h)^2 + k\)  
   — shows the vertex \((h, k)\) and direction of opening (up if \(a > 0\), down if \(a < 0\)).

The three forms are algebraically equivalent, but each highlights different features of the same parabola. Converting between forms is a key SAT skill.

\section*{Core Skills}
\begin{itemize}
  \item Identify \(a\), \(b\), and \(c\) in the standard form.
  \item Factor quadratics to find zeros (convert to factored form).
  \item Complete the square to convert to vertex form.
  \item Recognize how \(a\) affects the parabola’s width and direction.
  \item Interpret graphs and equations interchangeably between forms.
\end{itemize}

\section*{Example 1: Recognizing Forms}

Given \(y = 2x^2 - 8x + 6\):
\begin{itemize}
  \item Standard form: \(y = 2x^2 - 8x + 6\)
  \item Factored form: \(y = 2(x - 1)(x - 3)\)
  \item Vertex form: \(y = 2(x - 2)^2 - 2\)
\end{itemize}

Each form reveals something different:  
• Zeros: \(x = 1, 3\)• Vertex: \((2, -2)\)• Opens upward since \(a = 2 > 0\).

\section*{Example 2: Converting to Vertex Form}

Convert \(y = x^2 - 6x + 8\) to vertex form.

\[
y = (x^2 - 6x) + 8
\]
Complete the square:
\[
(x^2 - 6x + 9) - 9 + 8 = (x - 3)^2 - 1
\]
So:
\[
y = (x - 3)^2 - 1
\]

\textbf{Vertex:} \((3, -1)\), \textbf{opens upward.}

\section*{Example 3: Converting to Factored Form}

Convert \(y = x^2 - 5x + 6\) to factored form.

\[
x^2 - 5x + 6 = (x - 2)(x - 3)
\]
\textbf{Zeros:} \(x = 2, 3.\)

\textbf{Factored form:} \(\boxed{y = (x - 2)(x - 3)}\)

\section*{Example 4: Interpreting a Factored Equation}

Given \(y = -2(x + 1)(x - 4)\):
\begin{itemize}
  \item Zeros: \(x = -1, 4\)
  \item Opens downward (\(a = -2\))
  \item Axis of symmetry halfway between zeros: \(x = \dfrac{-1 + 4}{2} = 1.5\)
\end{itemize}

\textbf{Vertex:} occurs at \(x = 1.5\).

\section*{Example 5: Understanding Coefficient Effects}

Compare \(y = x^2\), \(y = 2x^2\), and \(y = \frac{1}{2}x^2\):
\begin{itemize}
  \item \(a > 1\): narrower parabola.
  \item \(0 < a < 1\): wider parabola.
  \item \(a < 0\): opens downward.
\end{itemize}

\textbf{Key insight:} \(a\) controls vertical stretch and direction.

\section*{Key Takeaways}
\begin{itemize}
  \item Standard form gives coefficients for the quadratic formula.
  \item Factored form gives zeros quickly.
  \item Vertex form gives vertex and direction directly.
  \item All forms describe the same parabola, just from different perspectives.
  \item Practice converting among forms to move fluidly between algebra and graphs.
\end{itemize}

\newpage

% ============================================================
% QUESTIONS — UNIT 6, TOPIC 1: FORMS OF QUADRATIC EQUATIONS
% ============================================================

\section*{Practice Questions: Forms of Quadratic Equations}

\subsection*{Part A: Identifying Forms and Coefficients}
\begin{enumerate}
  \item Identify \(a\), \(b\), and \(c\) in \(y = 3x^2 - 5x + 2.\)
  \item Identify \(a\), \(b\), and \(c\) in \(y = -2x^2 + 7x - 4.\)
  \item Which form is \(y = (x - 4)(x + 1)\)? Standard, vertex, or factored?
  \item Which form is \(y = 2(x + 3)^2 - 5\)? Standard, vertex, or factored?
  \item For \(y = x^2 + 6x + 5\), what is the leading coefficient?
\end{enumerate}

\subsection*{Part B: Converting Between Forms}
\begin{enumerate}
  \setcounter{enumi}{5}
  \item Expand \(y = (x - 2)(x - 5)\) into standard form.
  \item Convert \(y = x^2 - 8x + 16\) into vertex form.
  \item Factor \(y = x^2 - 7x + 10\).
  \item Convert \(y = (x + 2)^2 - 9\) into standard form.
  \item Convert \(y = x^2 - 10x + 21\) into factored form.
\end{enumerate}

\subsection*{Part C: Identifying Features from Each Form}
\begin{enumerate}
  \setcounter{enumi}{10}
  \item For \(y = 2(x - 1)(x - 5)\), find the zeros.
  \item For \(y = -3(x + 4)^2 + 2\), state the vertex and direction of opening.
  \item For \(y = x^2 - 6x + 9\), identify the vertex.
  \item For \(y = (x - 3)(x - 7)\), find the axis of symmetry.
  \item For \(y = 4(x + 2)^2 - 8\), identify the vertex and whether it is a minimum or maximum.
\end{enumerate}

\subsection*{Part D: Structural Reasoning and Graph Relationships}
\begin{enumerate}
  \setcounter{enumi}{15}
  \item The equation \(y = -x^2 + 6x - 5\) opens in which direction?
  \item Compare \(y = x^2\) and \(y = 3x^2\). Which is narrower?
  \item Compare \(y = x^2\) and \(y = \tfrac{1}{2}x^2\). Which is wider?
  \item Write the vertex form of \(y = x^2 + 4x + 1\).
  \item Rewrite \(y = (x - 1)(x - 4)\) in standard form, then find its axis of symmetry.
\end{enumerate}

\subsection*{Part E: Word Problems and SAT-Style Applications}
\begin{enumerate}
  \setcounter{enumi}{20}
  \item A ball is thrown upward with height \(h = -16t^2 + 32t + 48\). Identify \(a\), \(b\), and \(c\).
  \item The path of a rocket is modeled by \(h = -2(x - 5)^2 + 50\). What is the vertex, and what does it represent?
  \item A parabola has zeros at \(x = 2\) and \(x = 6\). Write an equation in factored form if \(a = 1.\)
  \item A parabola has vertex \((3, -4)\) and opens upward with \(a = 2\). Write its equation in vertex form.
  \item A quadratic in standard form has \(a = 1, b = -6, c = 8.\) Find its zeros by factoring.
\end{enumerate}

\newpage

% ============================================================
% SOLUTIONS — UNIT 6, TOPIC 1: FORMS OF QUADRATIC EQUATIONS
% ============================================================

\section*{Answer Key and Solutions: Forms of Quadratic Equations}

\subsection*{Part A Solutions: Identifying Forms and Coefficients}
\begin{enumerate}
  \item In \(y = 3x^2 - 5x + 2\):  
  \(a = 3,\; b = -5,\; c = 2.\)  
  \(\boxed{a=3,\; b=-5,\; c=2}\)

  \item In \(y = -2x^2 + 7x - 4\):  
  \(a = -2,\; b = 7,\; c = -4.\)  
  \(\boxed{a=-2,\; b=7,\; c=-4}\)

  \item \(y = (x - 4)(x + 1)\) is written as a product of linear factors.  
  \(\boxed{\text{Factored form}}\)

  \item \(y = 2(x + 3)^2 - 5\) matches \(y = a(x - h)^2 + k.\)  
  \(\boxed{\text{Vertex form}}\)

  \item For \(y = x^2 + 6x + 5\), the leading coefficient is the coefficient of \(x^2\), which is 1.  
  \(\boxed{1}\)
\end{enumerate}

\subsection*{Part B Solutions: Converting Between Forms}
\begin{enumerate}
  \setcounter{enumi}{5}
  \item Expand \(y = (x - 2)(x - 5)\):
  \[
  y = x^2 - 5x - 2x + 10 = x^2 - 7x + 10.
  \]
  \(\boxed{y = x^2 - 7x + 10}\)

  \item Convert \(y = x^2 - 8x + 16\) to vertex form.  
  Notice \(x^2 - 8x + 16 = (x - 4)^2.\)  
  \(\boxed{y = (x - 4)^2}\)  
  Vertex: \((4, 0)\).

  \item Factor \(y = x^2 - 7x + 10\).  
  We want two numbers that multiply to 10 and add to -7: \(-5\) and \(-2.\)  
  \[
  y = (x - 5)(x - 2).
  \]
  \(\boxed{y = (x - 5)(x - 2)}\)

  \item Convert \(y = (x + 2)^2 - 9\) to standard form.  
  First expand \((x + 2)^2 = x^2 + 4x + 4.\)  
  Then subtract 9:
  \[
  y = x^2 + 4x + 4 - 9 = x^2 + 4x - 5.
  \]
  \(\boxed{y = x^2 + 4x - 5}\)

  \item Convert \(y = x^2 - 10x + 21\) into factored form.  
  Two numbers that multiply to 21 and add to -10: \(-3\) and \(-7.\)  
  \[
  y = (x - 3)(x - 7).
  \]
  \(\boxed{y = (x - 3)(x - 7)}\)
\end{enumerate}

\subsection*{Part C Solutions: Identifying Features from Each Form}
\begin{enumerate}
  \setcounter{enumi}{10}
  \item For \(y = 2(x - 1)(x - 5)\):  
  Zeros occur when each factor is 0.  
  \(x - 1 = 0 \Rightarrow x = 1,\)  
  \(x - 5 = 0 \Rightarrow x = 5.\)  
  \(\boxed{x = 1 \text{ and } x = 5}\)

  \item For \(y = -3(x + 4)^2 + 2\):  
  Vertex form is \(y = a(x - h)^2 + k\). Here \(x + 4 = x - (-4)\).  
  So vertex is \((-4, 2)\).  
  Since \(a = -3 < 0\), it opens downward.  
  \(\boxed{\text{Vertex } (-4, 2), \text{ opens down}}\)

  \item For \(y = x^2 - 6x + 9\):  
  Recognize \((x^2 - 6x + 9) = (x - 3)^2.\)  
  Vertex form is \((x - 3)^2 + 0.\)  
  Vertex is \((3, 0)\).  
  \(\boxed{(3, 0)}\)

  \item For \(y = (x - 3)(x - 7)\):  
  Zeros are \(x = 3\) and \(x = 7.\)  
  Axis of symmetry is the midpoint between the zeros:  
  \[
  x = \frac{3 + 7}{2} = \frac{10}{2} = 5.
  \]
  \(\boxed{x = 5}\)

  \item For \(y = 4(x + 2)^2 - 8\):  
  Vertex form is \(a(x - h)^2 + k\). Here \(x + 2 = x - (-2)\).  
  Vertex is \((-2, -8)\).  
  Since \(a = 4 > 0\), it opens up, so the vertex is a minimum.  
  \(\boxed{\text{Vertex } (-2, -8), \text{ minimum}}\)
\end{enumerate}

\subsection*{Part D Solutions: Structural Reasoning and Graph Relationships}
\begin{enumerate}
  \setcounter{enumi}{15}
  \item For \(y = -x^2 + 6x - 5\):  
  The leading coefficient \(a = -1 < 0\).  
  A negative \(a\) means it opens downward.  
  \(\boxed{\text{Opens down}}\)

  \item Compare \(y = x^2\) and \(y = 3x^2\).  
  Larger \(|a|\) makes the parabola narrower.  
  \(3x^2\) is narrower.  
  \(\boxed{y = 3x^2 \text{ is narrower}}\)

  \item Compare \(y = x^2\) and \(y = \tfrac{1}{2}x^2\).  
  Smaller \(|a|\) between 0 and 1 makes it wider.  
  \(\tfrac{1}{2}x^2\) is wider.  
  \(\boxed{y = \tfrac{1}{2}x^2 \text{ is wider}}\)

  \item Write the vertex form of \(y = x^2 + 4x + 1\).  
  Complete the square:  
  \[
  x^2 + 4x + 1 = (x^2 + 4x + 4) - 4 + 1 = (x + 2)^2 - 3.
  \]
  \(\boxed{y = (x + 2)^2 - 3}\)  
  Vertex: \((-2, -3)\).

  \item Rewrite \(y = (x - 1)(x - 4)\) in standard form, then find its axis of symmetry.

  Expand:  
  \[
  y = (x - 1)(x - 4) = x^2 - 4x - x + 4 = x^2 - 5x + 4.
  \]

  Zeros are \(x = 1\) and \(x = 4\).  
  Axis of symmetry is the midpoint between 1 and 4:  
  \[
  x = \frac{1 + 4}{2} = \frac{5}{2} = 2.5.
  \]

  \(\boxed{y = x^2 - 5x + 4,\; \text{axis } x = \frac{5}{2}}\)
\end{enumerate}

\subsection*{Part E Solutions: Word Problems and SAT-Style Applications}
\begin{enumerate}
  \setcounter{enumi}{20}
  \item \(h = -16t^2 + 32t + 48.\)  
  This is standard form \(at^2 + bt + c.\)  
  \(\boxed{a = -16,\; b = 32,\; c = 48}\)

  \item \(h = -2(x - 5)^2 + 50.\)  
  This is vertex form \(a(x - h)^2 + k.\)  
  Vertex is \((5, 50)\).  
  Since \(a = -2 < 0\), it opens downward, so \((5, 50)\) is the maximum height/location.  
  \(\boxed{\text{Vertex } (5, 50), \text{ maximum point}}\)

  \item Zeros at \(x = 2\) and \(x = 6\), with \(a = 1.\)  
  Factored form uses zeros \(r_1, r_2\):  
  \[
  y = (x - 2)(x - 6).
  \]
  \(\boxed{y = (x - 2)(x - 6)}\)

  \item Vertex \((3, -4)\), opens up with \(a = 2.\)  
  Vertex form:  
  \[
  y = a(x - h)^2 + k = 2(x - 3)^2 - 4.
  \]
  \(\boxed{y = 2(x - 3)^2 - 4}\)

  \item Standard form has \(a = 1,\; b = -6,\; c = 8.\)  
  Solve \(x^2 - 6x + 8 = 0\) by factoring.  
  Numbers that multiply to 8 and add to -6: \(-2\) and \(-4.\)  
  \[
  x^2 - 6x + 8 = (x - 2)(x - 4) = 0.
  \]
  So \(x = 2,\; x = 4.\)  
  \(\boxed{x = 2 \text{ and } x = 4}\)
\end{enumerate}



\end{document}
