\documentclass[12pt]{article}

\usepackage{amsmath, amssymb}
\usepackage{geometry}
\usepackage{setspace}
\usepackage{titlesec}
\usepackage{lmodern}
\usepackage{xcolor}
\usepackage{enumitem}

\geometry{margin=1in}
\setstretch{1.2}
\titleformat{\section}{\normalfont\Large\bfseries}{\thesection}{1em}{}
\titleformat{\subsection}{\normalfont\large\bfseries}{\thesubsection}{1em}{}
\pagenumbering{gobble}

\begin{document}

\begin{center}
    \LARGE \textbf{Unit 6: Quadratic and Polynomial Functions} \\[6pt]
    \Large \textbf{Topic 4: Graphing Parabolas (Vertex and Intercepts)}
\end{center}

\vspace{1em}

\section*{Concept Summary}

A \textbf{parabola} is the graph of a quadratic function:
\[
y = ax^2 + bx + c.
\]
It is a U-shaped curve that opens upward when \(a > 0\) and downward when \(a < 0.\)

Key features of a parabola include its:
\begin{itemize}
  \item \textbf{Vertex:} the highest or lowest point.
  \item \textbf{Axis of symmetry:} the vertical line that passes through the vertex.
  \item \textbf{Y-intercept:} where the graph crosses the y-axis.
  \item \textbf{X-intercepts (zeros):} where the graph crosses the x-axis, found by solving \(ax^2 + bx + c = 0.\)
\end{itemize}

The vertex can be found from the formula:
\[
x = -\frac{b}{2a}, \quad y = f\!\left(-\frac{b}{2a}\right).
\]

In vertex form, \(y = a(x - h)^2 + k,\) the vertex is \((h, k)\).  
The axis of symmetry is \(x = h.\)

\subsection*{Why It Works}
Completing the square converts \(y = ax^2 + bx + c\) into vertex form, revealing how the graph shifts horizontally and vertically from the parent function \(y = x^2.\)

\section*{Core Skills}
\begin{itemize}
  \item Identify vertex, axis of symmetry, and intercepts from any form.
  \item Convert between standard and vertex forms.
  \item Determine whether the parabola opens up or down.
  \item Sketch parabolas accurately by plotting key points.
  \item Interpret graphs in real-world contexts (maximums/minimums).
\end{itemize}

\section*{Example 1: Finding Vertex and Intercepts}

Graph \(y = x^2 - 4x + 3.\)

\textbf{Step 1:} Find the vertex.  
\[
x = -\frac{b}{2a} = \frac{4}{2} = 2, \quad y = (2)^2 - 4(2) + 3 = 4 - 8 + 3 = -1.
\]
Vertex: \((2, -1)\).

\textbf{Step 2:} Find the intercepts.  
Y-intercept: when \(x = 0,\; y = 3.\)  
X-intercepts: solve \(x^2 - 4x + 3 = 0 \Rightarrow (x - 3)(x - 1) = 0 \Rightarrow x = 1, 3.\)

\textbf{Step 3:} Direction: \(a = 1 > 0\), so the parabola opens upward.

\textbf{Graph:} (Insert graph showing vertex at (2, –1), x-intercepts at 1 and 3, y-intercept at 3, symmetric about \(x = 2\).)

\section*{Example 2: Vertex Form and Transformations}

Graph \(y = 2(x + 1)^2 - 3.\)

\textbf{Vertex:} \((-1, -3)\).  
\textbf{Axis of symmetry:} \(x = -1.\)  
\textbf{Opening:} Upward (\(a = 2 > 0\)).  
\textbf{Stretch:} Narrower than \(y = x^2.\)

\textbf{Graph:} (Insert graph with vertex at (–1, –3), axis of symmetry \(x = -1\), opens upward, narrower.)

\section*{Example 3: Downward Opening Parabola}

Graph \(y = -x^2 + 2x + 3.\)

\textbf{Step 1:} Vertex: \(x = -\frac{b}{2a} = -\frac{2}{-2} = 1.\)  
\(y = -(1)^2 + 2(1) + 3 = -1 + 2 + 3 = 4.\)  
Vertex: \((1, 4)\).

\textbf{Step 2:} Intercepts:  
Y-intercept: when \(x = 0,\; y = 3.\)  
X-intercepts: solve \(-x^2 + 2x + 3 = 0\).  
Multiply by –1: \(x^2 - 2x - 3 = 0 \Rightarrow (x - 3)(x + 1) = 0.\)  
\(\boxed{x = 3, -1}\).

\textbf{Graph:} (Insert graph showing vertex at (1, 4), x-intercepts at –1 and 3, y-intercept at 3, opens downward.)

\section*{Example 4: Finding the Equation from a Graph}

A parabola has vertex \((2, -4)\) and passes through the point \((0, 4)\).  
Find its equation.

Use vertex form \(y = a(x - h)^2 + k.\)
\[
y = a(x - 2)^2 - 4.
\]
Substitute \((x, y) = (0, 4):\)
\[
4 = a(0 - 2)^2 - 4 \Rightarrow 4 = 4a - 4 \Rightarrow 8 = 4a \Rightarrow a = 2.
\]
\textbf{Equation:} \(\boxed{y = 2(x - 2)^2 - 4.}\)

\textbf{Graph:} (Insert graph showing vertex (2, –4), upward opening, passes through (0, 4).)

\section*{Example 5: Real-World Interpretation}

A rocket’s height is given by \(h = -16t^2 + 64t + 80.\)  
Find its vertex and interpret.

\[
t = -\frac{b}{2a} = -\frac{64}{-32} = 2.
\]
\(h(2) = -16(4) + 64(2) + 80 = -64 + 128 + 80 = 144.\)
\textbf{Vertex:} \((2, 144)\).

Interpretation: The rocket reaches a maximum height of 144 ft at 2 seconds.

\textbf{Graph:} (Insert graph of height vs. time showing vertex (2, 144), opens downward, intercepts at \(t = 0\) and \(t \approx 5.\))

\section*{Key Takeaways}
\begin{itemize}
  \item Use \(x = -\frac{b}{2a}\) to find the vertex in standard form.
  \item The sign of \(a\) determines if the parabola opens up or down.
  \item The axis of symmetry passes through the vertex.
  \item The vertex represents a minimum if \(a > 0\), or a maximum if \(a < 0.\)
  \item Plot vertex, intercepts, and symmetry to sketch accurate graphs.
\end{itemize}

\newpage

% ============================================================
% QUESTIONS — UNIT 6, TOPIC 4: GRAPHING PARABOLAS (VERTEX AND INTERCEPTS)
% ============================================================

\section*{Practice Questions: Graphing Parabolas (Vertex and Intercepts)}

\subsection*{Part A: Finding Vertex and Axis of Symmetry}
\begin{enumerate}
  \item Find the vertex of \(y = x^2 + 6x + 5.\)
  \item Find the vertex of \(y = 2x^2 - 4x + 1.\)
  \item Find the vertex and axis of symmetry for \(y = -x^2 + 8x - 7.\)
  \item Find the vertex and determine whether the parabola opens upward or downward: \(y = -2x^2 - 4x + 3.\)
  \item Find the vertex and minimum or maximum value of \(y = x^2 - 2x + 4.\)
\end{enumerate}

\subsection*{Part B: Intercepts and Shape}
\begin{enumerate}
  \setcounter{enumi}{5}
  \item Find the y-intercept and x-intercepts of \(y = x^2 - 5x + 6.\)
  \item Find the y-intercept and x-intercepts of \(y = -x^2 + 4x.\)
  \item Find the intercepts of \(y = 2x^2 + 3x - 5.\)
  \item Determine whether \(y = -3x^2 + 6x - 2\) opens up or down, and find its y-intercept.
  \item Find the intercepts and axis of symmetry of \(y = x^2 - 9.\)
\end{enumerate}

\subsection*{Part C: Vertex Form and Transformations}
\begin{enumerate}
  \setcounter{enumi}{10}
  \item Identify the vertex, axis of symmetry, and direction of opening for \(y = (x - 3)^2 - 2.\)
  \item Identify the vertex and direction for \(y = -2(x + 4)^2 + 1.\)
  \item Write the equation of a parabola with vertex \((2, -3)\) that opens upward and passes through \((4, 5)\).
  \item Write the equation of a parabola with vertex \((-1, 2)\) that opens downward and passes through \((1, -6)\).
  \item Rewrite \(y = x^2 + 6x + 10\) in vertex form.
\end{enumerate}

\subsection*{Part D: Graph-Based Reasoning}
(Insert labeled graph of \(y = x^2 - 4x + 3\) showing a U-shaped curve with vertex at (2, –1), x-intercepts at 1 and 3, y-intercept at 3.)
\begin{enumerate}
  \setcounter{enumi}{15}
  \item What is the vertex of the parabola shown?
  \item What is the axis of symmetry?
  \item What are the x-intercepts?
  \item What is the y-intercept?
  \item Does the parabola open upward or downward?
\end{enumerate}

\subsection*{Part E: Word Problems and SAT-Style Applications}
\begin{enumerate}
  \setcounter{enumi}{20}
  \item The height of a ball is modeled by \(h = -16t^2 + 32t + 48.\) Find the time at which the ball reaches its maximum height and the height itself.
  \item A parabola modeling profit is \(P = -2x^2 + 12x + 18.\) What is the maximum profit and at what value of \(x\) does it occur?
  \item A satellite dish is shaped like a parabola defined by \(y = 2x^2.\) How does the shape compare to \(y = x^2\)?
  \item A parabola has vertex \((3, 5)\) and opens downward. Write its equation if it passes through \((5, 1).\)
  \item A quadratic function models the path of a thrown object as \(h(t) = -5t^2 + 20t + 30.\) When will it hit the ground?
\end{enumerate}

\newpage

% ============================================================
% SOLUTIONS — UNIT 6, TOPIC 4: GRAPHING PARABOLAS (VERTEX AND INTERCEPTS)
% ============================================================

\section*{Answer Key and Solutions: Graphing Parabolas (Vertex and Intercepts)}

\subsection*{Part A Solutions: Finding Vertex and Axis of Symmetry}
\begin{enumerate}
  \item \(y = x^2 + 6x + 5\)

  For \(y = ax^2 + bx + c\), vertex \(x = -\frac{b}{2a}.\)

  Here \(a = 1,\; b = 6.\)

  \[
  x = -\frac{6}{2 \cdot 1} = -3.
  \]
  \[
  y(-3) = (-3)^2 + 6(-3) + 5 = 9 - 18 + 5 = -4.
  \]

  \(\boxed{\text{Vertex } (-3, -4)}\)

  \item \(y = 2x^2 - 4x + 1\)

  \(a = 2,\; b = -4.\)

  \[
  x = -\frac{-4}{2 \cdot 2} = \frac{4}{4} = 1.
  \]
  \[
  y(1) = 2(1)^2 - 4(1) + 1 = 2 - 4 + 1 = -1.
  \]

  \(\boxed{\text{Vertex } (1, -1)}\)

  \item \(y = -x^2 + 8x - 7\)

  \(a = -1,\; b = 8.\)

  \[
  x = -\frac{8}{2(-1)} = -\frac{8}{-2} = 4.
  \]
  \[
  y(4) = -(4)^2 + 8(4) - 7 = -16 + 32 - 7 = 9.
  \]

  \(\boxed{\text{Vertex } (4, 9), \text{ axis } x = 4}\)

  \item \(y = -2x^2 - 4x + 3\)

  \(a = -2,\; b = -4.\)

  \[
  x = -\frac{-4}{2(-2)} = \frac{4}{-4} = -1.
  \]
  \[
  y(-1) = -2(-1)^2 - 4(-1) + 3 = -2 + 4 + 3 = 5.
  \]

  Vertex is \((-1, 5)\). Since \(a = -2 < 0,\) it opens downward.

  \(\boxed{\text{Vertex } (-1, 5), \text{ opens down}}\)

  \item \(y = x^2 - 2x + 4\)

  \(a = 1,\; b = -2.\)

  \[
  x = -\frac{-2}{2 \cdot 1} = \frac{2}{2} = 1.
  \]
  \[
  y(1) = (1)^2 - 2(1) + 4 = 1 - 2 + 4 = 3.
  \]

  Vertex is \((1, 3).\) Because \(a = 1 > 0,\) this is a minimum, and the minimum value is 3.

  \(\boxed{\text{Vertex } (1, 3), \text{ min value } 3}\)
\end{enumerate}

\subsection*{Part B Solutions: Intercepts and Shape}
\begin{enumerate}
  \setcounter{enumi}{5}
  \item \(y = x^2 - 5x + 6\)

  Y-intercept: set \(x = 0\).  
  \(y = 6 \Rightarrow (0, 6).\)

  X-intercepts: solve \(x^2 - 5x + 6 = 0.\)  
  \((x - 2)(x - 3) = 0 \Rightarrow x = 2, 3.\)  
  Intercepts: \((2, 0)\) and \((3, 0).\)

  \(\boxed{\text{Y-int } (0, 6), \text{ X-ints } (2, 0), (3, 0)}\)

  \item \(y = -x^2 + 4x\)

  Y-intercept: \(x = 0 \Rightarrow y = 0 \Rightarrow (0, 0).\)

  X-intercepts: solve \(-x^2 + 4x = 0.\)  
  Factor: \(-x(x - 4) = 0 \Rightarrow x = 0, 4.\)  
  Intercepts: \((0, 0)\) and \((4, 0).\)

  \(\boxed{\text{Y-int } (0, 0), \text{ X-ints } (0, 0), (4, 0)}\)

  \item \(y = 2x^2 + 3x - 5\)

  Y-intercept: \(x = 0 \Rightarrow y = -5 \Rightarrow (0, -5).\)

  X-intercepts: solve \(2x^2 + 3x - 5 = 0.\)

  Quadratic formula with \(a = 2,\; b = 3,\; c = -5.\)

  \[
  x = \frac{-3 \pm \sqrt{3^2 - 4(2)(-5)}}{2 \cdot 2}
  = \frac{-3 \pm \sqrt{9 + 40}}{4}
  = \frac{-3 \pm \sqrt{49}}{4}
  = \frac{-3 \pm 7}{4}.
  \]

  So \(x = 1\) or \(x = -\tfrac{5}{2}.\)

  Intercepts: \((1, 0)\) and \(\left(-\tfrac{5}{2}, 0\right).\)

  \(\boxed{\text{Y-int } (0, -5), \text{ X-ints } (1, 0), (-\tfrac{5}{2}, 0)}\)

  \item \(y = -3x^2 + 6x - 2\)

  Opens up or down? The leading coefficient is \(-3 < 0.\)  
  So it opens downward.

  Y-intercept: \(x = 0 \Rightarrow y = -2 \Rightarrow (0, -2).\)

  \(\boxed{\text{Opens down, Y-int } (0, -2)}\)

  \item \(y = x^2 - 9\)

  Y-intercept: \(x = 0 \Rightarrow y = -9 \Rightarrow (0, -9).\)

  X-intercepts: solve \(x^2 - 9 = 0 \Rightarrow (x - 3)(x + 3) = 0.\)  
  So \(x = 3\) or \(-3.\)  
  Intercepts: \((3, 0)\) and \((-3, 0).\)

  Axis of symmetry: midpoint between \(-3\) and \(3\) is \(x = 0.\)

  \(\boxed{\text{Y-int } (0, -9), \text{ X-ints } (-3, 0), (3, 0), \text{ axis } x = 0}\)
\end{enumerate}

\subsection*{Part C Solutions: Vertex Form and Transformations}
\begin{enumerate}
  \setcounter{enumi}{10}
  \item \(y = (x - 3)^2 - 2\)

  Vertex form \(y = a(x - h)^2 + k.\)

  Here \(h = 3,\; k = -2,\; a = 1.\)

  Vertex: \((3, -2)\).  
  Axis of symmetry: \(x = 3.\)  
  Since \(a = 1 > 0\), it opens upward.

  \(\boxed{\text{Vertex } (3, -2), \text{ axis } x = 3, \text{ opens up}}\)

  \item \(y = -2(x + 4)^2 + 1\)

  Here \(h = -4,\; k = 1,\; a = -2.\)

  Vertex: \((-4, 1).\)  
  Since \(a < 0,\) it opens downward (narrower than \(y = x^2\) because \(|a| > 1\)).

  \(\boxed{\text{Vertex } (-4, 1), \text{ opens down}}\)

  \item Parabola with vertex \((2, -3)\), opens upward, passes through \((4, 5)\)

  Start with vertex form:
  \[
  y = a(x - 2)^2 - 3.
  \]
  Use \((4, 5)\):
  \[
  5 = a(4 - 2)^2 - 3 \Rightarrow 5 = 4a - 3 \Rightarrow 8 = 4a \Rightarrow a = 2.
  \]

  \(\boxed{y = 2(x - 2)^2 - 3}\)

  \item Parabola with vertex \((-1, 2)\), opens downward, passes through \((1, -6)\)

  Vertex form:
  \[
  y = a(x + 1)^2 + 2.
  \]
  Use \((1, -6)\):
  \[
  -6 = a(1 + 1)^2 + 2 \Rightarrow -6 = 4a + 2 \Rightarrow -8 = 4a \Rightarrow a = -2.
  \]

  \(\boxed{y = -2(x + 1)^2 + 2}\)

  \item Rewrite \(y = x^2 + 6x + 10\) in vertex form.

  Complete the square:
  \[
  x^2 + 6x + 10 = (x^2 + 6x + 9) + 1 = (x + 3)^2 + 1.
  \]

  \(\boxed{y = (x + 3)^2 + 1}\)

  Vertex: \((-3, 1)\).
\end{enumerate}

\subsection*{Part D Solutions: Graph-Based Reasoning}
(Reference: graph of \(y = x^2 - 4x + 3\), with vertex at (2, –1), x-intercepts at 1 and 3, y-intercept at 3, opens up.)

\begin{enumerate}
  \setcounter{enumi}{15}
  \item Vertex: \(\boxed{(2, -1)}\)

  \item Axis of symmetry: vertical line through the vertex.  
  \(\boxed{x = 2}\)

  \item X-intercepts: where the graph crosses the x-axis.  
  \(\boxed{(1, 0) \text{ and } (3, 0)}\)

  \item Y-intercept: where the graph crosses the y-axis.  
  From the given graph description, \(\boxed{(0, 3)}\)

  \item The parabola opens upward (U-shaped opening up).  
  \(\boxed{\text{Opens up}}\)
\end{enumerate}

\subsection*{Part E Solutions: Word Problems and SAT-Style Applications}
\begin{enumerate}
  \setcounter{enumi}{20}
  \item \(h = -16t^2 + 32t + 48\)

  Maximum height occurs at the vertex.  
  \(t = -\frac{b}{2a} = -\frac{32}{2(-16)} = -\frac{32}{-32} = 1.\)

  Height at \(t = 1\):  
  \[
  h(1) = -16(1)^2 + 32(1) + 48 = -16 + 32 + 48 = 64.
  \]

  \(\boxed{\text{Max height } 64 \text{ at } t = 1\text{ s}}\)

  \item \(P = -2x^2 + 12x + 18\)

  Vertex \(x\)-value:  
  \[
  x = -\frac{b}{2a} = -\frac{12}{2(-2)} = -\frac{12}{-4} = 3.
  \]

  Profit at \(x = 3\):  
  \[
  P(3) = -2(3)^2 + 12(3) + 18 = -18 + 36 + 18 = 36.
  \]

  \(\boxed{\text{Max profit } 36 \text{ at } x = 3}\)

  \item \(y = 2x^2\) vs \(y = x^2\)

  The coefficient \(a = 2\) is larger in absolute value than 1.  
  Larger \(|a|\) makes the parabola narrower.

  \(\boxed{y = 2x^2 \text{ is narrower (steeper)}}\)

  \item Vertex \((3, 5)\), opens down, passes through \((5, 1)\)

  Vertex form:
  \[
  y = a(x - 3)^2 + 5.
  \]
  Use point \((5, 1)\):
  \[
  1 = a(5 - 3)^2 + 5 \Rightarrow 1 = 4a + 5 \Rightarrow -4 = 4a \Rightarrow a = -1.
  \]

  \(\boxed{y = -(x - 3)^2 + 5}\)

  \item \(h(t) = -5t^2 + 20t + 30\). When does it hit the ground?

  Set \(h = 0\):
  \[
  -5t^2 + 20t + 30 = 0.
  \]

  Solve using quadratic formula with \(a = -5,\; b = 20,\; c = 30.\)

  \[
  t = \frac{-20 \pm \sqrt{20^2 - 4(-5)(30)}}{2(-5)}
  = \frac{-20 \pm \sqrt{400 + 600}}{-10}
  = \frac{-20 \pm \sqrt{1000}}{-10}
  \]
  \[
  = \frac{-20 \pm 10\sqrt{10}}{-10}
  = \frac{20 \pm 10\sqrt{10}}{10}
  = 2 \pm \sqrt{10}.
  \]

  So \(t = 2 + \sqrt{10}\) or \(t = 2 - \sqrt{10}.\)

  \(2 - \sqrt{10} < 0\) so it is not physical.

  \(\boxed{t = 2 + \sqrt{10} \text{ seconds}}\)
\end{enumerate}




\end{document}
