\documentclass[12pt]{article}

\usepackage{amsmath, amssymb}
\usepackage{geometry}
\usepackage{setspace}
\usepackage{titlesec}
\usepackage{lmodern}
\usepackage{xcolor}
\usepackage{enumitem}

\geometry{margin=1in}
\setstretch{1.2}
\titleformat{\section}{\normalfont\Large\bfseries}{\thesection}{1em}{}
\titleformat{\subsection}{\normalfont\large\bfseries}{\thesubsection}{1em}{}
\pagenumbering{gobble}

\begin{document}

\begin{center}
    \LARGE \textbf{Unit 6: Quadratic and Polynomial Functions} \\[6pt]
    \Large \textbf{Topic 2: Factoring and Solving Quadratics}
\end{center}

\vspace{1em}

\section*{Concept Summary}

A \textbf{quadratic equation} is an equation of the form
\[
ax^2 + bx + c = 0, \quad a \ne 0.
\]
One of the most efficient ways to solve quadratics is by \textbf{factoring}—rewriting the equation as a product of linear factors set equal to zero.

Factoring relies on the \textbf{Zero Product Property:}  
If \(A \cdot B = 0\), then \(A = 0\) or \(B = 0.\)

For example:
\[
x^2 - 5x + 6 = 0 \quad \Rightarrow \quad (x - 2)(x - 3) = 0 \quad \Rightarrow \quad x = 2 \text{ or } x = 3.
\]

Not all quadratics factor neatly, but recognizing common structures makes factoring fast:
\begin{itemize}
  \item Difference of squares: \(a^2 - b^2 = (a - b)(a + b)\)
  \item Perfect square trinomials: \(a^2 \pm 2ab + b^2 = (a \pm b)^2\)
  \item Common factors: factor out the greatest common factor (GCF) first.
\end{itemize}

Factoring is most useful when the coefficients are integers and the quadratic can be decomposed into two binomials easily.

\section*{Core Skills}
\begin{itemize}
  \item Factor out any GCF before proceeding.
  \item Recognize patterns: difference of squares and perfect square trinomials.
  \item Use the zero product property to find roots.
  \item Check solutions by substituting back into the original equation.
  \item Identify when factoring is not possible (then use quadratic formula).
\end{itemize}

\section*{Example 1: Simple Factoring}

Solve \(x^2 - 7x + 10 = 0.\)

Find two numbers that multiply to \(10\) and add to \(-7\): \(-5\) and \(-2.\)
\[
x^2 - 7x + 10 = (x - 5)(x - 2) = 0
\]
\textbf{Solutions:} \(x = 5, 2.\)

\section*{Example 2: Leading Coefficient Not 1}

Solve \(2x^2 + 7x + 3 = 0.\)

Multiply \(a \cdot c = 2 \times 3 = 6.\)  
Find two numbers that multiply to 6 and add to 7: \(6\) and \(1.\)
\[
2x^2 + 6x + x + 3 = 0
\]
Group:
\[
(2x^2 + 6x) + (x + 3) = 0 \Rightarrow 2x(x + 3) + 1(x + 3) = 0
\]
\[
(2x + 1)(x + 3) = 0
\]
\textbf{Solutions:} \(x = -\tfrac{1}{2}, -3.\)

\section*{Example 3: Difference of Squares}

Solve \(x^2 - 49 = 0.\)
\[
(x - 7)(x + 7) = 0
\]
\textbf{Solutions:} \(x = 7, -7.\)

\section*{Example 4: Perfect Square Trinomial}

Solve \(x^2 + 10x + 25 = 0.\)
\[
(x + 5)^2 = 0
\]
\textbf{Solution:} \(x = -5\) (double root).

\section*{Example 5: Factoring Out a Common Factor}

Solve \(3x^2 - 9x = 0.\)
\[
3x(x - 3) = 0
\]
\textbf{Solutions:} \(x = 0, 3.\)

\section*{Example 6: Application Problem}

The product of two consecutive integers is 72.  
Let the smaller integer be \(x\), so:
\[
x(x + 1) = 72
\]
\[
x^2 + x - 72 = 0
\]
\[
(x + 9)(x - 8) = 0
\]
\textbf{Solutions:} \(x = -9\) or \(x = 8.\)  
The integers are \((-9, -8)\) or \((8, 9).\)

\section*{Key Takeaways}
\begin{itemize}
  \item Always factor out the GCF first.
  \item Use pattern recognition to save time: difference of squares and perfect square trinomials.
  \item Apply the zero product property after factoring.
  \item When factoring fails, use the quadratic formula.
  \item Verify by substitution to ensure no extraneous solutions.
\end{itemize}

\newpage

% ============================================================
% QUESTIONS — UNIT 6, TOPIC 2: FACTORING AND SOLVING QUADRATICS
% ============================================================

\section*{Practice Questions: Factoring and Solving Quadratics}

\subsection*{Part A: Factoring Simple Quadratics}
\begin{enumerate}
  \item Solve \(x^2 - 5x + 6 = 0.\)
  \item Solve \(x^2 - 9x + 20 = 0.\)
  \item Solve \(x^2 + 3x - 10 = 0.\)
  \item Solve \(x^2 - x - 6 = 0.\)
  \item Solve \(x^2 + 2x - 15 = 0.\)
\end{enumerate}

\subsection*{Part B: Factoring with Leading Coefficients}
\begin{enumerate}
  \setcounter{enumi}{5}
  \item Solve \(2x^2 + 5x + 3 = 0.\)
  \item Solve \(3x^2 - 8x + 4 = 0.\)
  \item Solve \(4x^2 + 4x - 8 = 0.\)
  \item Solve \(5x^2 - 10x = 0.\)
  \item Solve \(6x^2 + 11x + 3 = 0.\)
\end{enumerate}

\subsection*{Part C: Special Factoring Patterns}
\begin{enumerate}
  \setcounter{enumi}{10}
  \item Solve \(x^2 - 36 = 0.\)
  \item Solve \(9x^2 - 25 = 0.\)
  \item Solve \(x^2 + 8x + 16 = 0.\)
  \item Solve \(x^2 - 12x + 36 = 0.\)
  \item Solve \(x^2 - 4x = 0.\)
\end{enumerate}

\subsection*{Part D: Common Factors and Mixed Structure}
\begin{enumerate}
  \setcounter{enumi}{15}
  \item Solve \(3x^2 - 6x = 0.\)
  \item Solve \(x^2 - 2x - 8 = 0.\)
  \item Solve \(2x^2 - 18 = 0.\)
  \item Solve \(x^3 - 9x = 0.\)
  \item Solve \(2x^2 - 10x + 12 = 0.\)
\end{enumerate}

\subsection*{Part E: Word Problems and SAT-Style Applications}
\begin{enumerate}
  \setcounter{enumi}{20}
  \item The product of two consecutive positive integers is 42. Find the integers.
  \item The area of a rectangle is 60. If the length is \(x + 5\) and the width is \(x - 3\), find \(x.\)
  \item A projectile’s height is given by \(h = -5t^2 + 20t.\) When does it hit the ground?
  \item The square of a number is 9 more than twice the number. Find the number.
  \item The product of two consecutive even integers is 48. Find the integers.
\end{enumerate}

\newpage

% ============================================================
% SOLUTIONS — UNIT 6, TOPIC 2: FACTORING AND SOLVING QUADRATICS
% ============================================================

\section*{Answer Key and Solutions: Factoring and Solving Quadratics}

\subsection*{Part A Solutions: Factoring Simple Quadratics}
\begin{enumerate}
  \item \(x^2 - 5x + 6 = 0\)  
  \((x - 2)(x - 3) = 0 \Rightarrow x = 2, 3.\)  
  \(\boxed{x = 2 \text{ or } x = 3}\)

  \item \(x^2 - 9x + 20 = 0\)  
  \((x - 4)(x - 5) = 0 \Rightarrow x = 4, 5.\)  
  \(\boxed{x = 4 \text{ or } x = 5}\)

  \item \(x^2 + 3x - 10 = 0\)  
  \((x + 5)(x - 2) = 0 \Rightarrow x = -5, 2.\)  
  \(\boxed{x = -5 \text{ or } x = 2}\)

  \item \(x^2 - x - 6 = 0\)  
  \((x - 3)(x + 2) = 0 \Rightarrow x = 3, -2.\)  
  \(\boxed{x = 3 \text{ or } x = -2}\)

  \item \(x^2 + 2x - 15 = 0\)  
  \((x + 5)(x - 3) = 0 \Rightarrow x = -5, 3.\)  
  \(\boxed{x = -5 \text{ or } x = 3}\)
\end{enumerate}

\subsection*{Part B Solutions: Factoring with Leading Coefficients}
\begin{enumerate}
  \setcounter{enumi}{5}
  \item \(2x^2 + 5x + 3 = 0\)  
  Factor: \((2x + 3)(x + 1) = 0.\)  
  Solutions: \(2x + 3 = 0 \Rightarrow x = -\tfrac{3}{2}\) and \(x + 1 = 0 \Rightarrow x = -1.\)  
  \(\boxed{x = -\tfrac{3}{2} \text{ or } x = -1}\)

  \item \(3x^2 - 8x + 4 = 0\)  
  Factor: \((3x - 2)(x - 2) = 0.\)  
  Solutions: \(3x - 2 = 0 \Rightarrow x = \tfrac{2}{3}\) and \(x - 2 = 0 \Rightarrow x = 2.\)  
  \(\boxed{x = \tfrac{2}{3} \text{ or } x = 2}\)

  \item \(4x^2 + 4x - 8 = 0\)  
  First factor out 4: \(4(x^2 + x - 2) = 0.\)  
  Then \((x + 2)(x - 1) = 0.\)  
  So \(x = -2, 1.\)  
  \(\boxed{x = -2 \text{ or } x = 1}\)

  \item \(5x^2 - 10x = 0\)  
  Factor out \(5x\): \(5x(x - 2) = 0.\)  
  So \(5x = 0 \Rightarrow x = 0\) or \(x - 2 = 0 \Rightarrow x = 2.\)  
  \(\boxed{x = 0 \text{ or } x = 2}\)

  \item \(6x^2 + 11x + 3 = 0\)  
  Factor: \((3x + 1)(2x + 3) = 0.\)  
  Solutions: \(3x + 1 = 0 \Rightarrow x = -\tfrac{1}{3}\) and \(2x + 3 = 0 \Rightarrow x = -\tfrac{3}{2}.\)  
  \(\boxed{x = -\tfrac{1}{3} \text{ or } x = -\tfrac{3}{2}}\)
\end{enumerate}

\subsection*{Part C Solutions: Special Factoring Patterns}
\begin{enumerate}
  \setcounter{enumi}{10}
  \item \(x^2 - 36 = 0\)  
  Difference of squares: \((x - 6)(x + 6) = 0.\)  
  \(\boxed{x = 6 \text{ or } x = -6}\)

  \item \(9x^2 - 25 = 0\)  
  Difference of squares: \((3x - 5)(3x + 5) = 0.\)  
  \(3x - 5 = 0 \Rightarrow x = \tfrac{5}{3}\)  
  \(3x + 5 = 0 \Rightarrow x = -\tfrac{5}{3}.\)  
  \(\boxed{x = \tfrac{5}{3} \text{ or } x = -\tfrac{5}{3}}\)

  \item \(x^2 + 8x + 16 = 0\)  
  Perfect square: \((x + 4)^2 = 0.\)  
  \(\boxed{x = -4}\) (double root)

  \item \(x^2 - 12x + 36 = 0\)  
  Perfect square: \((x - 6)^2 = 0.\)  
  \(\boxed{x = 6}\) (double root)

  \item \(x^2 - 4x = 0\)  
  Factor out \(x\): \(x(x - 4) = 0.\)  
  \(\boxed{x = 0 \text{ or } x = 4}\)
\end{enumerate}

\subsection*{Part D Solutions: Common Factors and Mixed Structure}
\begin{enumerate}
  \setcounter{enumi}{15}
  \item \(3x^2 - 6x = 0\)  
  Factor out \(3x\): \(3x(x - 2) = 0.\)  
  \(\boxed{x = 0 \text{ or } x = 2}\)

  \item \(x^2 - 2x - 8 = 0\)  
  \((x - 4)(x + 2) = 0.\)  
  \(\boxed{x = 4 \text{ or } x = -2}\)

  \item \(2x^2 - 18 = 0\)  
  Factor out 2: \(2(x^2 - 9) = 0 \Rightarrow x^2 - 9 = 0.\)  
  Difference of squares: \((x - 3)(x + 3) = 0.\)  
  \(\boxed{x = 3 \text{ or } x = -3}\)

  \item \(x^3 - 9x = 0\)  
  Factor out \(x\):  
  \(x(x^2 - 9) = x(x - 3)(x + 3) = 0.\)  
  \(\boxed{x = 0,\; x = 3,\; x = -3}\)

  \item \(2x^2 - 10x + 12 = 0\)  
  Factor out 2: \(2(x^2 - 5x + 6) = 0.\)  
  Then \(x^2 - 5x + 6 = (x - 2)(x - 3) = 0.\)  
  \(\boxed{x = 2 \text{ or } x = 3}\)
\end{enumerate}

\subsection*{Part E Solutions: Word Problems and SAT-Style Applications}
\begin{enumerate}
  \setcounter{enumi}{20}
  \item Product of two consecutive positive integers is 42.  
  Let the smaller be \(n\).  
  \(n(n+1) = 42 \Rightarrow n^2 + n - 42 = 0.\)  
  \((n - 6)(n + 7) = 0 \Rightarrow n = 6 \text{ or } n = -7.\)  
  Positive pair: \(\boxed{6 \text{ and } 7}\)

  \item Rectangle area: \((x + 5)(x - 3) = 60.\)  
  Expand: \(x^2 + 2x - 15 = 60 \Rightarrow x^2 + 2x - 75 = 0.\)  
  Quadratic formula:
  \[
  x = \frac{-2 \pm \sqrt{2^2 - 4(1)(-75)}}{2}
  = \frac{-2 \pm \sqrt{304}}{2}
  = \frac{-2 \pm 4\sqrt{19}}{2}
  = -1 \pm 2\sqrt{19}.
  \]
  For a physical rectangle, we need \(x > 3\).  
  \(-1 + 2\sqrt{19} \approx 7.7\) is valid.  
  \(\boxed{x = -1 + 2\sqrt{19}}\)

  \item Height: \(h = -5t^2 + 20t.\)  
  Hits the ground when \(h = 0\):  
  \(-5t^2 + 20t = 0 \Rightarrow -5t(t - 4) = 0.\)  
  \(t = 0\) or \(t = 4.\)  
  \(\boxed{t = 0 \text{ or } t = 4}\)

  \item Square of a number is 9 more than twice the number.  
  Let the number be \(n\).  
  \(n^2 = 2n + 9 \Rightarrow n^2 - 2n - 9 = 0.\)  
  Quadratic formula:
  \[
  n = \frac{2 \pm \sqrt{(-2)^2 - 4(1)(-9)}}{2}
  = \frac{2 \pm \sqrt{4 + 36}}{2}
  = \frac{2 \pm \sqrt{40}}{2}
  = \frac{2 \pm 2\sqrt{10}}{2}
  = 1 \pm \sqrt{10}.
  \]
  \(\boxed{n = 1 + \sqrt{10} \text{ or } n = 1 - \sqrt{10}}\)

  \item Product of two consecutive even integers is 48.  
  Let the smaller be \(k\), next is \(k + 2.\)  
  \(k(k+2) = 48 \Rightarrow k^2 + 2k - 48 = 0.\)  
  \((k + 8)(k - 6) = 0 \Rightarrow k = -8 \text{ or } k = 6.\)  
  Pairs: \((-8, -6)\) or \((6, 8).\)  
  \(\boxed{6 \text{ and } 8 \quad \text{(also } -8 \text{ and } -6\text{)}}\)
\end{enumerate}




\end{document}
