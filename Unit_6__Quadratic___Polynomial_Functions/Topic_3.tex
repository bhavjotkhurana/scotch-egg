\documentclass[12pt]{article}

\usepackage{amsmath, amssymb}
\usepackage{geometry}
\usepackage{setspace}
\usepackage{titlesec}
\usepackage{lmodern}
\usepackage{xcolor}
\usepackage{enumitem}

\geometry{margin=1in}
\setstretch{1.2}
\titleformat{\section}{\normalfont\Large\bfseries}{\thesection}{1em}{}
\titleformat{\subsection}{\normalfont\large\bfseries}{\thesubsection}{1em}{}
\pagenumbering{gobble}

\begin{document}

\begin{center}
    \LARGE \textbf{Unit 6: Quadratic and Polynomial Functions} \\[6pt]
    \Large \textbf{Topic 3: Quadratic Formula and Discriminant}
\end{center}

\vspace{1em}

\section*{Concept Summary}

When a quadratic equation cannot be easily factored, we use the \textbf{quadratic formula}:
\[
x = \frac{-b \pm \sqrt{b^2 - 4ac}}{2a}.
\]
It gives the exact solutions to any quadratic equation \(ax^2 + bx + c = 0\) where \(a \ne 0\).

The expression under the square root,
\[
\Delta = b^2 - 4ac,
\]
is called the \textbf{discriminant}. It determines the nature and number of the roots.

\subsection*{Meaning of the Discriminant}
\begin{itemize}
  \item \(\Delta > 0:\) two distinct real roots
  \item \(\Delta = 0:\) one real root (a repeated or double root)
  \item \(\Delta < 0:\) no real roots (two complex solutions)
\end{itemize}

On the SAT, problems often test whether you can use the discriminant to reason about the number or type of solutions without computing them.

\section*{Core Skills}
\begin{itemize}
  \item Identify \(a, b, c\) in standard form.
  \item Substitute into the quadratic formula correctly.
  \item Simplify radicals accurately.
  \item Interpret the discriminant to determine the number of real solutions.
  \item Apply the formula to both exact and approximate contexts.
\end{itemize}

\section*{Example 1: Using the Formula Directly}

Solve \(x^2 - 5x + 6 = 0.\)

Here \(a = 1,\; b = -5,\; c = 6.\)

\[
x = \frac{-(-5) \pm \sqrt{(-5)^2 - 4(1)(6)}}{2(1)} 
= \frac{5 \pm \sqrt{25 - 24}}{2} 
= \frac{5 \pm 1}{2}.
\]
\[
x = 3 \text{ or } 2.
\]
\textbf{Solutions:} \(\boxed{x = 2, 3}\)

\section*{Example 2: No Real Roots}

Solve \(x^2 + 4x + 8 = 0.\)

\(a = 1,\; b = 4,\; c = 8.\)
\[
x = \frac{-4 \pm \sqrt{4^2 - 4(1)(8)}}{2(1)} 
= \frac{-4 \pm \sqrt{16 - 32}}{2} 
= \frac{-4 \pm \sqrt{-16}}{2}.
\]
Since the discriminant is negative, no real roots exist.  
\textbf{Answer:} \(\boxed{\text{No real solutions.}}\)

\section*{Example 3: One Double Root}

Solve \(x^2 - 6x + 9 = 0.\)

\(a = 1,\; b = -6,\; c = 9.\)
\[
\Delta = (-6)^2 - 4(1)(9) = 36 - 36 = 0.
\]
\[
x = \frac{-(-6)}{2(1)} = \frac{6}{2} = 3.
\]
\textbf{One double root:} \(\boxed{x = 3}\)

\section*{Example 4: Simplifying with Square Roots}

Solve \(2x^2 - 3x - 2 = 0.\)

\(a = 2,\; b = -3,\; c = -2.\)
\[
x = \frac{-(-3) \pm \sqrt{(-3)^2 - 4(2)(-2)}}{2(2)} 
= \frac{3 \pm \sqrt{9 + 16}}{4}
= \frac{3 \pm \sqrt{25}}{4}
= \frac{3 \pm 5}{4}.
\]
\[
x = 2 \text{ or } -\tfrac{1}{2}.
\]
\textbf{Solutions:} \(\boxed{x = 2, -\tfrac{1}{2}}\)

\section*{Example 5: Interpreting the Discriminant}

For \(4x^2 + 4x + 1 = 0,\)
\[
\Delta = 4^2 - 4(4)(1) = 16 - 16 = 0.
\]
So there is one repeated real solution.  
\textbf{Root:} \(x = \frac{-4}{8} = -\tfrac{1}{2}.\)

\section*{Example 6: SAT Application}

A projectile’s height is given by \(h = -16t^2 + 32t + 48.\)  
When does it hit the ground (\(h = 0\))?

\[
-16t^2 + 32t + 48 = 0 \Rightarrow t = \frac{-32 \pm \sqrt{32^2 - 4(-16)(48)}}{2(-16)}.
\]
\[
t = \frac{-32 \pm \sqrt{1024 + 3072}}{-32}
= \frac{-32 \pm \sqrt{4096}}{-32}
= \frac{-32 \pm 64}{-32}.
\]
\[
t = 3 \text{ or } -1.5.
\]
Time cannot be negative, so \(\boxed{t = 3\text{ seconds}}\).

\section*{Key Takeaways}
\begin{itemize}
  \item The quadratic formula works for all quadratics—factorable or not.
  \item The discriminant \(b^2 - 4ac\) determines the number and type of roots.
  \item Always simplify radicals carefully and reduce fractions.
  \item If \(\Delta = 0\), expect one double root.
  \item If \(\Delta < 0\), the solutions are complex (no real roots on SAT).
\end{itemize}

\newpage

% ============================================================
% QUESTIONS — UNIT 6, TOPIC 3: QUADRATIC FORMULA AND DISCRIMINANT
% ============================================================

\section*{Practice Questions: Quadratic Formula and Discriminant}

\subsection*{Part A: Direct Application of the Quadratic Formula}
\begin{enumerate}
  \item Solve \(x^2 - 3x - 4 = 0\) using the quadratic formula.
  \item Solve \(x^2 - 2x - 8 = 0.\)
  \item Solve \(2x^2 - 5x + 2 = 0.\)
  \item Solve \(x^2 + 7x + 12 = 0.\)
  \item Solve \(3x^2 + 2x - 1 = 0.\)
\end{enumerate}

\subsection*{Part B: Working with the Discriminant}
\begin{enumerate}
  \setcounter{enumi}{5}
  \item Find the discriminant of \(x^2 + 4x + 3 = 0\) and describe the number of real solutions.
  \item Find the discriminant of \(x^2 + 2x + 1 = 0.\)
  \item Find the discriminant of \(x^2 + 6x + 10 = 0.\)
  \item Find the discriminant of \(4x^2 + 4x + 1 = 0.\)
  \item Find the discriminant of \(5x^2 - 2x + 7 = 0.\)
\end{enumerate}

\subsection*{Part C: Simplifying with Radicals}
\begin{enumerate}
  \setcounter{enumi}{10}
  \item Solve \(x^2 - 8x + 7 = 0.\)
  \item Solve \(2x^2 - 3x - 2 = 0.\)
  \item Solve \(x^2 - 2x - 3 = 0.\)
  \item Solve \(x^2 + x - 6 = 0.\)
  \item Solve \(x^2 - 4x + 1 = 0.\)
\end{enumerate}

\subsection*{Part D: Interpreting and Comparing Discriminants}
\begin{enumerate}
  \setcounter{enumi}{15}
  \item Compare the discriminants of \(x^2 - 4x + 4 = 0\) and \(x^2 - 4x + 5 = 0.\)
  \item For which value of \(k\) will \(x^2 + 4x + k = 0\) have exactly one real root?
  \item For which value of \(k\) will \(x^2 + 4x + k = 0\) have no real roots?
  \item A quadratic has one real solution if its discriminant equals 0. Find \(b\) if \(4x^2 + bx + 9 = 0\) has one solution.
  \item Determine the number of real roots for \(x^2 - 6x + 10 = 0.\)
\end{enumerate}

\subsection*{Part E: Word Problems and SAT-Style Applications}
\begin{enumerate}
  \setcounter{enumi}{20}
  \item A ball is thrown upward with height \(h = -16t^2 + 64t + 80.\) When does it hit the ground?
  \item The area of a garden is modeled by \(A = x^2 + 5x + 6.\) For what values of \(x\) is \(A = 0\)?
  \item A parabola crosses the x-axis at only one point. What must be true about its discriminant?
  \item A quadratic has two distinct real roots. What must be true about its discriminant?
  \item The equation \(2x^2 + kx + 8 = 0\) has no real solutions. What values of \(k\) satisfy this condition?
\end{enumerate}

\newpage

% ============================================================
% SOLUTIONS — UNIT 6, TOPIC 3: QUADRATIC FORMULA AND DISCRIMINANT
% ============================================================

\section*{Answer Key and Solutions: Quadratic Formula and Discriminant}

\subsection*{Part A Solutions: Direct Application of the Quadratic Formula}
\begin{enumerate}
  \item Solve \(x^2 - 3x - 4 = 0.\)

  \(a = 1,\; b = -3,\; c = -4.\)

  \[
  x = \frac{-b \pm \sqrt{b^2 - 4ac}}{2a}
  = \frac{-(-3) \pm \sqrt{(-3)^2 - 4(1)(-4)}}{2(1)}
  = \frac{3 \pm \sqrt{9 + 16}}{2}
  = \frac{3 \pm \sqrt{25}}{2}
  = \frac{3 \pm 5}{2}.
  \]

  \(\boxed{x = 4 \text{ or } x = -1}\)

  \item Solve \(x^2 - 2x - 8 = 0.\)

  \(a = 1,\; b = -2,\; c = -8.\)

  \[
  x = \frac{2 \pm \sqrt{(-2)^2 - 4(1)(-8)}}{2}
  = \frac{2 \pm \sqrt{4 + 32}}{2}
  = \frac{2 \pm \sqrt{36}}{2}
  = \frac{2 \pm 6}{2}.
  \]

  \(\boxed{x = 4 \text{ or } x = -2}\)

  \item Solve \(2x^2 - 5x + 2 = 0.\)

  \(a = 2,\; b = -5,\; c = 2.\)

  \[
  x = \frac{5 \pm \sqrt{(-5)^2 - 4(2)(2)}}{4}
  = \frac{5 \pm \sqrt{25 - 16}}{4}
  = \frac{5 \pm \sqrt{9}}{4}
  = \frac{5 \pm 3}{4}.
  \]

  \(\boxed{x = 2 \text{ or } x = \tfrac{1}{2}}\)

  \item Solve \(x^2 + 7x + 12 = 0.\)

  \(a = 1,\; b = 7,\; c = 12.\)

  \[
  x = \frac{-7 \pm \sqrt{7^2 - 4(1)(12)}}{2}
  = \frac{-7 \pm \sqrt{49 - 48}}{2}
  = \frac{-7 \pm \sqrt{1}}{2}
  = \frac{-7 \pm 1}{2}.
  \]

  \(\boxed{x = -3 \text{ or } x = -4}\)

  \item Solve \(3x^2 + 2x - 1 = 0.\)

  \(a = 3,\; b = 2,\; c = -1.\)

  \[
  x = \frac{-2 \pm \sqrt{2^2 - 4(3)(-1)}}{2 \cdot 3}
  = \frac{-2 \pm \sqrt{4 + 12}}{6}
  = \frac{-2 \pm \sqrt{16}}{6}
  = \frac{-2 \pm 4}{6}.
  \]

  \(\boxed{x = \tfrac{1}{3} \text{ or } x = -1}\)
\end{enumerate}

\subsection*{Part B Solutions: Working with the Discriminant}
\begin{enumerate}
  \setcounter{enumi}{5}
  \item For \(x^2 + 4x + 3 = 0\):  
  \(\Delta = b^2 - 4ac = 4^2 - 4(1)(3) = 16 - 12 = 4 > 0.\)  
  \(\boxed{\text{Two distinct real solutions}}\)

  \item For \(x^2 + 2x + 1 = 0\):  
  \(\Delta = 2^2 - 4(1)(1) = 4 - 4 = 0.\)  
  \(\boxed{\text{One real solution (double root)}}\)

  \item For \(x^2 + 6x + 10 = 0\):  
  \(\Delta = 6^2 - 4(1)(10) = 36 - 40 = -4 < 0.\)  
  \(\boxed{\text{No real solutions}}\)

  \item For \(4x^2 + 4x + 1 = 0\):  
  \(\Delta = 4^2 - 4(4)(1) = 16 - 16 = 0.\)  
  \(\boxed{\text{One real solution (double root)}}\)

  \item For \(5x^2 - 2x + 7 = 0\):  
  \(\Delta = (-2)^2 - 4(5)(7) = 4 - 140 = -136 < 0.\)  
  \(\boxed{\text{No real solutions}}\)
\end{enumerate}

\subsection*{Part C Solutions: Simplifying with Radicals}
\begin{enumerate}
  \setcounter{enumi}{10}
  \item Solve \(x^2 - 8x + 7 = 0.\)

  \(a = 1,\; b = -8,\; c = 7.\)

  \[
  x = \frac{8 \pm \sqrt{64 - 28}}{2}
  = \frac{8 \pm \sqrt{36}}{2}
  = \frac{8 \pm 6}{2}.
  \]

  \(\boxed{x = 7 \text{ or } x = 1}\)

  \item Solve \(2x^2 - 3x - 2 = 0.\)

  \(a = 2,\; b = -3,\; c = -2.\)

  \[
  x = \frac{3 \pm \sqrt{(-3)^2 - 4(2)(-2)}}{4}
  = \frac{3 \pm \sqrt{9 + 16}}{4}
  = \frac{3 \pm \sqrt{25}}{4}
  = \frac{3 \pm 5}{4}.
  \]

  \(\boxed{x = 2 \text{ or } x = -\tfrac{1}{2}}\)

  \item Solve \(x^2 - 2x - 3 = 0.\)

  \(a = 1,\; b = -2,\; c = -3.\)

  \[
  x = \frac{2 \pm \sqrt{4 - 4(1)(-3)}}{2}
  = \frac{2 \pm \sqrt{4 + 12}}{2}
  = \frac{2 \pm \sqrt{16}}{2}
  = \frac{2 \pm 4}{2}.
  \]

  \(\boxed{x = 3 \text{ or } x = -1}\)

  \item Solve \(x^2 + x - 6 = 0.\)

  \(a = 1,\; b = 1,\; c = -6.\)

  \[
  x = \frac{-1 \pm \sqrt{1^2 - 4(1)(-6)}}{2}
  = \frac{-1 \pm \sqrt{1 + 24}}{2}
  = \frac{-1 \pm \sqrt{25}}{2}
  = \frac{-1 \pm 5}{2}.
  \]

  \(\boxed{x = 2 \text{ or } x = -3}\)

  \item Solve \(x^2 - 4x + 1 = 0.\)

  \(a = 1,\; b = -4,\; c = 1.\)

  \[
  x = \frac{4 \pm \sqrt{(-4)^2 - 4(1)(1)}}{2}
  = \frac{4 \pm \sqrt{16 - 4}}{2}
  = \frac{4 \pm \sqrt{12}}{2}
  = \frac{4 \pm 2\sqrt{3}}{2}
  = 2 \pm \sqrt{3}.
  \]

  \(\boxed{x = 2 + \sqrt{3} \text{ or } x = 2 - \sqrt{3}}\)
\end{enumerate}

\subsection*{Part D Solutions: Interpreting and Comparing Discriminants}
\begin{enumerate}
  \setcounter{enumi}{15}
  \item Compare \(x^2 - 4x + 4 = 0\) and \(x^2 - 4x + 5 = 0.\)

  For \(x^2 - 4x + 4\):  
  \(\Delta = (-4)^2 - 4(1)(4) = 16 - 16 = 0\) → one real (double) root.

  For \(x^2 - 4x + 5\):  
  \(\Delta = (-4)^2 - 4(1)(5) = 16 - 20 = -4 < 0\) → no real roots.

  \(\boxed{\text{First: one real root. Second: no real roots.}}\)

  \item For what value of \(k\) does \(x^2 + 4x + k = 0\) have exactly one real root?

  Need \(\Delta = 0.\)  
  \(\Delta = 4^2 - 4(1)(k) = 16 - 4k = 0.\)  
  \(16 = 4k \Rightarrow k = 4.\)  
  \(\boxed{k = 4}\)

  \item For what value of \(k\) does \(x^2 + 4x + k = 0\) have no real roots?

  Need \(\Delta < 0.\)  
  \(16 - 4k < 0 \Rightarrow 16 < 4k \Rightarrow k > 4.\)  
  \(\boxed{k > 4}\)

  \item A quadratic \(4x^2 + bx + 9 = 0\) has one real solution. Find \(b.\)

  One real solution means \(\Delta = 0.\)

  \[
  \Delta = b^2 - 4(4)(9) = b^2 - 144 = 0
  \Rightarrow b^2 = 144
  \Rightarrow b = \pm 12.
  \]

  \(\boxed{b = 12 \text{ or } b = -12}\)

  \item Determine the number of real roots for \(x^2 - 6x + 10 = 0.\)

  \[
  \Delta = (-6)^2 - 4(1)(10) = 36 - 40 = -4 < 0.
  \]

  \(\boxed{\text{No real roots}}\)
\end{enumerate}

\subsection*{Part E Solutions: Word Problems and SAT-Style Applications}
\begin{enumerate}
  \setcounter{enumi}{20}
  \item A ball is thrown upward with height \(h = -16t^2 + 64t + 80.\) When does it hit the ground?

  Set \(h = 0\):  
  \(-16t^2 + 64t + 80 = 0.\)

  \(a = -16,\; b = 64,\; c = 80.\)

  \[
  t = \frac{-64 \pm \sqrt{64^2 - 4(-16)(80)}}{2(-16)}
  = \frac{-64 \pm \sqrt{4096 + 5120}}{-32}
  = \frac{-64 \pm \sqrt{9216}}{-32}
  = \frac{-64 \pm 96}{-32}.
  \]

  Two solutions: \(t = -1\) and \(t = 5.\)

  Negative time is not physical, so \(\boxed{t = 5 \text{ seconds}}\)

  \item The area of a garden is modeled by \(A = x^2 + 5x + 6.\) For what values of \(x\) is \(A = 0\)?

  Solve \(x^2 + 5x + 6 = 0.\)

  \(a = 1,\; b = 5,\; c = 6.\)

  \[
  x = \frac{-5 \pm \sqrt{25 - 24}}{2}
  = \frac{-5 \pm 1}{2}.
  \]

  \(\boxed{x = -2 \text{ or } x = -3}\)

  (In context, negative \(x\) might be nonphysical, but algebraically both are solutions.)

  \item A parabola crosses the x-axis at only one point. What must be true about its discriminant?

  Crossing at only one x-value means a double root.  
  \(\boxed{\Delta = 0}\)

  \item A quadratic has two distinct real roots. What must be true about its discriminant?

  Two different real roots means square root is positive.  
  \(\boxed{\Delta > 0}\)

  \item The equation \(2x^2 + kx + 8 = 0\) has no real solutions. What values of \(k\) satisfy this condition?

  Need \(\Delta < 0.\)

  \[
  \Delta = k^2 - 4(2)(8) = k^2 - 64 < 0
  \Rightarrow k^2 < 64
  \Rightarrow -8 < k < 8.
  \]

  \(\boxed{-8 < k < 8}\)
\end{enumerate}



\end{document}
