\documentclass[12pt]{article}

\usepackage{amsmath, amssymb}
\usepackage{geometry}
\usepackage{setspace}
\usepackage{titlesec}
\usepackage{lmodern}
\usepackage{xcolor}
\usepackage{enumitem}

\geometry{margin=1in}
\setstretch{1.2}
\titleformat{\section}{\normalfont\Large\bfseries}{\thesection}{1em}{}
\titleformat{\subsection}{\normalfont\large\bfseries}{\thesubsection}{1em}{}
\pagenumbering{gobble}

\begin{document}

\begin{center}
    \LARGE \textbf{Unit 6: Quadratic and Polynomial Functions} \\[6pt]
    \Large \textbf{Topic 6: Zeros, Factors, and Graphs of Polynomials}
\end{center}

\vspace{1em}

\section*{Concept Summary}

A \textbf{zero} (or \textbf{root}) of a polynomial function is an \(x\)-value that makes the function equal to zero:
\[
f(x) = 0.
\]
Each zero corresponds to an \textbf{x-intercept} on the graph of the polynomial.

A \textbf{factor} is a binomial or term that divides the polynomial evenly.  
If \((x - r)\) is a factor of \(f(x)\), then \(r\) is a zero of the function.  
This is known as the **Factor Theorem**.

\[
f(r) = 0 \quad \Leftrightarrow \quad (x - r) \text{ is a factor of } f(x).
\]

The graph of a polynomial touches or crosses the x-axis at its zeros, depending on their multiplicity:
\begin{itemize}
  \item If a zero has \textbf{odd multiplicity}, the graph crosses the x-axis.
  \item If a zero has \textbf{even multiplicity}, the graph touches the x-axis and turns around.
\end{itemize}

\subsection*{End Behavior}
The leading term (the term with the highest power of \(x\)) determines how the graph behaves at the ends:
\[
f(x) = a_nx^n + \dots
\]
\[
\begin{cases}
a_n > 0, n \text{ even: both ends up} \\
a_n < 0, n \text{ even: both ends down} \\
a_n > 0, n \text{ odd: left down, right up} \\
a_n < 0, n \text{ odd: left up, right down}
\end{cases}
\]

\section*{Core Skills}
\begin{itemize}
  \item Identify zeros from factored forms of polynomials.
  \item Determine multiplicity and predict graph behavior at each zero.
  \item Use the Factor Theorem to test if a given value is a zero.
  \item Sketch graphs based on zeros, multiplicities, and end behavior.
  \item Connect factored form, standard form, and graphical interpretations.
\end{itemize}

\section*{Example 1: Identifying Zeros from Factored Form}

Let \(f(x) = (x - 2)(x + 1)(x - 4)\).

\textbf{Step 1:} Set each factor equal to zero:
\[
x - 2 = 0 \Rightarrow x = 2, \quad x + 1 = 0 \Rightarrow x = -1, \quad x - 4 = 0 \Rightarrow x = 4.
\]

\textbf{Zeros:} \(-1, 2, 4.\)

\textbf{Graph:} (Insert graph showing intercepts at –1, 2, 4 and right end rising, left end falling since \(a > 0, n = 3.\))

\section*{Example 2: Multiplicity of Zeros}

Let \(f(x) = (x - 3)^2(x + 1).\)

\textbf{Zeros:} \(x = 3\) (multiplicity 2), \(x = -1\) (multiplicity 1).  
At \(x = 3\), the graph \textbf{touches} and turns around (even multiplicity).  
At \(x = -1\), the graph \textbf{crosses} (odd multiplicity).

\textbf{Graph:} (Insert graph showing touch at 3, cross at –1, both ends upward since leading coefficient positive and degree 3.)

\section*{Example 3: Testing for Factors (Factor Theorem)}

Let \(f(x) = x^3 - 6x^2 + 11x - 6.\)

Test if \(x - 1\) is a factor. Substitute \(x = 1\):
\[
f(1) = 1 - 6 + 11 - 6 = 0.
\]
Since \(f(1) = 0,\) \((x - 1)\) is a factor.

Similarly, test \(x = 2\):  
\(f(2) = 8 - 24 + 22 - 6 = 0.\)
Also a factor.  
\(x = 3:\; f(3) = 27 - 54 + 33 - 6 = 0.\)
All three are zeros.

\textbf{Factored form:} \(\boxed{f(x) = (x - 1)(x - 2)(x - 3)}\)

\textbf{Graph:} (Insert cubic graph crossing x-axis at 1, 2, 3, left down, right up.)

\section*{Example 4: End Behavior and Degree}

Determine the end behavior of:
\[
f(x) = -2x^4 + 5x^3 - x + 7.
\]
Leading term is \(-2x^4.\)

Since \(a = -2 < 0\) and degree is even,  
both ends point \textbf{downward.}

\section*{Example 5: Building a Polynomial from Given Zeros}

Find the polynomial with zeros \(x = -2, 1, 3.\)

\[
f(x) = (x + 2)(x - 1)(x - 3).
\]
Expand:
\[
(x + 2)(x^2 - 4x + 3) = x^3 - 4x^2 + 3x + 2x^2 - 8x + 6 = x^3 - 2x^2 - 5x + 6.
\]

\(\boxed{f(x) = x^3 - 2x^2 - 5x + 6}\)

\section*{Key Takeaways}
\begin{itemize}
  \item Zeros are x-values where \(f(x) = 0.\)
  \item If \((x - r)\) is a factor, then \(r\) is a zero (Factor Theorem).
  \item Even multiplicity → graph touches; odd multiplicity → graph crosses.
  \item Leading term determines end behavior.
  \item Factored form reveals zeros and structure quickly for sketching.
\end{itemize}

\newpage

% ============================================================
% QUESTIONS — UNIT 6, TOPIC 6: ZEROS, FACTORS, AND GRAPHS OF POLYNOMIALS
% ============================================================

\section*{Practice Questions: Zeros, Factors, and Graphs of Polynomials}

\subsection*{Part A: Identifying Zeros from Factored Form}
\begin{enumerate}
  \item Find the zeros of \(f(x) = (x - 3)(x + 2)(x - 1)\).
  \item Find the zeros of \(f(x) = (x + 4)(x - 5)\).
  \item Find the zeros of \(f(x) = (x - 1)(x - 1)(x + 3)\).
  \item Find the zeros and their multiplicities for \(f(x) = (x - 2)^3(x + 1)^2\).
  \item Find the zeros of \(f(x) = 2(x - 6)(x + 2)(x + 1)\).
\end{enumerate}

\subsection*{Part B: Using the Factor Theorem}
\begin{enumerate}
  \setcounter{enumi}{5}
  \item For \(f(x) = x^3 - 2x^2 - 5x + 6\), test whether \(x - 1\) is a factor.
  \item For \(f(x) = 2x^3 + 3x^2 - 8x - 12\), test whether \(x + 2\) is a factor.
  \item For \(f(x) = x^3 - 4x^2 + 4x\), test whether \(x = 0\) is a zero.
  \item For \(f(x) = x^3 - 3x^2 - 4x + 12\), determine if \(x = 3\) is a zero.
  \item For \(f(x) = 4x^3 - 12x^2 + 9x - 3\), test if \(x = \tfrac{3}{2}\) is a zero.
\end{enumerate}

\subsection*{Part C: Multiplicity and Graph Behavior}
\begin{enumerate}
  \setcounter{enumi}{10}
  \item Describe the behavior of the graph at \(x = 2\) for \(f(x) = (x - 2)^2(x + 1)\).
  \item Describe the behavior of the graph at \(x = -3\) for \(f(x) = (x + 3)^3(x - 1)\).
  \item Determine whether the graph crosses or touches the x-axis at each zero: \(f(x) = (x - 4)^4(x + 2)^2(x - 1)\).
  \item Identify zeros and multiplicities: \(f(x) = (x + 2)^2(x - 5)^3\).
  \item Explain how multiplicity affects the shape of a polynomial near a zero.
\end{enumerate}

\subsection*{Part D: End Behavior and Leading Terms}
\begin{enumerate}
  \setcounter{enumi}{15}
  \item Determine the end behavior of \(f(x) = 2x^5 + 3x^4 - x + 1.\)
  \item Determine the end behavior of \(f(x) = -x^6 + 4x^3 - 7.\)
  \item Describe the end behavior of \(f(x) = -3x^3 + 5x^2 - x.\)
  \item Describe the end behavior of \(f(x) = 4x^4 - 9x^2 + 7.\)
  \item For \(f(x) = -2x^5 + 8x^3 - x\), describe how the graph behaves as \(x \to \infty\) and \(x \to -\infty.\)
\end{enumerate}

\subsection*{Part E: SAT-Style Applications}
(Insert graphs for visual interpretation where noted below.)

\begin{enumerate}
  \setcounter{enumi}{20}
  \item A cubic function has zeros at \(x = -1, 2,\) and \(4.\) Write a possible factored form of the function.
  \item The graph of a polynomial touches the x-axis at \(x = 3\) and crosses at \(x = -1.\) Write a possible factored form.
  \item A quartic polynomial has end behavior with both ends up and zeros at \(x = -2\) (multiplicity 2) and \(x = 4\) (multiplicity 2). Write a possible equation.
  \item (Insert graph: quartic with x-intercepts at –2, 0, and 3, both ends up.) List all zeros and describe the end behavior.
  \item The graph of \(f(x)\) crosses the x-axis at \(x = -1\) and \(x = 4\), and touches at \(x = 2.\) Which of the following could be its factored form?  
  A) \((x + 1)(x - 2)(x - 4)\)  
  B) \((x + 1)(x - 2)^2(x - 4)\)  
  C) \((x + 1)^2(x - 2)(x - 4)\)  
  D) \((x + 1)(x - 4)^2(x - 2)\)
\end{enumerate}

\newpage

% ============================================================
% SOLUTIONS — UNIT 6, TOPIC 6: ZEROS, FACTORS, AND GRAPHS OF POLYNOMIALS
% ============================================================

\section*{Answer Key and Solutions: Zeros, Factors, and Graphs of Polynomials}

\subsection*{Part A Solutions: Identifying Zeros from Factored Form}
\begin{enumerate}
  \item \(f(x) = (x - 3)(x + 2)(x - 1)\)

  Set each factor to 0:
  \[
  x - 3 = 0 \Rightarrow x = 3,\quad
  x + 2 = 0 \Rightarrow x = -2,\quad
  x - 1 = 0 \Rightarrow x = 1.
  \]

  \(\boxed{x = -2,\; 1,\; 3}\)

  \item \(f(x) = (x + 4)(x - 5)\)

  \[
  x + 4 = 0 \Rightarrow x = -4,\quad
  x - 5 = 0 \Rightarrow x = 5.
  \]

  \(\boxed{x = -4,\; 5}\)

  \item \(f(x) = (x - 1)(x - 1)(x + 3) = (x - 1)^2(x + 3)\)

  \[
  x - 1 = 0 \Rightarrow x = 1 \text{ (multiplicity 2)}, \quad
  x + 3 = 0 \Rightarrow x = -3 \text{ (multiplicity 1)}.
  \]

  \(\boxed{x = 1 \text{ (double root)},\; x = -3}\)

  \item \(f(x) = (x - 2)^3(x + 1)^2\)

  \[
  x - 2 = 0 \Rightarrow x = 2 \text{ (mult 3)},\quad
  x + 1 = 0 \Rightarrow x = -1 \text{ (mult 2)}.
  \]

  \(\boxed{x = 2 \text{ (mult 3)},\; x = -1 \text{ (mult 2)}}\)

  \item \(f(x) = 2(x - 6)(x + 2)(x + 1)\)

  Constant 2 does not affect zeros.

  \[
  x - 6 = 0 \Rightarrow x = 6,\quad
  x + 2 = 0 \Rightarrow x = -2,\quad
  x + 1 = 0 \Rightarrow x = -1.
  \]

  \(\boxed{x = -2,\; -1,\; 6}\)
\end{enumerate}

\subsection*{Part B Solutions: Using the Factor Theorem}
\begin{enumerate}
  \setcounter{enumi}{5}
  \item \(f(x) = x^3 - 2x^2 - 5x + 6\). Test \(x - 1\).

  Factor Theorem: plug in \(x = 1.\)

  \[
  f(1) = 1 - 2 - 5 + 6 = 0.
  \]

  Since \(f(1) = 0\), \((x - 1)\) is a factor.

  \(\boxed{\text{Yes, } x - 1 \text{ is a factor}}\)

  \item \(f(x) = 2x^3 + 3x^2 - 8x - 12\). Test \(x + 2\).

  \(x + 2\) is a factor if \(f(-2) = 0.\)

  \[
  f(-2) = 2(-8) + 3(4) - 8(-2) - 12 = -16 + 12 + 16 - 12 = 0.
  \]

  \(\boxed{\text{Yes, } x + 2 \text{ is a factor}}\)

  \item \(f(x) = x^3 - 4x^2 + 4x\). Test \(x = 0\).

  \[
  f(0) = 0 - 0 + 0 = 0.
  \]

  So \(x = 0\) is a zero, and \(x\) is a factor.

  \(\boxed{x = 0 \text{ is a zero}}\)

  \item \(f(x) = x^3 - 3x^2 - 4x + 12\). Test \(x = 3\).

  \[
  f(3) = 27 - 27 - 12 + 12 = 0.
  \]

  So \(x = 3\) is a zero, and \((x - 3)\) is a factor.

  \(\boxed{x = 3 \text{ is a zero}}\)

  \item \(f(x) = 4x^3 - 12x^2 + 9x - 3\). Test \(x = \tfrac{3}{2}\).

  Substitute \(x = \tfrac{3}{2}\):

  First compute step by step:

  \[
  4\left(\tfrac{3}{2}\right)^3
  = 4 \cdot \tfrac{27}{8}
  = \tfrac{108}{8}
  = \tfrac{27}{2}.
  \]

  \[
  -12\left(\tfrac{3}{2}\right)^2
  = -12 \cdot \tfrac{9}{4}
  = -27.
  \]

  \[
  9\left(\tfrac{3}{2}\right)
  = \tfrac{27}{2}.
  \]

  Then minus 3:

  Add them:
  \[
  \tfrac{27}{2} - 27 + \tfrac{27}{2} - 3
  = \left(\tfrac{27}{2} + \tfrac{27}{2}\right) + (-27 - 3)
  = \tfrac{54}{2} - 30
  = 27 - 30
  = -3 \ne 0.
  \]

  \(\boxed{\text{No, } x = \tfrac{3}{2} \text{ is not a zero}}\)
\end{enumerate}

\subsection*{Part C Solutions: Multiplicity and Graph Behavior}
\begin{enumerate}
  \setcounter{enumi}{10}
  \item \(f(x) = (x - 2)^2(x + 1)\)

  At \(x = 2\), multiplicity is 2 (even).  
  The graph \textbf{touches} the x-axis at \(x = 2\) and turns around.

  \(\boxed{\text{Touches and turns at } x = 2}\)

  \item \(f(x) = (x + 3)^3(x - 1)\)

  At \(x = -3\), multiplicity is 3 (odd).  
  The graph \textbf{crosses} the x-axis at \(x = -3\), but flattens slightly there.

  \(\boxed{\text{Crosses at } x = -3 \text{ with flattening}}\)

  \item \(f(x) = (x - 4)^4(x + 2)^2(x - 1)\)

  Zeros:
  \[
  x = 4 \text{ (mult 4, even)},\quad
  x = -2 \text{ (mult 2, even)},\quad
  x = 1 \text{ (mult 1, odd)}.
  \]

  At even multiplicity zeros (4 and -2), the graph \textbf{touches} the axis.  
  At odd multiplicity zeros (1), the graph \textbf{crosses}.

  \(\boxed{\text{Touches at } x = 4, -2;\; crosses at } x = 1}\)

  \item \(f(x) = (x + 2)^2(x - 5)^3\)

  Zeros and multiplicities:
  \[
  x = -2 \text{ (mult 2, even)},\quad
  x = 5 \text{ (mult 3, odd)}.
  \]

  \(\boxed{x = -2 \text{ (touch)},\; x = 5 \text{ (cross)}}\)

  \item Multiplicity effect:

  If the multiplicity is even, the graph touches the x-axis and bounces.  
  If the multiplicity is odd, the graph crosses the x-axis.

  Higher multiplicity makes the graph flatter near that zero.

  \(\boxed{\text{Even mult = touch; odd mult = cross; larger mult = flatter}}\)
\end{enumerate}

\subsection*{Part D Solutions: End Behavior and Leading Terms}
\begin{enumerate}
  \setcounter{enumi}{15}
  \item \(f(x) = 2x^5 + 3x^4 - x + 1\)

  Leading term: \(2x^5.\)  
  Degree is odd, leading coefficient positive.

  As \(x \to \infty,\; f(x) \to \infty.\)  
  As \(x \to -\infty,\; f(x) \to -\infty.\)

  \(\boxed{\text{Left down, right up}}\)

  \item \(f(x) = -x^6 + 4x^3 - 7\)

  Leading term: \(-x^6.\)  
  Degree is even, leading coefficient negative.

  As \(x \to \infty,\; f(x) \to -\infty.\)  
  As \(x \to -\infty,\; f(x) \to -\infty.\)

  \(\boxed{\text{Both ends down}}\)

  \item \(f(x) = -3x^3 + 5x^2 - x\)

  Leading term: \(-3x^3.\)  
  Degree is odd, leading coefficient negative.

  As \(x \to \infty,\; f(x) \to -\infty.\)  
  As \(x \to -\infty,\; f(x) \to \infty.\)

  \(\boxed{\text{Left up, right down}}\)

  \item \(f(x) = 4x^4 - 9x^2 + 7\)

  Leading term: \(4x^4.\)  
  Degree is even, leading coefficient positive.

  As \(x \to \infty,\; f(x) \to \infty.\)  
  As \(x \to -\infty,\; f(x) \to \infty.\)

  \(\boxed{\text{Both ends up}}\)

  \item \(f(x) = -2x^5 + 8x^3 - x\)

  Leading term: \(-2x^5.\)  
  Degree 5 (odd), leading coefficient negative.

  As \(x \to \infty,\; f(x) \to -\infty.\)  
  As \(x \to -\infty,\; f(x) \to \infty.\)

  \(\boxed{\text{Left up, right down}}\)
\end{enumerate}

\subsection*{Part E Solutions: SAT-Style Applications}
\begin{enumerate}
  \setcounter{enumi}{20}
  \item A cubic has zeros at \(x = -1, 2, 4.\)

  A possible factored form (set leading coefficient to 1):
  \[
  f(x) = (x + 1)(x - 2)(x - 4).
  \]

  \(\boxed{f(x) = (x + 1)(x - 2)(x - 4)}\)

  \item Graph touches at \(x = 3\) and crosses at \(x = -1\).

  Touch at \(x = 3\) means even multiplicity, so \((x - 3)^2.\)  
  Cross at \(x = -1\) means odd multiplicity, so \((x + 1).\)

  \(\boxed{f(x) = (x - 3)^2(x + 1)}\)

  \item Quartic, both ends up, zeros at \(-2\) (mult 2) and \(4\) (mult 2)

  Even degree with positive leading coefficient gives both ends up.

  A possible equation:
  \[
  f(x) = (x + 2)^2(x - 4)^2.
  \]

  \(\boxed{f(x) = (x + 2)^2(x - 4)^2}\)

  \item (Graph described: quartic with x-intercepts at \(-2, 0, 3,\) both ends up.)

  Zeros: \(-2, 0, 3.\)

  If both ends are up, degree is even and leading coefficient positive.  
  To get an even degree with those three distinct zeros, at least one zero must have even multiplicity.  
  We cannot see multiplicities from text alone, so one valid possibility is:
  \[
  f(x) = x(x + 2)(x - 3)^2
  \]
  where \((x - 3)^2\) gives even degree overall and both ends up.

  \(\boxed{\text{Zeros: } x = -2,\; 0,\; 3 \text{ (3 with even multiplicity). End behavior: both ends up}}\)

  \item Crosses at \(x = -1\) and \(x = 4\), touches at \(x = 2\)

  Cross means odd multiplicity 1. Touch means even multiplicity (like squared).  
  So we want factors:
  \[
  (x + 1)(x - 4)(x - 2)^2.
  \]

  Option B is \((x + 1)(x - 2)^2(x - 4)\). This matches.

  \(\boxed{\text{Correct choice: B}}\)
\end{enumerate}



\end{document}

