\documentclass[12pt]{article}

\usepackage{amsmath, amssymb}
\usepackage{geometry}
\usepackage{setspace}
\usepackage{titlesec}
\usepackage{lmodern}
\usepackage{xcolor}
\usepackage{enumitem}

\geometry{margin=1in}
\setstretch{1.2}
\titleformat{\section}{\normalfont\Large\bfseries}{\thesection}{1em}{}
\titleformat{\subsection}{\normalfont\large\bfseries}{\thesubsection}{1em}{}
\pagenumbering{gobble}

\begin{document}

\begin{center}
    \LARGE \textbf{Unit 3: Ratios, Rates, and Percents} \\[6pt]
    \Large \textbf{Topic 4: Successive Percent Changes (Compound Percentage)}
\end{center}

\vspace{1em}

\section*{Concept Summary}

Successive percent changes multiply factors.  
If an amount \(A\) changes by \(p_1\%\) then by \(p_2\%\), the final value is
\[
A\cdot (1\pm p_1)\cdot (1\pm p_2),
\]
where each \(p_i\) is written as a decimal and the sign is plus for increase, minus for decrease.

Key facts:
\begin{itemize}
  \item Percent changes are not additive. A 20\% increase then a 20\% decrease does not return to the original.
  \item Order does not matter for multiplication of factors. \(A(1+p)(1-q) = A(1-q)(1+p)\).
  \item Repeated changes by the same percent use exponents: \(A(1+p)^n\) or \(A(1-p)^n\).
  \item To undo a single change by \(p\%\), divide by the factor: original \(=\dfrac{\text{new}}{1\pm p}\). To undo multiple, divide by the product of factors.
\end{itemize}

\section*{Core Skills}
\begin{itemize}
  \item Convert each percent to a multiplier \(1\pm p\) before computing.
  \item Multiply factors in sequence to get the net factor.
  \item For repeated yearly or stepwise changes, write a compact exponential model.
  \item Distinguish percent points from percent change when reading contexts.
\end{itemize}

\section*{Example 1: Increase then Decrease}

A price increases 25\% then decreases 20\%. Net factor
\[
(1+0.25)(1-0.20)=1.25\cdot0.80=1.00.
\]
Final equals original in this special case.

\section*{Example 2: Decrease then Decrease}

An item is discounted 30\% and then an extra 10\%.  
\[
\text{Final factor}=(1-0.30)(1-0.10)=0.70\cdot0.90=0.63.
\]
Final price is 63\% of the original.

\section*{Example 3: Two Increases}

Population grows by 8\% one year and 5\% the next.
\[
\text{Net factor}=(1.08)(1.05)=1.134.
\]
Overall increase is 13.4\%.

\section*{Example 4: Repeated Change Model}

A device loses 12\% of its value each year. Initial value \(V_0\). After \(n\) years
\[
V_n=V_0(1-0.12)^n=V_0(0.88)^n.
\]

\section*{Example 5: Reverse After Two Changes}

After a 10\% increase and then a 15\% decrease the final price is \$306.  
Original \(=\dfrac{306}{(1.10)(0.85)}=\dfrac{306}{0.935}\approx \$327.27\).

\section*{Example 6: Not Additive}

A 20\% increase then a 20\% decrease: factor \(1.2\cdot0.8=0.96\).  
Net change is a 4\% decrease, not zero.

\section*{Key Takeaways}
\begin{itemize}
  \item Translate every change into a multiplier and multiply.
  \item Use powers for repeated identical changes.
  \item To recover an original, divide by the product of all change factors.
  \item Do not add percent changes. Work with factors to avoid errors.
\end{itemize}

\newpage

% ============================================================
% QUESTIONS — TOPIC 4: SUCCESSIVE PERCENT CHANGES (COMPOUND PERCENTAGE)
% ============================================================

\section*{Practice Questions: Successive Percent Changes}

\subsection*{Part A: Two-Step Percent Changes}
Find the final value or total percent change.
\begin{enumerate}
  \item An item increases 20\% and then increases another 10\%.
  \item A phone’s price decreases 25\% and then increases 10\%.
  \item A stock rises 15\% one month and falls 10\% the next.
  \item A computer’s cost decreases 40\% and then 20\%.
  \item A shirt’s price increases 5\% and then decreases 5\%.
\end{enumerate}

\subsection*{Part B: Repeated Changes}
Use exponential form where applicable.
\begin{enumerate}
  \setcounter{enumi}{5}
  \item A car’s value drops 15\% per year. If the original price is \$24,000, find its value after 2 years.
  \item A town’s population grows 4\% per year. Starting at 50,000, what will it be after 3 years?
  \item A company’s revenue increases 10\% per quarter for 2 years. What is the growth factor after 8 quarters?
  \item A laptop’s value decreases 18\% per year. Express its value after \(n\) years.
  \item A population decreases 5\% per year. Write a formula for the population after \(t\) years if it starts at 80,000.
\end{enumerate}

\subsection*{Part C: Reverse Problems}
Find the original amount before the successive changes.
\begin{enumerate}
  \setcounter{enumi}{10}
  \item After an increase of 20\% and another increase of 10\%, the price is \$145.20. Find the original.
  \item A stock falls 10\% and then 5\% to a final price of \$171. Find the original price.
  \item A phone decreases 15\% and then increases 20\%, ending at \$138. Find the starting price.
  \item A population increases 12\% each of two years and ends at 125,440. Find the original population.
  \item After a 10\% tax and a 15\% discount, the final cost is \$153. Find the pre-tax original price.
\end{enumerate}

\subsection*{Part D: Comparing Percent Changes}
\begin{enumerate}
  \setcounter{enumi}{15}
  \item Which has a greater overall increase: 10\% then 20\%, or 15\% then 15\%?
  \item A decrease of 20\% followed by an increase of 20\% — net increase or decrease?
  \item A store raises prices by 25\% and then offers a 25\% discount. What is the net effect?
  \item An item is discounted twice: first 30\%, then 30\%. What single discount gives the same final price?
  \item An investment increases 8\% annually. How many years to grow by at least 25\%?
\end{enumerate}

\subsection*{Part E: SAT-Style Applications}
\begin{enumerate}
  \setcounter{enumi}{20}
  \item A car loses 20\% of its value each year. If it is worth \$16,000 now, what was it worth two years ago?
  \item An item is marked up by 30\% and then discounted by 10\%. What is the overall percent change?
  \item A stock decreases 25\% one year and then increases 40\% the next. What is the overall percent change?
  \item A company’s sales grow by 5\% per year. What is the total percent growth after 3 years?
  \item A machine’s value decreases 12\% per year. If it’s now worth \$6,912 after 3 years, what was the original value?
\end{enumerate}

\newpage

% ============================================================
% SOLUTIONS — UNIT 3, TOPIC 4: SUCCESSIVE PERCENT CHANGES (COMPOUND PERCENTAGE)
% ============================================================

\section*{Answer Key and Solutions: Successive Percent Changes}

\subsection*{Part A Solutions: Two-Step Percent Changes}
\begin{enumerate}
  \item Factor \(1.20\cdot1.10=1.32\). \(\boxed{32\%\ \text{increase}}\)
  \item Factor \(0.75\cdot1.10=0.825\). \(\boxed{17.5\%\ \text{decrease}}\)
  \item Factor \(1.15\cdot0.90=1.035\). \(\boxed{3.5\%\ \text{increase}}\)
  \item Factor \(0.60\cdot0.80=0.48\). \(\boxed{52\%\ \text{decrease}}\)
  \item Factor \(1.05\cdot0.95=0.9975\). \(\boxed{0.25\%\ \text{decrease}}\)
\end{enumerate}

\subsection*{Part B Solutions: Repeated Changes}
\begin{enumerate}
  \setcounter{enumi}{5}
  \item \(24000(0.85)^2=24000(0.7225)=\boxed{17340}\)
  \item \(50000(1.04)^3=50000(1.124864)=\boxed{56{,}243.2}\) \(\approx \boxed{56{,}243}\)
  \item Growth factor \((1.10)^8=\boxed{\approx 2.1436}\)
  \item \(V(n)=V_0(1-0.18)^n=\boxed{V_0(0.82)^n}\)
  \item \(P(t)=80000(1-0.05)^t=\boxed{80000(0.95)^t}\)
\end{enumerate}

\subsection*{Part C Solutions: Reverse Problems}
\begin{enumerate}
  \setcounter{enumi}{10}
  \item \(145.20=(1.20)(1.10)\cdot \text{orig}=1.32\cdot \text{orig}\Rightarrow \text{orig}=\boxed{110}\)
  \item \(171=(0.90)(0.95)\cdot \text{orig}=0.855\cdot \text{orig}\Rightarrow \text{orig}=\boxed{200}\)
  \item \(138=(0.85)(1.20)\cdot \text{orig}=1.02\cdot \text{orig}\Rightarrow \text{orig}=\boxed{135.29}\) (approx)
  \item \(125{,}440=(1.12)^2\cdot \text{orig}=1.2544\cdot \text{orig}\Rightarrow \text{orig}=\boxed{100{,}000}\)
  \item \(153=(1.10)(0.85)\cdot \text{orig}=0.935\cdot \text{orig}\Rightarrow \text{orig}=\boxed{163.69}\) (approx). Note: order mentioned does not affect the product.
\end{enumerate}

\subsection*{Part D Solutions: Comparing Percent Changes}
\begin{enumerate}
  \setcounter{enumi}{15}
  \item \(1.10\cdot1.20=1.32\) vs \(1.15^2=1.3225\). Greater is \(15\%\) then \(15\%\) by \(0.25\%\).
  \item \(0.80\cdot1.20=0.96\). \(\boxed{4\%\ \text{decrease}}\)
  \item \(1.25\cdot0.75=0.9375\). \(\boxed{6.25\%\ \text{decrease}}\)
  \item \(0.70\cdot0.70=0.49\). Single discount \(=1-0.49=\boxed{51\%}\).
  \item Solve \(1.08^n\ge1.25\Rightarrow n\ge \dfrac{\ln 1.25}{\ln 1.08}\approx 2.90\). Smallest integer \(n=\boxed{3}\).
\end{enumerate}

\subsection*{Part E Solutions: SAT-Style Applications}
\begin{enumerate}
  \setcounter{enumi}{20}
  \item Now \(= \text{past}\cdot(0.80)^2\Rightarrow \text{past}=\dfrac{16000}{0.64}=\boxed{25{,}000}\)
  \item \(1.30\cdot0.90=1.17\). \(\boxed{17\%\ \text{increase}}\)
  \item \(0.75\cdot1.40=1.05\). \(\boxed{5\%\ \text{increase}}\)
  \item \(1.05^3=1.157625\). \(\boxed{15.7625\%\ \text{increase}}\approx \boxed{15.8\%}\)
  \item Now \(= V_0(0.88)^3\Rightarrow V_0=\dfrac{6912}{0.681472}\approx \boxed{10{,}143}\)
\end{enumerate}


\end{document}
