\documentclass[12pt]{article}

\usepackage{amsmath, amssymb}
\usepackage{geometry}
\usepackage{setspace}
\usepackage{titlesec}
\usepackage{lmodern}
\usepackage{xcolor}
\usepackage{enumitem}

\geometry{margin=1in}
\setstretch{1.2}
\titleformat{\section}{\normalfont\Large\bfseries}{\thesection}{1em}{}
\titleformat{\subsection}{\normalfont\large\bfseries}{\thesubsection}{1em}{}
\pagenumbering{gobble}

\begin{document}

\begin{center}
    \LARGE \textbf{Unit 3: Ratios, Rates, and Percents} \\[6pt]
    \Large \textbf{Topic 1: Ratios and Equivalent Ratios}
\end{center}

\vspace{1em}

\section*{Concept Summary}

A \textbf{ratio} compares two quantities using division.  
It can be written in three forms:
\[
a:b, \quad \frac{a}{b}, \quad \text{or } a \text{ to } b
\]
where \(b \ne 0\).

Ratios can compare:
\begin{itemize}
    \item \textbf{Part to part:} boys to girls, red to blue.
    \item \textbf{Part to whole:} boys to total students.
    \item \textbf{Whole to part:} total students to boys.
\end{itemize}

Two ratios are \textbf{equivalent} if they have the same value when simplified:
\[
\frac{a}{b} = \frac{c}{d}
\quad \text{means} \quad ad = bc
\]
This property is the basis for solving proportion problems.

\section*{Core Skills}
\begin{itemize}
    \item Simplify ratios as you would fractions.
    \item Find missing terms using cross multiplication.
    \item Convert word statements to fraction or “:” form before solving.
    \item Express ratios in simplest form and interpret them with correct units.
\end{itemize}

\section*{Example 1: Simplifying Ratios}

Simplify the ratio \(18:24\).

\textbf{Step 1:} Divide both terms by their greatest common factor, 6.  
\[
\frac{18}{24} = \frac{3}{4}
\]
\textbf{Final Answer:} \(\boxed{3:4}\)

\section*{Example 2: Finding an Equivalent Ratio}

Find a ratio equivalent to \(4:5\) where the first term is 12.  
\[
\frac{4}{5} = \frac{12}{x}
\]
Cross multiply: \(4x = 60 \Rightarrow x = 15.\)  
\textbf{Final Answer:} \(\boxed{12:15}\)

\section*{Example 3: Solving a Proportion}

Solve for \(x\):
\[
\frac{3}{x} = \frac{9}{12}
\]
Cross multiply: \(3 \times 12 = 9x \Rightarrow 36 = 9x \Rightarrow x = 4.\)  
\textbf{Final Answer:} \(\boxed{x = 4}\)

\section*{Example 4: Interpreting Ratios in Context}

A recipe uses 2 cups of sugar for every 5 cups of flour. If 20 cups of flour are used, how much sugar is needed?

\[
\frac{2}{5} = \frac{x}{20}
\]
Cross multiply: \(5x = 40 \Rightarrow x = 8.\)  
\textbf{Final Answer:} \(\boxed{8\text{ cups of sugar}}\)

\section*{Key Takeaways}
\begin{itemize}
    \item Simplify ratios by dividing both quantities by the same factor.
    \item Equivalent ratios represent the same proportional relationship.
    \item Use cross multiplication to find missing terms in a proportion.
    \item Always include units and interpret ratios based on context.
\end{itemize}

\newpage

% ============================================================
% QUESTIONS — TOPIC 1: RATIOS AND EQUIVALENT RATIOS
% ============================================================

\section*{Practice Questions: Ratios and Equivalent Ratios}

\subsection*{Part A: Simplifying Ratios}
Simplify each ratio to its simplest form.
\begin{enumerate}
  \item \(8:12\)
  \item \(45:60\)
  \item \(18:27\)
  \item \(50:125\)
  \item \(120:200\)
\end{enumerate}

\subsection*{Part B: Finding Missing Terms in Equivalent Ratios}
Find the missing value \(x\) that makes each pair of ratios equivalent.
\begin{enumerate}
  \setcounter{enumi}{5}
  \item \(\frac{3}{5} = \frac{x}{20}\)
  \item \(\frac{7}{x} = \frac{21}{15}\)
  \item \(\frac{x}{9} = \frac{4}{6}\)
  \item \(\frac{8}{x} = \frac{10}{25}\)
  \item \(\frac{5}{12} = \frac{20}{x}\)
\end{enumerate}

\subsection*{Part C: Word Problems on Ratios}
\begin{enumerate}
  \setcounter{enumi}{10}
  \item A mixture contains red and blue paint in the ratio \(2:3\). If there are 15 liters of blue paint, how many liters of red paint are there?
  \item The ratio of students to teachers in a school is \(25:1\). If there are 800 students, how many teachers are there?
  \item A bag of marbles has red, green, and blue marbles in the ratio \(4:5:6\). If there are 30 blue marbles, how many red marbles are there?
  \item The ratio of boys to girls in a class is \(3:5\). If there are 32 girls, how many boys are there?
  \item A map uses a scale of \(1 \text{ cm} : 50 \text{ km}\). How many kilometers does 7 cm represent?
\end{enumerate}

\subsection*{Part D: Reasoning with Ratios}
\begin{enumerate}
  \setcounter{enumi}{15}
  \item Two numbers are in the ratio \(4:7\), and their sum is 66. Find the numbers.
  \item The ratio of the ages of Alice and Ben is \(3:5\). If Ben is 40 years old, how old is Alice?
  \item The sides of a rectangle are in the ratio \(2:3\). If the perimeter is 40 cm, find the length and width.
  \item The ratio of flour to sugar in a cake recipe is \(5:2\). If you have 8 cups of sugar, how much flour do you need to keep the same ratio?
  \item A class has 12 boys and 18 girls. Write three equivalent ratios for boys to girls.
\end{enumerate}

\subsection*{Part E: SAT-Style Applications}
\begin{enumerate}
  \setcounter{enumi}{20}
  \item The ratio of the number of apples to oranges in a basket is \(3:4\). If 10 oranges are added, the new ratio becomes \(3:5\). How many apples are there?
  \item In a certain alloy, the ratio of copper to zinc is \(5:3\). If 40 grams of copper are added and the ratio becomes \(3:2\), how much zinc was originally present?
  \item The ratio of girls to total students in a club is \(2:5\). If there are 18 girls, how many total students are in the club?
  \item A car travels 120 miles using 5 gallons of gas. At the same rate, how far can it travel using 8 gallons?
  \item Two quantities \(x\) and \(y\) are in the ratio \(7:9\). If \(x\) increases by 14 and \(y\) increases by 18, does the ratio remain the same? Explain.
\end{enumerate}

\newpage
% ============================================================
% SOLUTIONS — UNIT 3, TOPIC 1: RATIOS AND EQUIVALENT RATIOS
% ============================================================

\section*{Answer Key and Solutions: Ratios and Equivalent Ratios}

\subsection*{Part A Solutions: Simplifying Ratios}
\begin{enumerate}
  \item \(8:12 = \dfrac{8}{12}=\dfrac{2}{3} \Rightarrow \boxed{2:3}\)
  \item \(45:60 = \dfrac{45}{60}=\dfrac{3}{4} \Rightarrow \boxed{3:4}\)
  \item \(18:27 = \dfrac{18}{27}=\dfrac{2}{3} \Rightarrow \boxed{2:3}\)
  \item \(50:125 = \dfrac{50}{125}=\dfrac{2}{5} \Rightarrow \boxed{2:5}\)
  \item \(120:200 = \dfrac{120}{200}=\dfrac{3}{5} \Rightarrow \boxed{3:5}\)
\end{enumerate}

\subsection*{Part B Solutions: Missing Terms in Equivalent Ratios}
\begin{enumerate}
  \setcounter{enumi}{5}
  \item \(\dfrac{3}{5}=\dfrac{x}{20} \Rightarrow 5x=60 \Rightarrow \boxed{x=12}\)
  \item \(\dfrac{7}{x}=\dfrac{21}{15} \Rightarrow 7\cdot15=21x \Rightarrow 105=21x \Rightarrow \boxed{x=5}\)
  \item \(\dfrac{x}{9}=\dfrac{4}{6}=\dfrac{2}{3} \Rightarrow x=9\cdot\dfrac{2}{3}=\boxed{6}\)
  \item \(\dfrac{8}{x}=\dfrac{10}{25}=\dfrac{2}{5} \Rightarrow 2x=40 \Rightarrow \boxed{x=20}\)
  \item \(\dfrac{5}{12}=\dfrac{20}{x} \Rightarrow 5x=240 \Rightarrow \boxed{x=48}\)
\end{enumerate}

\subsection*{Part C Solutions: Word Problems on Ratios}
\begin{enumerate}
  \setcounter{enumi}{10}
  \item \(2:3=\dfrac{r}{15} \Rightarrow 3r=30 \Rightarrow \boxed{r=10\text{ L}}\)
  \item \(25:1=\dfrac{800}{t} \Rightarrow 25t=800 \Rightarrow \boxed{t=32}\)
  \item \(4:5:6\) with blue \(=30=6k \Rightarrow k=5 \Rightarrow \boxed{\text{red}=4k=20}\)
  \item \(3:5=\dfrac{b}{32} \Rightarrow 5b=96 \Rightarrow b=19.2\). Note: non-integer headcount; item yields a fractional result.
  \item \(1\text{ cm}:50\text{ km} \Rightarrow 7\text{ cm}: \boxed{350\text{ km}}\)
\end{enumerate}

\subsection*{Part D Solutions: Reasoning with Ratios}
\begin{enumerate}
  \setcounter{enumi}{15}
  \item \(4k+7k=66 \Rightarrow k=6 \Rightarrow \boxed{24\text{ and }42}\)
  \item \(3:5=\dfrac{A}{40} \Rightarrow 5A=120 \Rightarrow \boxed{A=24}\)
  \item Sides \(2k,3k\). Perimeter \(=2(2k+3k)=10k=40 \Rightarrow k=4 \Rightarrow \boxed{8\text{ cm and }12\text{ cm}}\)
  \item \(\dfrac{F}{S}=\dfrac{5}{2},\ S=8 \Rightarrow F=\dfrac{5}{2}\cdot8=\boxed{20\text{ cups}}\)
  \item Boys:girls \(=12:18=2:3\). Examples of equivalents: \(\boxed{2:3,\ 4:6,\ 6:9}\)
\end{enumerate}

\subsection*{Part E Solutions: SAT-Style Applications}
\begin{enumerate}
  \setcounter{enumi}{20}
  \item Let initial \(a:o=3k:4k\). New ratio \(a:(o+10)=3:5\).  
  \( \dfrac{3k}{4k+10}=\dfrac{3}{5} \Rightarrow 15k=12k+30 \Rightarrow k=10 \Rightarrow \boxed{a=30}\)
  \item Initial \(c:z=5:3=5k:3k\). After adding 40 g copper: \(\dfrac{5k+40}{3k}=\dfrac{3}{2}\) gives \(10k+80=9k\) so \(k=-80\).  
  Note: negative parameter indicates inconsistency with the stated change. No positive solution under these conditions.
  \item \(\dfrac{\text{girls}}{\text{total}}=\dfrac{2}{5}\). With 18 girls, total \(=\dfrac{5}{2}\cdot18=\boxed{45}\)
  \item Unit rate \(=120/5=24\) mpg. For 8 gallons: \(24\cdot8=\boxed{192\text{ miles}}\)
  \item \(x:y=7:9=7k:9k\). New ratio \(\dfrac{7k+14}{9k+18}=\dfrac{7(k+2)}{9(k+2)}=\dfrac{7}{9}\).  \newline
  \(\boxed{\text{Yes, ratio remains }7:9\ \text{since both parts increase proportionally}}\)
\end{enumerate}



\end{document}
