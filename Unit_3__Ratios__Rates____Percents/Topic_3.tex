\documentclass[12pt]{article}

\usepackage{amsmath, amssymb}
\usepackage{geometry}
\usepackage{setspace}
\usepackage{titlesec}
\usepackage{lmodern}
\usepackage{xcolor}
\usepackage{enumitem}

\geometry{margin=1in}
\setstretch{1.2}
\titleformat{\section}{\normalfont\Large\bfseries}{\thesection}{1em}{}
\titleformat{\subsection}{\normalfont\large\bfseries}{\thesubsection}{1em}{}
\pagenumbering{gobble}

\begin{document}

\begin{center}
    \LARGE \textbf{Unit 3: Ratios, Rates, and Percents} \\[6pt]
    \Large \textbf{Topic 3: Percent Increase, Percent Decrease, and Percent Error}
\end{center}

\vspace{1em}

\section*{Concept Summary}

Percent compares a change to an original amount.  
\[
\text{percent change}=\frac{\text{new} - \text{original}}{\text{original}}\times 100\%
\]
If the result is positive it is a percent increase. If negative it is a percent decrease.

Forward change models:
\[
\text{increase by }p\%:\ \ \text{new}=(1+p)\cdot \text{original}
\]
\[
\text{decrease by }p\%:\ \ \text{new}=(1-p)\cdot \text{original}
\]
Here \(p\) is written as a decimal. For example 12\% means \(p=0.12\).

Reverse change models (find the original):
\[
\text{original}=\frac{\text{new}}{1+p}\quad \text{after an increase}
\qquad
\text{original}=\frac{\text{new}}{1-p}\quad \text{after a decrease}
\]

Percent error measures accuracy of an approximation.
\[
\text{percent error}=\frac{|\text{approx} - \text{actual}|}{\text{actual}}\times 100\%
\]

\section*{Core Skills}
\begin{itemize}
  \item Translate words to the correct formula and identify the original amount.
  \item Convert percent to decimal before multiplying.
  \item Use growth factor \(1\pm p\) for one step changes.
  \item For reverse problems, divide by the factor \(1\pm p\).
  \item Distinguish percent of a number from percentage points.
\end{itemize}

\section*{Example 1: Percent Increase}

A jacket price rises from \$60 to \$75.  
\[
\frac{75-60}{60}\times 100\%=\frac{15}{60}\times 100\%=25\%
\]
\textbf{Answer:} 25 percent increase.

\section*{Example 2: Percent Decrease}

A population drops from 1{,}250 to 1{,}100.  
\[
\frac{1100-1250}{1250}\times 100\%=\frac{-150}{1250}\times 100\%=-12\%
\]
\textbf{Answer:} 12 percent decrease.

\section*{Example 3: Forward Calculation With Factor}

An item is discounted 30\% off the original price \$80.  
Factor \(=1-0.30=0.70\). New price \(=0.70\cdot 80=\$56\).

\section*{Example 4: Reverse Calculation After Decrease}

After a 20\% discount, the sale price is \$72. Find the original price.  
Sale price \(=\) \((1-0.20)\cdot \text{original}=0.80\cdot \text{original}\).  
\(\text{original}=\dfrac{72}{0.80}=\$90\).

\section*{Example 5: Tax or Markup}

A store marks up cost by 15\%. If the cost is \$120, the price is  
\((1+0.15)\cdot 120=1.15\cdot 120=\$138\).

\section*{Example 6: Percent Error}

A measurement is reported as 9.7 cm, but the actual length is 10.0 cm.
\[
\frac{|9.7-10.0|}{10.0}\times 100\%=\frac{0.3}{10}\times 100\%=3\%
\]
\textbf{Answer:} 3 percent error.

\section*{Key Takeaways}
\begin{itemize}
  \item Change divided by original times 100\% gives percent change.
  \item Use \(1+p\) for increases and \(1-p\) for decreases.
  \item To undo a percent change, divide by the same factor.
  \item Percent error uses actual in the denominator and absolute difference in the numerator.
\end{itemize}

\newpage

% ============================================================
% QUESTIONS — TOPIC 3: PERCENT INCREASE, DECREASE, AND ERROR
% ============================================================

\section*{Practice Questions: Percent Increase, Decrease, and Error}

\subsection*{Part A: Basic Percent Change}
Find the percent increase or decrease. Round to the nearest tenth of a percent if needed.
\begin{enumerate}
  \item From 40 to 50
  \item From 120 to 96
  \item From 75 to 90
  \item From 480 to 420
  \item From 20 to 25
\end{enumerate}

\subsection*{Part B: Using Growth or Decay Factors}
Compute the new amount using the given percent change.
\begin{enumerate}
  \setcounter{enumi}{5}
  \item Increase 15\% on \$240
  \item Decrease 30\% on \$80
  \item Increase 8\% on 500
  \item Decrease 12\% on 250
  \item Increase 6.5\% on 1,200
\end{enumerate}

\subsection*{Part C: Reverse Percent Problems}
Find the original amount before the percent change.
\begin{enumerate}
  \setcounter{enumi}{10}
  \item Sale price \$72 after 20\% discount
  \item Final price \$460 after 15\% tax
  \item Population 8,400 after a 12\% increase
  \item Sale price \$45 after 25\% discount
  \item Investment grew to \$3,150 after a 5\% increase
\end{enumerate}

\subsection*{Part D: Percent Error}
Find the percent error, rounding to one decimal place.
\begin{enumerate}
  \setcounter{enumi}{15}
  \item Measured 9 cm, actual 10 cm
  \item Measured 2.4 kg, actual 2.5 kg
  \item Estimated 480 miles, actual 500 miles
  \item Predicted 35 students, actual 32 students
  \item Measured 19.6 m, actual 20.0 m
\end{enumerate}

\subsection*{Part E: SAT-Style Applications}
\begin{enumerate}
  \setcounter{enumi}{20}
  \item A laptop originally priced at \$1,200 is discounted by 25\% and then increased by 10\%. Find the final price.
  \item A coat’s price increases from \$90 to \$108. By what percent did it increase?
  \item After a 40\% decrease, the value of a stock is \$180. What was its original value?
  \item A scientist’s measurement is 18.7 cm, while the accepted value is 19.0 cm. Find the percent error.
  \item The population of a town grew by 12\% one year and then decreased by 5\% the next. Find the overall percent change relative to the original population.
\end{enumerate}

\newpage
% ============================================================
% SOLUTIONS — UNIT 3, TOPIC 3: PERCENT INCREASE, DECREASE, AND ERROR
% ============================================================

\section*{Answer Key and Solutions: Percent Increase, Decrease, and Error}

\subsection*{Part A Solutions: Basic Percent Change}
\begin{enumerate}
  \item \(\dfrac{50-40}{40}\cdot100\% = \boxed{25\%\ \text{increase}}\)
  \item \(\dfrac{96-120}{120}\cdot100\% = \boxed{20\%\ \text{decrease}}\)
  \item \(\dfrac{90-75}{75}\cdot100\% = \boxed{20\%\ \text{increase}}\)
  \item \(\dfrac{420-480}{480}\cdot100\% = \boxed{12.5\%\ \text{decrease}}\)
  \item \(\dfrac{25-20}{20}\cdot100\% = \boxed{25\%\ \text{increase}}\)
\end{enumerate}

\subsection*{Part B Solutions: Using Growth or Decay Factors}
\begin{enumerate}
  \setcounter{enumi}{5}
  \item \(240(1+0.15)=\boxed{276}\)
  \item \(80(1-0.30)=\boxed{56}\)
  \item \(500(1+0.08)=\boxed{540}\)
  \item \(250(1-0.12)=\boxed{220}\)
  \item \(1200(1+0.065)=\boxed{1278}\)
\end{enumerate}

\subsection*{Part C Solutions: Reverse Percent Problems}
\begin{enumerate}
  \setcounter{enumi}{10}
  \item Sale price \(=0.80\cdot \text{original}\Rightarrow \text{original}=\dfrac{72}{0.80}=\boxed{90}\)
  \item Final \(=1.15\cdot \text{original}\Rightarrow \text{original}=\dfrac{460}{1.15}=\boxed{400}\)
  \item Final \(=1.12\cdot \text{original}\Rightarrow \text{original}=\dfrac{8400}{1.12}=\boxed{7500}\)
  \item Sale price \(=0.75\cdot \text{original}\Rightarrow \text{original}=\dfrac{45}{0.75}=\boxed{60}\)
  \item Final \(=1.05\cdot \text{original}\Rightarrow \text{original}=\dfrac{3150}{1.05}=\boxed{3000}\)
\end{enumerate}

\subsection*{Part D Solutions: Percent Error}
\begin{enumerate}
  \setcounter{enumi}{15}
  \item \(\dfrac{|9-10|}{10}\cdot100\%=\boxed{10.0\%}\)
  \item \(\dfrac{|2.4-2.5|}{2.5}\cdot100\%=\boxed{4.0\%}\)
  \item \(\dfrac{|480-500|}{500}\cdot100\%=\boxed{4.0\%}\)
  \item \(\dfrac{|35-32|}{32}\cdot100\%=9.375\%\approx\boxed{9.4\%}\)
  \item \(\dfrac{|19.6-20.0|}{20.0}\cdot100\%=\boxed{2.0\%}\)
\end{enumerate}

\subsection*{Part E Solutions: SAT-Style Applications}
\begin{enumerate}
  \setcounter{enumi}{20}
  \item \(1200\cdot0.75=900\). Then \(900\cdot1.10=\boxed{990}\)
  \item \(\dfrac{108-90}{90}\cdot100\%=\boxed{20\%}\)
  \item After 40\% decrease: final \(=0.60\cdot \text{original}\Rightarrow \text{original}=\dfrac{180}{0.60}=\boxed{300}\)
  \item \(\dfrac{|18.7-19.0|}{19.0}\cdot100\%=1.5789\%\approx\boxed{1.6\%}\)
  \item Net factor \(=1.12\cdot0.95=1.064\Rightarrow \boxed{6.4\%\ \text{increase}}\)
\end{enumerate}


\end{document}
