\documentclass[12pt]{article}

\usepackage{amsmath, amssymb}
\usepackage{geometry}
\usepackage{setspace}
\usepackage{titlesec}
\usepackage{lmodern}
\usepackage{xcolor}
\usepackage{enumitem}

\geometry{margin=1in}
\setstretch{1.2}
\titleformat{\section}{\normalfont\Large\bfseries}{\thesection}{1em}{}
\titleformat{\subsection}{\normalfont\large\bfseries}{\thesubsection}{1em}{}
\pagenumbering{gobble}

\begin{document}

\begin{center}
    \LARGE \textbf{Unit 3: Ratios, Rates, and Percents} \\[6pt]
    \Large \textbf{Topic 6: Direct and Inverse Variation}
\end{center}

\vspace{1em}

\section*{Concept Summary}

A \textbf{direct variation} is a proportional relationship of the form
\[
y = kx,
\]
where \(k\) is the constant of variation. As \(x\) increases, \(y\) increases proportionally.  
Graph: a straight line through the origin with slope \(k\).

An \textbf{inverse variation} is a relationship where one quantity increases as the other decreases, given by
\[
y = \frac{k}{x},
\]
where \(k\) is the constant of variation. The product \(xy = k\) is constant.  
Graph: a hyperbola in the first and third quadrants when \(k > 0\).

\section*{Core Skills}
\begin{itemize}
  \item Identify whether a situation or equation shows direct or inverse variation.
  \item Find the constant \(k\) from given data and use it to make predictions.
  \item Recognize \(y/x = \text{constant}\) as direct variation and \(xy = \text{constant}\) as inverse variation.
  \item Graph and interpret the meaning of \(k\) in context.
\end{itemize}

\section*{Example 1: Direct Variation}

If \(y\) varies directly with \(x\) and \(y=12\) when \(x=4\), find the equation.

\[
y = kx \Rightarrow 12 = 4k \Rightarrow k = 3 \Rightarrow \boxed{y = 3x}.
\]

\section*{Example 2: Inverse Variation}

If \(y\) varies inversely with \(x\) and \(y=8\) when \(x=2\), find the equation.

\[
y = \frac{k}{x} \Rightarrow 8 = \frac{k}{2} \Rightarrow k = 16 \Rightarrow \boxed{y = \frac{16}{x}}.
\]

\section*{Example 3: Identifying Variation Type}

Determine the type of variation:
\[
y = 5x + 2.
\]
This is not a variation because it does not pass through the origin.

\[
y = 10x \quad \text{(direct)} \qquad y = \frac{20}{x} \quad \text{(inverse)}.
\]

\section*{Example 4: Using Direct Variation to Predict}

If \(y = 7x\) and \(x=12\), find \(y\).
\[
y = 7(12) = \boxed{84}.
\]

\section*{Example 5: Using Inverse Variation to Predict}

If \(y = \frac{24}{x}\) and \(x=6\), find \(y\).
\[
y = \frac{24}{6} = \boxed{4}.
\]

\section*{Example 6: Real-World Context}

The time \(t\) to complete a job varies inversely with the number of workers \(n\).  
If 4 workers take 18 hours, find how long 6 workers will take.
\[
t = \frac{k}{n} \Rightarrow 18 = \frac{k}{4} \Rightarrow k = 72.
\]
\[
t = \frac{72}{6} = \boxed{12\text{ hours}}.
\]

\section*{Key Takeaways}
\begin{itemize}
  \item Direct variation: \(y = kx\). The ratio \(y/x\) is constant.
  \item Inverse variation: \(y = k/x\). The product \(xy\) is constant.
  \item Graphs of direct variation are lines through the origin; inverse variation graphs are hyperbolas.
  \item Use variation equations to find missing quantities and interpret proportional relationships.
\end{itemize}

\newpage

% ============================================================
% QUESTIONS — TOPIC 6: DIRECT AND INVERSE VARIATION
% ============================================================

\section*{Practice Questions: Direct and Inverse Variation}

\subsection*{Part A: Identifying Variation Type}
Decide whether each equation represents direct variation, inverse variation, or neither.
\begin{enumerate}
  \item \(y = 4x\)
  \item \(y = \dfrac{9}{x}\)
  \item \(y = 5x + 3\)
  \item \(xy = 10\)
  \item \(y = -7x\)
\end{enumerate}

\subsection*{Part B: Finding the Constant of Variation}
Find the constant \(k\) in each relationship.
\begin{enumerate}
  \setcounter{enumi}{5}
  \item \(y\) varies directly with \(x\); \(y = 20\) when \(x = 4\)
  \item \(y\) varies inversely with \(x\); \(y = 8\) when \(x = 5\)
  \item \(y\) varies directly with \(x\); \(y = 36\) when \(x = 12\)
  \item \(y\) varies inversely with \(x\); \(y = 15\) when \(x = 3\)
  \item \(y\) varies inversely with \(x\); \(x = 9\) when \(y = 2\)
\end{enumerate}

\subsection*{Part C: Writing Variation Equations}
\begin{enumerate}
  \setcounter{enumi}{10}
  \item \(y\) varies directly with \(x\); \(y = 18\) when \(x = 6\)
  \item \(y\) varies inversely with \(x\); \(y = 12\) when \(x = 8\)
  \item \(y\) varies directly with \(x\); \(y = 27\) when \(x = 9\)
  \item \(y\) varies inversely with \(x\); \(y = 4\) when \(x = 10\)
  \item \(y\) varies directly with \(x\); \(y = 5\) when \(x = 0.5\)
\end{enumerate}

\subsection*{Part D: Application Problems}
\begin{enumerate}
  \setcounter{enumi}{15}
  \item \(y\) varies directly with \(x\). If \(y = 18\) when \(x = 6\), find \(y\) when \(x = 10\).
  \item \(y\) varies inversely with \(x\). If \(y = 12\) when \(x = 8\), find \(y\) when \(x = 4\).
  \item The distance \(d\) traveled varies directly with time \(t\). If \(d = 72\) when \(t = 3\), find \(d\) when \(t = 5\).
  \item The pressure \(P\) of a gas varies inversely with volume \(V\). If \(P = 200\) when \(V = 8\), find \(P\) when \(V = 5\).
  \item The time \(t\) to paint a wall varies inversely with the number of painters \(n\). If 3 painters take 12 hours, how long would 4 painters take?
\end{enumerate}

\subsection*{Part E: SAT-Style Applications}
\begin{enumerate}
  \setcounter{enumi}{20}
  \item The intensity of light \(I\) varies inversely with the square of the distance \(d\). If \(I = 12\) when \(d = 2\), find \(I\) when \(d = 6\).
  \item The circumference \(C\) of a circle varies directly with its diameter \(d\). If \(C = 31.4\) when \(d = 10\), find \(C\) when \(d = 18\).
  \item The time \(T\) required to finish a job varies inversely with the number of workers \(n\). If 5 workers finish in 12 hours, how long would 8 workers take?
  \item The volume \(V\) of a gas varies directly with temperature \(T\) and inversely with pressure \(P\). If \(V = 400\) when \(T = 200\) and \(P = 50\), find \(V\) when \(T = 250\) and \(P = 40\).
  \item The gravitational force \(F\) varies directly with mass \(m\) and inversely with the square of the distance \(r\). If \(F = 80\) when \(m = 10\) and \(r = 2\), find \(F\) when \(m = 20\) and \(r = 4\).
\end{enumerate}

\newpage

% ============================================================
% SOLUTIONS — UNIT 3, TOPIC 6: DIRECT AND INVERSE VARIATION
% ============================================================

\section*{Answer Key and Solutions: Direct and Inverse Variation}

\subsection*{Part A Solutions: Identifying Variation Type}
\begin{enumerate}
  \item \(y=4x\): direct variation.
  \item \(y=\dfrac{9}{x}\): inverse variation.
  \item \(y=5x+3\): neither.
  \item \(xy=10 \Rightarrow y=\dfrac{10}{x}\): inverse variation.
  \item \(y=-7x\): direct variation.
\end{enumerate}

\subsection*{Part B Solutions: Constant of Variation}
\begin{enumerate}
  \setcounter{enumi}{5}
  \item Direct: \(y=kx\). \(20=4k \Rightarrow \boxed{k=5}\).
  \item Inverse: \(y=\dfrac{k}{x}\). \(8=\dfrac{k}{5} \Rightarrow \boxed{k=40}\).
  \item Direct: \(36=12k \Rightarrow \boxed{k=3}\).
  \item Inverse: \(15=\dfrac{k}{3} \Rightarrow \boxed{k=45}\).
  \item Inverse: \(2=\dfrac{k}{9} \Rightarrow \boxed{k=18}\).
\end{enumerate}

\subsection*{Part C Solutions: Writing Variation Equations}
\begin{enumerate}
  \setcounter{enumi}{10}
  \item Direct: \(18=6k \Rightarrow k=3 \Rightarrow \boxed{y=3x}\).
  \item Inverse: \(12=\dfrac{k}{8} \Rightarrow k=96 \Rightarrow \boxed{y=\dfrac{96}{x}}\).
  \item Direct: \(27=9k \Rightarrow k=3 \Rightarrow \boxed{y=3x}\).
  \item Inverse: \(4=\dfrac{k}{10} \Rightarrow k=40 \Rightarrow \boxed{y=\dfrac{40}{x}}\).
  \item Direct: \(5=k(0.5) \Rightarrow k=10 \Rightarrow \boxed{y=10x}\).
\end{enumerate}

\subsection*{Part D Solutions: Applications}
\begin{enumerate}
  \setcounter{enumi}{15}
  \item Direct: \(k=\dfrac{18}{6}=3\). For \(x=10\), \(y=3\cdot10=\boxed{30}\).
  \item Inverse: \(k=yx=12\cdot8=96\). For \(x=4\), \(y=\dfrac{96}{4}=\boxed{24}\).
  \item Direct: \(k=\dfrac{72}{3}=24\). For \(t=5\), \(d=24\cdot5=\boxed{120}\).
  \item Inverse: \(k=PV=200\cdot8=1600\). For \(V=5\), \(P=\dfrac{1600}{5}=\boxed{320}\).
  \item Inverse: \(k=tn=12\cdot3=36\). For \(n=4\), \(t=\dfrac{36}{4}=\boxed{9\text{ h}}\).
\end{enumerate}

\subsection*{Part E Solutions: SAT-Style Applications}
\begin{enumerate}
  \setcounter{enumi}{20}
  \item Inverse square: \(I=\dfrac{k}{d^2}\). \(12=\dfrac{k}{2^2} \Rightarrow k=48\). For \(d=6\), \(I=\dfrac{48}{36}=\boxed{\tfrac{4}{3}}\) \(\approx 1.33\).
  \item Direct: \(C=kd\). \(31.4=10k \Rightarrow k=3.14\). For \(d=18\), \(C=3.14\cdot18=\boxed{56.52}\).
  \item Inverse: \(T=\dfrac{k}{n}\). \(12=\dfrac{k}{5} \Rightarrow k=60\). For \(n=8\), \(T=\dfrac{60}{8}=\boxed{7.5\text{ h}}\).
  \item Combined: \(V=k\cdot \dfrac{T}{P}\). \(400=k\cdot \dfrac{200}{50}=4k \Rightarrow k=100\). New \(V=100\cdot \dfrac{250}{40}=\boxed{625}\).
  \item \(F=k\cdot\dfrac{m}{r^2}\). \(80=k\cdot\dfrac{10}{2^2}=k\cdot2.5 \Rightarrow k=32\). New \(F=32\cdot\dfrac{20}{4^2}=32\cdot\dfrac{20}{16}=\boxed{40}\).
\end{enumerate}


\end{document}
