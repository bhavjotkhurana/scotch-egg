\documentclass[12pt]{article}

\usepackage{amsmath, amssymb}
\usepackage{geometry}
\usepackage{setspace}
\usepackage{titlesec}
\usepackage{lmodern}
\usepackage{xcolor}
\usepackage{enumitem}

\geometry{margin=1in}
\setstretch{1.2}
\titleformat{\section}{\normalfont\Large\bfseries}{\thesection}{1em}{}
\titleformat{\subsection}{\normalfont\large\bfseries}{\thesubsection}{1em}{}
\pagenumbering{gobble}

\begin{document}

\begin{center}
    \LARGE \textbf{Unit 3: Ratios, Rates, and Percents} \\[6pt]
    \Large \textbf{Topic 5: Scale Drawings, Mixtures, and Density Problems}
\end{center}

\vspace{1em}

\section*{Concept Summary}

This topic connects ratios to real-world proportional reasoning:
\begin{itemize}
  \item \textbf{Scale drawings:} represent real dimensions proportionally.  
  Scale factor \(=\dfrac{\text{drawing}}{\text{actual}}\).  
  Multiply or divide by the scale to convert between the two.
  \item \textbf{Mixtures:} combine substances using ratio or weighted average relationships.
  \item \textbf{Density:} expresses mass per unit volume:  
  \[
  \text{density} = \frac{\text{mass}}{\text{volume}}.
  \]
  Rearrange as \(m = dV\) or \(V = \frac{m}{d}\).
\end{itemize}

\section*{Core Skills}
\begin{itemize}
  \item Use proportion equations for scale drawings: \(\dfrac{\text{drawing}}{\text{actual}}=\dfrac{x_1}{x_2}\).
  \item For mixtures, set up equations based on total quantity and total content (e.g. salt, acid, cost).
  \item For density, solve for missing variables given any two.
  \item Keep track of units consistently across quantities.
\end{itemize}

\section*{Example 1: Scale Drawing Conversion}

On a map, 1 cm represents 5 km. If two towns are 7.2 cm apart on the map, how far apart are they in reality?

\[
1:5 = 7.2:x \Rightarrow x = 7.2 \times 5 = \boxed{36\text{ km}}.
\]

\section*{Example 2: Scale Factor Reduction}

A model car is built at a scale of 1:20. The real car is 4 m long.  
\[
\text{model length} = \frac{1}{20}\times 4 = \boxed{0.2\text{ m}}.
\]

\section*{Example 3: Mixture Problem}

A chemist mixes 3 L of 20\% acid with 2 L of 50\% acid.  
\[
\text{total acid} = 3(0.20)+2(0.50)=0.6+1.0=1.6\text{ L}.
\]
Total volume \(=5\text{ L}\).  
\(\text{percent}=\frac{1.6}{5}\times100\%=32\%\).  
\textbf{Final mixture:} 32\% acid.

\section*{Example 4: Density Calculation}

A cube has a volume of \(200\text{ cm}^3\) and a mass of \(1.6\text{ g/cm}^3\).  
Mass \(=dV=1.6(200)=\boxed{320\text{ g}}\).

\section*{Example 5: Solving for Volume}

A liquid with density \(0.8\text{ g/cm}^3\) has a mass of \(400\text{ g}\).  
\[
V=\frac{m}{d}=\frac{400}{0.8}=500\text{ cm}^3.
\]
\textbf{Answer:} \(\boxed{500\text{ cm}^3}\).

\section*{Key Takeaways}
\begin{itemize}
  \item Scale drawings use direct proportion between model and real measurements.
  \item Mixtures combine quantities through weighted averages.
  \item Density links mass and volume linearly; rearrange as needed.
  \item Label all units and check proportional relationships carefully.
\end{itemize}

\newpage

% ============================================================
% QUESTIONS — TOPIC 5: SCALE DRAWINGS, MIXTURES, AND DENSITY PROBLEMS
% ============================================================

\section*{Practice Questions: Scale Drawings, Mixtures, and Density Problems}

\subsection*{Part A: Scale Drawings}
\begin{enumerate}
  \item A map has a scale of 1 cm : 6 km. What is the real distance if two cities are 8 cm apart?
  \item A model airplane is built at a scale of 1:50. If the actual wingspan is 30 m, what is the model’s wingspan?
  \item A floor plan is drawn so that 1 inch represents 4 feet. A room measures 3.5 inches by 4.25 inches on the plan. Find its real dimensions.
  \item On a map, 5 cm represents 20 km. What is the map distance for a real distance of 50 km?
  \item A model ship is 60 cm long. The real ship is 120 m long. What is the scale of the model?
\end{enumerate}

\subsection*{Part B: Mixture Problems}
\begin{enumerate}
  \setcounter{enumi}{5}
  \item How many liters of 40\% solution must be mixed with 60 liters of 20\% solution to make a 30\% solution?
  \item A 10\% salt solution is mixed with a 25\% salt solution to make 100 g of a 15\% solution. How much of each solution is needed?
  \item A 5\% sugar solution is combined with a 15\% sugar solution to make 20 L of a 10\% mixture. How much of each solution is used?
  \item A grocer mixes 30 lb of nuts costing \$6/lb with another type costing \$9/lb to make a blend worth \$7.50/lb. How many pounds of the \$9 nuts should be used?
  \item A chemist has 12 L of 30\% acid. How much water must be added to make a 20\% acid solution?
\end{enumerate}

\subsection*{Part C: Density Problems}
\begin{enumerate}
  \setcounter{enumi}{10}
  \item A cube of metal has a mass of 540 g and a volume of 200 cm\(^3\). Find its density.
  \item A wooden block has density \(0.8\text{ g/cm}^3\) and volume \(250\text{ cm}^3\). Find its mass.
  \item A liquid with density \(1.2\text{ g/cm}^3\) has a mass of 960 g. Find its volume.
  \item A rock has density \(2.5\text{ g/cm}^3\) and mass \(750\text{ g}\). Find its volume.
  \item A gold bar with density \(19.3\text{ g/cm}^3\) has a volume of \(100\text{ cm}^3\). Find its mass.
\end{enumerate}

\subsection*{Part D: Combined Reasoning}
\begin{enumerate}
  \setcounter{enumi}{15}
  \item A cylindrical water tank is shown on a drawing at 1:100 scale. The drawing shows a height of 12 cm. What is the real height?
  \item Two alloys contain 40\% and 70\% copper, respectively. How many kilograms of each should be combined to produce 60 kg of 50\% copper alloy?
  \item A liquid of density \(0.9\text{ g/cm}^3\) is poured into a container of volume 800 cm\(^3\). Find the mass of the liquid.
  \item A 20\% sugar syrup is diluted with pure water to make 10 L of 15\% syrup. How much of the 20\% syrup is used?
  \item A cube of aluminum (density \(2.7\text{ g/cm}^3\)) has a mass of 540 g. Find its edge length.
\end{enumerate}

\subsection*{Part E: SAT-Style Applications}
\begin{enumerate}
  \setcounter{enumi}{20}
  \item A rectangular room on a blueprint is drawn 2 inches wide and 3 inches long. The scale is 1 inch : 5 feet. Find the actual area of the room in square feet.
  \item An alloy is made by mixing 10 kg of 40\% copper with 20 kg of 20\% copper. What is the percent copper in the final mixture?
  \item A block of material weighs 6.3 kg and has a volume of 3,000 cm\(^3\). What is its density in g/cm\(^3\)?
  \item A model train 30 cm long is built to a 1:40 scale. How long is the real train in meters?
  \item A bottle of liquid has density \(1.1\text{ g/cm}^3\) and contains 500 cm\(^3\). What is its mass in grams and kilograms?
\end{enumerate}

\newpage

% ============================================================
% SOLUTIONS — UNIT 3, TOPIC 5: SCALE DRAWINGS, MIXTURES, AND DENSITY PROBLEMS
% ============================================================

\section*{Answer Key and Solutions: Scale Drawings, Mixtures, and Density}

\subsection*{Part A Solutions: Scale Drawings}
\begin{enumerate}
  \item \(1\text{ cm}:6\text{ km}\). \(8\text{ cm}\Rightarrow 8\cdot6=\boxed{48\text{ km}}\)
  \item Scale \(1:50\). Model \(=\dfrac{30}{50}=\boxed{0.6\text{ m}}\)
  \item \(1\text{ in}:4\text{ ft}\). Real \(=3.5\cdot4= \boxed{14\text{ ft}}\) by \(4.25\cdot4=\boxed{17\text{ ft}}\)
  \item \(5\text{ cm}\leftrightarrow 20\text{ km}\Rightarrow \dfrac{5}{20}=\dfrac{x}{50}\Rightarrow x=12.5\text{ cm}\). \(\boxed{12.5\text{ cm}}\)
  \item Model:real \(=60\text{ cm}:120\text{ m}=60:12000=1:200\). \(\boxed{1:200}\)
\end{enumerate}

\subsection*{Part B Solutions: Mixtures}
\begin{enumerate}
  \setcounter{enumi}{5}
  \item Let \(x\) L of 40\%. \(0.40x+0.20\cdot60=0.30(x+60)\Rightarrow 0.10x=6\Rightarrow \boxed{x=60\text{ L}}\)
  \item Let \(x\) g of 10\%, \(y\) g of 25\%. \(x+y=100,\ 0.10x+0.25y=15\Rightarrow x=\tfrac{200}{3},\ y=\tfrac{100}{3}\). \(\boxed{66.\overline{6}\text{ g and }33.\overline{3}\text{ g}}\)
  \item \(x+y=20,\ 0.05x+0.15y=2\Rightarrow \boxed{x=10\text{ L},\ y=10\text{ L}}\)
  \item \(\dfrac{30\cdot6+9y}{30+y}=7.5\Rightarrow 180+9y=225+7.5y\Rightarrow y= \boxed{30\text{ lb}}\)
  \item Acid \(=12\cdot0.30=3.6\text{ L}\). Want \(0.20V=3.6\Rightarrow V=18\). Water added \(=18-12=\boxed{6\text{ L}}\)
\end{enumerate}

\subsection*{Part C Solutions: Density}
\begin{enumerate}
  \setcounter{enumi}{10}
  \item \(d=\dfrac{m}{V}=\dfrac{540}{200}=\boxed{2.7\ \text{g/cm}^3}\)
  \item \(m=dV=0.8\cdot250=\boxed{200\text{ g}}\)
  \item \(V=\dfrac{m}{d}=\dfrac{960}{1.2}=\boxed{800\text{ cm}^3}\)
  \item \(V=\dfrac{750}{2.5}=\boxed{300\text{ cm}^3}\)
  \item \(m=dV=19.3\cdot100=\boxed{1930\text{ g}}=\boxed{1.93\text{ kg}}\)
\end{enumerate}

\subsection*{Part D Solutions: Combined Reasoning}
\begin{enumerate}
  \setcounter{enumi}{15}
  \item Scale \(1:100\). Real height \(=12\cdot100=\boxed{1200\text{ cm}}=\boxed{12\text{ m}}\)
  \item \(x+y=60,\ 0.40x+0.70y=30\Rightarrow x=40,\ y=20\). \(\boxed{40\text{ kg of 40\%},\ 20\text{ kg of 70\%}}\)
  \item \(m=dV=0.9\cdot800=\boxed{720\text{ g}}\)
  \item Let \(x\) L of 20\%. \(0.20x=0.15\cdot10=1.5\Rightarrow x=\boxed{7.5\text{ L}}\) of 20\% and \(2.5\) L water
  \item \(V=\dfrac{540}{2.7}=200\text{ cm}^3\). Edge \(a^3=200\Rightarrow a=\sqrt[3]{200}\approx \boxed{5.85\text{ cm}}\)
\end{enumerate}

\subsection*{Part E Solutions: SAT-Style Applications}
\begin{enumerate}
  \setcounter{enumi}{20}
  \item Real \(=2\cdot5=10\text{ ft}\) by \(3\cdot5=15\text{ ft}\). Area \(=10\cdot15=\boxed{150\text{ ft}^2}\)
  \item Copper \(=10\cdot0.40+20\cdot0.20=8\) kg out of \(30\) kg. Percent \(=\dfrac{8}{30}\cdot100\%=\boxed{26.\overline{6}\%}\)
  \item Convert \(6.3\text{ kg}=6300\text{ g}\). \(d=\dfrac{6300}{3000}=\boxed{2.1\ \text{g/cm}^3}\)
  \item Real length \(=30\cdot40=1200\text{ cm}= \boxed{12\text{ m}}\)
  \item \(m=dV=1.1\cdot500=\boxed{550\text{ g}}=\boxed{0.55\text{ kg}}\)
\end{enumerate}



\end{document}
