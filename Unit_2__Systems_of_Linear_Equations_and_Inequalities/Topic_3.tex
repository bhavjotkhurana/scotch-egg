\documentclass[12pt]{article}

\usepackage{amsmath, amssymb}
\usepackage{geometry}
\usepackage{setspace}
\usepackage{titlesec}
\usepackage{lmodern}
\usepackage{xcolor}
\usepackage{enumitem}

\geometry{margin=1in}
\setstretch{1.2}
\titleformat{\section}{\normalfont\Large\bfseries}{\thesection}{1em}{}
\titleformat{\subsection}{\normalfont\large\bfseries}{\thesubsection}{1em}{}
\pagenumbering{gobble}

\begin{document}

\begin{center}
    \LARGE \textbf{Unit 2: Systems of Linear Equations and Inequalities} \\[6pt]
    \Large \textbf{Topic 3: Interpreting Intersections and Solution Types}
\end{center}

\vspace{1em}

\section*{Concept Summary}

A system of two linear equations in two variables can have exactly one solution, no solution, or infinitely many solutions.

\begin{itemize}
    \item \textbf{One solution}: lines intersect at exactly one point. Slopes are different.
    \item \textbf{No solution}: lines are parallel and distinct. Slopes are equal, intercepts differ.
    \item \textbf{Infinitely many solutions}: lines are the same line. Both slope and intercept match after simplification.
\end{itemize}

Algebraic tests:
\[
\begin{aligned}
\text{If } &a_1x + b_1y = c_1,\quad a_2x + b_2y = c_2,\\
\text{compare ratios } &\frac{a_1}{a_2},\ \frac{b_1}{b_2},\ \frac{c_1}{c_2}.\\
&\text{One solution if ratios of }a\text{ and }b\text{ are not equal.}\\
&\text{No solution if } \frac{a_1}{a_2}=\frac{b_1}{b_2}\neq \frac{c_1}{c_2}.\\
&\text{Infinitely many if } \frac{a_1}{a_2}=\frac{b_1}{b_2}=\frac{c_1}{c_2}.
\end{aligned}
\]

Graph meaning:
\begin{itemize}
    \item Intersection point \((x,y)\) satisfies both equations.
    \item In contexts, \(x\) and \(y\) carry units. The coordinates answer the question in those units.
\end{itemize}

\section*{Core Skills}
\begin{itemize}
    \item Convert to slope intercept form to compare slopes and intercepts.
    \item Use elimination or substitution to detect contradictions like \(0=5\) or identities like \(0=0\).
    \item Interpret the ordered pair in context with correct units.
\end{itemize}

\section*{Example 1: One Solution}

\[
\begin{cases}
y = 2x + 1\\
y = -x + 10
\end{cases}
\]
Set equal: \(2x + 1 = -x + 10 \Rightarrow 3x = 9 \Rightarrow x = 3\). Then \(y = 2(3) + 1 = 7\).  
The lines have different slopes, so they meet once at \(\boxed{(3,7)}\).

\section*{Example 2: No Solution}

\[
\begin{cases}
2x - 3y = 6\\
4x - 6y = 10
\end{cases}
\]
Multiply the first equation by 2 to compare: \(4x - 6y = 12\). The second is \(4x - 6y = 10\).  
Same left side, different constants. This gives \(12 = 10\), a contradiction.  
Slopes match and intercepts differ, so \textbf{no solution}. Lines are parallel.

\section*{Example 3: Infinitely Many Solutions}

\[
\begin{cases}
3x + 6y = 12\\
x + 2y = 4
\end{cases}
\]
Multiply the second by 3: \(3x + 6y = 12\). The equations are identical.  
Every point on the line is a solution, so \textbf{infinitely many solutions}.

\section*{Example 4: Interpreting the Intersection in Context}

Two plans for streaming:
\[
\begin{cases}
\text{Plan A cost } C = 8 + 2m\\
\text{Plan B cost } C = 2 + 3m
\end{cases}
\]
where \(m\) is the number of movies and \(C\) is dollars.  
Set equal: \(8 + 2m = 2 + 3m \Rightarrow m = 6\). Then \(C = 8 + 2(6) = 20\).  
At \(m=6\) movies, both plans cost \(20\) dollars. The intersection \((6, 20)\) means 6 movies and 20 dollars.

\section*{Key Takeaways}
\begin{itemize}
    \item Different slopes gives one intersection and one solution.
    \item Equal slopes with different intercepts gives no solution.
    \item Proportional equations describe the same line and give infinitely many solutions.
    \item In word problems, attach units to the coordinates and state what the point means.
\end{itemize}

\newpage

% ============================================================
% QUESTIONS — TOPIC 3: INTERPRETING INTERSECTIONS AND SOLUTION TYPES
% ============================================================

\section*{Practice Questions: Interpreting Intersections and Solution Types}

\subsection*{Part A: Identify Solution Type by Inspection}
\begin{enumerate}
    \item \(\begin{cases} y = 2x + 1 \\ y = -x + 4 \end{cases}\)
    \item \(\begin{cases} y = 3x + 5 \\ y = 3x - 2 \end{cases}\)
    \item \(\begin{cases} 2x + 4y = 8 \\ x + 2y = 4 \end{cases}\)
    \item \(\begin{cases} y = -\tfrac{1}{2}x + 3 \\ y = 2x + 1 \end{cases}\)
    \item \(\begin{cases} 3x - 2y = 6 \\ 6x - 4y = 12 \end{cases}\)
\end{enumerate}

\subsection*{Part B: Determine Solution Type Algebraically}
\begin{enumerate}
    \setcounter{enumi}{5}
    \item \(\begin{cases} 2x + y = 10 \\ 4x + 2y = 20 \end{cases}\)
    \item \(\begin{cases} 3x - 2y = 5 \\ 6x - 4y = 9 \end{cases}\)
    \item \(\begin{cases} x + y = 8 \\ 2x + 2y = 18 \end{cases}\)
    \item \(\begin{cases} 4x + y = 12 \\ 8x + 2y = 26 \end{cases}\)
    \item \(\begin{cases} 5x - 3y = 7 \\ 10x - 6y = 14 \end{cases}\)
\end{enumerate}

\subsection*{Part C: Solve and Interpret the Intersection Point}
\begin{enumerate}
    \setcounter{enumi}{10}
    \item \(\begin{cases} y = 2x + 5 \\ y = -x + 8 \end{cases}\)
    \item \(\begin{cases} 3x + y = 12 \\ 2x - y = 3 \end{cases}\)
    \item \(\begin{cases} 4x - y = 11 \\ x + y = 4 \end{cases}\)
    \item \(\begin{cases} y = -2x + 6 \\ y = x - 3 \end{cases}\)
    \item \(\begin{cases} 2x + 3y = 12 \\ 3x + 2y = 12 \end{cases}\)
\end{enumerate}

\subsection*{Part D: Word Problems and Interpretation}
\begin{enumerate}
    \setcounter{enumi}{15}
    \item A gym charges \$30 per month plus \$5 per class. Another gym charges \$10 per month plus \$8 per class. At what number of classes will the cost be the same, and what will the cost be?
    \item One phone plan charges \$50 plus \$0.10 per minute, while another charges \$20 plus \$0.25 per minute. Find the number of minutes where both cost the same and what the total cost is.
    \item Two taxi services charge as follows:  
    Service A: \$3 plus \$1.50 per mile  
    Service B: \$6 plus \$1.00 per mile  
    After how many miles will both cost the same?
    \item The total cost of 3 pens and 2 notebooks is \$11. The total cost of 6 pens and 4 notebooks is \$22. How many solutions does this system have, and why?
    \item Two linear equations represent the same budget line in an economics problem. What kind of system is this, and what does that mean about the feasible combinations?
\end{enumerate}



\end{document}
