\documentclass[12pt]{article}

\usepackage{amsmath, amssymb}
\usepackage{geometry}
\usepackage{setspace}
\usepackage{titlesec}
\usepackage{lmodern}
\usepackage{xcolor}
\usepackage{enumitem}

\geometry{margin=1in}
\setstretch{1.2}
\titleformat{\section}{\normalfont\Large\bfseries}{\thesection}{1em}{}
\titleformat{\subsection}{\normalfont\large\bfseries}{\thesubsection}{1em}{}
\pagenumbering{gobble}

\begin{document}

\begin{center}
    \LARGE \textbf{Unit 2: Systems of Linear Equations and Inequalities} \\[6pt]
    \Large \textbf{Topic 2: Solving Systems by Elimination}
\end{center}

\vspace{1em}

\section*{Concept Summary}

The \textbf{elimination method} is a way to solve a system of equations by adding or subtracting the equations so that one variable is removed (eliminated). This allows us to solve for the remaining variable.

For example:
\[
\begin{cases}
2x + y = 10 \\
x - y = 4
\end{cases}
\]

If we add these two equations, \(y\) is eliminated:
\[
(2x + y) + (x - y) = 10 + 4 \Rightarrow 3x = 14 \Rightarrow x = \tfrac{14}{3}.
\]

Then substitute \(x = \tfrac{14}{3}\) back into one equation to find \(y\).

\section*{Core Skills}
\begin{itemize}
    \item Line up equations vertically by variables and equal signs.
    \item Decide whether to add or subtract to eliminate a variable.
    \item Multiply one or both equations if coefficients need adjustment.
    \item Always simplify and substitute back to check the solution.
\end{itemize}

\section*{Example 1: Direct Elimination}

Solve the system:
\[
\begin{cases}
x + y = 9 \\
x - y = 3
\end{cases}
\]

\textbf{Step 1:} Add the two equations to eliminate \(y\):
\[
(x + y) + (x - y) = 9 + 3 \Rightarrow 2x = 12 \Rightarrow x = 6.
\]

\textbf{Step 2:} Substitute \(x = 6\) into \(x + y = 9\):
\[
6 + y = 9 \Rightarrow y = 3.
\]

\textbf{Final Answer:} \(\boxed{(6, 3)}\)

\section*{Example 2: Multiplying to Eliminate}

Solve the system:
\[
\begin{cases}
3x + 2y = 16 \\
x - 2y = 4
\end{cases}
\]

\textbf{Step 1:} Add the two equations to eliminate \(y\):
\[
(3x + 2y) + (x - 2y) = 16 + 4 \Rightarrow 4x = 20 \Rightarrow x = 5.
\]

\textbf{Step 2:} Substitute \(x = 5\) into \(x - 2y = 4\):
\[
5 - 2y = 4 \Rightarrow -2y = -1 \Rightarrow y = \tfrac{1}{2}.
\]

\textbf{Final Answer:} \(\boxed{(5, \tfrac{1}{2})}\)

\section*{Example 3: Multiplying Both Equations}

Solve the system:
\[
\begin{cases}
2x + 3y = 12 \\
3x + 4y = 17
\end{cases}
\]

\textbf{Step 1:} Multiply the first equation by 3 and the second by 2 to make the \(x\)-coefficients match:
\[
\begin{cases}
6x + 9y = 36 \\
6x + 8y = 34
\end{cases}
\]

\textbf{Step 2:} Subtract the second equation from the first:
\[
(6x + 9y) - (6x + 8y) = 36 - 34 \Rightarrow y = 2.
\]

\textbf{Step 3:} Substitute \(y = 2\) into \(2x + 3y = 12\):
\[
2x + 6 = 12 \Rightarrow 2x = 6 \Rightarrow x = 3.
\]

\textbf{Final Answer:} \(\boxed{(3, 2)}\)

\section*{Key Takeaways}
\begin{itemize}
    \item Use elimination when variables are already aligned or easy to match.
    \item Multiply equations to create equal and opposite coefficients if needed.
    \item Always check the solution in both equations.
\end{itemize}

\newpage

% ============================================================
% QUESTIONS — TOPIC 2: SOLVING SYSTEMS BY ELIMINATION
% ============================================================

\section*{Practice Questions: Solving Systems by Elimination}

\subsection*{Part A: Direct Elimination}
\begin{enumerate}
    \item \(\begin{cases} x + y = 10 \\ x - y = 2 \end{cases}\)
    \item \(\begin{cases} 3x + y = 11 \\ 3x - y = 7 \end{cases}\)
    \item \(\begin{cases} 2x + y = 8 \\ 2x - y = 4 \end{cases}\)
    \item \(\begin{cases} x + 2y = 9 \\ x - 2y = 3 \end{cases}\)
    \item \(\begin{cases} 4x + y = 15 \\ 4x - y = 5 \end{cases}\)
\end{enumerate}

\subsection*{Part B: Multiplying One Equation}
\begin{enumerate}
    \setcounter{enumi}{5}
    \item \(\begin{cases} 2x + y = 10 \\ 3x + 2y = 16 \end{cases}\)
    \item \(\begin{cases} x + 3y = 12 \\ 2x - 3y = 6 \end{cases}\)
    \item \(\begin{cases} 5x - 2y = 8 \\ 3x + 4y = 20 \end{cases}\)
    \item \(\begin{cases} 2x + 5y = 19 \\ 4x - 5y = 1 \end{cases}\)
    \item \(\begin{cases} 6x + y = 14 \\ 4x - y = 10 \end{cases}\)
\end{enumerate}

\subsection*{Part C: Multiplying Both Equations}
\begin{enumerate}
    \setcounter{enumi}{10}
    \item \(\begin{cases} 2x + 3y = 12 \\ 3x + 4y = 17 \end{cases}\)
    \item \(\begin{cases} 4x - 3y = 5 \\ 3x - 2y = 4 \end{cases}\)
    \item \(\begin{cases} 5x + 2y = 18 \\ 3x + 4y = 14 \end{cases}\)
    \item \(\begin{cases} 2x - y = 7 \\ 3x + 2y = 13 \end{cases}\)
    \item \(\begin{cases} 4x + 5y = 23 \\ 6x - 5y = 7 \end{cases}\)
\end{enumerate}

\subsection*{Part D: Word Problems}
\begin{enumerate}
    \setcounter{enumi}{15}
    \item The sum of two numbers is 20. Their difference is 6. Find the two numbers.
    \item A rectangle has a perimeter of 50 cm. The length is 5 cm more than the width. Find its dimensions.
    \item The sum of the ages of a father and his son is 60. The father is 24 years older than his son. Find their ages.
    \item A movie ticket and a snack together cost \$22. Two movie tickets and one snack cost \$35. Find the price of each item.
    \item The total value of 5-dollar and 10-dollar bills is \$120. There are 16 bills in all. How many of each type are there?
\end{enumerate}

% ============================================================
% SOLUTIONS — TOPIC 2: SOLVING SYSTEMS BY ELIMINATION
% ============================================================

\section*{Answer Key and Solutions: Solving Systems by Elimination}

\subsection*{Part A Solutions: Direct Elimination}
\begin{enumerate}
  \item \(\begin{cases} x + y = 10 \\ x - y = 2 \end{cases}\)  
  Add equations: \(2x = 12 \Rightarrow x = 6\). Then \(6 + y = 10 \Rightarrow y = 4\).  
  \(\boxed{(6, 4)}\)

  \item \(\begin{cases} 3x + y = 11 \\ 3x - y = 7 \end{cases}\)  
  Add: \(6x = 18 \Rightarrow x = 3\). Then \(3x + y = 11 \Rightarrow 9 + y = 11 \Rightarrow y = 2\).  
  \(\boxed{(3, 2)}\)

  \item \(\begin{cases} 2x + y = 8 \\ 2x - y = 4 \end{cases}\)  
  Add: \(4x = 12 \Rightarrow x = 3\). Then \(2x + y = 8 \Rightarrow 6 + y = 8 \Rightarrow y = 2\).  
  \(\boxed{(3, 2)}\)

  \item \(\begin{cases} x + 2y = 9 \\ x - 2y = 3 \end{cases}\)  
  Add: \(2x = 12 \Rightarrow x = 6\). Then \(6 + 2y = 9 \Rightarrow 2y = 3 \Rightarrow y = \tfrac{3}{2}\).  
  \(\boxed{(6, \tfrac{3}{2})}\)

  \item \(\begin{cases} 4x + y = 15 \\ 4x - y = 5 \end{cases}\)  
  Add: \(8x = 20 \Rightarrow x = \tfrac{5}{2}\). Then \(4x + y = 15 \Rightarrow 10 + y = 15 \Rightarrow y = 5\).  
  \(\boxed{(\tfrac{5}{2}, 5)}\)
\end{enumerate}

\subsection*{Part B Solutions: Multiplying One Equation}
\begin{enumerate}
  \setcounter{enumi}{5}
  \item \(\begin{cases} 2x + y = 10 \\ 3x + 2y = 16 \end{cases}\)  
  Multiply first by \(-2\): \(-4x - 2y = -20\). Add to second: \(-x = -4 \Rightarrow x = 4\). Then \(2x + y = 10 \Rightarrow 8 + y = 10 \Rightarrow y = 2\).  
  \(\boxed{(4, 2)}\)

  \item \(\begin{cases} x + 3y = 12 \\ 2x - 3y = 6 \end{cases}\)  
  Add: \(3x = 18 \Rightarrow x = 6\). Then \(6 + 3y = 12 \Rightarrow 3y = 6 \Rightarrow y = 2\).  
  \(\boxed{(6, 2)}\)

  \item \(\begin{cases} 5x - 2y = 8 \\ 3x + 4y = 20 \end{cases}\)  
  Multiply first by \(2\): \(10x - 4y = 16\). Add to second: \(13x = 36 \Rightarrow x = \tfrac{36}{13}\).  
  Then \(5x - 2y = 8 \Rightarrow 5(\tfrac{36}{13}) - 2y = 8 \Rightarrow \tfrac{180}{13} - 2y = 8\).  
  So \(-2y = 8 - \tfrac{180}{13} = -\tfrac{76}{13} \Rightarrow y = \tfrac{38}{13}\).  
  \(\boxed{(\tfrac{36}{13}, \tfrac{38}{13})}\)

  \item \(\begin{cases} 2x + 5y = 19 \\ 4x - 5y = 1 \end{cases}\)  
  Add: \(6x = 20 \Rightarrow x = \tfrac{10}{3}\). Then \(2x + 5y = 19 \Rightarrow \tfrac{20}{3} + 5y = 19 \Rightarrow 5y = \tfrac{37}{3} \Rightarrow y = \tfrac{37}{15}\).  
  \(\boxed{(\tfrac{10}{3}, \tfrac{37}{15})}\)

  \item \(\begin{cases} 6x + y = 14 \\ 4x - y = 10 \end{cases}\)  
  Add: \(10x = 24 \Rightarrow x = \tfrac{12}{5}\). Then \(6x + y = 14 \Rightarrow \tfrac{72}{5} + y = 14 \Rightarrow y = -\tfrac{2}{5}\).  
  \(\boxed{(\tfrac{12}{5}, -\tfrac{2}{5})}\)
\end{enumerate}

\subsection*{Part C Solutions: Multiplying Both Equations}
\begin{enumerate}
  \setcounter{enumi}{10}
  \item \(\begin{cases} 2x + 3y = 12 \\ 3x + 4y = 17 \end{cases}\)  
  Multiply first by \(3\): \(6x + 9y = 36\). Multiply second by \(2\): \(6x + 8y = 34\). Subtract: \(y = 2\).  
  Then \(2x + 3(2) = 12 \Rightarrow 2x = 6 \Rightarrow x = 3\).  
  \(\boxed{(3, 2)}\)

  \item \(\begin{cases} 4x - 3y = 5 \\ 3x - 2y = 4 \end{cases}\)  
  Multiply first by \(2\): \(8x - 6y = 10\). Multiply second by \(-3\): \(-9x + 6y = -12\). Add: \(-x = -2 \Rightarrow x = 2\).  
  Then \(3x - 2y = 4 \Rightarrow 6 - 2y = 4 \Rightarrow y = 1\).  
  \(\boxed{(2, 1)}\)

  \item \(\begin{cases} 5x + 2y = 18 \\ 3x + 4y = 14 \end{cases}\)  
  Multiply first by \(2\): \(10x + 4y = 36\). Subtract second: \(7x = 22 \Rightarrow x = \tfrac{22}{7}\).  
  Then \(5x + 2y = 18 \Rightarrow 5(\tfrac{22}{7}) + 2y = 18 \Rightarrow \tfrac{110}{7} + 2y = 18\).  
  So \(2y = \tfrac{16}{7} \Rightarrow y = \tfrac{8}{7}\).  
  \(\boxed{(\tfrac{22}{7}, \tfrac{8}{7})}\)

  \item \(\begin{cases} 2x - y = 7 \\ 3x + 2y = 13 \end{cases}\)  
  Multiply first by \(2\): \(4x - 2y = 14\). Add to second: \(7x = 27 \Rightarrow x = \tfrac{27}{7}\).  
  Then \(2x - y = 7 \Rightarrow 2(\tfrac{27}{7}) - y = 7 \Rightarrow \tfrac{54}{7} - y = 7\).  
  So \(y = \tfrac{54}{7} - 7 = \tfrac{5}{7}\).  
  \(\boxed{(\tfrac{27}{7}, \tfrac{5}{7})}\)

  \item \(\begin{cases} 4x + 5y = 23 \\ 6x - 5y = 7 \end{cases}\)  
  Add: \(10x = 30 \Rightarrow x = 3\). Then \(4(3) + 5y = 23 \Rightarrow 12 + 5y = 23 \Rightarrow y = \tfrac{11}{5}\).  
  \(\boxed{(3, \tfrac{11}{5})}\)
\end{enumerate}

\subsection*{Part D Solutions: Word Problems}
\begin{enumerate}
  \setcounter{enumi}{15}
  \item Let the numbers be \(a\) and \(b\).  
  \(a + b = 20\), \(a - b = 6\). Add: \(2a = 26 \Rightarrow a = 13\). Then \(b = 7\).  
  \(\boxed{13 \text{ and } 7}\)

  \item Let length \(L\), width \(W\).  
  Perimeter: \(2L + 2W = 50 \Rightarrow L + W = 25\). Also \(L = W + 5\).  
  Substitute: \(W + 5 + W = 25 \Rightarrow 2W = 20 \Rightarrow W = 10\). Then \(L = 15\).  
  \(\boxed{L = 15 \text{ cm},\ W = 10 \text{ cm}}\)

  \item Let son \(s\), father \(f\).  
  \(s + f = 60\), \(f = s + 24\). Substitute: \(s + (s + 24) = 60 \Rightarrow 2s = 36 \Rightarrow s = 18\). Then \(f = 42\).  
  \(\boxed{18 \text{ and } 42}\)

  \item Let movie \(M\), snack \(S\).  
  \(M + S = 22\), \(2M + S = 35\). Subtract first from second: \(M = 13\). Then \(S = 9\).  
  \(\boxed{M = \$13,\ S = \$9}\)

  \item Let number of 5-dollar bills be \(x\), 10-dollar bills be \(y\).  
  \(x + y = 16\), \(5x + 10y = 120\). Substitute \(x = 16 - y\):  
  \(5(16 - y) + 10y = 120 \Rightarrow 80 - 5y + 10y = 120 \Rightarrow 5y = 40 \Rightarrow y = 8\).  
  Then \(x = 8\).  
  \(\boxed{8 \text{ five-dollar bills and } 8 \text{ ten-dollar bills}}\)
\end{enumerate}


\end{document}
