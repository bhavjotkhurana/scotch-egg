\documentclass[14pt]{extarticle}

% ---------- PACKAGES ----------
\usepackage{amsmath, amssymb}
\usepackage{geometry}
\usepackage{setspace}
\usepackage{titlesec}
\usepackage{xcolor}
\usepackage{helvet}
\usepackage{enumitem}


% ---------- PAGE SETUP ----------
\geometry{margin=1in}
\setstretch{1.5}
\setlength{\parindent}{0pt}
\setlength{\parskip}{0.75em}
\setlist{itemsep=0.75\baselineskip, topsep=0.5\baselineskip}
\titleformat{\section}{\normalfont\Large\bfseries}{\thesection}{1em}{}
\titleformat{\subsection}{\normalfont\large\bfseries}{\thesubsection}{1em}{}
\renewcommand{\familydefault}{\sfdefault}
\renewcommand{\emph}[1]{\textbf{#1}}


% ---------- DOCUMENT ----------
\begin{document}
\raggedright
\pagenumbering{gobble}

\begin{center}
    \LARGE \textbf{Unit 1: Linear Relationships and Equations} \\[6pt]
    \Large \textbf{Topic 2: Solving Linear Equations}
\end{center}

\vspace{1em}

\section*{Concept Summary}

A \textbf{linear equation} is an equation where the variable has an exponent of 1.  
When written in one variable, it has the general form:
\[
ax + b = c
\]
where \(a\), \(b\), and \(c\) are constants.

The goal in solving a linear equation is to \textbf{isolate the variable} on one side of the equation.  
We do this by using the properties of equality:
\begin{itemize}
    \item You can add or subtract the same number from both sides.
    \item You can multiply or divide both sides by the same nonzero number.
\end{itemize}

The solution is the value of the variable that makes the equation true.

\section*{Core Skills}
\begin{itemize}
    \item Simplify each side of the equation before solving.
    \item Use inverse operations to isolate the variable.
    \item Check the solution by substituting it back into the original equation.
\end{itemize}

\section*{Example 1: One-Step Equation}

Solve for \(x\):
\[
x - 7 = 12
\]

\textbf{Step 1: Add 7 to both sides.}
\[
x - 7 + 7 = 12 + 7
\]
\[
x = 19
\]

\textbf{Check:} Substitute \(x = 19\) into the equation.
\[
19 - 7 = 12 \quad \checkmark
\]

\textbf{Final Answer:} \(\boxed{x = 19}\)

\section*{Example 2: Two-Step Equation}

Solve for \(x\):
\[
4x - 5 = 11
\]

\textbf{Step 1: Add 5 to both sides.}
\[
4x = 16
\]

\textbf{Step 2: Divide both sides by 4.}
\[
x = \frac{16}{4}
\]
\[
\boxed{x = 4}
\]

\textbf{Check:} Substitute \(x = 4\) into the equation.
\[
4(4) - 5 = 16 - 5 = 11 \quad \checkmark
\]

\section*{Key Takeaways}
\begin{itemize}
    \item Perform the same operation on both sides of the equation.
    \item Simplify carefully before isolating the variable.
    \item Always verify your solution.
\end{itemize}

\newpage

% ============================================================
% QUESTIONS — TOPIC 2: SOLVING LINEAR EQUATIONS
% ============================================================

\section*{Practice Questions: Solving Linear Equations}

\subsection*{Part A: One-Step Equations}
\begin{enumerate}
    \item Solve for \(x\): \(x + 9 = 14\)
    \item Solve for \(x\): \(x - 5 = -2\)
    \item Solve for \(x\): \(4x = 20\)
    \item Solve for \(x\): \(\dfrac{x}{3} = 7\)
    \item Solve for \(x\): \(-8x = 32\)
\end{enumerate}

\subsection*{Part B: Two-Step Equations}
\begin{enumerate}
    \setcounter{enumi}{5}
    \item Solve for \(x\): \(3x + 4 = 13\)
    \item Solve for \(x\): \(5x - 6 = 19\)
    \item Solve for \(x\): \(\dfrac{x}{2} + 7 = 12\)
    \item Solve for \(x\): \(8x - 9 = 23\)
    \item Solve for \(x\): \(4x + 5 = -11\)
\end{enumerate}

\subsection*{Part C: Equations with Variables on Both Sides}
\begin{enumerate}
    \setcounter{enumi}{10}
    \item Solve for \(x\): \(2x + 7 = 3x - 5\)
    \item Solve for \(x\): \(5x - 8 = 2x + 7\)
    \item Solve for \(x\): \(9x + 4 = 4x + 19\)
    \item Solve for \(x\): \(6x - 10 = 2x + 14\)
    \item Solve for \(x\): \(3x + 9 = -x + 25\)
\end{enumerate}

\subsection*{Part D: Equations with Parentheses and Fractions}
\begin{enumerate}
    \setcounter{enumi}{15}
    \item Solve for \(x\): \(3(x - 2) = 9\)
    \item Solve for \(x\): \(2(x + 4) = 10\)
    \item Solve for \(x\): \(5(x - 1) = 3x + 9\)
    \item Solve for \(x\): \(\dfrac{2x - 3}{5} = 7\)
    \item Solve for \(x\): \(\dfrac{x + 2}{4} = \dfrac{x - 1}{2}\)
\end{enumerate}

\subsection*{Part E: Word Problems}
\begin{enumerate}
    \setcounter{enumi}{20}
    \item The sum of a number and 8 is equal to 15. What is the number?
    \item A number decreased by 6 equals twice that number minus 12. Find the number.
    \item Five more than three times a number is equal to 20. What is the number?
    \item When 4 is subtracted from half a number, the result is 6. What is the number?
    \item The perimeter \(P\) of a rectangle is given by \(P = 2L + 2W\). If \(P = 30\) and \(W = 5\), find the length \(L\).
\end{enumerate}

\newpage

% ============================================================
% SOLUTIONS — TOPIC 2: SOLVING LINEAR EQUATIONS
% ============================================================

\section*{Answer Key and Solutions: Solving Linear Equations}

\subsection*{Part A Solutions: One-Step Equations}
\begin{enumerate}
    \item \(x + 9 = 14 \Rightarrow x = 14 - 9 = \boxed{5}\)
    \item \(x - 5 = -2 \Rightarrow x = -2 + 5 = \boxed{3}\)
    \item \(4x = 20 \Rightarrow x = \dfrac{20}{4} = \boxed{5}\)
    \item \(\dfrac{x}{3} = 7 \Rightarrow x = 7 \cdot 3 = \boxed{21}\)
    \item \(-8x = 32 \Rightarrow x = \dfrac{32}{-8} = \boxed{-4}\)
\end{enumerate}

\subsection*{Part B Solutions: Two-Step Equations}
\begin{enumerate}
    \setcounter{enumi}{5}
    \item \(3x + 4 = 13 \Rightarrow 3x = 9 \Rightarrow x = \boxed{3}\)
    \item \(5x - 6 = 19 \Rightarrow 5x = 25 \Rightarrow x = \boxed{5}\)
    \item \(\dfrac{x}{2} + 7 = 12 \Rightarrow \dfrac{x}{2} = 5 \Rightarrow x = \boxed{10}\)
    \item \(8x - 9 = 23 \Rightarrow 8x = 32 \Rightarrow x = \boxed{4}\)
    \item \(4x + 5 = -11 \Rightarrow 4x = -16 \Rightarrow x = \boxed{-4}\)
\end{enumerate}

\subsection*{Part C Solutions: Variables on Both Sides}
\begin{enumerate}
    \setcounter{enumi}{10}
    \item \(2x + 7 = 3x - 5 \Rightarrow 7 = x - 5 \Rightarrow x = \boxed{12}\)
    \item \(5x - 8 = 2x + 7 \Rightarrow 3x = 15 \Rightarrow x = \boxed{5}\)
    \item \(9x + 4 = 4x + 19 \Rightarrow 5x = 15 \Rightarrow x = \boxed{3}\)
    \item \(6x - 10 = 2x + 14 \Rightarrow 4x = 24 \Rightarrow x = \boxed{6}\)
    \item \(3x + 9 = -x + 25 \Rightarrow 4x = 16 \Rightarrow x = \boxed{4}\)
\end{enumerate}

\subsection*{Part D Solutions: Parentheses and Fractions}
\begin{enumerate}
    \setcounter{enumi}{15}
    \item \(3(x - 2) = 9 \Rightarrow 3x - 6 = 9 \Rightarrow 3x = 15 \Rightarrow x = \boxed{5}\)
    \item \(2(x + 4) = 10 \Rightarrow 2x + 8 = 10 \Rightarrow 2x = 2 \Rightarrow x = \boxed{1}\)
    \item \(5(x - 1) = 3x + 9 \Rightarrow 5x - 5 = 3x + 9 \Rightarrow 2x = 14 \Rightarrow x = \boxed{7}\)
    \item \(\dfrac{2x - 3}{5} = 7 \Rightarrow 2x - 3 = 35 \Rightarrow 2x = 38 \Rightarrow x = \boxed{19}\)
    \item \(\dfrac{x + 2}{4} = \dfrac{x - 1}{2} \Rightarrow 2(x + 2) = 4(x - 1) \Rightarrow 2x + 4 = 4x - 4 \Rightarrow 8 = 2x \Rightarrow x = \boxed{4}\)
\end{enumerate}

\subsection*{Part E Solutions: SAT-Style Word Problems}
\begin{enumerate}
    \setcounter{enumi}{20}
    \item Let the number be \(n\). \(n + 8 = 15 \Rightarrow n = \boxed{7}\)
    \item Let the number be \(n\). \(n - 6 = 2n - 12 \Rightarrow 6 = n \Rightarrow \boxed{6}\)
    \item Let the number be \(n\). \(3n + 5 = 20 \Rightarrow 3n = 15 \Rightarrow n = \boxed{5}\)
    \item Let the number be \(n\). \(\dfrac{n}{2} - 4 = 6 \Rightarrow \dfrac{n}{2} = 10 \Rightarrow n = \boxed{20}\)
    \item \(P = 2L + 2W\). With \(P = 30\) and \(W = 5\): \(30 = 2L + 10 \Rightarrow 20 = 2L \Rightarrow L = \boxed{10}\)
\end{enumerate}


\end{document}
