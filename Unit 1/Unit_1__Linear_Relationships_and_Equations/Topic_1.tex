\documentclass[14pt]{extarticle}

% ---------- PACKAGES ----------
\usepackage{amsmath, amssymb}
\usepackage{geometry}
\usepackage{setspace}
\usepackage{titlesec}
\usepackage{xcolor}
\usepackage{helvet}
\usepackage{enumitem}

% ---------- PAGE SETUP ----------
\geometry{margin=1in}
\setstretch{1.5}
\setlength{\parindent}{0pt}
\setlength{\parskip}{0.75em}
\setlist{itemsep=0.75\baselineskip, topsep=0.5\baselineskip}
\titleformat{\section}{\normalfont\Large\bfseries}{\thesection}{1em}{}
\titleformat{\subsection}{\normalfont\large\bfseries}{\thesubsection}{1em}{}
\renewcommand{\familydefault}{\sfdefault}
\renewcommand{\emph}[1]{\textbf{#1}}

% ---------- DOCUMENT ----------
\begin{document}
\raggedright
\pagenumbering{gobble}

\begin{center}
    \LARGE \textbf{Unit 1: Linear Relationships and Equations} \\[6pt]
    \Large \textbf{Topic 1: Variables, Expressions, and Equations}
\end{center}

\vspace{1em}

\section*{Concept Summary}

Mathematical expressions and equations are the building blocks of algebra.  
\textbf{Variables} represent unknown values. We often use letters like \(x\), \(y\), or \(n\) to stand for a number that can change.  

An \textbf{expression} is a combination of numbers, variables, and operations, but it does not include an equals sign.  
\[
3x + 5 \quad \text{or} \quad 2(a - 4)
\]

An \textbf{equation} shows that two expressions are equal.  
\[
3x + 5 = 17
\]

To \textbf{solve an equation} means to find the value of the variable that makes the equation true.  
The goal is to isolate the variable on one side of the equation using opposite operations.

\section*{Core Skills}
\begin{itemize}
    \item Simplify expressions by combining like terms and using the distributive property.
    \item Translate word phrases into algebraic expressions.
    \item Solve simple linear equations by isolating the variable.
\end{itemize}

\section*{Example 1: Simplifying an Expression}

Simplify the expression:
\[
3(2y - 4) + y
\]

\textbf{Step 1: Apply the distributive property.}
\[
3 \times 2y = 6y, \quad 3 \times (-4) = -12
\]
So the expression becomes:
\[
6y - 12 + y
\]

\textbf{Step 2: Combine like terms.}
\[
6y + y = 7y
\]

\textbf{Final Answer:}
\[
\boxed{7y - 12}
\]

\section*{Example 2: Solving a Simple Equation}

Solve for \(x\):
\[
2x + 5 = 19
\]

\textbf{Step 1: Subtract 5 from both sides.}
\[
2x + 5 - 5 = 19 - 5
\]
\[
2x = 14
\]

\textbf{Step 2: Divide both sides by 2.}
\[
x = \frac{14}{2}
\]
\[
\boxed{x = 7}
\]

\textbf{Check:}
Substitute \(x = 7\) back into the original equation:
\[
2(7) + 5 = 19 \quad \checkmark
\]

\section*{Key Takeaways}
\begin{itemize}
    \item Expressions do \emph{not} have an equals sign; equations do.
    \item Always perform the same operation on both sides of an equation to keep it balanced.
    \item Checking your solution helps confirm accuracy.
\end{itemize}

\newpage

% ============================================================
% QUESTIONS
% ============================================================

\section*{Practice Questions}

\subsection*{Part A: Simplifying Expressions}
\begin{enumerate}
    \item Simplify: \(4x + 3x - 5\)
    \item Simplify: \(5(2y - 1) + 3y\)
    \item Simplify: \(2(a + 3) - 4(a - 1)\)
    \item Simplify: \(7m - 2(3m - 5)\)
    \item Simplify: \(3x + 2y - (4x - 3y)\)
\end{enumerate}

\subsection*{Part B: Evaluating Expressions}
\begin{enumerate}
    \setcounter{enumi}{5}
    \item If \(x = 4\), evaluate \(3x + 5\).
    \item If \(a = 2\) and \(b = -3\), evaluate \(2a - 3b\).
    \item If \(p = -1\), evaluate \(4p^2 - 3p + 2\).
    \item If \(x = 5\) and \(y = -2\), evaluate \(2x - 3y\).
    \item If \(m = 6\), evaluate \(\dfrac{2m - 4}{m}\).
\end{enumerate}

\subsection*{Part C: Solving Linear Equations}
\begin{enumerate}
    \setcounter{enumi}{10}
    \item Solve for \(x\): \(x + 7 = 12\)
    \item Solve for \(x\): \(4x - 5 = 11\)
    \item Solve for \(y\): \(3y + 2 = 17\)
    \item Solve for \(x\): \(2x - 3 = 9\)
    \item Solve for \(x\): \(5x + 8 = 23\)
\end{enumerate}

\subsection*{Part D: Multi-Step and Fractional Equations}
\begin{enumerate}
    \setcounter{enumi}{15}
    \item Solve for \(x\): \(3(x - 2) = 9\)
    \item Solve for \(x\): \(2(x + 4) = 5x - 6\)
    \item Solve for \(x\): \(\dfrac{2x - 3}{5} = 3\)
    \item Solve for \(x\): \(\dfrac{x + 2}{4} = \dfrac{x - 1}{2}\)
    \item Solve for \(x\): \(7x - 4 = 2x + 11\)
\end{enumerate}

\newpage

% ============================================================
% SOLUTIONS
% ============================================================

\section*{Answer Key and Solutions}

\subsection*{Part A Solutions}
\begin{enumerate}
    \item \(4x + 3x - 5 = 7x - 5\)
    \item \(5(2y - 1) + 3y = 10y - 5 + 3y = 13y - 5\)
    \item \(2(a + 3) - 4(a - 1) = 2a + 6 - 4a + 4 = -2a + 10\)
    \item \(7m - 2(3m - 5) = 7m - 6m + 10 = m + 10\)
    \item \(3x + 2y - (4x - 3y) = 3x + 2y - 4x + 3y = -x + 5y\)
\end{enumerate}

\subsection*{Part B Solutions}
\begin{enumerate}
    \setcounter{enumi}{5}
    \item \(3(4) + 5 = 12 + 5 = 17\)
    \item \(2(2) - 3(-3) = 4 + 9 = 13\)
    \item \(4(-1)^2 - 3(-1) + 2 = 4 + 3 + 2 = 9\)
    \item \(2(5) - 3(-2) = 10 + 6 = 16\)
    \item \(\dfrac{2(6) - 4}{6} = \dfrac{8}{6} = \dfrac{4}{3}\)
\end{enumerate}

\subsection*{Part C Solutions}
\begin{enumerate}
    \setcounter{enumi}{10}
    \item \(x + 7 = 12 \Rightarrow x = 5\)
    \item \(4x - 5 = 11 \Rightarrow 4x = 16 \Rightarrow x = 4\)
    \item \(3y + 2 = 17 \Rightarrow 3y = 15 \Rightarrow y = 5\)
    \item \(2x - 3 = 9 \Rightarrow 2x = 12 \Rightarrow x = 6\)
    \item \(5x + 8 = 23 \Rightarrow 5x = 15 \Rightarrow x = 3\)
\end{enumerate}

\subsection*{Part D Solutions}
\begin{enumerate}
    \setcounter{enumi}{15}
    \item \(3(x - 2) = 9 \Rightarrow 3x - 6 = 9 \Rightarrow 3x = 15 \Rightarrow x = 5\)
    \item \(2(x + 4) = 5x - 6 \Rightarrow 2x + 8 = 5x - 6 \Rightarrow 14 = 3x \Rightarrow x = \dfrac{14}{3}\)
    \item \(\dfrac{2x - 3}{5} = 3 \Rightarrow 2x - 3 = 15 \Rightarrow 2x = 18 \Rightarrow x = 9\)
    \item \(\dfrac{x + 2}{4} = \dfrac{x - 1}{2} \Rightarrow 2(x + 2) = 4(x - 1) \Rightarrow 2x + 4 = 4x - 4 \Rightarrow 8 = 2x \Rightarrow x = 4\)
    \item \(7x - 4 = 2x + 11 \Rightarrow 5x = 15 \Rightarrow x = 3\)
\end{enumerate}

\end{document}
