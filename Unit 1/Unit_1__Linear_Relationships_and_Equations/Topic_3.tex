\documentclass[14pt]{extarticle}

% ---------- PACKAGES ----------
\usepackage{amsmath, amssymb}
\usepackage{geometry}
\usepackage{setspace}
\usepackage{titlesec}
\usepackage{xcolor}
\usepackage{helvet}
\usepackage{enumitem}

% ---------- PAGE SETUP ----------
\geometry{margin=1in}
\setstretch{1.5}
\setlength{\parindent}{0pt}
\setlength{\parskip}{0.75em}
\setlist{itemsep=0.75\baselineskip, topsep=0.5\baselineskip}
\titleformat{\section}{\normalfont\Large\bfseries}{\thesection}{1em}{}
\titleformat{\subsection}{\normalfont\large\bfseries}{\thesubsection}{1em}{}
\renewcommand{\familydefault}{\sfdefault}
\renewcommand{\emph}[1]{\textbf{#1}}

% ---------- DOCUMENT ----------
\begin{document}
\raggedright
\pagenumbering{gobble}

\begin{center}
    \LARGE \textbf{Unit 1: Linear Relationships and Equations} \\[6pt]
    \Large \textbf{Topic 3: Multi-Step and Fractional Equations}
\end{center}

\vspace{1em}

\section*{Concept Summary}

Many SAT algebra questions involve equations that take more than one step to solve.  
These are called \textbf{multi-step equations}. They may include parentheses, fractions, or variables on both sides.  

The goal is still the same: \textbf{isolate the variable}.  
To do that, we simplify both sides of the equation by:
\begin{enumerate}
    \item Using the distributive property to remove parentheses.  
    \item Combining like terms on each side.  
    \item Using addition, subtraction, multiplication, or division to isolate the variable.
\end{enumerate}

When fractions appear, it is often easiest to eliminate them first by multiplying both sides by a common denominator.

\section*{Core Skills}
\begin{itemize}
    \item Use the distributive property to expand expressions.  
    \item Combine like terms carefully.  
    \item Multiply through by the least common denominator (LCD) to remove fractions.  
    \item Check your final answer by substituting it back into the equation.
\end{itemize}

\section*{Example 1: Multi-Step Equation}

Solve for \(x\):
\[
3(x - 2) = 9
\]

\textbf{Step 1: Distribute.}
\[
3x - 6 = 9
\]

\textbf{Step 2: Add 6 to both sides.}
\[
3x = 15
\]

\textbf{Step 3: Divide by 3.}
\[
x = 5
\]

\textbf{Check:}
\[
3(5 - 2) = 9 \Rightarrow 3(3) = 9 \quad \checkmark
\]

\[
\boxed{x = 5}
\]

\section*{Example 2: Fractional Equation}

Solve for \(x\):
\[
\dfrac{2x - 3}{5} = 3
\]

\textbf{Step 1: Multiply both sides by 5 to remove the denominator.}
\[
2x - 3 = 15
\]

\textbf{Step 2: Add 3 to both sides.}
\[
2x = 18
\]

\textbf{Step 3: Divide by 2.}
\[
x = 9
\]

\textbf{Check:}
\[
\dfrac{2(9) - 3}{5} = \dfrac{18 - 3}{5} = \dfrac{15}{5} = 3 \quad \checkmark
\]

\[
\boxed{x = 9}
\]

\section*{Key Takeaways}
\begin{itemize}
    \item Simplify both sides before isolating the variable.  
    \item Use the distributive property correctly when parentheses are involved.  
    \item When fractions are present, multiply by the LCD to clear denominators.  
    \item Always check your answer to avoid sign or arithmetic errors.
\end{itemize}

\newpage

% ============================================================
% QUESTIONS — TOPIC 3: MULTI-STEP AND FRACTIONAL EQUATIONS
% ============================================================

\section*{Practice Questions: Multi-Step and Fractional Equations}

\subsection*{Part A: Multi-Step Equations}
\begin{enumerate}
    \item Solve for \(x\): \(2(x + 3) = 14\)
    \item Solve for \(x\): \(4x - 7 = 13\)
    \item Solve for \(x\): \(5x + 6 = 3x + 16\)
    \item Solve for \(x\): \(3(x - 4) + 5 = 14\)
    \item Solve for \(x\): \(6x - 3 = 2x + 13\)
\end{enumerate}

\subsection*{Part B: Parentheses and Distribution}
\begin{enumerate}
    \setcounter{enumi}{5}
    \item Solve for \(x\): \(2(3x - 5) = 14\)
    \item Solve for \(x\): \(5(x + 2) - 3 = 17\)
    \item Solve for \(x\): \(4(x - 3) + 2x = 18\)
    \item Solve for \(x\): \(3(x + 1) = 2(x + 5)\)
    \item Solve for \(x\): \(7x - 4(2x - 3) = 10\)
\end{enumerate}

\subsection*{Part C: Fractional Equations (One Denominator)}
\begin{enumerate}
    \setcounter{enumi}{10}
    \item Solve for \(x\): \(\dfrac{x - 4}{3} = 2\)
    \item Solve for \(x\): \(\dfrac{2x + 5}{4} = 7\)
    \item Solve for \(x\): \(\dfrac{3x - 2}{5} = 4\)
    \item Solve for \(x\): \(\dfrac{5x - 1}{2} = 9\)
    \item Solve for \(x\): \(\dfrac{x + 3}{6} = 2\)
\end{enumerate}

\subsection*{Part D: Fractional Equations (Two Denominators)}
\begin{enumerate}
    \setcounter{enumi}{15}
    \item Solve for \(x\): \(\dfrac{x + 1}{4} = \dfrac{x - 3}{2}\)
    \item Solve for \(x\): \(\dfrac{3x - 5}{6} = \dfrac{x + 1}{3}\)
    \item Solve for \(x\): \(\dfrac{2x - 1}{5} = \dfrac{x + 4}{10}\)
    \item Solve for \(x\): \(\dfrac{x - 2}{3} = \dfrac{2x + 4}{6}\)
    \item Solve for \(x\): \(\dfrac{5x + 2}{8} = \dfrac{x + 6}{4}\)
\end{enumerate}

\subsection*{Part E: SAT-Style Word and Context Problems}
\begin{enumerate}
    \setcounter{enumi}{20}
    \item Twice a number minus 6 equals 14. What is the number?
    \item When 3 is added to half a number, the result is 11. Find the number.
    \item The sum of 4 times a number and 7 is equal to 23. Find the number.
    \item A plumber charges a \$50 service fee plus \$30 per hour. If a customer paid \$200, how many hours did the plumber work?
    \item A taxi ride costs \$2.50 plus \$0.75 per mile. If the total cost was \$11.75, how many miles was the ride?
\end{enumerate}
\newpage

% ============================================================
% SOLUTIONS — TOPIC 3: MULTI-STEP AND FRACTIONAL EQUATIONS
% ============================================================

\section*{Answer Key and Solutions: Multi-Step and Fractional Equations}

\subsection*{Part A Solutions: Multi-Step Equations}
\begin{enumerate}
    \item \(2(x + 3) = 14 \Rightarrow x + 3 = 7 \Rightarrow x = \boxed{4}\)
    \item \(4x - 7 = 13 \Rightarrow 4x = 20 \Rightarrow x = \boxed{5}\)
    \item \(5x + 6 = 3x + 16 \Rightarrow 2x = 10 \Rightarrow x = \boxed{5}\)
    \item \(3(x - 4) + 5 = 14 \Rightarrow 3x - 12 + 5 = 14 \Rightarrow 3x - 7 = 14 \Rightarrow 3x = 21 \Rightarrow x = \boxed{7}\)
    \item \(6x - 3 = 2x + 13 \Rightarrow 4x = 16 \Rightarrow x = \boxed{4}\)
\end{enumerate}

\subsection*{Part B Solutions: Parentheses and Distribution}
\begin{enumerate}
    \setcounter{enumi}{5}
    \item \(2(3x - 5) = 14 \Rightarrow 6x - 10 = 14 \Rightarrow 6x = 24 \Rightarrow x = \boxed{4}\)
    \item \(5(x + 2) - 3 = 17 \Rightarrow 5x + 10 - 3 = 17 \Rightarrow 5x + 7 = 17 \Rightarrow 5x = 10 \Rightarrow x = \boxed{2}\)
    \item \(4(x - 3) + 2x = 18 \Rightarrow 4x - 12 + 2x = 18 \Rightarrow 6x = 30 \Rightarrow x = \boxed{5}\)
    \item \(3(x + 1) = 2(x + 5) \Rightarrow 3x + 3 = 2x + 10 \Rightarrow x = \boxed{7}\)
    \item \(7x - 4(2x - 3) = 10 \Rightarrow 7x - 8x + 12 = 10 \Rightarrow -x + 12 = 10 \Rightarrow -x = -2 \Rightarrow x = \boxed{2}\)
\end{enumerate}

\subsection*{Part C Solutions: Fractional Equations (One Denominator)}
\begin{enumerate}
    \setcounter{enumi}{10}
    \item \(\dfrac{x - 4}{3} = 2 \Rightarrow x - 4 = 6 \Rightarrow x = \boxed{10}\)
    \item \(\dfrac{2x + 5}{4} = 7 \Rightarrow 2x + 5 = 28 \Rightarrow 2x = 23 \Rightarrow x = \boxed{\dfrac{23}{2}}\)
    \item \(\dfrac{3x - 2}{5} = 4 \Rightarrow 3x - 2 = 20 \Rightarrow 3x = 22 \Rightarrow x = \boxed{\dfrac{22}{3}}\)
    \item \(\dfrac{5x - 1}{2} = 9 \Rightarrow 5x - 1 = 18 \Rightarrow 5x = 19 \Rightarrow x = \boxed{\dfrac{19}{5}}\)
    \item \(\dfrac{x + 3}{6} = 2 \Rightarrow x + 3 = 12 \Rightarrow x = \boxed{9}\)
\end{enumerate}

\subsection*{Part D Solutions: Fractional Equations (Two Denominators)}
\begin{enumerate}
    \setcounter{enumi}{15}
    \item \(\dfrac{x + 1}{4} = \dfrac{x - 3}{2} \Rightarrow 2(x + 1) = 4(x - 3) \Rightarrow 2x + 2 = 4x - 12 \Rightarrow 14 = 2x \Rightarrow x = \boxed{7}\)
    \item \(\dfrac{3x - 5}{6} = \dfrac{x + 1}{3} \Rightarrow 3(3x - 5) = 6(x + 1) \Rightarrow 9x - 15 = 6x + 6 \Rightarrow 3x = 21 \Rightarrow x = \boxed{7}\)
    \item \(\dfrac{2x - 1}{5} = \dfrac{x + 4}{10} \Rightarrow 10(2x - 1) = 5(x + 4) \Rightarrow 20x - 10 = 5x + 20 \Rightarrow 15x = 30 \Rightarrow x = \boxed{2}\)
    \item \(\dfrac{x - 2}{3} = \dfrac{2x + 4}{6}\). Reduce right side: \(\dfrac{2x + 4}{6} = \dfrac{x + 2}{3}\). Then \(\dfrac{x - 2}{3} = \dfrac{x + 2}{3} \Rightarrow x - 2 = x + 2\), which is impossible. \textbf{No solution}.
    \item \(\dfrac{5x + 2}{8} = \dfrac{x + 6}{4} \Rightarrow 4(5x + 2) = 8(x + 6) \Rightarrow 20x + 8 = 8x + 48 \Rightarrow 12x = 40 \Rightarrow x = \boxed{\dfrac{10}{3}}\)
\end{enumerate}

\subsection*{Part E Solutions: SAT-Style Word and Context Problems}
\begin{enumerate}
    \setcounter{enumi}{20}
    \item Let the number be \(n\). \(2n - 6 = 14 \Rightarrow 2n = 20 \Rightarrow n = \boxed{10}\)
    \item Let the number be \(n\). \(\dfrac{n}{2} + 3 = 11 \Rightarrow \dfrac{n}{2} = 8 \Rightarrow n = \boxed{16}\)
    \item Let the number be \(n\). \(4n + 7 = 23 \Rightarrow 4n = 16 \Rightarrow n = \boxed{4}\)
    \item Cost model: \(C = 50 + 30h\). Given \(C = 200\): \(50 + 30h = 200 \Rightarrow 30h = 150 \Rightarrow h = \boxed{5}\)
    \item Cost model: \(C = 2.50 + 0.75m\). Given \(C = 11.75\): \(0.75m = 11.75 - 2.50 = 9.25 \Rightarrow m = \dfrac{9.25}{0.75} = \dfrac{37}{3} \Rightarrow \boxed{\dfrac{37}{3}\text{ miles}}\)
\end{enumerate}


\end{document}
