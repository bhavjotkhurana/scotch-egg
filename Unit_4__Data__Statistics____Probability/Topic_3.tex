\documentclass[12pt]{article}

\usepackage{amsmath, amssymb}
\usepackage{geometry}
\usepackage{setspace}
\usepackage{titlesec}
\usepackage{lmodern}
\usepackage{xcolor}
\usepackage{enumitem}

\geometry{margin=1in}
\setstretch{1.2}
\titleformat{\section}{\normalfont\Large\bfseries}{\thesection}{1em}{}
\titleformat{\subsection}{\normalfont\large\bfseries}{\thesubsection}{1em}{}
\pagenumbering{gobble}

\begin{document}

\begin{center}
    \LARGE \textbf{Unit 4: Data, Statistics, and Probability} \\[6pt]
    \Large \textbf{Topic 3: Weighted Averages and Combined Means}
\end{center}

\vspace{1em}

\section*{Concept Summary}

A \textbf{weighted average} gives different importance (weights) to each data value.  
\[
\text{Weighted mean} = \frac{w_1x_1 + w_2x_2 + \cdots + w_nx_n}{w_1 + w_2 + \cdots + w_n}.
\]

Weights may represent frequency, percentage, or relative importance.

When combining groups with different averages, use the formula:
\[
\text{Combined mean} = \frac{n_1\bar{x}_1 + n_2\bar{x}_2}{n_1 + n_2}.
\]

\section*{Core Skills}
\begin{itemize}
  \item Multiply each value by its corresponding weight.
  \item Divide by the sum of weights to find the weighted mean.
  \item For group combinations, use total sums or averages with counts.
  \item Recognize that a weighted average leans toward the group with the larger weight.
\end{itemize}

\section*{Example 1: Simple Weighted Average}

A student has grades:  
Homework 90 (weight 40\%), Tests 80 (weight 60\%).  
\[
\text{Average} = \frac{0.4(90)+0.6(80)}{1} = 36 + 48 = \boxed{84}.
\]

\section*{Example 2: Combining Groups}

Group A: average 70, 10 people.  
Group B: average 90, 20 people.  
\[
\text{Combined mean} = \frac{10(70)+20(90)}{30} = \frac{700+1800}{30} = \boxed{83.3}.
\]

\section*{Example 3: Weighted Average by Frequency}

\[
\begin{array}{c|ccc}
\text{Score} & 60 & 80 & 100\\ \hline
\text{Frequency} & 3 & 4 & 3
\end{array}
\]
Weighted mean:
\[
\frac{3(60)+4(80)+3(100)}{3+4+3} = \frac{180+320+300}{10} = \boxed{80}.
\]

\section*{Example 4: Weighted Average by Percent}

A student’s grade is 85 on homework (20\%), 90 on quizzes (30\%), and 80 on tests (50\%).  
\[
\text{Average} = 0.2(85) + 0.3(90) + 0.5(80) = 17 + 27 + 40 = \boxed{84}.
\]

\section*{Example 5: SAT-Style Context}

Two factories produce widgets. Factory A makes 500 widgets with average cost \$10.  
Factory B makes 200 widgets with average cost \$12.  
\[
\text{Combined cost} = \frac{500(10)+200(12)}{700} = \frac{7400}{700} = \boxed{10.57}.
\]

\section*{Key Takeaways}
\begin{itemize}
  \item Weighted average = total of “value × weight” ÷ total weight.
  \item Combined mean depends on both averages and group sizes.
  \item Greater weight pulls the overall mean closer to that group’s mean.
  \item Useful for class averages, grade calculations, and mixture problems.
\end{itemize}

\newpage

% ============================================================
% QUESTIONS — TOPIC 3: WEIGHTED AVERAGES AND COMBINED MEANS
% ============================================================

\section*{Practice Questions: Weighted Averages and Combined Means}

\subsection*{Part A: Weighted Average Basics}
\begin{enumerate}
  \item A grade is based on 40\% homework (average 80) and 60\% tests (average 90). Find the overall average.
  \item A quiz is worth 25\% of a student’s grade and the final exam 75\%. If the quiz score is 88 and the exam score is 76, find the final grade.
  \item Compute the weighted mean: \(x_1=70, x_2=90, w_1=2, w_2=3.\)
  \item A student’s grades are: 85 (10\%), 75 (30\%), 95 (60\%). Find the weighted average.
  \item An investor earns 4\% on \$2000 and 8\% on \$3000. Find the overall rate of return.
\end{enumerate}

\subsection*{Part B: Combining Groups}
\begin{enumerate}
  \setcounter{enumi}{5}
  \item Class A: 20 students, average 70. Class B: 30 students, average 80. Find the combined average.
  \item Two sets of data have averages 40 and 60 with 5 and 15 items respectively. Find the combined mean.
  \item The mean of Group X (50 people) is 120. The mean of Group Y (150 people) is 130. Find the combined mean.
  \item A factory produces 300 units at \$10 each and 200 units at \$15 each. Find the average unit price.
  \item Group 1: average 75, size 12; Group 2: average 90, size 8. Find combined mean.
\end{enumerate}

\subsection*{Part C: Frequency and Percent Data}
\begin{enumerate}
  \setcounter{enumi}{10}
  \item 
  \[
  \begin{array}{c|ccc}
  \text{Score} & 50 & 70 & 90\\ \hline
  \text{Frequency} & 2 & 5 & 3
  \end{array}
  \]
  Find the weighted mean.

  \item 
  \[
  \begin{array}{c|cccc}
  \text{Score} & 60 & 70 & 80 & 90\\ \hline
  \text{Frequency} & 1 & 3 & 4 & 2
  \end{array}
  \]
  Find the mean score.

  \item A student’s semester grade is based on Homework 20\%, Tests 50\%, and Final 30\%. Scores are 92, 84, and 78 respectively. Find the weighted average.
  \item A car travels 40 km at 60 km/h and 60 km at 40 km/h. Find its average speed for the trip.
  \item A business has two departments: Dept A (60 employees, avg salary \$50,000), Dept B (40 employees, avg salary \$70,000). Find the overall average salary.
\end{enumerate}

\subsection*{Part D: Conceptual and SAT-Style Problems}
\begin{enumerate}
  \setcounter{enumi}{15}
  \item The mean of a data set is 20. If another data set of equal size has mean 40, what is the mean when they are combined?
  \item The mean of 10 numbers is 12. Another 20 numbers have mean 18. What is the combined mean?
  \item If one test counts twice as much as another, how would you compute the weighted mean of the two test scores?
  \item Two classes have the same mean test score. One has twice as many students. If they are combined, what happens to the overall mean?
  \item A student’s scores are 80, 90, 100, weighted 1, 2, 3 respectively. Find the weighted average.
\end{enumerate}

\newpage

% ============================================================
% SOLUTIONS — UNIT 4, TOPIC 3: WEIGHTED AVERAGES AND COMBINED MEANS
% ============================================================

\section*{Answer Key and Solutions: Weighted Averages and Combined Means}

\subsection*{Part A Solutions: Weighted Average Basics}
\begin{enumerate}
  \item \(0.40(80)+0.60(90)=32+54=\boxed{86}\)
  \item \(0.25(88)+0.75(76)=22+57=\boxed{79}\)
  \item \(\dfrac{2\cdot70+3\cdot90}{2+3}=\dfrac{140+270}{5}=\boxed{82}\)
  \item \(0.10(85)+0.30(75)+0.60(95)=8.5+22.5+57=\boxed{88}\)
  \item Overall rate \(=\dfrac{0.04(2000)+0.08(3000)}{2000+3000}=\dfrac{80+240}{5000}=0.064=\boxed{6.4\%}\)
\end{enumerate}

\subsection*{Part B Solutions: Combining Groups}
\begin{enumerate}
  \setcounter{enumi}{5}
  \item \(\dfrac{20\cdot70+30\cdot80}{20+30}=\dfrac{1400+2400}{50}=\boxed{76}\)
  \item \(\dfrac{5\cdot40+15\cdot60}{5+15}=\dfrac{200+900}{20}=\boxed{55}\)
  \item \(\dfrac{50\cdot120+150\cdot130}{200}=\dfrac{6000+19500}{200}=\boxed{127.5}\)
  \item \(\dfrac{300\cdot10+200\cdot15}{300+200}=\dfrac{3000+3000}{500}=\boxed{12}\)
  \item \(\dfrac{12\cdot75+8\cdot90}{12+8}=\dfrac{900+720}{20}=\boxed{81}\)
\end{enumerate}

\subsection*{Part C Solutions: Frequency and Percent Data}
\begin{enumerate}
  \setcounter{enumi}{10}
  \item \(\dfrac{2\cdot50+5\cdot70+3\cdot90}{2+5+3}=\dfrac{100+350+270}{10}=\boxed{72}\)
  \item \(\dfrac{1\cdot60+3\cdot70+4\cdot80+2\cdot90}{10}=\dfrac{60+210+320+180}{10}=\boxed{77}\)
  \item \(0.20(92)+0.50(84)+0.30(78)=18.4+42+23.4=\boxed{83.8}\)
  \item Average speed \(=\dfrac{\text{total distance}}{\text{total time}}=\dfrac{40+60}{\tfrac{40}{60}+\tfrac{60}{40}}=\dfrac{100}{\tfrac{2}{3}+1.5}=\dfrac{100}{\tfrac{13}{6}}=\dfrac{600}{13}\approx\boxed{46.15\ \text{km/h}}\)
  \item \(\dfrac{60\cdot 50{,}000+40\cdot 70{,}000}{60+40}=\dfrac{3{,}000{,}000+2{,}800{,}000}{100}=\boxed{\$58{,}000}\)
\end{enumerate}

\subsection*{Part D Solutions: Conceptual and SAT-Style}
\begin{enumerate}
  \setcounter{enumi}{15}
  \item Equal sizes \(\Rightarrow\) combined mean \(=\dfrac{20+40}{2}=\boxed{30}\)
  \item \(\dfrac{10\cdot12+20\cdot18}{30}=\dfrac{120+360}{30}=\boxed{16}\)
  \item If one test counts twice: weighted mean \(=\dfrac{1\cdot s_1+2\cdot s_2}{1+2}=\boxed{\dfrac{s_1+2s_2}{3}}\)
  \item Same mean and any sizes \(\Rightarrow\) overall mean \(\boxed{\text{unchanged}}\)
  \item \(\dfrac{1\cdot80+2\cdot90+3\cdot100}{1+2+3}=\dfrac{80+180+300}{6}=\dfrac{560}{6}=\boxed{93\tfrac{1}{3}}\ (\approx \boxed{93.33})\)
\end{enumerate}



\end{document}
