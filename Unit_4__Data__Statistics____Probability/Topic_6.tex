\documentclass[12pt]{article}

\usepackage{amsmath, amssymb}
\usepackage{geometry}
\usepackage{setspace}
\usepackage{titlesec}
\usepackage{lmodern}
\usepackage{xcolor}
\usepackage{enumitem}

\geometry{margin=1in}
\setstretch{1.2}
\titleformat{\section}{\normalfont\Large\bfseries}{\thesection}{1em}{}
\titleformat{\subsection}{\normalfont\large\bfseries}{\thesubsection}{1em}{}
\pagenumbering{gobble}

\begin{document}

\begin{center}
    \LARGE \textbf{Unit 4: Data, Statistics, and Probability} \\[6pt]
    \Large \textbf{Topic 6: Understanding Variability and Standard Deviation (Conceptual)}
\end{center}

\vspace{1em}

\section*{Concept Summary}

\textbf{Variability} measures how spread out data values are. Two data sets can have the same mean but very different variability.

\begin{itemize}
  \item The \textbf{range} (max–min) is a simple measure of spread.
  \item The \textbf{standard deviation (SD)} measures the average distance of data points from the mean.
  \item A small SD means data are close to the mean; a large SD means they are widely spread.
\end{itemize}

The formula for the population standard deviation is:
\[
\sigma = \sqrt{\frac{\sum (x - \mu)^2}{N}}
\]
and for a sample:
\[
s = \sqrt{\frac{\sum (x - \bar{x})^2}{n-1}}.
\]
In this unit, focus on interpreting—not calculating—standard deviation.

\section*{Core Skills}
\begin{itemize}
  \item Recognize what larger or smaller standard deviation indicates.
  \item Compare two data sets to decide which has greater spread.
  \item Understand that adding or multiplying all data values affects the standard deviation:
  \begin{itemize}
    \item Adding a constant \(k\) does not change SD.
    \item Multiplying by \(c\) multiplies SD by \(|c|\).
  \end{itemize}
  \item Interpret SD in the context of SAT word problems and graphs.
\end{itemize}

\section*{Example 1: Comparing Variability}

Set A: \(2, 4, 6, 8, 10\)  
Set B: \(4, 5, 6, 7, 8\)  
Both have mean 6, but Set A is more spread out. So Set A has a larger SD.

\section*{Example 2: Effect of Outliers}

Set X: \(5, 6, 7, 8, 9\)  
Set Y: \(5, 6, 7, 8, 20\)  
Set Y’s mean is higher and its SD much larger because of the outlier 20.

\section*{Example 3: Graph Interpretation}

\textit{Insert placeholder graph: two bell-shaped curves centered at the same mean; one narrow, one wide.}

The wider curve has greater SD because its values are more spread from the mean.

\section*{Example 4: Scaling and Shifting}

If every test score increases by 5 points, SD is unchanged.  
If every score doubles, SD doubles.

\section*{Example 5: Real-World Context}

Set A: daily high temperatures (degrees F) over one week in a tropical climate.  
Set B: daily highs in a northern climate with big changes.  
Set B has higher SD because of greater temperature variation.

\section*{Key Takeaways}
\begin{itemize}
  \item SD measures spread, not center.
  \item Larger SD → more variability.
  \item Adding constants shifts data, not spread.
  \item Multiplying constants changes the spread proportionally.
  \item Outliers increase SD significantly.
\end{itemize}

\newpage

% ============================================================
% QUESTIONS — TOPIC 6: UNDERSTANDING VARIABILITY AND STANDARD DEVIATION (CONCEPTUAL)
% ============================================================

\section*{Practice Questions: Understanding Variability and Standard Deviation}

\subsection*{Part A: Conceptual Understanding}
\begin{enumerate}
  \item What does the standard deviation of a data set measure?
  \item If two data sets have the same mean but different standard deviations, what does that tell you?
  \item How does adding a constant (like +5) to all values affect the SD?
  \item How does multiplying all data values by 3 affect the SD?
  \item What effect does an outlier have on SD?
\end{enumerate}

\subsection*{Part B: Comparing Data Sets}
\begin{enumerate}
  \setcounter{enumi}{5}
  \item Set A: \(1, 3, 5, 7, 9\); Set B: \(4, 5, 6, 5, 6\). Which has the larger SD?
  \item Set X: \(10, 10, 10, 10, 10\); Set Y: \(8, 9, 10, 11, 12\). Which has greater variability?
  \item Set P: \(2, 4, 6, 8\); Set Q: \(20, 40, 60, 80\). How does SD of Q compare to SD of P?
  \item Two classes took the same quiz. Class 1’s scores are tightly grouped around the mean; Class 2’s are spread widely. Which class has a higher SD?
  \item Which of these is likely to have the smallest SD:  
  (a) Heights of adult men, (b) Daily temperatures in one city over a week, (c) Scores on a random guessing quiz?
\end{enumerate}

\subsection*{Part C: Interpreting Graphs and Scenarios}
\begin{enumerate}
  \setcounter{enumi}{10}
  \item \textit{Insert placeholder: two bell curves with same mean; Curve A is narrower, Curve B is wider.}  
  Which has the larger SD, and why?
  \item A histogram shows that most data values are close to the mean. What does that say about SD?
  \item Two data sets each have mean 50. One has SD = 2, the other SD = 10. Which is more consistent?
  \item If the SD of a sample is 0, what does that mean about the data?
  \item Why does the presence of an extreme value (outlier) increase the SD?
\end{enumerate}

\subsection*{Part D: Real-World and SAT-Style Problems}
\begin{enumerate}
  \setcounter{enumi}{15}
  \item Test A: scores clustered around 75; Test B: scores from 40 to 100. Which test shows more consistency?
  \item A class’s quiz scores have mean 80, SD 5. Another class’s scores have mean 80, SD 15. Which class’s performance was more consistent?
  \item The mean weekly income for two towns is the same, but Town A’s SD is higher. What does that imply?
  \item Two SAT math sections have similar averages, but one has a much larger SD. What does that say about the range of student abilities?
  \item If all students’ SAT scores increase by 100 points next year, what happens to the SD of scores?
\end{enumerate}

\newpage

% ============================================================
% SOLUTIONS — UNIT 4, TOPIC 6: UNDERSTANDING VARIABILITY AND STANDARD DEVIATION (CONCEPTUAL)
% ============================================================

\section*{Answer Key and Solutions: Variability and Standard Deviation}

\subsection*{Part A Solutions: Conceptual Understanding}
\begin{enumerate}
  \item SD measures typical distance of data values from the mean; it quantifies spread.
  \item Same mean, different SD means one set is more spread out than the other.
  \item Adding a constant shifts all values equally, SD is unchanged.
  \item Multiplying by 3 scales all distances from the mean by 3, SD triples.
  \item An outlier increases SD because it adds large squared deviations from the mean.
\end{enumerate}

\subsection*{Part B Solutions: Comparing Data Sets}
\begin{enumerate}
  \setcounter{enumi}{5}
  \item Set A has larger SD. Its values are farther from the mean than B’s clustered values.
  \item Set Y has greater variability. X has SD 0, Y has positive SD.
  \item SD of Q is 10 times SD of P, since Q is P multiplied by 10.
  \item Class 2 has higher SD due to wider spread.
  \item Smallest SD: heights of adult men. The other contexts vary more from trial to trial.
\end{enumerate}

\subsection*{Part C Solutions: Interpreting Graphs and Scenarios}
\begin{enumerate}
  \setcounter{enumi}{10}
  \item Curve B (wider) has larger SD because values are more dispersed from the mean.
  \item Most values near the mean implies a small SD.
  \item The set with SD 2 is more consistent; values cluster tightly around 50.
  \item SD 0 means all data values are identical.
  \item Because SD uses squared deviations, a single extreme value contributes a large term, raising SD.
\end{enumerate}

\subsection*{Part D Solutions: Real-World and SAT-Style}
\begin{enumerate}
  \setcounter{enumi}{15}
  \item Test A shows more consistency; scores cluster around 75, so smaller SD.
  \item The class with SD 5 is more consistent; the class with SD 15 is more variable.
  \item Higher SD in Town A implies incomes are more spread out around the mean in Town A.
  \item Larger SD indicates a wider range of abilities or performance on that section.
  \item Adding the same 100 to all scores shifts the mean up by 100 but leaves SD unchanged.
\end{enumerate}


\end{document}
