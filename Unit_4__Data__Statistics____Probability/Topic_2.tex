\documentclass[12pt]{article}

\usepackage{amsmath, amssymb}
\usepackage{geometry}
\usepackage{setspace}
\usepackage{titlesec}
\usepackage{lmodern}
\usepackage{xcolor}
\usepackage{enumitem}

\geometry{margin=1in}
\setstretch{1.2}
\titleformat{\section}{\normalfont\Large\bfseries}{\thesection}{1em}{}
\titleformat{\subsection}{\normalfont\large\bfseries}{\thesubsection}{1em}{}
\pagenumbering{gobble}

\begin{document}

\begin{center}
    \LARGE \textbf{Unit 4: Data, Statistics, and Probability} \\[6pt]
    \Large \textbf{Topic 2: Mean, Median, Mode, and Range}
\end{center}

\vspace{1em}

\section*{Concept Summary}

For a data set \(x_1,x_2,\dots,x_n\):
\begin{itemize}
  \item \textbf{Mean (average):} \(\displaystyle \bar{x}=\frac{x_1+\cdots+x_n}{n}\).
  \item \textbf{Median:} middle value after sorting. If \(n\) is even, average of the two middle values.
  \item \textbf{Mode:} most frequent value(s).
  \item \textbf{Range:} max minus min.
\end{itemize}

Outliers pull the \textbf{mean} more than the \textbf{median}. Median is more resistant to extreme values.

\section*{Core Skills}
\begin{itemize}
  \item Sort the data before finding the median and range.
  \item Compute mean quickly using totals or by using balance reasoning.
  \item Identify modes from a list or simple frequency table.
  \item Decide which measure best represents the center given outliers or skew.
\end{itemize}

\section*{Example 1: All Four Measures}

Data: \(7, 9, 4, 6, 9\). Sorted: \(4,6,7,9,9\).
\[
\bar{x}=\frac{35}{5}=7,\quad \text{median}=7,\quad \text{mode}=9,\quad \text{range}=9-4=5.
\]

\section*{Example 2: Even Count Median}

Data: \(12, 15, 18, 20\). Sorted already.  
Median \(=\frac{15+18}{2}=16.5\).  
Mean \(=\frac{65}{4}=16.25\). Range \(=20-12=8\). No mode.

\section*{Example 3: Effect of an Outlier}

Test scores: \(70, 74, 76, 78, 80, 82, 100\).  
Mean \(=\frac{560}{7}=80\). Median is the 4th value \(=78\).  
The 100 raises the mean above the median. Median better represents the typical score.

\section*{Example 4: From a Frequency Table}

\[
\begin{array}{c|cccc}
\text{Value} & 2 & 3 & 4 & 5\\ \hline
\text{Frequency} & 1 & 2 & 3 & 2
\end{array}
\]
Total count \(n=8\). Sum \(=2(1)+3(2)+4(3)+5(2)=2+6+12+10=30\).  
Mean \(=30/8=3.75\). Mode \(=4\) (highest frequency).  
Sorted positions 4th and 5th are both 4, so median \(=4\). Range \(=5-2=3\).

\section*{Example 5: Shifts and Scales}

If each data value increases by \(k\), mean and median increase by \(k\), range unchanged.  
If each data value is multiplied by \(c>0\), mean, median, and range are multiplied by \(c\).

\section*{Key Takeaways}
\begin{itemize}
  \item Use the median when data are skewed or have outliers.
  \item Mean equals total divided by count. Keep track of \(n\).
  \item Mode describes most common value. There can be more than one or none.
  \item Range measures spread but is sensitive to extremes.
\end{itemize}

\newpage

% ============================================================
% QUESTIONS — TOPIC 2: MEAN, MEDIAN, MODE, AND RANGE
% ============================================================

\section*{Practice Questions: Mean, Median, Mode, and Range}

\subsection*{Part A: Basic Computation}
Find the mean, median, mode, and range for each data set.
\begin{enumerate}
  \item \(4, 7, 9, 3, 6\)
  \item \(10, 12, 14, 16, 18, 20\)
  \item \(2, 5, 8, 8, 10, 12\)
  \item \(15, 20, 25, 30, 35\)
  \item \(9, 9, 9, 12, 15, 18\)
\end{enumerate}

\subsection*{Part B: Interpreting Data}
\begin{enumerate}
  \setcounter{enumi}{5}
  \item The ages of employees in a small company are: 22, 25, 25, 26, 27, 29, 60.  
  Which measure (mean or median) better represents the typical age? Explain.
  \item A dataset has mean 50 and median 48. Is the distribution likely skewed left, right, or symmetric?
  \item A dataset of scores: 75, 80, 82, 83, 85, 90, 100. Identify the outlier and describe its effect on the mean.
  \item If 5 is added to every number in a dataset, how do the mean and median change?
  \item If every value in a dataset is doubled, what happens to the mean, median, and range?
\end{enumerate}

\subsection*{Part C: Frequency Tables}
\begin{enumerate}
  \setcounter{enumi}{10}
  \item 
  \[
  \begin{array}{c|cccc}
  \text{Value} & 2 & 3 & 4 & 5\\ \hline
  \text{Frequency} & 1 & 2 & 4 & 3
  \end{array}
  \]
  Find mean, median, and mode.

  \item 
  \[
  \begin{array}{c|ccc}
  \text{Value} & 10 & 20 & 30\\ \hline
  \text{Frequency} & 2 & 5 & 3
  \end{array}
  \]
  Find the mean and identify the mode.

  \item 
  \[
  \begin{array}{c|cccc}
  \text{Value} & 1 & 2 & 3 & 4\\ \hline
  \text{Frequency} & 3 & 2 & 2 & 1
  \end{array}
  \]
  Find mean, median, and range.

  \item 
  \[
  \begin{array}{c|ccccc}
  \text{Score} & 60 & 70 & 80 & 90 & 100\\ \hline
  \text{Frequency} & 2 & 4 & 6 & 4 & 2
  \end{array}
  \]
  Find mean and median.

  \item 
  \[
  \begin{array}{c|cccc}
  \text{Number of pets} & 0 & 1 & 2 & 3\\ \hline
  \text{Frequency} & 3 & 6 & 4 & 2
  \end{array}
  \]
  Find the most common number of pets (mode) and the mean.
\end{enumerate}

\subsection*{Part D: Word Problems}
\begin{enumerate}
  \setcounter{enumi}{15}
  \item The mean of five numbers is 12. Four of the numbers are 8, 10, 14, and 18. Find the fifth number.
  \item The average of 6 test scores is 80. If one score (60) is dropped, what is the average of the remaining 5 scores?
  \item A class of 10 students has an average height of 160 cm. If one new student with height 180 cm joins, what is the new mean height?
  \item The median of 7 numbers is 10. If the largest number increases by 20, what happens to the median?
  \item A dataset has mean 40 and range 12. If each value increases by 3, what are the new mean and range?
\end{enumerate}

\subsection*{Part E: SAT-Style Applications}
\begin{enumerate}
  \setcounter{enumi}{20}
  \item A student’s five quiz scores are 85, 90, 80, 100, and 75. If the lowest score is dropped, what is the mean of the remaining scores?
  \item The mean of four numbers is 25. When a fifth number is added, the new mean becomes 30. What is the fifth number?
  \item The median of a set of 9 numbers is 50. If the largest value increases by 10, what happens to the median?
  \item In a dataset, the mode is 8 and the range is 12. If all values are doubled, what are the new mode and range?
  \item The mean of three numbers is 20. Two of the numbers are 15 and 25. Find the third number.
\end{enumerate}

\newpage

% ============================================================
% SOLUTIONS — UNIT 4, TOPIC 2: MEAN, MEDIAN, MODE, AND RANGE
% ============================================================

\section*{Answer Key and Solutions: Mean, Median, Mode, and Range}

\subsection*{Part A Solutions: Basic Computation}
\begin{enumerate}
  \item Data \(4,7,9,3,6\). Sum \(=29\Rightarrow \text{mean}=29/5=5.8\). Sorted \(3,4,6,7,9\Rightarrow \text{median}=6\). \(\text{Mode}=\) none. \(\text{Range}=9-3=6\).
  \item Sum \(=90\Rightarrow \text{mean}=90/6=15\). Median \(=(14+16)/2=15\). Mode none. Range \(=20-10=10\).
  \item Sum \(=45\Rightarrow \text{mean}=45/6=7.5\). Median \(=(8+8)/2=8\). Mode \(=8\). Range \(=12-2=10\).
  \item Sum \(=125\Rightarrow \text{mean}=25\). Median \(=25\). Mode none. Range \(=35-15=20\).
  \item Sum \(=72\Rightarrow \text{mean}=12\). Median \(=(9+12)/2=10.5\). Mode \(=9\). Range \(=18-9=9\).
\end{enumerate}

\subsection*{Part B Solutions: Interpreting Data}
\begin{enumerate}
  \setcounter{enumi}{5}
  \item Median better. The 60 is an outlier that pulls the mean up.
  \item Mean \(>\) median suggests right skew.
  \item Outlier \(=100\). It increases the mean more than the median.
  \item Mean increases by 5. Median increases by 5.
  \item Mean and median both double. Range doubles.
\end{enumerate}

\subsection*{Part C Solutions: Frequency Tables}
\begin{enumerate}
  \setcounter{enumi}{10}
  \item \(n=1+2+4+3=10\). Sum \(=2(1)+3(2)+4(4)+5(3)=39\Rightarrow \text{mean}=3.9\). Mode \(=4\). Median is average of 5th and 6th values, both 4, so median \(=4\).
  \item \(n=10\). Sum \(=10(2)+20(5)+30(3)=210\Rightarrow \text{mean}=21\). Mode \(=20\) (highest frequency 5).
  \item \(n=8\). Sum \(=1(3)+2(2)+3(2)+4(1)=17\Rightarrow \text{mean}=17/8=2.125\). Median is average of 4th and 5th values, both 2, so median \(=2\). Range \(=4-1=3\).
  \item \(n=18\). Sum \(=60(2)+70(4)+80(6)+90(4)+100(2)=1440\Rightarrow \text{mean}=1440/18=80\). Median is between 9th and 10th values; both are 80, so median \(=80\).
  \item \(n=15\). Mode \(=1\) (freq 6). Mean \(=\dfrac{0(3)+1(6)+2(4)+3(2)}{15}=\dfrac{20}{15}=\dfrac{4}{3}\approx1.33\).
\end{enumerate}

\subsection*{Part D Solutions: Word Problems}
\begin{enumerate}
  \setcounter{enumi}{15}
  \item Total \(=5\cdot12=60\). Known sum \(=8+10+14+18=50\). Fifth \(=60-50=\boxed{10}\).
  \item Total \(=6\cdot80=480\). Remove 60 gives \(420\) over 5 scores. New mean \(=420/5=\boxed{84}\).
  \item Old total \(=10\cdot160=1600\). New total \(=1600+180=1780\). New mean \(=1780/11\approx\boxed{161.8\text{ cm}}\).
  \item The median is the 4th of 7 numbers. Changing the largest leaves the median \(\boxed{\text{unchanged at }10}\).
  \item New mean \(=40+3=\boxed{43}\). Range unchanged \(\boxed{12}\).
\end{enumerate}

\subsection*{Part E Solutions: SAT-Style Applications}
\begin{enumerate}
  \setcounter{enumi}{20}
  \item Sum \(=85+90+80+100+75=430\). Drop 75 \(\Rightarrow\) \(355/4=\boxed{88.75}\).
  \item Four-number total \(=4\cdot25=100\). Five-number total \(=5\cdot30=150\). Fifth \(=150-100=\boxed{50}\).
  \item Median is the 5th value. Increasing largest does not change it. \(\boxed{\text{Unchanged at }50}\).
  \item Mode doubles to \(\boxed{16}\). Range doubles to \(\boxed{24}\).
  \item Total \(=3\cdot20=60\). Known \(=15+25=40\). Third \(=60-40=\boxed{20}\).
\end{enumerate}


\end{document}
