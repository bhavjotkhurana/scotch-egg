\documentclass[12pt]{article}

\usepackage{amsmath, amssymb}
\usepackage{geometry}
\usepackage{setspace}
\usepackage{titlesec}
\usepackage{lmodern}
\usepackage{xcolor}
\usepackage{enumitem}

\geometry{margin=1in}
\setstretch{1.2}
\titleformat{\section}{\normalfont\Large\bfseries}{\thesection}{1em}{}
\titleformat{\subsection}{\normalfont\large\bfseries}{\thesubsection}{1em}{}
\pagenumbering{gobble}

\begin{document}

\begin{center}
    \LARGE \textbf{Unit 4: Data, Statistics, and Probability} \\[6pt]
    \Large \textbf{Topic 1: Reading Tables, Graphs, and Scatterplots}
\end{center}

\vspace{1em}

\section*{Concept Summary}

The SAT often presents data in \textbf{tables, bar graphs, line graphs, or scatterplots} and asks you to interpret or compare values.  
Your task is to \textbf{read carefully}, identify patterns, and connect the question to what’s shown.

\subsection*{Common Data Displays}
\begin{itemize}
  \item \textbf{Tables:} Show exact numerical values, often for multiple categories.
  \item \textbf{Bar graphs:} Compare quantities across categories.
  \item \textbf{Line graphs:} Show trends or change over time.
  \item \textbf{Scatterplots:} Show relationships between two variables. Look for positive, negative, or no correlation.
\end{itemize}

\section*{Core Skills}
\begin{itemize}
  \item Identify which variable is on each axis.
  \item Read exact values (e.g., from a table or bar height).
  \item Estimate trends and slopes for graphs.
  \item Distinguish between correlation and causation.
  \item Recognize outliers and general patterns in scatterplots.
\end{itemize}

\section*{Example 1: Reading a Table}

A table lists the number of books sold by a store each month:

\[
\begin{array}{c|cccccc}
\text{Month} & \text{Jan} & \text{Feb} & \text{Mar} & \text{Apr} & \text{May} & \text{Jun}\\ \hline
\text{Books Sold} & 120 & 150 & 180 & 210 & 240 & 270
\end{array}
\]

The number of books increases by 30 each month.  
\textbf{Question:} How many books were sold in March?  
\textbf{Answer:} 180.

\section*{Example 2: Interpreting a Bar Graph}

A bar graph (insert here: “bar graph showing the number of students choosing subjects Math, Science, English, History, and Art”)  
shows the tallest bar at Science with 50 students, and the shortest bar at Art with 20 students.  
\textbf{Question:} Which subject is most popular?  
\textbf{Answer:} Science.

\section*{Example 3: Interpreting a Line Graph}

A line graph (insert here: “line graph showing average temperature over 12 months”)  
shows a steady rise from January to July, peaking at 85°F, then declining.  
\textbf{Question:} During which months does temperature increase the fastest?  
\textbf{Answer:} Between March and June.

\section*{Example 4: Reading a Scatterplot}

A scatterplot (insert here: “scatterplot of hours studied vs. SAT Math score”)  
shows points roughly following an upward trend.  
\textbf{Interpretation:} As hours studied increase, SAT scores tend to increase — this is a \textbf{positive correlation}.

\section*{Example 5: Identifying Outliers}

In the same scatterplot, one point shows a student who studied 20 hours but scored very low.  
This point is an \textbf{outlier}, meaning it does not fit the general pattern.

\section*{Example 6: Misinterpretation Warning}

Correlation does not imply causation.  
Even if two variables move together, one does not necessarily cause the other (for example, ice cream sales and drowning incidents both increase in summer, but one does not cause the other).

\section*{Key Takeaways}
\begin{itemize}
  \item Identify variable types and axes before answering.
  \item Use the graph or table for exact or approximate values as needed.
  \item Check for overall trends, correlations, and anomalies.
  \item Be cautious when interpreting correlations — do not assume cause and effect.
\end{itemize}

\newpage

% ============================================================
% QUESTIONS — TOPIC 1: READING TABLES, GRAPHS, AND SCATTERPLOTS
% ============================================================

\section*{Practice Questions: Reading Tables, Graphs, and Scatterplots}

\subsection*{Part A: Reading Tables}
\begin{enumerate}
  \item The table below shows sales for a company over six months.  
  \[
  \begin{array}{c|cccccc}
  \text{Month} & \text{Jan} & \text{Feb} & \text{Mar} & \text{Apr} & \text{May} & \text{Jun}\\ \hline
  \text{Sales (\$1000s)} & 20 & 25 & 30 & 40 & 50 & 55
  \end{array}
  \]
  What is the total sales for the six months?

  \item Using the same table, what is the percent increase in sales from January to June?

  \item A new employee earns the following weekly salaries (in dollars): 500, 520, 540, 560, 580.  
  What is the average weekly salary?

  \item A table lists population growth:  
  \[
  \begin{array}{c|cccc}
  \text{Year} & 2018 & 2019 & 2020 & 2021\\ \hline
  \text{Population} & 12{,}000 & 12{,}600 & 13{,}230 & 13{,}890
  \end{array}
  \]
  What is the percent increase from 2018 to 2019?

  \item In a survey, 40 students chose their favorite sport: 10 basketball, 15 soccer, 8 baseball, 7 swimming.  
  What percent chose soccer?
\end{enumerate}

\subsection*{Part B: Reading and Interpreting Graphs}
(Insert line graph showing company revenue increasing steadily from 2018 to 2023, with a small dip in 2021.)

\begin{enumerate}
  \setcounter{enumi}{5}
  \item During which year did the company experience a decline in revenue?
  \item Between which two years did revenue increase the most?
  \item If revenue in 2018 was \$1.2 million and 2023 revenue is \$2.1 million, what is the total increase?
  \item What is the average rate of increase per year?
  \item If the trend continues, estimate revenue for 2024.
\end{enumerate}

\subsection*{Part C: Bar Graphs}
(Insert bar graph comparing number of students choosing: Math = 45, Science = 50, English = 35, History = 30, Art = 20.)

\begin{enumerate}
  \setcounter{enumi}{10}
  \item Which subject is the most popular?
  \item What is the difference between the number of students in Science and Art?
  \item What percent of students chose Math?
  \item If 10 more students choose English next year, what will be the new total for English?
  \item What is the average number of students across all five subjects?
\end{enumerate}

\subsection*{Part D: Scatterplots}
(Insert scatterplot showing hours studied on the x-axis and SAT Math score on the y-axis, with a generally upward trend.)

\begin{enumerate}
  \setcounter{enumi}{15}
  \item What type of correlation is shown between hours studied and SAT score?
  \item Describe what an outlier would look like on this scatterplot.
  \item If one student studied 15 hours and scored 700, what does that suggest about the trend?
  \item If another student studied 5 hours and scored 500, how does this fit with the pattern?
  \item Does this graph prove that studying more causes higher scores? Why or why not?
\end{enumerate}

\subsection*{Part E: SAT-Style Applications}
(Insert mixed data set: line graph of temperature vs. month and a scatterplot of hours of sleep vs. productivity.)

\begin{enumerate}
  \setcounter{enumi}{20}
  \item The line graph shows temperatures increasing from March to July, peaking at 90°F. What month likely had the greatest rate of temperature increase?
  \item The scatterplot of hours of sleep vs. productivity shows points clustered around an upward trend. What kind of relationship exists between these variables?
  \item In a bar graph showing average test scores by class period, the highest bar corresponds to Period 3. What does this indicate about that class?
  \item A table shows rainfall amounts for four cities. Which city had the largest decrease compared to last year?
  \item On a scatterplot of study time vs. test scores, one student studied 20 hours but scored very low. What does this point represent?
\end{enumerate}

\newpage

% ============================================================
% SOLUTIONS — UNIT 4, TOPIC 1: READING TABLES, GRAPHS, AND SCATTERPLOTS
% ============================================================

\section*{Answer Key and Solutions: Reading Tables, Graphs, and Scatterplots}

\subsection*{Part A Solutions: Reading Tables}
\begin{enumerate}
  \item Total sales: \(20+25+30+40+50+55=220\) (\$1000s) \(\Rightarrow \boxed{\$220{,}000}\)
  \item Percent increase Jan \(\to\) Jun: \(\dfrac{55-20}{20}\cdot100\%=\boxed{175\%}\)
  \item Average: \(\dfrac{500+520+540+560+580}{5}=\dfrac{2700}{5}=\boxed{540}\)
  \item \(\dfrac{12{,}600-12{,}000}{12{,}000}\cdot100\%=\boxed{5\%}\)
  \item \(\dfrac{15}{40}\cdot100\%=\boxed{37.5\%}\)
\end{enumerate}

\subsection*{Part B Solutions: Reading and Interpreting Graphs}
\textit{Note: answers rely on the inserted line graph.}
\begin{enumerate}
  \setcounter{enumi}{5}
  \item Year of decline: \(\boxed{2021}\) (small dip shown).
  \item Largest increase: the steepest upward segment; from the mock, \(\boxed{2021\to 2022}\) (confirm from the graph).
  \item Total increase: \(2.1-1.2=\boxed{0.9\text{ million dollars}}\).
  \item Average rate per year: \(\dfrac{0.9}{5}=\boxed{0.18\text{ million per year}}=\boxed{\$180{,}000/\text{yr}}\).
  \item Linear projection: \(2.1+0.18=\boxed{2.28\text{ million}}\) (assumes same average rate).
\end{enumerate}

\subsection*{Part C Solutions: Bar Graphs}
\textit{Note: uses the provided category counts.}
\begin{enumerate}
  \setcounter{enumi}{10}
  \item \(\boxed{\text{Science}}\) (50).
  \item \(50-20=\boxed{30}\).
  \item Total \(=45+50+35+30+20=180\). Math percent \(=45/180= \boxed{25\%}\).
  \item New English total \(=35+10=\boxed{45}\).
  \item Average \(=180/5=\boxed{36}\).
\end{enumerate}

\subsection*{Part D Solutions: Scatterplots}
\textit{Note: relies on the inserted scatterplot.}
\begin{enumerate}
  \setcounter{enumi}{15}
  \item \(\boxed{\text{Positive correlation}}\).
  \item An outlier is a point far from the overall trend, e.g., \(\boxed{\text{high hours but low score (or vice versa)}}\).
  \item \((15,700)\) lies on/near the upward trend \(\Rightarrow\) \(\boxed{\text{consistent with positive association}}\).
  \item \((5,500)\) is near the cluster for low hours \(\Rightarrow\) \(\boxed{\text{fits the pattern}}\).
  \item \(\boxed{\text{No. Correlation }\neq\text{ causation; other factors may influence scores.}}\)
\end{enumerate}

\subsection*{Part E Solutions: SAT-Style Applications}
\textit{Note: confirm with the inserted visuals/tables.}
\begin{enumerate}
  \setcounter{enumi}{20}
  \item Greatest rate of increase: the month-to-month with steepest slope; typically \(\boxed{\text{May}\to\text{June}}\) (verify on the graph).
  \item \(\boxed{\text{Positive relationship}}\) (productivity tends to rise with sleep).
  \item Period 3 has the highest bar \(\Rightarrow\) \(\boxed{\text{highest average score among periods}}\).
  \item \(\boxed{\text{City with largest negative change}}\) (compute “this year minus last year” and choose most negative).
  \item \(\boxed{\text{An outlier}}\) (studied many hours but scored unusually low).
\end{enumerate}



\end{document}
