\documentclass[12pt]{article}

\usepackage{amsmath, amssymb}
\usepackage{geometry}
\usepackage{setspace}
\usepackage{titlesec}
\usepackage{lmodern}
\usepackage{xcolor}
\usepackage{enumitem}

\geometry{margin=1in}
\setstretch{1.2}
\titleformat{\section}{\normalfont\Large\bfseries}{\thesection}{1em}{}
\titleformat{\subsection}{\normalfont\large\bfseries}{\thesubsection}{1em}{}
\pagenumbering{gobble}

\begin{document}

\begin{center}
    \LARGE \textbf{Unit 4: Data, Statistics, and Probability} \\[6pt]
    \Large \textbf{Topic 7: Interpreting Trend Lines and Regression (Conceptual)}
\end{center}

\vspace{1em}

\section*{Concept Summary}

A \textbf{trend line} (or line of best fit) shows the general direction of data in a scatterplot.  
It helps predict values and understand the relationship between two variables.

\begin{itemize}
  \item \textbf{Positive correlation:} as one variable increases, the other tends to increase.
  \item \textbf{Negative correlation:} as one variable increases, the other tends to decrease.
  \item \textbf{No correlation:} points show no clear pattern.
\end{itemize}

A line of best fit is often written as:
\[
\hat{y} = mx + b
\]
where \(m\) is the slope (rate of change) and \(b\) is the \(y\)-intercept (predicted \(y\) when \(x=0\)).

\section*{Core Skills}
\begin{itemize}
  \item Identify direction (positive, negative, none) of correlation.
  \item Interpret slope and intercept in real-world context.
  \item Use the line to make predictions or estimates.
  \item Recognize that correlation does not imply causation.
\end{itemize}

\section*{Example 1: Positive Correlation}

\textit{Insert placeholder: scatterplot showing upward trend between hours studied and test score.}  
As study hours increase, scores increase.  
The trend line shows a positive slope.

\section*{Example 2: Negative Correlation}

\textit{Insert placeholder: scatterplot showing downward trend between absences and GPA.}  
As absences increase, GPA decreases.  
Negative slope, negative correlation.

\section*{Example 3: No Correlation}

\textit{Insert placeholder: random scatter with no clear pattern.}  
No consistent relationship between the two variables.

\section*{Example 4: Interpreting the Equation of a Trend Line}

For the trend line \(\hat{y} = 5x + 60\):  
\begin{itemize}
  \item \(m = 5\): each additional study hour increases score by about 5 points.  
  \item \(b = 60\): if 0 hours studied, predicted score is 60.
\end{itemize}

\section*{Example 5: Making Predictions}

If \(x = 4\) hours, predicted score \(= 5(4)+60=80\).

\section*{Example 6: Correlation Coefficient \(r\) (Conceptual)}

The correlation coefficient \(r\) measures strength and direction of a linear relationship:  
\[
-1 \le r \le 1
\]
\begin{itemize}
  \item \(r \approx 1\): strong positive correlation  
  \item \(r \approx -1\): strong negative correlation  
  \item \(r \approx 0\): little or no linear relationship
\end{itemize}

\section*{Example 7: Causation Reminder}

Even with strong correlation, one variable may not cause the other.  
Example: Ice cream sales and drownings both rise in summer—correlated due to a third factor (temperature).

\section*{Key Takeaways}
\begin{itemize}
  \item Trend lines describe relationships and allow prediction.
  \item The slope shows rate of change; the intercept gives baseline prediction.
  \item \(r\) measures strength of linear correlation.
  \item Correlation $\neq$ causation; always interpret context carefully.
\end{itemize}

\newpage

% ============================================================
% QUESTIONS — TOPIC 7: INTERPRETING TREND LINES AND REGRESSION (CONCEPTUAL)
% ============================================================

\section*{Practice Questions: Interpreting Trend Lines and Regression}

\subsection*{Part A: Understanding Correlation}
\begin{enumerate}
  \item What does a positive correlation between two variables mean?
  \item What does a negative correlation mean?
  \item What does it mean if there is no correlation?
  \item Can two variables be strongly correlated without one causing the other? Explain.
  \item Which value of \(r\) shows the strongest relationship: 0.2, -0.7, or 0.9?
\end{enumerate}

\subsection*{Part B: Interpreting Scatterplots}
\textit{For each question, imagine a scatterplot (insert placeholder where needed).}
\begin{enumerate}
  \setcounter{enumi}{5}
  \item \textit{Insert placeholder: scatterplot sloping upward.}  
  What type of correlation is shown?
  \item \textit{Insert placeholder: scatterplot sloping downward.}  
  What type of correlation is shown?
  \item \textit{Insert placeholder: random scatter of points.}  
  What type of correlation is shown?
  \item Which scatterplot would have the correlation coefficient \(r = -0.95\)?
  \item Which scatterplot would have \(r = 0\)?
\end{enumerate}

\subsection*{Part C: Interpreting Equations of Trend Lines}
\begin{enumerate}
  \setcounter{enumi}{10}
  \item The trend line is \(\hat{y} = 4x + 20\). Interpret the slope.
  \item Interpret the intercept in that same line.
  \item If \(x = 5\), predict \(\hat{y}\).
  \item The line \(\hat{y} = -2x + 100\) models GPA vs. absences. What does the slope mean here?
  \item For \(\hat{y} = 6x + 50\), what happens to \(y\) when \(x\) increases by 3?
\end{enumerate}

\subsection*{Part D: Correlation Coefficient and Strength}
\begin{enumerate}
  \setcounter{enumi}{15}
  \item Which of the following shows the weakest linear relationship: \(r = 0.1, r = -0.3, r = 0.8\)?
  \item If \(r = -1\), what does that tell you about the data?
  \item If \(r = 0\), what does that tell you?
  \item Which of the following is most likely for the relationship between height and weight: \(r = 0.8\), \(r = -0.5\), \(r = 0\)?
  \item Which is more strongly correlated: \(r = 0.6\) or \(r = -0.9\)?
\end{enumerate}

\subsection*{Part E: SAT-Style Applications}
\begin{enumerate}
  \setcounter{enumi}{20}
  \item A trend line for the relationship between hours studied (\(x\)) and test score (\(y\)) is \(\hat{y} = 10x + 50\). Predict the score for a student who studies 4 hours.
  \item A regression line \(\hat{y} = -3x + 90\) represents the relationship between number of absences and test score. Interpret the slope.
  \item Two scatterplots have \(r = 0.9\) and \(r = -0.9\). How do the relationships differ?
  \item Which situation is likely to have a negative correlation?  
  (a) Hours of sleep vs. alertness  
  (b) Distance driven vs. gas left  
  (c) Hours studied vs. test score
  \item Why does a strong correlation not always mean one variable causes the other?
\end{enumerate}

\newpage

% ============================================================
% SOLUTIONS — UNIT 4, TOPIC 7: INTERPRETING TREND LINES AND REGRESSION (CONCEPTUAL)
% ============================================================

\section*{Answer Key and Solutions: Trend Lines and Regression}

\subsection*{Part A Solutions: Understanding Correlation}
\begin{enumerate}
  \item As \(x\) increases, \(y\) tends to increase.
  \item As \(x\) increases, \(y\) tends to decrease.
  \item No consistent linear relationship between \(x\) and \(y\).
  \item Yes. A third variable can drive both or it can be coincidence. Correlation does not imply causation.
  \item \(0.9\) is strongest in magnitude.
\end{enumerate}

\subsection*{Part B Solutions: Interpreting Scatterplots}
\begin{enumerate}
  \setcounter{enumi}{5}
  \item Positive correlation.
  \item Negative correlation.
  \item No correlation.
  \item The strong downward plot.
  \item The random scatter plot.
\end{enumerate}

\subsection*{Part C Solutions: Interpreting Equations}
\begin{enumerate}
  \setcounter{enumi}{10}
  \item Slope 4 means each 1 unit increase in \(x\) raises predicted \(y\) by 4.
  \item Intercept 20 means when \(x=0\), predicted \(y=20\).
  \item \(\hat{y}=4(5)+20=40\).
  \item Each additional absence decreases predicted GPA by 2 points.
  \item Increase in \(x\) by 3 raises \(y\) by \(6\cdot3=18\).
\end{enumerate}

\subsection*{Part D Solutions: \(r\) and Strength}
\begin{enumerate}
  \setcounter{enumi}{15}
  \item \(r=0.1\) is weakest by magnitude.
  \item Perfect negative linear relationship. All points on a line with negative slope.
  \item No linear association.
  \item \(r=0.8\) is most plausible. Height and weight are positively related.
  \item \(|-0.9|>|0.6|\) so \(r=-0.9\) is stronger.
\end{enumerate}

\subsection*{Part E Solutions: SAT-Style Applications}
\begin{enumerate}
  \setcounter{enumi}{20}
  \item \(\hat{y}=10(4)+50=90\).
  \item Each additional absence lowers predicted score by 3 points.
  \item Both are strong. \(0.9\) is strong positive, \(-0.9\) is strong negative.
  \item \(\boxed{\text{b}}\) Distance driven vs gas left.
  \item There may be lurking variables or coincidence. Direction and strength do not establish cause.
\end{enumerate}


\end{document}
