\documentclass[12pt]{article}

\usepackage{amsmath, amssymb}
\usepackage{geometry}
\usepackage{setspace}
\usepackage{titlesec}
\usepackage{lmodern}
\usepackage{xcolor}
\usepackage{enumitem}

\geometry{margin=1in}
\setstretch{1.2}
\titleformat{\section}{\normalfont\Large\bfseries}{\thesection}{1em}{}
\titleformat{\subsection}{\normalfont\large\bfseries}{\thesubsection}{1em}{}
\pagenumbering{gobble}

\begin{document}

\begin{center}
    \LARGE \textbf{Unit 4: Data, Statistics, and Probability} \\[6pt]
    \Large \textbf{Topic 5: Conditional Probability}
\end{center}

\vspace{1em}

\section*{Concept Summary}

\textbf{Conditional probability} measures the chance of an event given that another event has occurred.  
\[
P(A\mid B)=\frac{P(A\cap B)}{P(B)},\quad \text{provided } P(B)>0.
\]

Key relationships:
\begin{itemize}
  \item \textbf{Multiplication rule:} \(P(A\cap B)=P(A\mid B)\,P(B)=P(B\mid A)\,P(A)\).
  \item \textbf{Independence:} \(A\) and \(B\) are independent if \(P(A\mid B)=P(A)\) which is equivalent to \(P(A\cap B)=P(A)P(B)\).
  \item \textbf{Law of Total Probability:} If \(B_1,\dots,B_k\) partition the sample space, then \(P(A)=\sum_i P(A\mid B_i)P(B_i)\).
\end{itemize}

\section*{Core Skills}
\begin{itemize}
  \item Translate word statements into \(P(\cdot)\) notation with the correct condition.
  \item Use two way tables to compute conditional probabilities by dividing the appropriate cell by the given row or column total.
  \item Use tree diagrams with branch probabilities to track multi step processes.
  \item Test independence by checking \(P(A\mid B)=P(A)\) or \(P(A\cap B)=P(A)P(B)\).
\end{itemize}

\section*{Example 1: From a Two Way Table}

A survey records whether students prefer tea or coffee and whether they are juniors or seniors.

\[
\begin{array}{c|ccc}
 & \text{Tea} & \text{Coffee} & \text{Total}\\ \hline
\text{Junior} & 24 & 16 & 40\\
\text{Senior} & 21 & 39 & 60\\ \hline
\text{Total} & 45 & 55 & 100
\end{array}
\]

(a) \(P(\text{Tea}\mid \text{Senior})=\dfrac{21}{60}=\dfrac{7}{20}\).  
(b) \(P(\text{Senior}\mid \text{Coffee})=\dfrac{39}{55}\).  
(c) \(P(\text{Tea and Junior})=\dfrac{24}{100}=0.24\).

\section*{Example 2: Independence Check}

Using the table above, \(P(\text{Tea})=45/100=0.45\).  
\(P(\text{Tea}\mid \text{Senior})=21/60=0.35 \ne 0.45\).  
Not equal, so Tea preference and Senior status are not independent.

\section*{Example 3: Tree Diagram for a Multi Step Process}
\textit{Insert a small two level tree diagram here: first branch is rain vs no rain with probabilities 0.3 and 0.7. From each, a branch for traffic jam with conditional probabilities 0.5 if rain and 0.2 if no rain.}

Compute \(P(\text{Jam})\):  
\[
P(\text{Jam})=P(\text{Jam}\mid \text{Rain})P(\text{Rain})+P(\text{Jam}\mid \text{NoRain})P(\text{NoRain})
=0.5(0.3)+0.2(0.7)=0.15+0.14=0.29.
\]

\section*{Example 4: Using the Multiplication Rule}

A bag has 5 red and 7 blue marbles. Draw 2 without replacement.  
\[
P(\text{both red})=P(\text{R}_1)\,P(\text{R}_2\mid \text{R}_1)=\frac{5}{12}\cdot\frac{4}{11}=\frac{20}{132}=\frac{5}{33}.
\]

\section*{Example 5: Bayes Type Update}

A test is positive with probability 0.9 if a person has a condition and 0.1 if not. Prevalence is 5\%.  
\[
P(\text{Has}\mid +)=\frac{0.9\cdot0.05}{0.9\cdot0.05+0.1\cdot0.95}
=\frac{0.045}{0.045+0.095}=\frac{0.045}{0.14}\approx 0.321.
\]
Interpretation: despite a high true positive rate, the low prevalence means a positive result is only about 32.1\% likely to indicate the condition.

\section*{Example 6: SAT Style Card Context}

From a standard deck, one card is drawn and observed to be a face card. What is the probability it is a heart?  
There are 12 face cards total and 3 are hearts.  
\[
P(\text{Heart}\mid \text{Face})=\frac{3}{12}=\frac{1}{4}.
\]

\section*{Key Takeaways}
\begin{itemize}
  \item Read \(P(A\mid B)\) as “probability of \(A\) within the world where \(B\) is known to have occurred.”
  \item For tables, divide the intersection cell by the given condition total.
  \item For sequential processes, multiply along branches and add across paths.
  \item Independence means conditioning does not change the probability.
\end{itemize}

\newpage

% ============================================================
% QUESTIONS — TOPIC 5: CONDITIONAL PROBABILITY
% ============================================================

\section*{Practice Questions: Conditional Probability}

\subsection*{Part A: Understanding Conditional Probability}
\begin{enumerate}
  \item Define \(P(A\mid B)\) in words.
  \item In a class, 60\% of students play sports and 40\% do not. Of those who play sports, 70\% have a GPA above 3.0. Find \(P(\text{GPA}>3.0\mid \text{Sports})\).
  \item A bag contains 4 red and 6 blue marbles. One marble is drawn at random. If a blue marble is drawn, what is \(P(\text{Blue})\)?
  \item A coin is flipped twice. Find \(P(\text{2nd flip is heads}\mid \text{1st flip is heads})\).
  \item A box contains 3 green and 2 yellow balls. If one ball is drawn without replacement and is green, find \(P(\text{next ball is green}\mid \text{first green})\).
\end{enumerate}

\subsection*{Part B: Two-Way Table Problems}
\textit{Use the following table for Questions 6–10.}

\[
\begin{array}{c|ccc}
 & \text{Male} & \text{Female} & \text{Total} \\ \hline
\text{Left-handed} & 8 & 12 & 20\\
\text{Right-handed} & 32 & 48 & 80\\ \hline
\text{Total} & 40 & 60 & 100
\end{array}
\]

\begin{enumerate}
  \setcounter{enumi}{5}
  \item Find \(P(\text{Left-handed})\).
  \item Find \(P(\text{Left-handed}\mid \text{Female})\).
  \item Find \(P(\text{Female}\mid \text{Left-handed})\).
  \item Find \(P(\text{Right-handed and Male})\).
  \item Are gender and handedness independent? Explain using probabilities.
\end{enumerate}

\subsection*{Part C: Tree Diagram Problems}
\textit{Use this description: A machine produces 90\% good items and 10\% defective. A test identifies 95\% of good items as “pass” and 80\% of defective items as “fail.”}

\begin{enumerate}
  \setcounter{enumi}{10}
  \item Draw a tree diagram to represent the situation. (Insert placeholder: "Tree diagram showing good/defective → pass/fail branches.")
  \item Find \(P(\text{Pass})\).
  \item Find \(P(\text{Good}\mid \text{Pass})\).
  \item Find \(P(\text{Defective}\mid \text{Fail})\).
  \item Is a failed test more likely to be defective or good? Support with probabilities.
\end{enumerate}

\subsection*{Part D: Word Problems}
\begin{enumerate}
  \setcounter{enumi}{15}
  \item In a city, 30\% of people own dogs, and 60\% of dog owners own cats. Find \(P(\text{Owns cat}\mid \text{Owns dog})\).
  \item A card is drawn from a deck. Given that it’s a red card, what is \(P(\text{Heart})\)?
  \item A student passes math with probability 0.8, and passes English with probability 0.7. If the events are independent, what is \(P(\text{Pass both})\)?
  \item A student passes English with probability 0.7. If 90\% of math passers also pass English, find \(P(\text{English}\mid \text{Math pass})\).
  \item 10\% of light bulbs are defective. If 95\% of non-defective bulbs and 60\% of defective bulbs pass inspection, find \(P(\text{Defective}\mid \text{Pass})\).
\end{enumerate}

\subsection*{Part E: SAT-Style Applications}
\begin{enumerate}
  \setcounter{enumi}{20}
  \item A student answers a question correctly 80\% of the time when they study, and 30\% of the time when they do not. If they study 70\% of the time, what is \(P(\text{Correct})\)?
  \item From a deck, given that a card is a face card, what is \(P(\text{Spade})\)?
  \item A box has 4 red and 6 blue balls. One is drawn, not replaced, and another drawn. Find \(P(\text{2nd red}\mid \text{1st red})\).
  \item If \(P(A)=0.4\), \(P(B)=0.5\), and \(P(A\cap B)=0.2\), find \(P(A\mid B)\).
  \item A test for a disease is 95\% accurate, and 2\% of the population has the disease. Find \(P(\text{Disease}\mid \text{Positive})\).
\end{enumerate}

\newpage

% ============================================================
% SOLUTIONS — UNIT 4, TOPIC 5: CONDITIONAL PROBABILITY
% ============================================================

\section*{Answer Key and Solutions: Conditional Probability}

\subsection*{Part A Solutions: Understanding Conditional Probability}
\begin{enumerate}
  \item \(P(A\mid B)\): the probability that \(A\) occurs given that \(B\) is known to have occurred.
  \item \(P(\text{GPA}>3.0\mid \text{Sports})=\boxed{0.70}\).
  \item Given the draw is blue, \(P(\text{Blue})=\boxed{1}\).
  \item Coin flips are independent: \(P(\text{2nd H}\mid \text{1st H})=\boxed{\tfrac{1}{2}}\).
  \item After first green: remaining \(=2\) green out of \(4\) total \(\Rightarrow \boxed{\tfrac{1}{2}}\).
\end{enumerate}

\subsection*{Part B Solutions: Two-Way Table}
\[
\begin{array}{c|ccc}
 & \text{Male} & \text{Female} & \text{Total} \\ \hline
\text{Left} & 8 & 12 & 20\\
\text{Right} & 32 & 48 & 80\\ \hline
\text{Total} & 40 & 60 & 100
\end{array}
\]
\begin{enumerate}
  \setcounter{enumi}{5}
  \item \(P(\text{Left})=20/100=\boxed{0.20}\).
  \item \(P(\text{Left}\mid \text{Female})=12/60=\boxed{0.20}\).
  \item \(P(\text{Female}\mid \text{Left})=12/20=\boxed{0.60}\).
  \item \(P(\text{Right and Male})=32/100=\boxed{0.32}\).
  \item Check independence: \(P(\text{Left})=0.20\), \(P(\text{Left}\mid \text{Female})=0.20\), \(P(\text{Left}\mid \text{Male})=8/40=0.20\). Equal, so \(\boxed{\text{independent}}\).
\end{enumerate}

\subsection*{Part C Solutions: Tree Diagram Setting}
Given: \(P(G)=0.9,\ P(D)=0.1\), \(P(\text{Pass}\mid G)=0.95\), \(P(\text{Fail}\mid D)=0.80\Rightarrow P(\text{Pass}\mid D)=0.20\).
\begin{enumerate}
  \setcounter{enumi}{10}
  \item Tree description provided in the question prompt.
  \item \(P(\text{Pass})=0.95(0.9)+0.20(0.1)=0.855+0.02=\boxed{0.875}\).
  \item \(P(G\mid \text{Pass})=\dfrac{0.95\cdot0.9}{0.875}=\dfrac{0.855}{0.875}\approx\boxed{0.977}\).
  \item \(P(D\mid \text{Fail})=\dfrac{0.80\cdot0.10}{0.05\cdot0.90+0.80\cdot0.10}=\dfrac{0.08}{0.125}=\boxed{0.64}\).
  \item A fail is more likely defective: \(P(D\mid \text{Fail})=0.64 > P(G\mid \text{Fail})=0.36\). \(\boxed{\text{Defective}}\).
\end{enumerate}

\subsection*{Part D Solutions: Word Problems}
\begin{enumerate}
  \setcounter{enumi}{15}
  \item \(P(\text{Cat}\mid \text{Dog})=\boxed{0.60}\).
  \item Given red, hearts among red cards \(=13/26=\boxed{\tfrac{1}{2}}\).
  \item Independent: \(0.8\cdot0.7=\boxed{0.56}\).
  \item Given: \(P(\text{English}\mid \text{Math})=\boxed{0.90}\).
  \item \(P(\text{Def}\mid \text{Pass})=\dfrac{0.10\cdot0.60}{0.10\cdot0.60+0.90\cdot0.95}
  =\dfrac{0.06}{0.915}\approx\boxed{0.0656}\ (\text{about }6.6\%)\).
\end{enumerate}

\subsection*{Part E Solutions: SAT-Style Applications}
\begin{enumerate}
  \setcounter{enumi}{20}
  \item \(P(\text{Correct})=0.8(0.7)+0.3(0.3)=0.56+0.09=\boxed{0.65}\).
  \item Face cards total \(=12\). Spade faces \(=3\Rightarrow 3/12=\boxed{\tfrac{1}{4}}\).
  \item After first red, remaining red \(=3\) of \(9\Rightarrow \boxed{\tfrac{1}{3}}\).
  \item \(P(A\mid B)=\dfrac{P(A\cap B)}{P(B)}=\dfrac{0.2}{0.5}=\boxed{0.4}\).
  \item Assume 95\% sensitivity and 95\% specificity. Then
  \[
  P(D\mid +)=\frac{0.95\cdot0.02}{0.95\cdot0.02+0.05\cdot0.98}
  =\frac{0.019}{0.068}\approx\boxed{0.279}\ (\text{about }27.9\%).
  \]
\end{enumerate}



\end{document}
