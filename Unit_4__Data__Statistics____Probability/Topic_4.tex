\documentclass[12pt]{article}

\usepackage{amsmath, amssymb}
\usepackage{geometry}
\usepackage{setspace}
\usepackage{titlesec}
\usepackage{lmodern}
\usepackage{xcolor}
\usepackage{enumitem}

\geometry{margin=1in}
\setstretch{1.2}
\titleformat{\section}{\normalfont\Large\bfseries}{\thesection}{1em}{}
\titleformat{\subsection}{\normalfont\large\bfseries}{\thesubsection}{1em}{}
\pagenumbering{gobble}

\begin{document}

\begin{center}
    \LARGE \textbf{Unit 4: Data, Statistics, and Probability} \\[6pt]
    \Large \textbf{Topic 4: Basic and Compound Probability}
\end{center}

\vspace{1em}

\section*{Concept Summary}

\textbf{Probability} measures how likely an event is to occur:
\[
P(\text{event}) = \frac{\text{number of favorable outcomes}}{\text{total possible outcomes}}.
\]

Probabilities range from \(0\) (impossible) to \(1\) (certain).  
For independent events, probabilities multiply.  
For mutually exclusive events, probabilities add.

\subsection*{Basic Rules}
\begin{itemize}
  \item \(P(A') = 1 - P(A)\)
  \item \(P(A\text{ or }B) = P(A) + P(B) - P(A\text{ and }B)\)
  \item \(P(A\text{ and }B) = P(A)\cdot P(B)\) if \(A\) and \(B\) are independent
\end{itemize}

\section*{Core Skills}
\begin{itemize}
  \item Compute simple probabilities for dice, cards, and spinners.
  \item Distinguish between “or” (addition) and “and” (multiplication).
  \item Identify whether events are independent or dependent.
  \item Use complementary probability for “at least one” problems.
\end{itemize}

\section*{Example 1: Basic Probability}

A fair die is rolled.  
\(P(\text{even}) = \dfrac{3}{6} = \boxed{\tfrac{1}{2}}\).

\section*{Example 2: Independent Events}

Flip a coin and roll a die.  
\(P(\text{heads and 4}) = \tfrac{1}{2}\times\tfrac{1}{6}=\boxed{\tfrac{1}{12}}\).

\section*{Example 3: Dependent Events}

A box has 3 red and 2 blue balls. Draw 2 without replacement.  
\(P(\text{both red})=\tfrac{3}{5}\times\tfrac{2}{4}=\boxed{\tfrac{3}{10}}\).

\section*{Example 4: “At Least One”}

Two coin flips. \(P(\text{at least one head})=1-P(\text{no heads})=1-(\tfrac{1}{2})^2=\boxed{\tfrac{3}{4}}\).

\section*{Example 5: “Or” Probabilities}

In a standard deck, \(P(\text{heart or king})=\tfrac{13}{52}+\tfrac{4}{52}-\tfrac{1}{52}=\boxed{\tfrac{16}{52}=\tfrac{4}{13}}\).

\section*{Example 6: Compound Probability (Tree Diagram Idea)}

A coin is flipped twice. The possible outcomes are HH, HT, TH, TT.  
\(P(\text{exactly one head})=\tfrac{2}{4}=\boxed{\tfrac{1}{2}}\).

\section*{Key Takeaways}
\begin{itemize}
  \item Add probabilities for “or,” multiply for “and” (independent).
  \item Use complements for “at least” or “none” types.
  \item For dependent events, adjust denominators as items are removed.
  \item All probabilities must sum to 1 for the entire sample space.
\end{itemize}

\newpage

% ============================================================
% QUESTIONS — TOPIC 4: BASIC AND COMPOUND PROBABILITY
% ============================================================

\section*{Practice Questions: Basic and Compound Probability}

\subsection*{Part A: Basic Probability}
\begin{enumerate}
  \item A coin is flipped once. What is \(P(\text{heads})\)?
  \item A die is rolled. What is \(P(\text{rolling a 5})\)?
  \item A spinner has 8 equal sections labeled 1–8. What is \(P(\text{even number})\)?
  \item From a deck of 52 cards, what is \(P(\text{drawing a heart})\)?
  \item A bag contains 4 red, 3 blue, and 3 green marbles. What is \(P(\text{green})\)?
\end{enumerate}

\subsection*{Part B: “And” and “Or” Events}
\begin{enumerate}
  \setcounter{enumi}{5}
  \item A die is rolled twice. What is \(P(\text{rolling a 3 and then a 5})\)?
  \item A card is drawn from a deck. What is \(P(\text{heart or queen})\)?
  \item A coin is flipped twice. What is \(P(\text{two heads})\)?
  \item A coin is flipped twice. What is \(P(\text{head or tail on first flip})\)?
  \item A jar contains 5 red and 5 blue chips. Two are drawn without replacement. What is \(P(\text{both red})\)?
\end{enumerate}

\subsection*{Part C: Complement and “At Least” Problems}
\begin{enumerate}
  \setcounter{enumi}{10}
  \item A die is rolled once. What is \(P(\text{not rolling a 6})\)?
  \item Two coins are flipped. What is \(P(\text{at least one head})\)?
  \item A student guesses on a 4-option multiple-choice question. What is \(P(\text{wrong answer})\)?
  \item A bag contains 6 white and 4 black marbles. One is drawn. What is \(P(\text{not black})\)?
  \item A coin is flipped three times. What is \(P(\text{at least one tail})\)?
\end{enumerate}

\subsection*{Part D: Dependent and Independent Events}
\begin{enumerate}
  \setcounter{enumi}{15}
  \item Two cards are drawn from a deck \textbf{without replacement}. What is \(P(\text{both aces})\)?
  \item Two cards are drawn \textbf{with replacement}. What is \(P(\text{both aces})\)?
  \item A bag contains 3 red and 2 blue balls. Two are drawn without replacement. Find \(P(\text{red then blue})\).
  \item A coin is flipped and a die is rolled. What is \(P(\text{head and even number})\)?
  \item In a class, 60\% of students are female and 40\% male. If one student is selected, what is \(P(\text{female})\)?
\end{enumerate}

\subsection*{Part E: SAT-Style Applications}
\begin{enumerate}
  \setcounter{enumi}{20}
  \item A jar contains 10 marbles: 4 red, 3 blue, 3 green. Two marbles are drawn without replacement. What is \(P(\text{both blue})\)?
  \item A student randomly guesses on two true/false questions. What is \(P(\text{both correct})\)?
  \item A fair die is rolled twice. What is \(P(\text{sum of 7})\)?
  \item A computer password has two randomly chosen digits (0–9). What is \(P(\text{both digits are even})\)?
  \item The probability of rain today is 0.3 and tomorrow 0.4 (independent). What is \(P(\text{rains both days})\)?
\end{enumerate}

\newpage

% ============================================================
% SOLUTIONS — UNIT 4, TOPIC 4: BASIC AND COMPOUND PROBABILITY
% ============================================================

\section*{Answer Key and Solutions: Basic and Compound Probability}

\subsection*{Part A Solutions: Basic Probability}
\begin{enumerate}
  \item \(P(\text{heads})=\tfrac{1}{2}\).
  \item \(P(5)=\tfrac{1}{6}\).
  \item Even outcomes \(\{2,4,6,8\}\): \(4/8=\tfrac{1}{2}\).
  \item Hearts \(=13/52=\tfrac{1}{4}\).
  \item Green \(=3/(4+3+3)=\tfrac{3}{10}\).
\end{enumerate}

\subsection*{Part B Solutions: “And” and “Or”}
\begin{enumerate}
  \setcounter{enumi}{5}
  \item Independent: \(\tfrac{1}{6}\cdot\tfrac{1}{6}=\tfrac{1}{36}\).
  \item \(P(\heartsuit \text{ or Q})=\tfrac{13}{52}+\tfrac{4}{52}-\tfrac{1}{52}=\tfrac{16}{52}=\tfrac{4}{13}\).
  \item \((\tfrac{1}{2})^2=\tfrac{1}{4}\).
  \item First flip is head or tail with certainty: \(1\).
  \item Without replacement: \(\tfrac{5}{10}\cdot\tfrac{4}{9}=\tfrac{20}{90}=\tfrac{2}{9}\).
\end{enumerate}

\subsection*{Part C Solutions: Complement and “At Least”}
\begin{enumerate}
  \setcounter{enumi}{10}
  \item \(1-\tfrac{1}{6}=\tfrac{5}{6}\).
  \item \(1-(\tfrac{1}{2})^2=\tfrac{3}{4}\).
  \item \(1-\tfrac{1}{4}=\tfrac{3}{4}\).
  \item Not black \(=\) white \(= \tfrac{6}{10}=\tfrac{3}{5}\).
  \item \(1-(\tfrac{1}{2})^3=1-\tfrac{1}{8}=\tfrac{7}{8}\).
\end{enumerate}

\subsection*{Part D Solutions: Dependent and Independent}
\begin{enumerate}
  \setcounter{enumi}{15}
  \item Without replacement: \(\tfrac{4}{52}\cdot\tfrac{3}{51}=\tfrac{12}{2652}=\boxed{\tfrac{1}{221}}\).
  \item With replacement: \((\tfrac{4}{52})^2=(\tfrac{1}{13})^2=\tfrac{1}{169}\).
  \item \(\tfrac{3}{5}\cdot\tfrac{2}{4}=\tfrac{6}{20}=\tfrac{3}{10}\).
  \item Independent: \(\tfrac{1}{2}\cdot\tfrac{3}{6}=\tfrac{1}{4}\).
  \item \(0.60\).
\end{enumerate}

\subsection*{Part E Solutions: SAT-Style Applications}
\begin{enumerate}
  \setcounter{enumi}{20}
  \item \(\tfrac{3}{10}\cdot\tfrac{2}{9}=\tfrac{6}{90}=\tfrac{1}{15}\).
  \item \((\tfrac{1}{2})^2=\tfrac{1}{4}\).
  \item Favorable sums of 7: 6 outcomes of 36 \(\Rightarrow \tfrac{6}{36}=\tfrac{1}{6}\).
  \item Even digits \(\{0,2,4,6,8\}\): \((\tfrac{5}{10})^2=(\tfrac{1}{2})^2=\tfrac{1}{4}\).
  \item Independent: \(0.3\cdot0.4=0.12\).
\end{enumerate}


\end{document}
