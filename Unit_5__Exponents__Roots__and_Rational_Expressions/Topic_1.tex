% =========================
% SAT Resources Topic Template
% =========================
\documentclass[12pt]{article}

% ---------- PACKAGES ----------
\usepackage{amsmath, amssymb}
\usepackage{geometry}
\usepackage{setspace}
\usepackage{titlesec}
\usepackage{lmodern}
\usepackage{xcolor}
\usepackage{enumitem}
\usepackage{ifthen}

% ---------- PAGE SETUP ----------
\geometry{margin=1in}
\setstretch{1.2}
\titleformat{\section}{\normalfont\Large\bfseries}{\thesection}{1em}{}
\titleformat{\subsection}{\normalfont\large\bfseries}{\thesubsection}{1em}{}
\setlist[enumerate]{label=\textbf{(\arabic*)}, itemsep=4pt, topsep=4pt}
\setlist[itemize]{itemsep=2pt, topsep=4pt}
\pagenumbering{gobble}

% ---------- TOGGLES ----------
\newboolean{showsolutions}
\setboolean{showsolutions}{true}

% ---------- LIGHTWEIGHT MACROS ----------
\newcommand{\UnitTitle}{Unit 5: Exponents, Roots, and Rational Expressions}
\newcommand{\TopicTitle}{Topic 1: Laws of Exponents}

\newcommand{\TopicMeta}[3]{%
  \textcolor{gray}{\footnotesize \textbf{SAT Domain:} #1 \quad \textbf{Calculator:} #2 \quad \textbf{Progression:} #3}
}
\newcommand{\qtag}[1]{\textcolor{gray}{\footnotesize [#1]}}

\newenvironment{Example}[1]{%
  \subsection*{Example #1}%
}{\vspace{0.25em}}

% ---------- DOCUMENT ----------
\begin{document}

\begin{center}
  \LARGE \textbf{\UnitTitle}\\[6pt]
  \Large \textbf{\TopicTitle}\\[4pt]
\end{center}

\vspace{0.75em}

% =========================
% MINI-LESSON
% =========================
\section*{Concept Summary}
Exponents represent repeated multiplication of the same base. For any real number \(a\) and integer \(n > 0\),
\[
a^n = \underbrace{a \times a \times \cdots \times a}_{n \text{ times}}.
\]
The properties of exponents allow us to simplify and compare exponential expressions efficiently. These laws apply when the bases are the same or when the exponents follow specific relationships.

Each law stems from basic multiplication and division principles. Understanding *why* they work helps prevent common errors—especially when simplifying or solving equations that include exponential terms.

\section*{Core Skills}
\begin{itemize}
  \item Apply the product rule: \(a^m \cdot a^n = a^{m+n}\)
  \item Apply the quotient rule: \(\dfrac{a^m}{a^n} = a^{m-n}\)
  \item Apply the power rule: \((a^m)^n = a^{mn}\)
  \item Apply the power of a product rule: \((ab)^n = a^n b^n\)
  \item Apply the power of a quotient rule: \(\left(\dfrac{a}{b}\right)^n = \dfrac{a^n}{b^n}\)
\end{itemize}

\begin{Example}{1}
Simplify \(x^3 \cdot x^5\) \qtag{Core rule, product}
\[
x^3 \cdot x^5 = x^{3+5} = x^8
\]
\textbf{Check:} Multiplying \(x^3 = x \times x \times x\) by \(x^5\) gives eight total factors of \(x\). \(\checkmark\)

\textbf{Final Answer:} \(\boxed{x^8}\)
\end{Example}

\begin{Example}{2}
Simplify \(\dfrac{y^7}{y^2}\) \qtag{Quotient rule}
\[
\dfrac{y^7}{y^2} = y^{7-2} = y^5
\]
\textbf{Check:} Cancelling two factors of \(y\) from numerator and denominator leaves five. \(\checkmark\)

\textbf{Final Answer:} \(\boxed{y^5}\)
\end{Example}

\begin{Example}{3}
Simplify \((2a^3)^4\) \qtag{Power rule, power of product}
\[
(2a^3)^4 = 2^4 \cdot (a^3)^4 = 16a^{12}
\]
\textbf{Check:} Each factor of \(2a^3\) multiplies four times, so the exponents multiply. \(\checkmark\)

\textbf{Final Answer:} \(\boxed{16a^{12}}\)
\end{Example}

\begin{Example}{4}
Simplify \(\left(\dfrac{x^2y^3}{z}\right)^2\) \qtag{Power of quotient}
\[
\left(\dfrac{x^2y^3}{z}\right)^2 = \dfrac{(x^2)^2 (y^3)^2}{z^2} = \dfrac{x^4y^6}{z^2}
\]
\textbf{Final Answer:} \(\boxed{\dfrac{x^4y^6}{z^2}}\)
\end{Example}

\begin{Example}{5}
Simplify \(m^4n^3 \cdot m^2n^5\) \qtag{Multiple bases}
\[
m^4n^3 \cdot m^2n^5 = m^{4+2}n^{3+5} = m^6n^8
\]
\textbf{Final Answer:} \(\boxed{m^6n^8}\)
\end{Example}

\section*{Key Takeaways}
\begin{itemize}
  \item Combine exponents only when the bases match.
  \item Subtract exponents when dividing like bases.
  \item Multiply exponents when taking a power of a power.
  \item Distribute exponents across products or quotients, not sums.
  \item Always check that no bases equal zero when dividing. 
\end{itemize}

\newpage

% ============================================================
% QUESTIONS — UNIT 5, TOPIC 1: LAWS OF EXPONENTS
% ============================================================

\section*{Practice Questions: Laws of Exponents}

\subsection*{Part A: Core Skills (One-Step Simplifications)}
\begin{enumerate}
  \item Simplify \(x^3 \cdot x^4\)
  \item Simplify \(\dfrac{y^8}{y^3}\)
  \item Simplify \((a^2)^4\)
  \item Simplify \((2b)^3\)
  \item Simplify \(\left(\dfrac{p}{q}\right)^2\)
\end{enumerate}

\subsection*{Part B: Two-Step Simplifications}
\begin{enumerate}
  \setcounter{enumi}{5}
  \item Simplify \((m^2n^3)^2\)
  \item Simplify \(\dfrac{(x^5)^3}{x^2}\)
  \item Simplify \(\left(\dfrac{3a^2}{b}\right)^2\)
  \item Simplify \(4x^2 \cdot (2x^3)^2\)
  \item Simplify \((r^3s)^2 \cdot s^4\)
\end{enumerate}

\subsection*{Part C: Multiple Variables and Structure Recognition}
\begin{enumerate}
  \setcounter{enumi}{10}
  \item Simplify \(\dfrac{x^6y^3}{x^2y}\)
  \item Simplify \(\dfrac{(a^3b)^2}{a^2b^4}\)
  \item Simplify \((x^2y^3)^2 \cdot (x^4y)^3\)
  \item Simplify \(\dfrac{(2a^2b^3)^2}{(4ab)^3}\)
  \item Simplify \(\dfrac{(3x^2y^4)^2}{(9xy^2)^2}\)
\end{enumerate}

\subsection*{Part D: Parentheses, Fractions, and Edge Cases}
\begin{enumerate}
  \setcounter{enumi}{15}
  \item Simplify \(\dfrac{(x^2)^3}{x^5}\)
  \item Simplify \((ab^2)^0\)
  \item Simplify \((2x^3y^2)^1\)
  \item Simplify \((a^4b^{-2})^2\) \textit{(introduces negative exponents, preview for next topic)}
  \item Simplify \(\dfrac{(3x^2)^3}{(3x^4)^2}\)
\end{enumerate}

\subsection*{Part E: Word Problems and SAT-Style Applications}
\begin{enumerate}
  \setcounter{enumi}{20}
  \item The area of a square is \(s^2\). If each side is doubled, write an expression for the new area in terms of \(s\).
  \item The population of a city doubles every decade. If the current population is \(p\), write an expression for the population after 3 decades.
  \item The volume of a cube is given by \(V = s^3\). If each side is multiplied by \(k\), express the new volume in terms of \(V\) and \(k\).
  \item The intensity of light varies inversely with the square of the distance \(d\). If the distance doubles, how does the intensity change?
  \item The kinetic energy of an object is proportional to the square of its velocity. If the velocity is multiplied by 3, by what factor does the energy change?
\end{enumerate}

\newpage

% ============================================================
% SOLUTIONS — UNIT 5, TOPIC 1: LAWS OF EXPONENTS
% ============================================================

\section*{Answer Key and Solutions: Laws of Exponents}

\subsection*{Part A Solutions: Core Skills (One-Step Simplifications)}
\begin{enumerate}
  \item \(x^3 \cdot x^4 = x^{3+4} = \boxed{x^7}\)
  \item \(\dfrac{y^8}{y^3} = y^{8-3} = \boxed{y^5}\)
  \item \((a^2)^4 = a^{2 \cdot 4} = \boxed{a^8}\)
  \item \((2b)^3 = 2^3 b^3 = \boxed{8b^3}\)
  \item \(\left(\dfrac{p}{q}\right)^2 = \dfrac{p^2}{q^2} = \boxed{\dfrac{p^2}{q^2}}\)
\end{enumerate}

\subsection*{Part B Solutions: Two-Step Simplifications}
\begin{enumerate}
  \setcounter{enumi}{5}
  \item \((m^2 n^3)^2 = m^{2 \cdot 2} n^{3 \cdot 2} = \boxed{m^4 n^6}\)
  \item \(\dfrac{(x^5)^3}{x^2} = \dfrac{x^{15}}{x^2} = x^{15-2} = \boxed{x^{13}}\)
  \item \(\left(\dfrac{3a^2}{b}\right)^2 = \dfrac{(3a^2)^2}{b^2} = \dfrac{9a^4}{b^2} = \boxed{\dfrac{9a^4}{b^2}}\)
  \item \(4x^2 \cdot (2x^3)^2 = 4x^2 \cdot (4x^6) = 16x^8 = \boxed{16x^8}\)
  \item \((r^3 s)^2 \cdot s^4 = r^{6} s^{2} \cdot s^4 = r^6 s^{2+4} = \boxed{r^6 s^6}\)
\end{enumerate}

\subsection*{Part C Solutions: Multiple Variables and Structure Recognition}
\begin{enumerate}
  \setcounter{enumi}{10}
  \item \(\dfrac{x^6 y^3}{x^2 y} = x^{6-2} y^{3-1} = \boxed{x^4 y^2}\)
  \item \(\dfrac{(a^3 b)^2}{a^2 b^4} = \dfrac{a^6 b^2}{a^2 b^4} = a^{6-2} b^{2-4} = a^4 b^{-2} = \boxed{\dfrac{a^4}{b^2}}\)
  \item \((x^2 y^3)^2 \cdot (x^4 y)^3 = (x^4 y^6) \cdot (x^{12} y^3) = x^{4+12} y^{6+3} = \boxed{x^{16} y^9}\)
  \item \(\dfrac{(2a^2 b^3)^2}{(4ab)^3} = \dfrac{4a^4 b^6}{64a^3 b^3} = \dfrac{1}{16} a^{4-3} b^{6-3} = \boxed{\dfrac{ab^3}{16}}\)
  \item \(\dfrac{(3x^2 y^4)^2}{(9xy^2)^2} = \dfrac{9x^4 y^8}{81x^2 y^4} = \dfrac{1}{9} x^{4-2} y^{8-4} = \boxed{\dfrac{x^2 y^4}{9}}\)
\end{enumerate}

\subsection*{Part D Solutions: Parentheses, Fractions, and Edge Cases}
\begin{enumerate}
  \setcounter{enumi}{15}
  \item \(\dfrac{(x^2)^3}{x^5} = \dfrac{x^6}{x^5} = x^{6-5} = \boxed{x}\)
  \item \((ab^2)^0 = \boxed{1}\) \quad (any nonzero base to the zero power is 1)
  \item \((2x^3 y^2)^1 = \boxed{2x^3 y^2}\)
  \item \((a^4 b^{-2})^2 = a^{8} b^{-4} = \boxed{\dfrac{a^8}{b^4}}\)
  \item \(\dfrac{(3x^2)^3}{(3x^4)^2} = \dfrac{27x^6}{9x^8} = 3x^{6-8} = 3x^{-2} = \boxed{\dfrac{3}{x^2}}\)
\end{enumerate}

\subsection*{Part E Solutions: Word Problems and SAT-Style Applications}
\begin{enumerate}
  \setcounter{enumi}{20}
  \item Side doubled: new side is \(2s\). New area \(=(2s)^2 = 4s^2 = \boxed{4s^2}\)
  \item Population doubles every decade. After 3 decades: \(p \cdot 2^3 = \boxed{8p}\)
  \item Original \(V = s^3\). New side is \(ks\), so new volume \(=(ks)^3 = k^3 s^3 = \boxed{k^3 V}\)
  \item Intensity \(\propto \dfrac{1}{d^2}\). If distance becomes \(2d\), intensity \(\propto \dfrac{1}{(2d)^2} = \dfrac{1}{4d^2}\). New intensity is \(\boxed{\tfrac{1}{4}\text{ of the original}}\)
  \item Energy \(\propto v^2\). If velocity becomes \(3v\), energy \(\propto (3v)^2 = 9v^2\). So energy is multiplied by \(\boxed{9}\)
\end{enumerate}




\end{document}
