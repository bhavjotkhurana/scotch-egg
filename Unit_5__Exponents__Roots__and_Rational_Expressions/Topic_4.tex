\documentclass[12pt]{article}

\usepackage{amsmath, amssymb}
\usepackage{geometry}
\usepackage{setspace}
\usepackage{titlesec}
\usepackage{lmodern}
\usepackage{xcolor}
\usepackage{enumitem}

\geometry{margin=1in}
\setstretch{1.2}
\titleformat{\section}{\normalfont\Large\bfseries}{\thesection}{1em}{}
\titleformat{\subsection}{\normalfont\large\bfseries}{\thesubsection}{1em}{}
\pagenumbering{gobble}

\begin{document}

\begin{center}
    \LARGE \textbf{Unit 5: Exponents, Roots, and Rational Expressions} \\[6pt]
    \Large \textbf{Topic 4: Domain Restrictions and Undefined Values}
\end{center}

\vspace{1em}

\section*{Concept Summary}

A \textbf{domain} is the set of all input values (\(x\)) for which an expression or function is defined.  
When working with rational expressions, radicals, or denominators, certain values must be excluded from the domain because they make the expression undefined.

1. **Division by zero is undefined:**  
   Any value that makes a denominator equal to zero must be excluded.
   \[
   \frac{1}{x-2} \text{ is undefined at } x = 2.
   \]

2. **Even roots of negative numbers are undefined for real values:**  
   For example, \(\sqrt{x-3}\) is defined only when \(x - 3 \ge 0\).

3. **Numerators do not affect the domain**, only denominators and even roots do.

The SAT often tests whether you can identify these restrictions when simplifying or interpreting rational or radical expressions.

\section*{Core Skills}
\begin{itemize}
  \item Identify values that make denominators zero.
  \item Determine where expressions under even roots are nonnegative.
  \item Express domain restrictions in set notation or interval notation.
  \item Retain restrictions even after simplification.
  \item Check all variable conditions before evaluating or graphing.
\end{itemize}

\section*{Example 1: Denominator Restriction}

Find the domain of \(\dfrac{3x + 1}{x - 4}\).

\[
x - 4 = 0 \Rightarrow x = 4
\]
The expression is undefined at \(x = 4\).

\textbf{Domain:} all real numbers except 4  
\[
\boxed{x \ne 4}
\]

\section*{Example 2: Multiple Denominators}

Find the domain of \(\dfrac{x + 2}{x^2 - 9}\).

\[
x^2 - 9 = (x - 3)(x + 3)
\]
Undefined when \(x = 3\) or \(x = -3\).

\textbf{Domain:} all real \(x\) except \(x = \pm 3\).  
\[
\boxed{x \ne 3, -3}
\]

\section*{Example 3: Even Root Restriction}

Find the domain of \(\sqrt{x - 5}\).

\[
x - 5 \ge 0 \Rightarrow x \ge 5
\]

\textbf{Domain:} \([5, \infty)\)

\section*{Example 4: Radical in a Denominator}

Find the domain of \(\dfrac{1}{\sqrt{x - 2}}\).

\[
x - 2 > 0 \Rightarrow x > 2
\]
The denominator cannot be zero or negative.

\textbf{Domain:} \((2, \infty)\)

\section*{Example 5: Rational Expression After Simplification}

Simplify and find the domain of \(\dfrac{x^2 - 9}{x - 3}\).

\[
\dfrac{(x - 3)(x + 3)}{x - 3} = x + 3
\]
Even though the simplified form is \(x + 3\), the restriction \(x \ne 3\) remains because the original denominator was zero at \(x = 3\).

\textbf{Final Answer:} \(\boxed{x + 3, \; x \ne 3}\)

\section*{Example 6: Combined Root and Denominator}

Find the domain of \(\dfrac{\sqrt{x + 4}}{x - 2}\).

\[
x + 4 \ge 0 \Rightarrow x \ge -4, \quad x - 2 \ne 0 \Rightarrow x \ne 2
\]
\textbf{Domain:} all \(x \ge -4\) except \(x = 2.\)

\[
\boxed{x \ge -4, \; x \ne 2}
\]

\section*{Key Takeaways}
\begin{itemize}
  \item Denominators cannot be zero; exclude those \(x\)-values.
  \item Even roots require nonnegative radicands (\(x \ge 0\)).
  \item Simplifying does not remove restrictions from the original expression.
  \item Odd roots and numerators impose no restrictions.
  \item Always check both radicals and denominators together for valid domains.
\end{itemize}

\newpage

% ============================================================
% QUESTIONS — UNIT 5, TOPIC 4: DOMAIN RESTRICTIONS AND UNDEFINED VALUES
% ============================================================

\section*{Practice Questions: Domain Restrictions and Undefined Values}

\subsection*{Part A: Identifying Denominator Restrictions}
\begin{enumerate}
  \item State the value of \(x\) that makes \(\dfrac{1}{x - 5}\) undefined.
  \item Find the domain of \(\dfrac{2x + 3}{x + 4}\).
  \item Determine where \(\dfrac{x^2 + 1}{x^2 - 16}\) is undefined.
  \item Find all \(x\)-values excluded from \(\dfrac{5x}{x(x - 3)}\).
  \item Identify the restriction in \(\dfrac{7x + 2}{3x - 6}\).
\end{enumerate}

\subsection*{Part B: Even Root and Denominator Conditions}
\begin{enumerate}
  \setcounter{enumi}{5}
  \item Find the domain of \(\sqrt{x - 8}\).
  \item Determine where \(\sqrt{2x + 6}\) is defined.
  \item Find the domain of \(\dfrac{1}{\sqrt{x}}\).
  \item Find the domain of \(\dfrac{1}{\sqrt{x - 3}}\).
  \item Determine all real \(x\) that satisfy \(\sqrt{x^2 - 9}\) being real.
\end{enumerate}

\subsection*{Part C: Mixed Rational and Radical Expressions}
\begin{enumerate}
  \setcounter{enumi}{10}
  \item Find the domain of \(\dfrac{\sqrt{x + 5}}{x}\).
  \item Find the domain of \(\dfrac{\sqrt{x - 1}}{x - 4}\).
  \item Determine the domain of \(\dfrac{1}{x^2 - 9}\).
  \item Find the domain of \(\dfrac{\sqrt{x + 2}}{x + 7}\).
  \item Find the domain of \(\dfrac{\sqrt{x^2 - 4}}{x - 2}\).
\end{enumerate}

\subsection*{Part D: Simplified Expressions with Hidden Restrictions}
\begin{enumerate}
  \setcounter{enumi}{15}
  \item Simplify \(\dfrac{x^2 - 9}{x - 3}\) and state the domain.
  \item Simplify \(\dfrac{x^2 - 4x}{x}\) and state the domain.
  \item Simplify \(\dfrac{x^2 + 3x}{x}\) and state the domain.
  \item Simplify \(\dfrac{x^2 - 16}{x + 4}\) and state the domain.
  \item Simplify \(\dfrac{x^3 - 8}{x - 2}\) and state the domain.
\end{enumerate}

\subsection*{Part E: Word Problems and SAT-Style Applications}
\begin{enumerate}
  \setcounter{enumi}{20}
  \item A function is defined by \(f(x) = \dfrac{3x + 1}{x - 2}\). What is the domain of \(f(x)\)?
  \item The period of a pendulum is \(T = 2\pi \sqrt{\dfrac{L}{g}}\). For which values of \(L\) is \(T\) defined?
  \item The formula for current in a circuit is \(I = \dfrac{V}{R}\). What restriction must be placed on \(R\)?
  \item A function is defined by \(g(x) = \dfrac{1}{\sqrt{x - 4}}\). Find the domain of \(g(x)\).
  \item The height of a projectile is modeled by \(h(t) = -16t^2 + 64t + 80\). Does this function have any domain restrictions in real numbers?
\end{enumerate}

\newpage

% ============================================================
% SOLUTIONS — UNIT 5, TOPIC 4: DOMAIN RESTRICTIONS AND UNDEFINED VALUES
% ============================================================

\section*{Answer Key and Solutions: Domain Restrictions and Undefined Values}

\subsection*{Part A Solutions: Identifying Denominator Restrictions}
\begin{enumerate}
  \item \(\dfrac{1}{x - 5}\) is undefined when the denominator is 0.  
  \(x - 5 = 0 \Rightarrow x = 5.\)  
  \(\boxed{x \ne 5}\)

  \item Domain of \(\dfrac{2x + 3}{x + 4}\): denominator cannot be 0.  
  \(x + 4 = 0 \Rightarrow x = -4.\)  
  \(\boxed{x \ne -4}\)

  \item \(\dfrac{x^2 + 1}{x^2 - 16}\) is undefined when \(x^2 - 16 = 0\).  
  \(x^2 - 16 = (x - 4)(x + 4).\)  
  So \(x = 4\) or \(x = -4.\)  
  \(\boxed{x \ne 4,\; x \ne -4}\)

  \item \(\dfrac{5x}{x(x - 3)}\): denominator is \(x(x-3)\).  
  Denominator is 0 when \(x = 0\) or \(x = 3.\)  
  \(\boxed{x \ne 0,\; x \ne 3}\)

  \item \(\dfrac{7x + 2}{3x - 6}\): denominator cannot be 0.  
  \(3x - 6 = 0 \Rightarrow x = 2.\)  
  \(\boxed{x \ne 2}\)
\end{enumerate}

\subsection*{Part B Solutions: Even Root and Denominator Conditions}
\begin{enumerate}
  \setcounter{enumi}{5}
  \item \(\sqrt{x - 8}\) is defined when the radicand is \(\ge 0\).  
  \(x - 8 \ge 0 \Rightarrow x \ge 8.\)  
  \(\boxed{x \ge 8}\)

  \item \(\sqrt{2x + 6}\) is defined when \(2x + 6 \ge 0.\)  
  \(2x + 6 \ge 0 \Rightarrow 2x \ge -6 \Rightarrow x \ge -3.\)  
  \(\boxed{x \ge -3}\)

  \item \(\dfrac{1}{\sqrt{x}}\):  
  For \(\sqrt{x}\), need \(x \ge 0\).  
  Also, denominator \(\sqrt{x} \ne 0\), so \(x > 0.\)  
  \(\boxed{x > 0}\)

  \item \(\dfrac{1}{\sqrt{x - 3}}\):  
  Need \(x - 3 \ge 0 \Rightarrow x \ge 3.\)  
  Also denominator cannot be 0, so \(x > 3.\)  
  \(\boxed{x > 3}\)

  \item \(\sqrt{x^2 - 9}\) is defined when \(x^2 - 9 \ge 0.\)  
  \(x^2 - 9 \ge 0 \Rightarrow (x - 3)(x + 3) \ge 0.\)  
  This holds for \(x \le -3\) or \(x \ge 3.\)  
  \(\boxed{x \le -3 \text{ or } x \ge 3}\)
\end{enumerate}

\subsection*{Part C Solutions: Mixed Rational and Radical Expressions}
\begin{enumerate}
  \setcounter{enumi}{10}
  \item \(\dfrac{\sqrt{x + 5}}{x}\):  
  For \(\sqrt{x+5}\), need \(x + 5 \ge 0 \Rightarrow x \ge -5.\)  
  Also denominator \(x \ne 0.\)  
  \(\boxed{x \ge -5,\; x \ne 0}\)

  \item \(\dfrac{\sqrt{x - 1}}{x - 4}\):  
  For \(\sqrt{x-1}\), need \(x - 1 \ge 0 \Rightarrow x \ge 1.\)  
  Also denominator \(x - 4 \ne 0 \Rightarrow x \ne 4.\)  
  \(\boxed{x \ge 1,\; x \ne 4}\)

  \item \(\dfrac{1}{x^2 - 9}\):  
  Denominator cannot be 0.  
  \(x^2 - 9 = 0 \Rightarrow x = \pm 3.\)  
  \(\boxed{x \ne 3,\; x \ne -3}\)

  \item \(\dfrac{\sqrt{x + 2}}{x + 7}\):  
  Need \(x + 2 \ge 0 \Rightarrow x \ge -2.\)  
  Also \(x + 7 \ne 0 \Rightarrow x \ne -7.\)  
  Since \(-7 < -2\), both matter.  
  \(\boxed{x \ge -2,\; x \ne -7}\)

  \item \(\dfrac{\sqrt{x^2 - 4}}{x - 2}\):  
  For \(\sqrt{x^2 - 4}\), need \(x^2 - 4 \ge 0 \Rightarrow (x-2)(x+2) \ge 0.\)  
  That holds when \(x \le -2\) or \(x \ge 2.\)  
  Also denominator \(x - 2 \ne 0 \Rightarrow x \ne 2.\)  
  Combine: \(x \le -2\) or \(x > 2.\)  
  \(\boxed{x \le -2 \text{ or } x > 2}\)
\end{enumerate}

\subsection*{Part D Solutions: Simplified Expressions with Hidden Restrictions}
\begin{enumerate}
  \setcounter{enumi}{15}
  \item \(\dfrac{x^2 - 9}{x - 3}
  = \dfrac{(x-3)(x+3)}{x-3}
  = x + 3.\)
  
  Domain restriction comes from the original denominator \(x - 3 \ne 0\).  
  \(\boxed{x + 3,\; x \ne 3}\)

  \item \(\dfrac{x^2 - 4x}{x}
  = \dfrac{x(x - 4)}{x}
  = x - 4.\)
  
  Original denominator \(x \ne 0.\)  
  \(\boxed{x - 4,\; x \ne 0}\)

  \item \(\dfrac{x^2 + 3x}{x}
  = \dfrac{x(x + 3)}{x}
  = x + 3.\)
  
  Original denominator \(x \ne 0.\)  
  \(\boxed{x + 3,\; x \ne 0}\)

  \item \(\dfrac{x^2 - 16}{x + 4}
  = \dfrac{(x - 4)(x + 4)}{x + 4}
  = x - 4.\)
  
  Original denominator \(x + 4 \ne 0 \Rightarrow x \ne -4.\)  
  \(\boxed{x - 4,\; x \ne -4}\)

  \item \(\dfrac{x^3 - 8}{x - 2}\).  
  Factor numerator as difference of cubes:  
  \(x^3 - 8 = (x - 2)(x^2 + 2x + 4).\)

  \[
  \dfrac{(x - 2)(x^2 + 2x + 4)}{x - 2} = x^2 + 2x + 4.
  \]

  Original denominator requires \(x \ne 2.\)  
  \(\boxed{x^2 + 2x + 4,\; x \ne 2}\)
\end{enumerate}

\subsection*{Part E Solutions: Word Problems and SAT-Style Applications}
\begin{enumerate}
  \setcounter{enumi}{20}
  \item \(f(x) = \dfrac{3x + 1}{x - 2}\).  
  Denominator cannot be 0.  
  \(x - 2 = 0 \Rightarrow x = 2.\)  
  \(\boxed{x \ne 2}\)

  \item \(T = 2\pi \sqrt{\dfrac{L}{g}}\).  
  Inside the square root, \(\dfrac{L}{g}\) must be nonnegative.  
  \(L \ge 0.\)  
  Physically length cannot be 0 or negative, but mathematically \(L = 0\) gives \(T = 0\).  
  \(\boxed{L \ge 0}\)

  \item \(I = \dfrac{V}{R}\).  
  Denominator cannot be 0.  
  \(\boxed{R \ne 0}\)

  \item \(g(x) = \dfrac{1}{\sqrt{x - 4}}\).  
  Need \(x - 4 > 0\) because the radicand must be positive and cannot be 0 in the denominator.  
  \(x > 4.\)  
  \(\boxed{x > 4}\)

  \item \(h(t) = -16t^2 + 64t + 80\).  
  This is a polynomial. Polynomials are defined for all real inputs.  
  \(\boxed{\text{All real } t}\)
\end{enumerate}



\end{document}
