\documentclass[12pt]{article}

\usepackage{amsmath, amssymb}
\usepackage{geometry}
\usepackage{setspace}
\usepackage{titlesec}
\usepackage{lmodern}
\usepackage{xcolor}
\usepackage{enumitem}

\geometry{margin=1in}
\setstretch{1.2}
\titleformat{\section}{\normalfont\Large\bfseries}{\thesection}{1em}{}
\titleformat{\subsection}{\normalfont\large\bfseries}{\thesubsection}{1em}{}
\pagenumbering{gobble}

\begin{document}

\begin{center}
    \LARGE \textbf{Unit 5: Exponents, Roots, and Rational Expressions} \\[6pt]
    \Large \textbf{Topic 2: Negative and Fractional Exponents}
\end{center}

\vspace{1em}

\section*{Concept Summary}

Exponents can be extended beyond positive integers.  
A \textbf{negative exponent} represents the reciprocal of a base raised to a positive power:
\[
a^{-n} = \frac{1}{a^n}, \quad a \ne 0.
\]
A \textbf{fractional exponent} represents a root:
\[
a^{\frac{1}{n}} = \sqrt[n]{a}, \quad a^{\frac{m}{n}} = \sqrt[n]{a^m} = (\sqrt[n]{a})^m.
\]
These definitions make the laws of exponents consistent for all rational exponents.

Understanding why these work helps connect exponent and root operations.  
For example, \(a^{\frac{1}{2}}\) must mean “the number that, when squared, gives \(a\)” so that \(a^{\frac{1}{2}\cdot 2} = a^1\).  
Similarly, \(a^{-n}\) keeps the multiplication rule valid because \(a^m \cdot a^{-m} = a^{m-m} = a^0 = 1.\)

\section*{Core Skills}
\begin{itemize}
  \item Rewrite negative exponents as reciprocals.
  \item Rewrite fractional exponents as roots.
  \item Simplify expressions combining integer, negative, and fractional exponents.
  \item Convert between radical and exponential forms.
  \item Apply exponent laws consistently across all exponent types.
\end{itemize}

\section*{Example 1: Negative Exponent as Reciprocal}

Simplify \(x^{-3}\).

\[
x^{-3} = \frac{1}{x^3}
\]

\textbf{Interpretation:} A negative exponent inverts the base.  
\textbf{Final Answer:} \(\boxed{\dfrac{1}{x^3}}\)

\section*{Example 2: Fractional Exponent as Root}

Simplify \(16^{\frac{1}{2}}\).

\[
16^{\frac{1}{2}} = \sqrt{16} = 4
\]

\textbf{Interpretation:} The denominator 2 indicates a square root.  
\textbf{Final Answer:} \(\boxed{4}\)

\section*{Example 3: Mixed Fractional Exponent}

Simplify \(27^{\frac{2}{3}}\).

\[
27^{\frac{2}{3}} = (\sqrt[3]{27})^2 = 3^2 = 9
\]

\textbf{Interpretation:} Cube root first, then square.  
\textbf{Final Answer:} \(\boxed{9}\)

\section*{Example 4: Negative and Fractional Combined}

Simplify \(8^{-\frac{2}{3}}\).

\[
8^{-\frac{2}{3}} = \frac{1}{8^{\frac{2}{3}}} = \frac{1}{(\sqrt[3]{8})^2} = \frac{1}{4}
\]

\textbf{Final Answer:} \(\boxed{\dfrac{1}{4}}\)

\section*{Example 5: Algebraic Expression with Mixed Exponents}

Simplify \(\dfrac{x^{\frac{3}{2}}}{x^{\frac{1}{2}}}\).

\[
x^{\frac{3}{2} - \frac{1}{2}} = x^{1} = x
\]

\textbf{Final Answer:} \(\boxed{x}\)

\section*{Example 6: Converting Between Radical and Exponential Form}

Rewrite \(\sqrt[4]{x^3}\) using exponents.

\[
\sqrt[4]{x^3} = x^{\frac{3}{4}}
\]

\textbf{Final Answer:} \(\boxed{x^{\frac{3}{4}}}\)

\section*{Key Takeaways}
\begin{itemize}
  \item \(a^{-n} = \frac{1}{a^n}\) and \(a^{\frac{1}{n}} = \sqrt[n]{a}\).
  \item Exponent rules (\(a^m \cdot a^n = a^{m+n}\), etc.) hold for all rational exponents.
  \item Negative means “reciprocal”; fractional means “root.”
  \item Simplify by rewriting all terms with positive exponents before combining.
  \item When in doubt, rewrite radicals as fractional powers for consistency.
\end{itemize}

\newpage

% ============================================================
% QUESTIONS — UNIT 5, TOPIC 2: NEGATIVE AND FRACTIONAL EXPONENTS
% ============================================================

\section*{Practice Questions: Negative and Fractional Exponents}

\subsection*{Part A: Core Skills (One-Step Simplifications)}
\begin{enumerate}
  \item Simplify \(x^{-4}\)
  \item Simplify \(\dfrac{1}{a^{-3}}\)
  \item Simplify \(25^{\frac{1}{2}}\)
  \item Simplify \(8^{\frac{1}{3}}\)
  \item Simplify \(\left(\dfrac{1}{16}\right)^{-\frac{1}{2}}\)
\end{enumerate}

\subsection*{Part B: Two-Step Simplifications}
\begin{enumerate}
  \setcounter{enumi}{5}
  \item Simplify \(27^{\frac{2}{3}}\)
  \item Simplify \(81^{-\frac{3}{4}}\)
  \item Simplify \(\dfrac{a^{\frac{5}{2}}}{a^{\frac{3}{2}}}\)
  \item Simplify \((x^{-2})^{\frac{3}{2}}\)
  \item Simplify \(\dfrac{1}{b^{-\frac{1}{3}}}\)
\end{enumerate}

\subsection*{Part C: Mixed Variables and Exponent Operations}
\begin{enumerate}
  \setcounter{enumi}{10}
  \item Simplify \(a^{-\frac{3}{2}}b^{\frac{1}{2}}\)
  \item Simplify \(\dfrac{x^{\frac{3}{4}}}{x^{-\frac{1}{2}}}\)
  \item Simplify \(\left(\dfrac{y^{-2}}{y^3}\right)^{\frac{1}{2}}\)
  \item Simplify \((m^{\frac{2}{3}}n^{-\frac{1}{3}})^3\)
  \item Simplify \(\dfrac{(a^{-1}b^{2})^{\frac{1}{2}}}{b^{-\frac{1}{2}}}\)
\end{enumerate}

\subsection*{Part D: Parentheses, Fractions, and Edge Cases}
\begin{enumerate}
  \setcounter{enumi}{15}
  \item Simplify \(\left(\dfrac{1}{x^2}\right)^{-\frac{1}{2}}\)
  \item Simplify \((a^{-3}b^2)^0\)
  \item Simplify \((4x^{-1})^{\frac{1}{2}}\)
  \item Simplify \(\left(\dfrac{p^{-2}}{q}\right)^{-\frac{1}{2}}\)
  \item Simplify \((9a^{-4})^{-\frac{1}{2}}\)
\end{enumerate}

\subsection*{Part E: Word Problems and SAT-Style Applications}
\begin{enumerate}
  \setcounter{enumi}{20}
  \item The side length of a cube is \(s\). Its surface area is proportional to \(s^2\) and volume to \(s^3\). Write the ratio of surface area to volume as a single expression using exponents.
  \item The intensity of light is proportional to \(d^{-2}\), where \(d\) is the distance. If the distance is tripled, by what factor does intensity change?
  \item A bacteria population grows according to \(P = P_0 \cdot 2^{t/3}\). What is the meaning of the exponent \(\frac{t}{3}\)?
  \item The time for a planet to orbit the sun is proportional to \(r^{\frac{3}{2}}\), where \(r\) is its average distance. If \(r\) is doubled, how does the orbital time change?
  \item The volume of a shape is \(V = kx^{-\frac{1}{2}}\). What happens to \(V\) when \(x\) becomes 4 times larger?
\end{enumerate}

\newpage

% ============================================================
% SOLUTIONS — UNIT 5, TOPIC 2: NEGATIVE AND FRACTIONAL EXPONENTS
% ============================================================

\section*{Answer Key and Solutions: Negative and Fractional Exponents}

\subsection*{Part A Solutions: Core Skills (One-Step Simplifications)}
\begin{enumerate}
  \item \(x^{-4} = \dfrac{1}{x^4}\). A negative exponent means take the reciprocal. \(\boxed{\dfrac{1}{x^4}}\)

  \item \(\dfrac{1}{a^{-3}} = \dfrac{1}{\frac{1}{a^3}} = a^3\). \(\boxed{a^3}\)

  \item \(25^{\frac{1}{2}} = \sqrt{25} = \boxed{5}\)

  \item \(8^{\frac{1}{3}} = \sqrt[3]{8} = 2\). \(\boxed{2}\)

  \item \(\left(\dfrac{1}{16}\right)^{-\frac{1}{2}} = \dfrac{1}{\left(\dfrac{1}{16}\right)^{\frac{1}{2}}}
  = \dfrac{1}{\dfrac{1}{4}} = \boxed{4}\)
\end{enumerate}

\subsection*{Part B Solutions: Two-Step Simplifications}
\begin{enumerate}
  \setcounter{enumi}{5}
  \item \(27^{\frac{2}{3}} = \left(\sqrt[3]{27}\right)^2 = 3^2 = \boxed{9}\)

  \item \(81^{-\frac{3}{4}} = \dfrac{1}{81^{\frac{3}{4}}}
  = \dfrac{1}{\left(81^{\frac{1}{4}}\right)^3}
  = \dfrac{1}{3^3}
  = \boxed{\dfrac{1}{27}}\)

  \item \(\dfrac{a^{\frac{5}{2}}}{a^{\frac{3}{2}}}
  = a^{\frac{5}{2} - \frac{3}{2}}
  = a^{\frac{2}{2}}
  = a^1
  = \boxed{a}\)

  \item \((x^{-2})^{\frac{3}{2}}
  = x^{-2 \cdot \frac{3}{2}}
  = x^{-3}
  = \dfrac{1}{x^3}
  = \boxed{\dfrac{1}{x^3}}\)

  \item \(\dfrac{1}{b^{-\frac{1}{3}}}
  = \dfrac{1}{\frac{1}{b^{\frac{1}{3}}}}
  = b^{\frac{1}{3}}
  = \boxed{b^{\frac{1}{3}}}\)
\end{enumerate}

\subsection*{Part C Solutions: Mixed Variables and Exponent Operations}
\begin{enumerate}
  \setcounter{enumi}{10}
  \item \(a^{-\frac{3}{2}} b^{\frac{1}{2}}
  = \dfrac{b^{\frac{1}{2}}}{a^{\frac{3}{2}}}
  = \boxed{\dfrac{b^{1/2}}{a^{3/2}}}\)

  \item \(\dfrac{x^{\frac{3}{4}}}{x^{-\frac{1}{2}}}
  = x^{\frac{3}{4} - \left(-\frac{1}{2}\right)}
  = x^{\frac{3}{4} + \frac{1}{2}}
  = x^{\frac{3}{4} + \frac{2}{4}}
  = x^{\frac{5}{4}}
  = \boxed{x^{5/4}}\)

  \item \(\left(\dfrac{y^{-2}}{y^3}\right)^{\frac{1}{2}}
  = \left(y^{-2-3}\right)^{\frac{1}{2}}
  = (y^{-5})^{\frac{1}{2}}
  = y^{-\frac{5}{2}}
  = \dfrac{1}{y^{5/2}}
  = \boxed{\dfrac{1}{y^{5/2}}}\)

  \item \((m^{\frac{2}{3}} n^{-\frac{1}{3}})^3
  = m^{\frac{2}{3} \cdot 3} n^{-\frac{1}{3} \cdot 3}
  = m^{2} n^{-1}
  = \dfrac{m^2}{n}
  = \boxed{\dfrac{m^2}{n}}\)

  \item \(\dfrac{(a^{-1} b^{2})^{\frac{1}{2}}}{b^{-\frac{1}{2}}}
  = \dfrac{a^{-\frac{1}{2}} b^{1}}{b^{-\frac{1}{2}}}
  = a^{-\frac{1}{2}} b^{1 - \left(-\frac{1}{2}\right)}
  = a^{-\frac{1}{2}} b^{\frac{3}{2}}
  = \dfrac{b^{3/2}}{a^{1/2}}
  = \boxed{\dfrac{b^{3/2}}{a^{1/2}}}\)
\end{enumerate}

\subsection*{Part D Solutions: Parentheses, Fractions, and Edge Cases}
\begin{enumerate}
  \setcounter{enumi}{15}
  \item \(\left(\dfrac{1}{x^2}\right)^{-\frac{1}{2}}
  = (x^{-2})^{-\frac{1}{2}}
  = x^{-2 \cdot \left(-\frac{1}{2}\right)}
  = x^{1}
  = \boxed{x}\)

  \item \((a^{-3} b^2)^0 = \boxed{1}\)  
  Any nonzero base to the zero power is 1.

  \item \((4x^{-1})^{\frac{1}{2}}
  = 4^{\frac{1}{2}} (x^{-1})^{\frac{1}{2}}
  = 2 x^{-\frac{1}{2}}
  = \dfrac{2}{x^{1/2}}
  = \boxed{\dfrac{2}{\sqrt{x}}}\)

  \item \(\left(\dfrac{p^{-2}}{q}\right)^{-\frac{1}{2}}
  = (p^{-2} q^{-1})^{-\frac{1}{2}}
  = p^{-2 \cdot \left(-\frac{1}{2}\right)} q^{-1 \cdot \left(-\frac{1}{2}\right)}
  = p^{1} q^{\frac{1}{2}}
  = p \sqrt{q}
  = \boxed{p\sqrt{q}}\)

  \item \((9a^{-4})^{-\frac{1}{2}}
  = 9^{-\frac{1}{2}} a^{-4 \cdot \left(-\frac{1}{2}\right)}
  = 9^{-\frac{1}{2}} a^{2}
  = \dfrac{1}{\sqrt{9}} a^{2}
  = \dfrac{1}{3} a^{2}
  = \boxed{\dfrac{a^2}{3}}\)
\end{enumerate}

\subsection*{Part E Solutions: Word Problems and SAT-Style Applications}
\begin{enumerate}
  \setcounter{enumi}{20}
  \item Surface area \(\propto s^2\), volume \(\propto s^3\).  
  Ratio \(= \dfrac{s^2}{s^3} = s^{-1} = \boxed{\dfrac{1}{s}}\)

  \item Intensity \(\propto d^{-2}\).  
  If \(d\) becomes \(3d\): \((3d)^{-2} = 3^{-2} d^{-2} = \dfrac{1}{9} d^{-2}\).  
  So intensity becomes \(\boxed{\dfrac{1}{9}\text{ of the original}}\)

  \item \(P = P_0 \cdot 2^{t/3}\).  
  The exponent \(\frac{t}{3}\) means the population doubles once every 3 units of time.  
  After time \(t\), there have been \(\frac{t}{3}\) doublings.  
  \(\boxed{\text{Doubles every 3 time units}}\)

  \item Time \(\propto r^{\frac{3}{2}}\).  
  If \(r\) becomes \(2r\): \((2r)^{\frac{3}{2}} = 2^{\frac{3}{2}} r^{\frac{3}{2}}\).  
  \(2^{\frac{3}{2}} = 2 \sqrt{2}\).  
  So orbital time is multiplied by \(\boxed{2\sqrt{2}}\)

  \item \(V = kx^{-\frac{1}{2}}\).  
  If \(x\) becomes \(4x\):  
  \(V_{\text{new}} = k(4x)^{-\frac{1}{2}} = k \cdot 4^{-\frac{1}{2}} \cdot x^{-\frac{1}{2}}
  = k \cdot \dfrac{1}{2} \cdot x^{-\frac{1}{2}}
  = \dfrac{1}{2} (kx^{-\frac{1}{2}})\).  
  So volume is \(\boxed{\text{half as large}}\)
\end{enumerate}


\end{document}
