\documentclass[12pt]{article}

\usepackage{amsmath, amssymb}
\usepackage{geometry}
\usepackage{setspace}
\usepackage{titlesec}
\usepackage{lmodern}
\usepackage{xcolor}
\usepackage{enumitem}

\geometry{margin=1in}
\setstretch{1.2}
\titleformat{\section}{\normalfont\Large\bfseries}{\thesection}{1em}{}
\titleformat{\subsection}{\normalfont\large\bfseries}{\thesubsection}{1em}{}
\pagenumbering{gobble}

\begin{document}

\begin{center}
    \LARGE \textbf{Unit 5: Exponents, Roots, and Rational Expressions} \\[6pt]
    \Large \textbf{Topic 3: Simplifying Rational Expressions}
\end{center}

\vspace{1em}

\section*{Concept Summary}

A \textbf{rational expression} is a fraction in which the numerator and denominator are polynomials.  
Simplifying a rational expression follows the same logic as simplifying numerical fractions — divide out common factors, not terms.

For example:
\[
\frac{x^2 + 3x}{x} \ne x + 3
\]
because \(x^2 + 3x\) can be factored, while \(x\) in the denominator only cancels with matching factors:
\[
\frac{x^2 + 3x}{x} = \frac{x(x + 3)}{x} = x + 3.
\]
Here, cancellation was valid because both numerator and denominator shared a factor of \(x\).

Always factor completely before simplifying, and remember that any value making the denominator zero must be excluded from the domain.

\section*{Core Skills}
\begin{itemize}
  \item Factor numerators and denominators completely.
  \item Identify and cancel common factors (not terms).
  \item Simplify using exponent laws where possible.
  \item Rewrite complex fractions into single rational expressions.
  \item Recognize and note restrictions where denominators are zero.
\end{itemize}

\section*{Example 1: Basic Common Factor Cancellation}

Simplify \(\dfrac{6x^2}{3x}\).

\[
\dfrac{6x^2}{3x} = \dfrac{6}{3} \cdot \dfrac{x^2}{x} = 2x
\]

\textbf{Final Answer:} \(\boxed{2x}\), where \(x \ne 0.\)

\section*{Example 2: Factoring Before Cancelling}

Simplify \(\dfrac{x^2 + 5x}{x}\).

\[
\dfrac{x^2 + 5x}{x} = \dfrac{x(x + 5)}{x} = x + 5
\]

\textbf{Final Answer:} \(\boxed{x + 5}\), where \(x \ne 0.\)

\section*{Example 3: Polynomial with Common Binomial Factor}

Simplify \(\dfrac{(x + 2)(x - 3)}{(x - 3)}\).

\[
(x - 3)\text{ cancels out.} \quad \boxed{x + 2}
\]

\textbf{Restriction:} \(x \ne 3.\)

\section*{Example 4: Quadratic Factorization}

Simplify \(\dfrac{x^2 - 9}{x^2 - 3x}\).

\[
\dfrac{(x - 3)(x + 3)}{x(x - 3)} = \dfrac{x + 3}{x}
\]

\textbf{Final Answer:} \(\boxed{\dfrac{x + 3}{x}}\), where \(x \ne 0, 3.\)

\section*{Example 5: Simplifying with Exponents}

Simplify \(\dfrac{x^5}{x^2}\).

\[
x^{5 - 2} = x^3
\]

\textbf{Final Answer:} \(\boxed{x^3}\), where \(x \ne 0.\)

\section*{Example 6: Complex Rational Expression}

Simplify \(\dfrac{\frac{x}{x+1}}{\frac{x^2}{x+1}}\).

\[
\dfrac{\frac{x}{x+1}}{\frac{x^2}{x+1}} = \frac{x}{x+1} \cdot \frac{x+1}{x^2} = \frac{1}{x}
\]

\textbf{Final Answer:} \(\boxed{\dfrac{1}{x}}\), where \(x \ne 0, -1.\)

\section*{Key Takeaways}
\begin{itemize}
  \item Always factor completely before cancelling.
  \item Cancel only common factors, never terms connected by addition or subtraction.
  \item Record any values that make the denominator zero as restrictions.
  \item Simplifying rational expressions mirrors numerical fraction simplification.
  \item Use exponent rules to handle variable powers efficiently.
\end{itemize}

\newpage

% ============================================================
% QUESTIONS — UNIT 5, TOPIC 3: SIMPLIFYING RATIONAL EXPRESSIONS
% ============================================================

\section*{Practice Questions: Simplifying Rational Expressions}

\subsection*{Part A: Core Skills (Direct Simplification)}
\begin{enumerate}
  \item Simplify \(\dfrac{6x^3}{3x}\)
  \item Simplify \(\dfrac{10y^2}{5y}\)
  \item Simplify \(\dfrac{x^4}{x^2}\)
  \item Simplify \(\dfrac{12a^5b}{4a^2b}\)
  \item Simplify \(\dfrac{9m^3n^2}{3mn}\)
\end{enumerate}

\subsection*{Part B: Factoring Before Simplifying}
\begin{enumerate}
  \setcounter{enumi}{5}
  \item Simplify \(\dfrac{x^2 + 2x}{x}\)
  \item Simplify \(\dfrac{y^2 + 3y}{y}\)
  \item Simplify \(\dfrac{a^2 - 9}{a - 3}\)
  \item Simplify \(\dfrac{x^2 + 4x + 4}{x + 2}\)
  \item Simplify \(\dfrac{p^2 + 5p}{p}\)
\end{enumerate}

\subsection*{Part C: Common Factors and Structure Recognition}
\begin{enumerate}
  \setcounter{enumi}{10}
  \item Simplify \(\dfrac{(x + 3)(x - 2)}{(x - 2)}\)
  \item Simplify \(\dfrac{(y - 4)(y + 1)}{y - 4}\)
  \item Simplify \(\dfrac{(a + 5)(a - 1)}{(a + 5)}\)
  \item Simplify \(\dfrac{(b^2 - 16)}{(b - 4)}\)
  \item Simplify \(\dfrac{(x + 7)(x + 2)}{(x + 2)(x - 1)}\)
\end{enumerate}

\subsection*{Part D: Complex and Multi-Term Rational Expressions}
\begin{enumerate}
  \setcounter{enumi}{15}
  \item Simplify \(\dfrac{x^2 - 9}{x^2 - 3x}\)
  \item Simplify \(\dfrac{x^2 + 5x + 6}{x + 3}\)
  \item Simplify \(\dfrac{a^2 - 4a + 4}{a - 2}\)
  \item Simplify \(\dfrac{b^3 - b^2}{b^2}\)
  \item Simplify \(\dfrac{x^2 - 1}{x^2 + 2x + 1}\)
\end{enumerate}

\subsection*{Part E: Word Problems and SAT-Style Applications}
\begin{enumerate}
  \setcounter{enumi}{20}
  \item A rectangle has length \(x + 3\) and width \(x\). Write the ratio of length to width as a simplified rational expression.
  \item The speed of an object is given by \(v = \dfrac{d}{t}\). If both distance and time are doubled, simplify the new expression for \(v\) relative to the original.
  \item The volume of a cylinder is \(V = \pi r^2h\). Write \(\dfrac{V}{\pi r h}\) as a simplified expression.
  \item The efficiency of a machine is \(E = \dfrac{\text{output}}{\text{input}}\). If both output and input are multiplied by \(k\), simplify the new expression for \(E\).
  \item A formula gives \(y = \dfrac{x^2 + 3x}{x}\). Simplify and state any restrictions.
\end{enumerate}

\newpage

% ============================================================
% SOLUTIONS — UNIT 5, TOPIC 3: SIMPLIFYING RATIONAL EXPRESSIONS
% ============================================================

\section*{Answer Key and Solutions: Simplifying Rational Expressions}

\subsection*{Part A Solutions: Core Skills (Direct Simplification)}
\begin{enumerate}
  \item \(\dfrac{6x^3}{3x}
  = \dfrac{6}{3} \cdot \dfrac{x^3}{x}
  = 2x^{2}
  = \boxed{2x^2}, \; x \ne 0\)

  \item \(\dfrac{10y^2}{5y}
  = \dfrac{10}{5} \cdot \dfrac{y^2}{y}
  = 2y
  = \boxed{2y}, \; y \ne 0\)

  \item \(\dfrac{x^4}{x^2}
  = x^{4-2}
  = x^2
  = \boxed{x^2}, \; x \ne 0\)

  \item \(\dfrac{12a^5 b}{4a^2 b}
  = \dfrac{12}{4} \cdot \dfrac{a^5}{a^2} \cdot \dfrac{b}{b}
  = 3a^{3}
  = \boxed{3a^3}, \; a \ne 0,\; b \ne 0\)

  \item \(\dfrac{9m^3 n^2}{3mn}
  = \dfrac{9}{3} \cdot \dfrac{m^3}{m} \cdot \dfrac{n^2}{n}
  = 3m^{2} n
  = \boxed{3m^2 n}, \; m \ne 0,\; n \ne 0\)
\end{enumerate}

\subsection*{Part B Solutions: Factoring Before Simplifying}
\begin{enumerate}
  \setcounter{enumi}{5}
  \item \(\dfrac{x^2 + 2x}{x}
  = \dfrac{x(x+2)}{x}
  = x+2
  = \boxed{x+2}, \; x \ne 0\)

  \item \(\dfrac{y^2 + 3y}{y}
  = \dfrac{y(y+3)}{y}
  = y+3
  = \boxed{y+3}, \; y \ne 0\)

  \item \(\dfrac{a^2 - 9}{a - 3}
  = \dfrac{(a-3)(a+3)}{a-3}
  = a+3
  = \boxed{a+3}, \; a \ne 3\)

  \item \(\dfrac{x^2 + 4x + 4}{x + 2}
  = \dfrac{(x+2)(x+2)}{x+2}
  = x+2
  = \boxed{x+2}, \; x \ne -2\)

  \item \(\dfrac{p^2 + 5p}{p}
  = \dfrac{p(p+5)}{p}
  = p+5
  = \boxed{p+5}, \; p \ne 0\)
\end{enumerate}

\subsection*{Part C Solutions: Common Factors and Structure Recognition}
\begin{enumerate}
  \setcounter{enumi}{10}
  \item \(\dfrac{(x+3)(x-2)}{(x-2)}
  = x+3
  = \boxed{x+3}, \; x \ne 2\)

  \item \(\dfrac{(y-4)(y+1)}{y-4}
  = y+1
  = \boxed{y+1}, \; y \ne 4\)

  \item \(\dfrac{(a+5)(a-1)}{(a+5)}
  = a-1
  = \boxed{a-1}, \; a \ne -5\)

  \item \(\dfrac{b^2 - 16}{b - 4}
  = \dfrac{(b-4)(b+4)}{b-4}
  = b+4
  = \boxed{b+4}, \; b \ne 4\)

  \item \(\dfrac{(x+7)(x+2)}{(x+2)(x-1)}
  = \dfrac{x+7}{x-1}
  = \boxed{\dfrac{x+7}{x-1}}, \; x \ne -2,\; x \ne 1\)
\end{enumerate}

\subsection*{Part D Solutions: Complex and Multi-Term Rational Expressions}
\begin{enumerate}
  \setcounter{enumi}{15}
  \item \(\dfrac{x^2 - 9}{x^2 - 3x}
  = \dfrac{(x-3)(x+3)}{x(x-3)}
  = \dfrac{x+3}{x}
  = \boxed{\dfrac{x+3}{x}}, \; x \ne 0,\; x \ne 3\)

  \item \(\dfrac{x^2 + 5x + 6}{x + 3}
  = \dfrac{(x+2)(x+3)}{x+3}
  = x+2
  = \boxed{x+2}, \; x \ne -3\)

  \item \(\dfrac{a^2 - 4a + 4}{a - 2}
  = \dfrac{(a-2)(a-2)}{a-2}
  = a-2
  = \boxed{a-2}, \; a \ne 2\)

  \item \(\dfrac{b^3 - b^2}{b^2}
  = \dfrac{b^2(b-1)}{b^2}
  = b-1
  = \boxed{b-1}, \; b \ne 0\)

  \item \(\dfrac{x^2 - 1}{x^2 + 2x + 1}
  = \dfrac{(x-1)(x+1)}{(x+1)(x+1)}
  = \dfrac{x-1}{x+1}
  = \boxed{\dfrac{x-1}{x+1}}, \; x \ne -1\)
\end{enumerate}

\subsection*{Part E Solutions: Word Problems and SAT-Style Applications}
\begin{enumerate}
  \setcounter{enumi}{20}
  \item Length \(= x+3\), width \(= x\).  
  Ratio \(= \dfrac{x+3}{x}
  = \boxed{\dfrac{x+3}{x}}, \; x \ne 0\)

  \item Original speed \(v = \dfrac{d}{t}\).  
  If both distance and time are doubled: new speed \(= \dfrac{2d}{2t} = \dfrac{d}{t} = v\).  
  So \(\boxed{\text{speed stays the same}}\)

  \item \(V = \pi r^2 h\).  
  \(\dfrac{V}{\pi r h}
  = \dfrac{\pi r^2 h}{\pi r h}
  = r
  = \boxed{r}, \; r \ne 0,\; h \ne 0\)

  \item \(E = \dfrac{\text{output}}{\text{input}}\).  
  If both output and input are multiplied by \(k\):  
  New \(E = \dfrac{k \cdot \text{output}}{k \cdot \text{input}}
  = \dfrac{\text{output}}{\text{input}}
  = E\).  
  So \(\boxed{\text{efficiency stays the same}}\)

  \item \(y = \dfrac{x^2 + 3x}{x}
  = \dfrac{x(x+3)}{x}
  = x+3\).  
  So \(y = \boxed{x+3}, \; x \ne 0\)
\end{enumerate}



\end{document}
