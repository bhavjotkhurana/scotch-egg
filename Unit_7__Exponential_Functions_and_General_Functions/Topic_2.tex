\documentclass[12pt]{article}

\usepackage{amsmath, amssymb}
\usepackage{geometry}
\usepackage{setspace}
\usepackage{titlesec}
\usepackage{lmodern}
\usepackage{xcolor}
\usepackage{enumitem}

\geometry{margin=1in}
\setstretch{1.2}
\titleformat{\section}{\normalfont\Large\bfseries}{\thesection}{1em}{}
\titleformat{\subsection}{\normalfont\large\bfseries}{\thesubsection}{1em}{}
\pagenumbering{gobble}

\begin{document}

\begin{center}
    \LARGE \textbf{Unit 7: Exponential and General Functions} \\[6pt]
    \Large \textbf{Topic 2: Comparing Linear and Exponential Growth}
\end{center}

\vspace{1em}

\section*{Concept Summary}

Linear and exponential functions both model change over time, but in very different ways:

\[
\text{Linear: } y = mx + b \qquad \text{Exponential: } y = a b^x
\]

A \textbf{linear function} increases (or decreases) by a constant \emph{amount} with each step in \(x\).  
An \textbf{exponential function} increases (or decreases) by a constant \emph{factor} (percentage) with each step in \(x\).

\[
\begin{cases}
\text{Linear: } \Delta y = \text{constant} \\
\text{Exponential: } \dfrac{y_2}{y_1} = \text{constant ratio}
\end{cases}
\]

Early on, a linear function might appear to grow faster, but eventually an exponential function surpasses it because it compounds multiplicatively.

\subsection*{When to Use Each Model}
\begin{itemize}
  \item Use a \textbf{linear model} when a fixed amount is added or subtracted (e.g., salary increases by \$1000 each year).
  \item Use an \textbf{exponential model} when change is proportional to the current value (e.g., population grows 5\% per year).
\end{itemize}

Graphically:
\begin{itemize}
  \item Linear graphs are straight lines.
  \item Exponential graphs curve upward (growth) or downward (decay).
\end{itemize}

(Insert graph: one straight line showing constant slope, one exponential curve crossing it and rising faster.)

\section*{Core Skills}
\begin{itemize}
  \item Identify whether change is additive (linear) or multiplicative (exponential).
  \item Write equations for both models given a situation or data.
  \item Compare values over time to see when exponential overtakes linear.
  \item Interpret meaning of slope (\(m\)) and growth factor (\(b\)).
  \item Recognize patterns in tables and graphs that distinguish the two.
\end{itemize}

\section*{Example 1: Distinguishing Linear vs. Exponential}

A quantity increases by 10 each year.

\[
y = 50 + 10x
\]
Constant difference → linear.

Another quantity increases by 10\% each year.

\[
y = 50(1.10)^x
\]
Constant ratio → exponential.

\textbf{Answer:} First is linear, second is exponential.

\section*{Example 2: Comparing Growth Over Time}

Compare \(L(x) = 100 + 20x\) and \(E(x) = 100(1.15)^x.\)

Compute a few values:

\[
\begin{array}{c|ccccc}
x & 0 & 2 & 4 & 6 & 8 \\ \hline
L(x) & 100 & 140 & 180 & 220 & 260 \\
E(x) & 100 & 132 & 174 & 229 & 301
\end{array}
\]

At first, linear grows faster, but around \(x = 5\), exponential surpasses it.

\textbf{Interpretation:} Exponential growth dominates over time.

\section*{Example 3: Real-World Comparison}

A worker earns \$30,000 and gets a \$2,000 raise yearly (linear):
\[
S_L(t) = 30{,}000 + 2000t
\]

A startup grows revenue by 10\% annually (exponential):
\[
S_E(t) = 30{,}000(1.10)^t
\]

After 5 years:
\[
S_L(5) = 40{,}000, \quad S_E(5) = 30{,}000(1.10)^5 = 48{,}315.
\]
The exponential model surpasses the linear one.

\textbf{Answer:} Exponential wins in the long term.

\section*{Example 4: Identifying Model Type from Data}

\[
\begin{array}{c|ccccc}
x & 0 & 1 & 2 & 3 & 4 \\ \hline
y & 10 & 20 & 30 & 40 & 50
\end{array}
\]
Constant difference of +10 → linear.

\[
\begin{array}{c|ccccc}
x & 0 & 1 & 2 & 3 & 4 \\ \hline
y & 10 & 15 & 22.5 & 33.75 & 50.6
\end{array}
\]
Constant ratio (multiply by 1.5) → exponential.

\section*{Example 5: Crossover Point (When They Are Equal)}

Solve \(L(x) = E(x)\) for \(100 + 20x = 100(1.15)^x.\)

This cannot be solved algebraically without a calculator, but graphically they intersect near \(x \approx 5.2.\)

Beyond that point, exponential dominates.

(Insert graph showing line and curve intersecting near \(x = 5\).)

\section*{Key Takeaways}
\begin{itemize}
  \item Linear: constant difference; Exponential: constant ratio.
  \item Linear growth is steady; exponential growth accelerates.
  \item On a graph, exponential curves eventually overtake linear lines.
  \item In data, check differences vs. ratios to identify the model.
  \item On the SAT, expect comparison questions between linear and exponential models.
\end{itemize}

\newpage

% ============================================================
% QUESTIONS — UNIT 7, TOPIC 2: COMPARING LINEAR AND EXPONENTIAL GROWTH
% ============================================================

\section*{Practice Questions: Comparing Linear and Exponential Growth}

\subsection*{Part A: Identifying the Type of Growth}
\begin{enumerate}
  \item Determine whether \(y = 50 + 4x\) is linear or exponential.
  \item Determine whether \(y = 200(1.05)^x\) is linear or exponential.
  \item Classify \(y = 300 - 25x\) and \(y = 300(0.9)^x\) as linear or exponential.
  \item Explain why \(y = 100(1.1)^x\) is exponential, not linear.
  \item Explain how you can tell from a table if a function is linear or exponential.
\end{enumerate}

\subsection*{Part B: Comparing Equations}
\begin{enumerate}
  \setcounter{enumi}{5}
  \item Compare \(L(x) = 100 + 15x\) and \(E(x) = 100(1.15)^x\).  
  Which grows faster at first? Which grows faster later?
  \item For \(L(x) = 50 + 10x\) and \(E(x) = 50(1.2)^x\), find \(L(4)\) and \(E(4)\) and compare.
  \item Write one example of a linear model and one of an exponential model that both start at 200.
  \item A savings account grows by \$200 per year. Another account grows by 6\% per year from \$1000.  
  Write both models and describe which will be greater after 10 years.
  \item A car’s value decreases linearly by \$1500 per year starting at \$20,000, while another model decreases by 10\% each year.  
  Write both equations and describe which loses value faster over time.
\end{enumerate}

\subsection*{Part C: Evaluating and Comparing Values}
\begin{enumerate}
  \setcounter{enumi}{10}
  \item Find \(L(5)\) for \(L(x) = 300 + 25x\) and \(E(5)\) for \(E(x) = 300(1.07)^x.\)
  \item For \(L(x) = 100 + 30x\) and \(E(x) = 100(1.25)^x,\) find when \(E(x)\) first exceeds \(L(x)\) (estimate).
  \item Compute and compare:  
  \(L(8) = 50 + 8(10)\) and \(E(8) = 50(1.15)^8.\)
  \item Use \(L(x) = 500 + 50x\) and \(E(x) = 500(1.08)^x.\)  
  Find both values at \(x = 0, 5, 10.\)
  \item The quantity \(Q(t) = 100 + 10t\) and \(R(t) = 100(1.1)^t.\)  
  At what \(t\) does \(R(t)\) become larger than \(Q(t)\)? (Estimate numerically.)
\end{enumerate}

\subsection*{Part D: Graph and Table Interpretation}
(Insert graphs showing a line and an exponential curve intersecting around \(x=5\).)
\begin{enumerate}
  \setcounter{enumi}{15}
  \item Which graph shows a constant rate of change: the line or the curve?
  \item Which graph represents a constant percent change?
  \item Where does the exponential curve overtake the linear line?
  \item In the table below, determine whether data is linear or exponential.  

  \[
  \begin{array}{c|cccc}
  x & 0 & 1 & 2 & 3 \\ \hline
  y & 20 & 25 & 30 & 35
  \end{array}
  \]
  \[
  \begin{array}{c|cccc}
  x & 0 & 1 & 2 & 3 \\ \hline
  y & 10 & 15 & 22.5 & 33.75
  \end{array}
  \]
  Identify which is linear and which is exponential.
  \item Describe how the slopes or ratios appear different in graphs of linear vs exponential functions.
\end{enumerate}

\subsection*{Part E: SAT-Style Applications}
(Insert two graphs: one straight line showing linear salary growth, one curved line showing exponential revenue growth.)

\begin{enumerate}
  \setcounter{enumi}{20}
  \item A salary increases by \$2500 each year, starting at \$40,000. Write the linear model.
  \item A different salary starts at \$40,000 and increases by 5\% per year. Write the exponential model.
  \item After 5 years, which salary is higher?
  \item Approximately how many years will it take for the exponential salary to exceed the linear salary?
  \item On the SAT, which clue in a word problem signals exponential growth rather than linear growth?
\end{enumerate}

\newpage

% ============================================================
% SOLUTIONS — UNIT 7, TOPIC 2: COMPARING LINEAR AND EXPONENTIAL GROWTH
% ============================================================

\section*{Answer Key and Solutions: Comparing Linear and Exponential Growth}

\subsection*{Part A Solutions: Identifying the Type of Growth}
\begin{enumerate}
  \item \(y = 50 + 4x\)

  Of the form \(mx + b\). Change is +4 each 1 unit in \(x\).  
  \(\boxed{\text{Linear}}\)

  \item \(y = 200(1.05)^x\)

  Of the form \(a b^x\). Change is by multiplying by 1.05 each step.  
  \(\boxed{\text{Exponential}}\)

  \item
  \[
  y = 300 - 25x \quad \rightarrow \quad \text{Linear (constant decrease of 25)}
  \]
  \[
  y = 300(0.9)^x \quad \rightarrow \quad \text{Exponential (keeps 90\% each step)}
  \]

  \(\boxed{\text{First is linear, second is exponential}}\)

  \item \(y = 100(1.1)^x\)

  The output is repeatedly multiplied by 1.1, which is a constant percent increase of 10\%.  
  Linear would add the same amount each step, not multiply.  \newline
  \(\boxed{\text{Exponential because it grows by 10\% each step}}\)

  \item In a table:
  \begin{itemize}
    \item Linear: \(y\) goes up by the same difference each step.
    \item Exponential: the ratio \(\frac{y_{\text{next}}}{y_{\text{current}}}\) is the same.
  \end{itemize}

  \(\boxed{\text{Check differences for linear, ratios for exponential}}\)
\end{enumerate}

\subsection*{Part B Solutions: Comparing Equations}
\begin{enumerate}
  \setcounter{enumi}{5}
  \item \(L(x) = 100 + 15x,\; E(x) = 100(1.15)^x\)

  At first (small \(x\)), \(L\) adds 15 per step, which is big compared to early exponential growth.  
  Later, \(E\) compounds by 15\% each step and overtakes.

  \(\boxed{\text{Linear is faster first, exponential is faster later}}\)

  \item \(L(x) = 50 + 10x,\; E(x) = 50(1.2)^x\)

  At \(x = 4\):
  \[
  L(4) = 50 + 10(4) = 90
  \]
  \[
  E(4) = 50(1.2)^4 = 50(2.0736) \approx 103.68
  \]

  \(\boxed{E(4) \approx 103.7 \text{ is larger than } L(4)=90}\)

  \item Linear example starting at 200:  
  \[
  L(x) = 200 + 5x
  \]

    
  Exponential example starting at 200:  
  \[
  E(x) = 200(1.03)^x
  \]


  \(\boxed{L(x)=200+5x,\; E(x)=200(1.03)^x}\)

  \item Savings account 1: grows by \$200/year from \$1000. Linear.
  \[
  A(t) = 1000 + 200t
  \]

  Savings account 2: grows 6\% per year from \$1000. Exponential.
  \[
  B(t) = 1000(1.06)^t
  \]

  After 10 years:
  \[
  A(10) = 1000 + 200(10) = 3000
  \]
  \[
  B(10) = 1000(1.06)^{10} \approx 1000(1.7908) \approx 1790.8
  \]

  \(\boxed{\text{After 10 years, linear is larger}}\)

  \item Car model 1: loses \$1500 per year, starting \$20{,}000. Linear.
  \[
  V_L(t) = 20{,}000 - 1500t
  \]

  Car model 2: loses 10\% per year. Keeps 90\%.
  \[
  V_E(t) = 20{,}000(0.9)^t
  \]

  Over time:
  \begin{itemize}
    \item Early: 10\% of 20{,}000 is 2000, which is more than 1500. So exponential decay drops faster at first.
    \item Later: As value shrinks, 10\% of the remaining amount becomes smaller than 1500, so linear eventually loses value faster.
  \end{itemize}

  \(\boxed{\text{Exponential loses faster first, linear loses faster later}}\)
\end{enumerate}

\subsection*{Part C Solutions: Evaluating and Comparing Values}
\begin{enumerate}
  \setcounter{enumi}{10}
  \item \(L(x) = 300 + 25x,\; E(x) = 300(1.07)^x\)

  At \(x = 5\):
  \[
  L(5) = 300 + 25(5) = 425
  \]
  \[
  E(5) = 300(1.07)^5 \approx 300(1.4026) \approx 420.8
  \]

  \(\boxed{L(5)=425 \text{ is slightly greater than } E(5)\approx 420.8}\)

  \item \(L(x) = 100 + 30x,\; E(x) = 100(1.25)^x\)

  We check a few \(x\):

  \[
  x=2: L=160,\; E=100(1.25)^2=156.25
  \]
  \[
  x=3: L=190,\; E=100(1.25)^3=195.31
  \]

  \(E\) first exceeds \(L\) between \(x=2\) and \(x=3\). At \(x=3,\; E > L.\)

  \(\boxed{\text{Around } x=3}\)

  \item
  \[
  L(8) = 50 + 8(10) = 130
  \]
  \[
  E(8) = 50(1.15)^8
  \]
  \((1.15)^8 \approx 3.0590.\)
  \[
  E(8) \approx 50 \times 3.0590 \approx 152.95
  \]

  \(\boxed{L(8)=130,\; E(8)\approx 153;\; \text{exponential is larger}}\)

  \item Use \(L(x)=500+50x,\; E(x)=500(1.08)^x\).

  \(x=0:\)
  \[
  L(0)=500,\quad E(0)=500(1)^0=500.
  \]

  \(x=5:\)
  \[
  L(5)=500+50(5)=750
  \]
  \[
  E(5)=500(1.08)^5 \approx 500(1.4693)\approx 734.65
  \]

  \(x=10:\)
  \[
  L(10)=500+50(10)=1000
  \]
  \[
  E(10)=500(1.08)^{10} \approx 500(2.1589)\approx 1079.45
  \]

  Summary:
  \[
  x=0:\; \text{tie at 500}
  \]
  \[
  x=5:\; L > E
  \]
  \[
  x=10:\; E > L
  \]

  \(\boxed{\text{Linear wins early, exponential wins later}}\)

  \item \(Q(t) = 100 + 10t,\; R(t) = 100(1.1)^t\)

  Check values:

  \(t=2:\)
  \[
  Q=120,\quad R=100(1.1)^2=121
  \Rightarrow R>Q
  \]

  \(t=1:\)
  \[
  Q=110,\quad R=110
  \Rightarrow \text{tie}
  \]

  So they are equal at \(t=1\), and exponential becomes larger just after that.

  \(\boxed{\text{After } t=1,\; R>Q}\)
\end{enumerate}

\subsection*{Part D Solutions: Graph and Table Interpretation}
(Reference: line with constant slope, exponential curve crossing and then rising faster.)

\begin{enumerate}
  \setcounter{enumi}{15}
  \item Constant rate of change means constant slope. That is the line.

  \(\boxed{\text{The line has constant rate of change}}\)

  \item Constant percent change means multiply by same factor repeatedly. That is the exponential curve.

  \(\boxed{\text{The curve has constant percent change}}\)

  \item The exponential curve overtakes the line at the intersection point. From the prompt, this is around \(x = 5\).

  \(\boxed{\text{Near } x \approx 5}\)

  \item First table:
  \[
  \begin{array}{c|cccc}
  x & 0 & 1 & 2 & 3 \\ \hline
  y & 20 & 25 & 30 & 35
  \end{array}
  \]
  Differences are +5 each step → linear.

  Second table:
  \[
  \begin{array}{c|cccc}
  x & 0 & 1 & 2 & 3 \\ \hline
  y & 10 & 15 & 22.5 & 33.75
  \end{array}
  \]
  Ratios are \(\times 1.5\) each step → exponential.

  \(\boxed{\text{First is linear, second is exponential}}\)

  \item In a linear graph, slope is constant and the line is straight.  
  In an exponential graph, the slope keeps increasing (for growth), so the curve gets steeper.

  \(\boxed{\text{Linear: constant slope. Exponential: slope keeps changing}}\)
\end{enumerate}

\subsection*{Part E Solutions: SAT-Style Applications}
\begin{enumerate}
  \setcounter{enumi}{20}
  \item Salary with constant \$2500 yearly raise, starting \$40{,}000.

  This is linear:
  \[
  S_L(t) = 40{,}000 + 2500t.
  \]

  \(\boxed{S_L(t)=40{,}000+2500t}\)

  \item Salary starting \$40{,}000, grows 5\% per year.

  Growth factor 1.05:
  \[
  S_E(t) = 40{,}000(1.05)^t.
  \]

  \(\boxed{S_E(t)=40{,}000(1.05)^t}\)

  \item After 5 years:

  Linear:
  \[
  S_L(5)=40{,}000+2500(5)=52{,}500.
  \]

  Exponential:
  \[
  S_E(5)=40{,}000(1.05)^5 \approx 40{,}000(1.2763)\approx 51{,}052.
  \]

  \(\boxed{After 5 years, linear salary is higher}\)

  \item When does exponential exceed linear?

  We compare \(40{,}000 + 2500t\) and \(40{,}000(1.05)^t\).

  Check:
  \[
  t=5:\; 52{,}500 \text{ vs } 51{,}052 \quad (\text{linear bigger})
  \]
  \[
  t=10:\; S_L(10)=40{,}000+25{,}000=65{,}000
  \]
  \[
  S_E(10)=40{,}000(1.05)^{10}\approx 40{,}000(1.6289)\approx 65{,}156
  \]

  Around \(t \approx 10\) years, exponential passes linear.

  \(\boxed{\text{About 10 years}}\)

  \item SAT clue for exponential:

  Words like \(\%\) per year, doubles every 6 days, loses 12\% each cycle, or "multiplied by the same factor each period" indicate exponential.  
  Words like "increases by 400 each month" indicate linear.

  \(\boxed{\text{Percent growth/decay or repeated multiply = exponential}}\)
\end{enumerate}



\end{document}
