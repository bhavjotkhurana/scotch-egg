\documentclass[12pt]{article}

\usepackage{amsmath, amssymb}
\usepackage{geometry}
\usepackage{setspace}
\usepackage{titlesec}
\usepackage{lmodern}
\usepackage{xcolor}
\usepackage{enumitem}

\geometry{margin=1in}
\setstretch{1.2}
\titleformat{\section}{\normalfont\Large\bfseries}{\thesection}{1em}{}
\titleformat{\subsection}{\normalfont\large\bfseries}{\thesubsection}{1em}{}
\pagenumbering{gobble}

\begin{document}

\begin{center}
    \LARGE \textbf{Unit 7: Exponential and General Functions} \\[6pt]
    \Large \textbf{Topic 3: Function Notation and Evaluation}
\end{center}

\vspace{1em}

\section*{Concept Summary}

A \textbf{function} is a rule that assigns each input \(x\) exactly one output \(y\).  
We often write functions using \textbf{function notation}:
\[
y = f(x)
\]
which means “\(f\) of \(x\).”  
This notation emphasizes that \(y\) depends on \(x\).

To \textbf{evaluate a function}, substitute a given value for \(x\) and simplify.

Example:
\[
f(x) = 3x + 2, \quad f(4) = 3(4) + 2 = 14.
\]

The same rule applies for more complex expressions or multiple steps.  
SAT questions often ask for \(f(a)\), \(f(x+1)\), or \(f(2x)\), which require substitution and simplification rather than solving.

\section*{Core Skills}
\begin{itemize}
  \item Interpret \(f(x)\) as the value of the function at input \(x\).
  \item Substitute specific or algebraic values into function definitions.
  \item Simplify expressions like \(f(x+h)\) or \(f(2x)\).
  \item Understand \(f(x)\) vs. \(f(a)\) as general rule vs. specific value.
  \item Recognize and evaluate composite expressions inside functions.
\end{itemize}

\section*{Example 1: Basic Evaluation}

Given \(f(x) = 2x + 5,\) find \(f(3).\)

\[
f(3) = 2(3) + 5 = 6 + 5 = 11.
\]

\(\boxed{f(3) = 11}\)

\section*{Example 2: Substituting an Expression}

If \(f(x) = 4x - 7,\) find \(f(x + 2).\)

\[
f(x + 2) = 4(x + 2) - 7 = 4x + 8 - 7 = 4x + 1.
\]

\(\boxed{f(x + 2) = 4x + 1}\)

\section*{Example 3: Function in Context}

A taxi fare is modeled by \(C(t) = 3t + 5,\) where \(t\) is miles.  
Find \(C(10)\).

\[
C(10) = 3(10) + 5 = 35.
\]
The cost for 10 miles is \(\boxed{\$35}.\)

\section*{Example 4: Evaluating a Quadratic Function}

Given \(f(x) = x^2 - 4x + 1,\) find \(f(-2)\).

\[
f(-2) = (-2)^2 - 4(-2) + 1 = 4 + 8 + 1 = 13.
\]

\(\boxed{f(-2) = 13}\)

\section*{Example 5: Algebraic Substitution}

Given \(f(x) = 3x^2 + 2x - 1,\) find \(f(2x).\)

\[
f(2x) = 3(2x)^2 + 2(2x) - 1 = 3(4x^2) + 4x - 1 = 12x^2 + 4x - 1.
\]

\(\boxed{f(2x) = 12x^2 + 4x - 1}\)

\section*{Example 6: Working with Multiple Functions}

If \(f(x) = x + 2\) and \(g(x) = 3x - 1,\) find:
\[
f(3), \quad g(3), \quad f(g(3)).
\]

Compute step by step:
\[
g(3) = 3(3) - 1 = 8.
\]
\[
f(3) = 3 + 2 = 5.
\]
\[
f(g(3)) = f(8) = 8 + 2 = 10.
\]

\(\boxed{f(3) = 5,\; g(3) = 8,\; f(g(3)) = 10}\)

\section*{Key Takeaways}
\begin{itemize}
  \item \(f(x)\) is just notation for the output of function \(f\) at input \(x\).
  \item Substitute carefully—replace every \(x\) with the given input.
  \item Algebraic inputs like \(x+1\) or \(2x\) require expanding and simplifying.
  \item On SAT, evaluation may be numeric or symbolic—read carefully.
  \item For composed or nested functions, compute inside-out.
\end{itemize}

\newpage

% ============================================================
% QUESTIONS — UNIT 7, TOPIC 3: FUNCTION NOTATION AND EVALUATION
% ============================================================

\section*{Practice Questions: Function Notation and Evaluation}

\subsection*{Part A: Direct Evaluation}
\begin{enumerate}
  \item If \(f(x) = 3x + 4,\) find \(f(2).\)
  \item If \(f(x) = 7x - 5,\) find \(f(0).\)
  \item If \(f(x) = 2x^2 + 1,\) find \(f(3).\)
  \item If \(f(x) = x^2 - 4x + 6,\) find \(f(-2).\)
  \item If \(f(x) = 5 - x,\) find \(f(8).\)
\end{enumerate}

\subsection*{Part B: Substituting Expressions}
\begin{enumerate}
  \setcounter{enumi}{5}
  \item Given \(f(x) = 2x + 3,\) find \(f(x + 4).\)
  \item Given \(f(x) = x^2 + 2x,\) find \(f(x - 1).\)
  \item If \(f(x) = 4x - 7,\) find \(f(2x).\)
  \item If \(f(x) = x^2 - 3x,\) find \(f(2x + 1).\)
  \item Given \(f(x) = 3x^2 - 5x + 2,\) find \(f(x + 2).\)
\end{enumerate}

\subsection*{Part C: Function Tables and Interpretation}
\begin{enumerate}
  \setcounter{enumi}{10}
  \item The function \(f(x) = 2x + 1\) is evaluated for several inputs:  
  Complete the table.  

  \[
  \begin{array}{c|ccccc}
  x & -2 & -1 & 0 & 1 & 2 \\ \hline
  f(x) & ? & ? & ? & ? & ?
  \end{array}
  \]

  \item For \(f(x) = x^2 - 2x,\) fill in the missing outputs:  

  \[
  \begin{array}{c|ccccc}
  x & 0 & 1 & 2 & 3 & 4 \\ \hline
  f(x) & ? & ? & ? & ? & ?
  \end{array}
  \]

  \item What is the input \(x\) if \(f(x) = 3x + 5 = 20?\)
  \item If \(f(x) = x^2 + 1,\) for what \(x\) does \(f(x) = 10?\)
  \item The output of \(f(x) = 2x + 3\) is 15. What is \(x?\)
\end{enumerate}

\subsection*{Part D: Function Operations}
\begin{enumerate}
  \setcounter{enumi}{15}
  \item If \(f(x) = x + 2\) and \(g(x) = 3x - 4,\) find \(f(3)\) and \(g(3).\)
  \item Using the same functions, find \(f(g(x)).\)
  \item If \(f(x) = 2x\) and \(g(x) = x + 5,\) find \(g(f(x)).\)
  \item Given \(f(x) = x^2\) and \(g(x) = x + 3,\) find \(f(g(x))\) and \(g(f(x)).\)
  \item If \(f(x) = 5x - 1\) and \(g(x) = x^2,\) find \(f(g(2))\) and \(g(f(2)).\)
\end{enumerate}

\subsection*{Part E: SAT-Style Applications}
\begin{enumerate}
  \setcounter{enumi}{20}
  \item The cost of a ride is modeled by \(C(m) = 2.5m + 4,\) where \(m\) is miles.  
  What is the cost for a 6-mile ride?
  \item The temperature (in °F) after \(t\) hours is given by \(T(t) = 70 - 5t.\)  
  Find \(T(3)\) and interpret the result.
  \item A function \(h(x)\) is defined by \(h(x) = x^2 - 9.\)  
  What is \(h(a + 3)?\)
  \item If \(f(x) = 4x + 1\) and \(g(x) = x - 2,\) find \(f(g(5)).\)
  \item The function \(P(t) = 100(1.05)^t\) represents a population.  
  Find \(P(2)\) and explain what it means in context.
\end{enumerate}

\newpage

% ============================================================
% SOLUTIONS — UNIT 7, TOPIC 3: FUNCTION NOTATION AND EVALUATION
% ============================================================

\section*{Answer Key and Solutions: Function Notation and Evaluation}

\subsection*{Part A Solutions: Direct Evaluation}
\begin{enumerate}
  \item \(f(x) = 3x + 4\)

  \[
  f(2) = 3(2) + 4 = 6 + 4 = 10.
  \]

  \(\boxed{f(2) = 10}\)

  \item \(f(x) = 7x - 5\)

  \[
  f(0) = 7(0) - 5 = -5.
  \]

  \(\boxed{f(0) = -5}\)

  \item \(f(x) = 2x^2 + 1\)

  \[
  f(3) = 2(3)^2 + 1 = 2(9) + 1 = 18 + 1 = 19.
  \]

  \(\boxed{f(3) = 19}\)

  \item \(f(x) = x^2 - 4x + 6\)

  \[
  f(-2) = (-2)^2 - 4(-2) + 6 = 4 + 8 + 6 = 18.
  \]

  \(\boxed{f(-2) = 18}\)

  \item \(f(x) = 5 - x\)

  \[
  f(8) = 5 - 8 = -3.
  \]

  \(\boxed{f(8) = -3}\)
\end{enumerate}

\subsection*{Part B Solutions: Substituting Expressions}
\begin{enumerate}
  \setcounter{enumi}{5}
  \item \(f(x) = 2x + 3\)

  \[
  f(x + 4) = 2(x + 4) + 3 = 2x + 8 + 3 = 2x + 11.
  \]

  \(\boxed{f(x + 4) = 2x + 11}\)

  \item \(f(x) = x^2 + 2x\)

  \[
  f(x - 1) = (x - 1)^2 + 2(x - 1)
  = (x^2 - 2x + 1) + (2x - 2)
  = x^2 - 1.
  \]

  \(\boxed{f(x - 1) = x^2 - 1}\)

  \item \(f(x) = 4x - 7\)

  \[
  f(2x) = 4(2x) - 7 = 8x - 7.
  \]

  \(\boxed{f(2x) = 8x - 7}\)

  \item \(f(x) = x^2 - 3x\)

  \[
  f(2x + 1) = (2x + 1)^2 - 3(2x + 1)
  = (4x^2 + 4x + 1) - (6x + 3)
  = 4x^2 - 2x - 2.
  \]

  \(\boxed{f(2x + 1) = 4x^2 - 2x - 2}\)

  \item \(f(x) = 3x^2 - 5x + 2\)

  \[
  f(x + 2) = 3(x + 2)^2 - 5(x + 2) + 2
  = 3(x^2 + 4x + 4) - 5x - 10 + 2
  \]
  \[
  = 3x^2 + 12x + 12 - 5x - 10 + 2
  = 3x^2 + 7x + 4.
  \]

  \(\boxed{f(x + 2) = 3x^2 + 7x + 4}\)
\end{enumerate}

\subsection*{Part C Solutions: Function Tables and Interpretation}
\begin{enumerate}
  \setcounter{enumi}{10}
  \item \(f(x) = 2x + 1\)

  Evaluate each:
  \[
  f(-2)=2(-2)+1=-4+1=-3
  \]
  \[
  f(-1)=2(-1)+1=-2+1=-1
  \]
  \[
  f(0)=2(0)+1=1
  \]
  \[
  f(1)=2(1)+1=3
  \]
  \[
  f(2)=2(2)+1=5
  \]

  Table:
  \[
  \begin{array}{c|ccccc}
  x & -2 & -1 & 0 & 1 & 2 \\ \hline
  f(x) & -3 & -1 & 1 & 3 & 5
  \end{array}
  \]

  \(\boxed{f(x) = -3,\,-1,\,1,\,3,\,5}\)

  \item \(f(x) = x^2 - 2x\)

  \[
  f(0)=0^2-2(0)=0
  \]
  \[
  f(1)=1^2-2(1)=1-2=-1
  \]
  \[
  f(2)=4-4=0
  \]
  \[
  f(3)=9-6=3
  \]
  \[
  f(4)=16-8=8
  \]

  Table:
  \[
  \begin{array}{c|ccccc}
  x & 0 & 1 & 2 & 3 & 4 \\ \hline
  f(x) & 0 & -1 & 0 & 3 & 8
  \end{array}
  \]

  \(\boxed{0,\,-1,\,0,\,3,\,8}\)

  \item Solve \(f(x) = 3x + 5 = 20\)

  \[
  3x + 5 = 20 \Rightarrow 3x = 15 \Rightarrow x = 5.
  \]

  \(\boxed{x = 5}\)

  \item \(f(x) = x^2 + 1,\; f(x) = 10\)

  \[
  x^2 + 1 = 10 \Rightarrow x^2 = 9 \Rightarrow x = \pm 3.
  \]

  \(\boxed{x = 3 \text{ or } x = -3}\)

  \item \(f(x) = 2x + 3 = 15\)

  \[
  2x + 3 = 15 \Rightarrow 2x = 12 \Rightarrow x = 6.
  \]

  \(\boxed{x = 6}\)
\end{enumerate}

\subsection*{Part D Solutions: Function Operations}
\begin{enumerate}
  \setcounter{enumi}{15}
  \item \(f(x) = x + 2,\; g(x) = 3x - 4\)

  \[
  f(3)=3+2=5,\quad g(3)=3(3)-4=9-4=5.
  \]

  \(\boxed{f(3)=5,\; g(3)=5}\)

  \item Find \(f(g(x))\) with \(f(x)=x+2,\; g(x)=3x-4\)

  \[
  f(g(x)) = f(3x - 4) = (3x - 4) + 2 = 3x - 2.
  \]

  \(\boxed{f(g(x)) = 3x - 2}\)

  \item \(f(x) = 2x,\; g(x) = x + 5\)

  \[
  g(f(x)) = g(2x) = 2x + 5.
  \]

  \(\boxed{g(f(x)) = 2x + 5}\)

  \item \(f(x) = x^2,\; g(x) = x + 3\)

  \[
  f(g(x)) = f(x + 3) = (x + 3)^2 = x^2 + 6x + 9
  \]
  \[
  g(f(x)) = g(x^2) = x^2 + 3
  \]

  \(\boxed{f(g(x)) = x^2 + 6x + 9,\; g(f(x)) = x^2 + 3}\)

  \item \(f(x) = 5x - 1,\; g(x) = x^2\)

  First find \(g(2)\):
  \[
  g(2) = 2^2 = 4.
  \]
  Then
  \[
  f(g(2)) = f(4) = 5(4) - 1 = 20 - 1 = 19.
  \]

  Now find \(f(2)\):
  \[
  f(2) = 5(2) - 1 = 10 - 1 = 9.
  \]
  Then
  \[
  g(f(2)) = g(9) = 9^2 = 81.
  \]

  \(\boxed{f(g(2)) = 19,\; g(f(2)) = 81}\)
\end{enumerate}

\subsection*{Part E Solutions: SAT-Style Applications}
\begin{enumerate}
  \setcounter{enumi}{20}
  \item Cost function \(C(m) = 2.5m + 4\)

  For \(m = 6\):
  \[
  C(6) = 2.5(6) + 4 = 15 + 4 = 19.
  \]

  \(\boxed{\$19 \text{ for a 6-mile ride}}\)

  \item \(T(t) = 70 - 5t\)

  \[
  T(3) = 70 - 5(3) = 70 - 15 = 55.
  \]

  Interpretation: After 3 hours, the temperature is 55°F.

  \(\boxed{T(3)=55,\; \text{temperature is 55°F after 3 hours}}\)

  \item \(h(x) = x^2 - 9\)

  \[
  h(a + 3) = (a + 3)^2 - 9 = a^2 + 6a + 9 - 9 = a^2 + 6a.
  \]

  \(\boxed{h(a + 3) = a^2 + 6a}\)

  \item \(f(x) = 4x + 1,\; g(x) = x - 2\)

  First:
  \[
  g(5) = 5 - 2 = 3.
  \]
  Then:
  \[
  f(g(5)) = f(3) = 4(3) + 1 = 13.
  \]

  \(\boxed{f(g(5)) = 13}\)

  \item \(P(t) = 100(1.05)^t\)

  \[
  P(2) = 100(1.05)^2 = 100(1.1025) = 110.25.
  \]

  Meaning: After 2 time periods, the population is about 110.25 (about 10 percent growth overall).

  \(\boxed{P(2) \approx 110.25,\; \text{population after 2 units of time}}\)
\end{enumerate}



\end{document}
