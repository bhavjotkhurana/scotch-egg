\documentclass[12pt]{article}

\usepackage{amsmath, amssymb}
\usepackage{geometry}
\usepackage{setspace}
\usepackage{titlesec}
\usepackage{lmodern}
\usepackage{xcolor}
\usepackage{enumitem}

\geometry{margin=1in}
\setstretch{1.2}
\titleformat{\section}{\normalfont\Large\bfseries}{\thesection}{1em}{}
\titleformat{\subsection}{\normalfont\large\bfseries}{\thesubsection}{1em}{}
\pagenumbering{gobble}

\begin{document}

\begin{center}
    \LARGE \textbf{Unit 7: Exponential and General Functions} \\[6pt]
    \Large \textbf{Topic 5: Composition and Inverses of Functions}
\end{center}

\vspace{1em}

\section*{Concept Summary}

Functions can be \textbf{combined} to form new ones, or \textbf{reversed} to find inverses.  
Composition means applying one function to the output of another:
\[
(f \circ g)(x) = f(g(x)).
\]

The order matters — generally \(f(g(x)) \ne g(f(x))\).

An \textbf{inverse function}, written \(f^{-1}(x)\), reverses the action of \(f\).  
It “undoes” \(f(x)\), so:
\[
f^{-1}(f(x)) = x \quad \text{and} \quad f(f^{-1}(x)) = x.
\]

To find an inverse algebraically:
\begin{enumerate}
  \item Replace \(f(x)\) with \(y\).
  \item Swap \(x\) and \(y\).
  \item Solve for \(y\).
  \item Replace \(y\) with \(f^{-1}(x)\).
\end{enumerate}

A function has an inverse only if it is \textbf{one-to-one} (each input gives a unique output).  
Graphically, a function and its inverse are reflections across the line \(y = x\).

(Insert graph notes: show \(f(x)\) and \(f^{-1}(x)\) as mirror images across \(y = x\).)

\section*{Core Skills}
\begin{itemize}
  \item Evaluate compositions like \(f(g(x))\) and \(g(f(x))\).
  \item Understand order in function composition.
  \item Find inverses algebraically using variable swap and solve.
  \item Recognize inverse pairs in tables or graphs.
  \item Test if two functions are inverses using \(f(g(x)) = x\).
\end{itemize}

\section*{Example 1: Function Composition}

If \(f(x) = 2x + 3\) and \(g(x) = x - 1,\) find:
\[
(f \circ g)(x) = f(g(x)) = f(x - 1) = 2(x - 1) + 3 = 2x + 1.
\]

\[
(g \circ f)(x) = g(f(x)) = g(2x + 3) = (2x + 3) - 1 = 2x + 2.
\]

\(\boxed{f(g(x)) = 2x + 1,\; g(f(x)) = 2x + 2}\)

\section*{Example 2: Composition with Quadratic Function}

If \(f(x) = x^2\) and \(g(x) = 3x + 2,\)
\[
(f \circ g)(x) = f(3x + 2) = (3x + 2)^2 = 9x^2 + 12x + 4.
\]
\[
(g \circ f)(x) = g(x^2) = 3x^2 + 2.
\]
\(\boxed{f(g(x)) = 9x^2 + 12x + 4,\; g(f(x)) = 3x^2 + 2}\)

\section*{Example 3: Finding an Inverse}

Find the inverse of \(f(x) = 3x - 5.\)

Step 1: Replace \(f(x)\) with \(y\): \(y = 3x - 5.\)

Step 2: Swap \(x\) and \(y\):
\[
x = 3y - 5.
\]

Step 3: Solve for \(y\):
\[
3y = x + 5 \Rightarrow y = \frac{x + 5}{3}.
\]

\(\boxed{f^{-1}(x) = \frac{x + 5}{3}}\)

\section*{Example 4: Verifying Inverses}

Given \(f(x) = 2x + 7\) and \(g(x) = \frac{x - 7}{2},\)
check if they are inverses.

Compute \(f(g(x))\):
\[
f(g(x)) = 2\left(\frac{x - 7}{2}\right) + 7 = x - 7 + 7 = x.
\]
and \(g(f(x))\):
\[
g(f(x)) = \frac{(2x + 7) - 7}{2} = \frac{2x}{2} = x.
\]

\(\boxed{f \text{ and } g \text{ are inverses}}\)

\section*{Example 5: Nonlinear Inverse}

Find the inverse of \(f(x) = \dfrac{x + 4}{3}.\)

\[
y = \frac{x + 4}{3} \quad \Rightarrow \quad x = \frac{y + 4}{3}.
\]
Multiply both sides by 3:
\[
3x = y + 4 \Rightarrow y = 3x - 4.
\]

\(\boxed{f^{-1}(x) = 3x - 4}\)

\section*{Example 6: Graphical Idea (description only)}

Graph \(f(x) = 2x + 1\) and its inverse \(f^{-1}(x) = \frac{x - 1}{2}\).  
They intersect at the point where \(x = y\).  
(Insert note: Graph shows both lines reflecting across \(y = x\).)

\section*{Key Takeaways}
\begin{itemize}
  \item Composition means applying one function to another.
  \item Order matters: \(f(g(x)) \neq g(f(x))\) in general.
  \item To find an inverse, swap \(x\) and \(y\) and solve for \(y\).
  \item The graph of an inverse reflects across \(y = x\).
  \item Verify inverses using composition: \(f(g(x)) = g(f(x)) = x.\)
\end{itemize}

\newpage

% ============================================================
% QUESTIONS — UNIT 7, TOPIC 5: COMPOSITION AND INVERSES OF FUNCTIONS
% ============================================================

\section*{Practice Questions: Composition and Inverses of Functions}

\subsection*{Part A: Evaluating Function Compositions}
\begin{enumerate}
  \item If \(f(x) = 2x + 3\) and \(g(x) = x - 1,\) find \(f(g(x))\).
  \item If \(f(x) = x^2\) and \(g(x) = 4x,\) find \(f(g(x))\) and \(g(f(x))\).
  \item If \(f(x) = 3x - 2\) and \(g(x) = x^2 + 1,\) find \(f(g(x))\).
  \item If \(f(x) = x + 5\) and \(g(x) = \dfrac{x}{2},\) find \(g(f(x))\).
  \item If \(f(x) = x^2 + 1\) and \(g(x) = 2x - 3,\) find \(f(g(x))\).
\end{enumerate}

\subsection*{Part B: Mixed Compositions (numeric and symbolic)}
\begin{enumerate}
  \setcounter{enumi}{5}
  \item Given \(f(x) = 3x + 2\) and \(g(x) = x^2,\) find \(f(g(2))\).
  \item For the same functions, find \(g(f(2))\).
  \item If \(f(x) = x - 4\) and \(g(x) = 5x,\) find \(f(g(x))\).
  \item If \(f(x) = \sqrt{x}\) and \(g(x) = x + 3,\) find \(f(g(x))\).
  \item If \(f(x) = x^2 - 2\) and \(g(x) = x + 1,\) find \(f(g(x))\).
\end{enumerate}

\subsection*{Part C: Finding Inverse Functions}
\begin{enumerate}
  \setcounter{enumi}{10}
  \item Find the inverse of \(f(x) = 2x + 5.\)
  \item Find the inverse of \(f(x) = \dfrac{x - 4}{3}.\)
  \item Find the inverse of \(f(x) = 5x - 7.\)
  \item Find the inverse of \(f(x) = \dfrac{x + 6}{2}.\)
  \item Find the inverse of \(f(x) = 3x + 1.\)
\end{enumerate}

\subsection*{Part D: Checking and Applying Inverses}
\begin{enumerate}
  \setcounter{enumi}{15}
  \item \(f(x) = 4x - 3,\; g(x) = \dfrac{x + 3}{4}\). Verify that \(f\) and \(g\) are inverses.
  \item \(f(x) = 2x + 1,\; g(x) = \dfrac{x - 1}{2}\). Compute \(f(g(x))\) to confirm.
  \item \(f(x) = 5x - 10.\) Find \(f^{-1}(x)\), then check \(f(f^{-1}(x)) = x.\)
  \item If \(f(x) = 3x + 2,\) what is \(f^{-1}(10)\)?
  \item If \(f(x) = \dfrac{x - 4}{2},\) find \(f^{-1}(8)\).
\end{enumerate}

\subsection*{Part E: SAT-Style Applications}
(Insert sketch later: line \(y = f(x)\) and its inverse reflected across \(y = x\).)

\begin{enumerate}
  \setcounter{enumi}{20}
  \item A function \(f(t) = 4t + 7\) gives the cost in dollars for \(t\) tickets.  
  What does \(f^{-1}(50)\) represent in context?
  \item The temperature in °C relates to °F by \(F(C) = \dfrac{9}{5}C + 32.\)  
  Find the inverse function \(C(F)\).
  \item The function \(f(x) = 2x - 5\) is graphed. What line is its inverse symmetric about?
  \item A function \(g(x)\) doubles a number and then subtracts 1.  
  Write \(g(x)\) and find its inverse.
  \item On a graph, \(f(x)\) and \(f^{-1}(x)\) intersect at what kind of point?
\end{enumerate}

\newpage

% ============================================================
% SOLUTIONS — UNIT 7, TOPIC 5: COMPOSITION AND INVERSES OF FUNCTIONS
% ============================================================

\section*{Answer Key and Solutions: Composition and Inverses of Functions}

\subsection*{Part A Solutions: Evaluating Function Compositions}
\begin{enumerate}
  \item \(f(x) = 2x + 3,\; g(x) = x - 1\)

  \[
  f(g(x)) = f(x - 1) = 2(x - 1) + 3 = 2x - 2 + 3 = 2x + 1.
  \]

  \(\boxed{f(g(x)) = 2x + 1}\)

  \item \(f(x) = x^2,\; g(x) = 4x\)

  \[
  f(g(x)) = f(4x) = (4x)^2 = 16x^2.
  \]
  \[
  g(f(x)) = g(x^2) = 4(x^2) = 4x^2.
  \]

  \(\boxed{f(g(x)) = 16x^2,\; g(f(x)) = 4x^2}\)

  \item \(f(x) = 3x - 2,\; g(x) = x^2 + 1\)

  \[
  f(g(x)) = f(x^2 + 1) = 3(x^2 + 1) - 2 = 3x^2 + 3 - 2 = 3x^2 + 1.
  \]

  \(\boxed{f(g(x)) = 3x^2 + 1}\)

  \item \(f(x) = x + 5,\; g(x) = \dfrac{x}{2}\)

  \[
  g(f(x)) = g(x + 5) = \dfrac{x + 5}{2}.
  \]

  \(\boxed{g(f(x)) = \dfrac{x + 5}{2}}\)

  \item \(f(x) = x^2 + 1,\; g(x) = 2x - 3\)

  \[
  f(g(x)) = f(2x - 3) = (2x - 3)^2 + 1 = 4x^2 - 12x + 9 + 1 = 4x^2 - 12x + 10.
  \]

  \(\boxed{f(g(x)) = 4x^2 - 12x + 10}\)
\end{enumerate}

\subsection*{Part B Solutions: Mixed Compositions (numeric and symbolic)}
\begin{enumerate}
  \setcounter{enumi}{5}
  \item \(f(x) = 3x + 2,\; g(x) = x^2\)

  First find \(g(2)\):
  \[
  g(2) = 2^2 = 4.
  \]

  Then:
  \[
  f(g(2)) = f(4) = 3(4) + 2 = 12 + 2 = 14.
  \]

  \(\boxed{f(g(2)) = 14}\)

  \item Same functions. Find \(g(f(2))\).

  First:
  \[
  f(2) = 3(2) + 2 = 8.
  \]

  Then:
  \[
  g(f(2)) = g(8) = 8^2 = 64.
  \]

  \(\boxed{g(f(2)) = 64}\)

  \item \(f(x) = x - 4,\; g(x) = 5x\)

  \[
  f(g(x)) = f(5x) = 5x - 4.
  \]

  \(\boxed{f(g(x)) = 5x - 4}\)

  \item \(f(x) = \sqrt{x},\; g(x) = x + 3\)

  \[
  f(g(x)) = f(x + 3) = \sqrt{x + 3}.
  \]

  \(\boxed{f(g(x)) = \sqrt{x + 3}}\)

  \item \(f(x) = x^2 - 2,\; g(x) = x + 1\)

  \[
  f(g(x)) = f(x + 1) = (x + 1)^2 - 2 = x^2 + 2x + 1 - 2 = x^2 + 2x - 1.
  \]

  \(\boxed{f(g(x)) = x^2 + 2x - 1}\)
\end{enumerate}

\subsection*{Part C Solutions: Finding Inverse Functions}
\begin{enumerate}
  \setcounter{enumi}{10}
  \item \(f(x) = 2x + 5\)

  Let \(y = 2x + 5.\) Swap \(x\) and \(y\):
  \[
  x = 2y + 5 \Rightarrow 2y = x - 5 \Rightarrow y = \frac{x - 5}{2}.
  \]

  \(\boxed{f^{-1}(x) = \dfrac{x - 5}{2}}\)

  \item \(f(x) = \dfrac{x - 4}{3}\)

  Let \(y = \dfrac{x - 4}{3}.\) Swap:
  \[
  x = \dfrac{y - 4}{3}.
  \]
  Multiply by 3:
  \[
  3x = y - 4 \Rightarrow y = 3x + 4.
  \]

  \(\boxed{f^{-1}(x) = 3x + 4}\)

  Note: a cleaner start is \(y = (x - 4)/3,\) then switch \(x\) and \(y\): \(x = (y - 4)/3.\)

  \item \(f(x) = 5x - 7\)

  Let \(y = 5x - 7.\) Swap:
  \[
  x = 5y - 7 \Rightarrow 5y = x + 7 \Rightarrow y = \frac{x + 7}{5}.
  \]

  \(\boxed{f^{-1}(x) = \dfrac{x + 7}{5}}\)

  \item \(f(x) = \dfrac{x + 6}{2}\)

  Let \(y = \dfrac{x + 6}{2}.\) Swap:
  \[
  x = \dfrac{y + 6}{2}.
  \]
  Multiply by 2:
  \[
  2x = y + 6 \Rightarrow y = 2x - 6.
  \]

  \(\boxed{f^{-1}(x) = 2x - 6}\)

  \item \(f(x) = 3x + 1\)

  Let \(y = 3x + 1.\) Swap:
  \[
  x = 3y + 1 \Rightarrow 3y = x - 1 \Rightarrow y = \frac{x - 1}{3}.
  \]

  \(\boxed{f^{-1}(x) = \dfrac{x - 1}{3}}\)
\end{enumerate}

\subsection*{Part D Solutions: Checking and Applying Inverses}
\begin{enumerate}
  \setcounter{enumi}{15}
  \item \(f(x) = 4x - 3,\; g(x) = \dfrac{x + 3}{4}\)

  Check \(f(g(x))\):
  \[
  f(g(x)) = 4\left(\frac{x + 3}{4}\right) - 3 = x + 3 - 3 = x.
  \]

  Check \(g(f(x))\):
  \[
  g(f(x)) = \frac{(4x - 3) + 3}{4} = \frac{4x}{4} = x.
  \]

  Since both compositions give \(x\), they are inverses.

  \(\boxed{f \text{ and } g \text{ are inverses}}\)

  \item \(f(x) = 2x + 1,\; g(x) = \dfrac{x - 1}{2}\)

  Compute \(f(g(x))\):
  \[
  f\bigg(\frac{x - 1}{2}\bigg) = 2\left(\frac{x - 1}{2}\right) + 1 = (x - 1) + 1 = x.
  \]

  \(\boxed{f(g(x)) = x \Rightarrow \text{yes, they are inverses}}\)

  \item \(f(x) = 5x - 10\)

  First, find inverse.

  Let \(y = 5x - 10.\) Swap:
  \[
  x = 5y - 10 \Rightarrow 5y = x + 10 \Rightarrow y = \frac{x + 10}{5}.
  \]

  So
  \[
  f^{-1}(x) = \frac{x + 10}{5}.
  \]

  Now check:
  \[
  f(f^{-1}(x)) = 5\left(\frac{x + 10}{5}\right) - 10 = (x + 10) - 10 = x.
  \]

  \(\boxed{f^{-1}(x) = \dfrac{x + 10}{5} \text{ and } f(f^{-1}(x)) = x}\)

  \item \(f(x) = 3x + 2.\) Find \(f^{-1}(10)\).

  First find inverse:
  \[
  y = 3x + 2 \Rightarrow x = 3y + 2 \Rightarrow 3y = x - 2 \Rightarrow y = \frac{x - 2}{3}.
  \]
  \[
  f^{-1}(x) = \frac{x - 2}{3}.
  \]

  Then:
  \[
  f^{-1}(10) = \frac{10 - 2}{3} = \frac{8}{3}.
  \]

  \(\boxed{f^{-1}(10) = \dfrac{8}{3}}\)

  \item \(f(x) = \dfrac{x - 4}{2}\)

  Solve for the inverse:
  \[
  y = \frac{x - 4}{2} \Rightarrow x = \frac{y - 4}{2}.
  \]
  Multiply:
  \[
  2x = y - 4 \Rightarrow y = 2x + 4.
  \]

  \[
  f^{-1}(x) = 2x + 4.
  \]

  Now evaluate:
  \[
  f^{-1}(8) = 2(8) + 4 = 16 + 4 = 20.
  \]

  \(\boxed{f^{-1}(8) = 20}\)
\end{enumerate}

\subsection*{Part E Solutions: SAT-Style Applications}
\begin{enumerate}
  \setcounter{enumi}{20}
  \item \(f(t) = 4t + 7\)

  \(f(t)\) is cost in dollars for \(t\) tickets.  
  \(f^{-1}(50)\) asks: for a total cost of \$50, how many tickets were bought.

  \(\boxed{\text{Number of tickets that cost \$50 total}}\)

  \item \(F(C) = \dfrac{9}{5}C + 32\)

  To find \(C(F)\), solve for \(C\) in terms of \(F\):
  \[
  F = \frac{9}{5}C + 32 \Rightarrow F - 32 = \frac{9}{5}C
  \Rightarrow C = \frac{5}{9}(F - 32).
  \]

  \(\boxed{C(F) = \dfrac{5}{9}(F - 32)}\)

  \item The line is always the same for a function and its inverse: \(y = x.\)

  \(\boxed{\text{They are symmetric across } y = x}\)

  \item "Doubles a number then subtracts 1"

  That gives:
  \[
  g(x) = 2x - 1.
  \]

  Find inverse:

  Let \(y = 2x - 1.\) Swap:
  \[
  x = 2y - 1 \Rightarrow 2y = x + 1 \Rightarrow y = \frac{x + 1}{2}.
  \]

  \(\boxed{g(x) = 2x - 1,\; g^{-1}(x) = \dfrac{x + 1}{2}}\)

  \item A function and its inverse intersect where \(f(x) = x.\)

  That means the input equals the output.  
  On the graph, these intersection points are on the line \(y = x.\)

  \(\boxed{\text{They intersect on the line } y = x \text{ where } f(x)=x}\)
\end{enumerate}



\end{document}
