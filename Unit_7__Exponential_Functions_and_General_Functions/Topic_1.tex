\documentclass[12pt]{article}

\usepackage{amsmath, amssymb}
\usepackage{geometry}
\usepackage{setspace}
\usepackage{titlesec}
\usepackage{lmodern}
\usepackage{xcolor}
\usepackage{enumitem}

\geometry{margin=1in}
\setstretch{1.2}
\titleformat{\section}{\normalfont\Large\bfseries}{\thesection}{1em}{}
\titleformat{\subsection}{\normalfont\large\bfseries}{\thesubsection}{1em}{}
\pagenumbering{gobble}

\begin{document}

\begin{center}
    \LARGE \textbf{Unit 7: Exponential and General Functions} \\[6pt]
    \Large \textbf{Topic 1: Exponential Growth and Decay}
\end{center}

\vspace{1em}

\section*{Concept Summary}

Exponential functions model situations where a quantity changes by a constant \textbf{percentage rate} over equal time intervals.  
They have the general form:
\[
y = a \, b^x,
\]
where:
\begin{itemize}
  \item \(a\) is the initial value (when \(x = 0\)),
  \item \(b\) is the growth or decay factor,
  \item \(x\) is the time or independent variable.
\end{itemize}

\[
\begin{cases}
b > 1 & \text{Exponential Growth} \\
0 < b < 1 & \text{Exponential Decay}
\end{cases}
\]

Each increase of 1 unit in \(x\) multiplies the value of \(y\) by \(b\), not adds to it.  
This is the key difference from linear functions, which change by a constant amount.

The percentage rate of change can be written as:
\[
b = 1 + r \quad \text{for growth}, \qquad b = 1 - r \quad \text{for decay},
\]
where \(r\) is the rate expressed as a decimal.

\subsection*{Exponential Growth Example}
Population, compound interest, or bacteria doubling over time follow growth patterns.

\[
P(t) = P_0 (1 + r)^t
\]

\subsection*{Exponential Decay Example}
Radioactive decay or depreciation of value follow decay patterns.

\[
A(t) = A_0 (1 - r)^t
\]

\section*{Core Skills}
\begin{itemize}
  \item Identify \(a\), \(b\), and the growth/decay rate \(r\).
  \item Distinguish between linear (additive) and exponential (multiplicative) change.
  \item Write exponential models from verbal descriptions or data.
  \item Evaluate exponential expressions for given time intervals.
  \item Interpret real-world meanings of constants \(a\), \(b\), and \(r\).
\end{itemize}

\section*{Example 1: Identifying Growth or Decay}

Determine whether each function represents growth or decay:
\[
\text{(a) } y = 200(1.05)^x, \quad \text{(b) } y = 500(0.92)^x.
\]

(a) \(b = 1.05 > 1\) → Growth (5% increase per unit).  
(b) \(b = 0.92 < 1\) → Decay (8% decrease per unit).

\textbf{Answer:} (a) Growth; (b) Decay.

\section*{Example 2: Writing an Exponential Model}

A population of 1000 bacteria doubles every 4 hours.  
\[
P(t) = 1000(2)^{t/4}.
\]
Each 4-hour period multiplies the amount by 2.  
\textbf{Interpretation:} \(a = 1000\), doubling time = 4 hours.

\section*{Example 3: Finding the Growth Rate}

An investment grows according to \(A = 500(1.06)^t.\)  
Here \(r = 0.06,\) so growth rate = 6% per time unit.  
After 5 years:
\[
A = 500(1.06)^5 = 500(1.3382) = 669.1.
\]
\(\boxed{A = \$669.10}\)

\section*{Example 4: Exponential Decay Model}

A car’s value is modeled by \(V = 30{,}000(0.85)^t,\) where \(t\) is years after purchase.  
Each year, it retains 85% of its previous value → 15% loss annually.  
After 3 years:
\[
V = 30{,}000(0.85)^3 = 30{,}000(0.614) = 18{,}420.
\]
\(\boxed{V = \$18{,}420}\)

\section*{Example 5: Comparing Two Models}

Compare \(y = 100(1.03)^x\) and \(y = 100(0.97)^x.\)

The first model grows 3% each period; the second decays 3%.  
If \(x = 10\):
\[
y_1 = 100(1.03)^{10} = 134.4, \quad y_2 = 100(0.97)^{10} = 73.7.
\]

\textbf{Interpretation:} Growth doubles faster than decay reduces.

\section*{Example 6: Graph Behavior}

(Insert exponential graphs here:  
— One for growth with \(b > 1\) rising rapidly;  
— One for decay with \(0 < b < 1\) decreasing toward 0.)

Key features:
\begin{itemize}
  \item Both pass through \((0, a)\).
  \item Growth curves rise to the right.
  \item Decay curves fall to the right.
  \item The x-axis (\(y = 0\)) is a horizontal asymptote.
\end{itemize}

\section*{Key Takeaways}
\begin{itemize}
  \item Exponential growth multiplies by a constant factor \(b > 1\).
  \item Exponential decay multiplies by \(0 < b < 1\).
  \item Rate \(r\) links to \(b\) through \(b = 1 \pm r.\)
  \item Graphs approach but never reach 0 (asymptote at \(y = 0\)).
  \item Recognizing \(b\) and interpreting context are key SAT skills.
\end{itemize}

\newpage

% ============================================================
% QUESTIONS — UNIT 7, TOPIC 1: EXPONENTIAL GROWTH AND DECAY
% ============================================================

\section*{Practice Questions: Exponential Growth and Decay}

\subsection*{Part A: Identifying Growth or Decay}
\begin{enumerate}
  \item Determine whether each represents growth or decay:  
  \(y = 250(1.08)^x\)
  \item Determine whether \(y = 600(0.97)^x\) represents growth or decay.
  \item Classify \(y = 1200(1.02)^x\) and \(y = 500(0.6)^x\) as growth or decay.
  \item Identify the growth rate \(r\) for \(y = 200(1.15)^x.\)
  \item Identify the decay rate \(r\) for \(y = 400(0.9)^x.\)
\end{enumerate}

\subsection*{Part B: Writing and Interpreting Models}
\begin{enumerate}
  \setcounter{enumi}{5}
  \item A population of 500 bacteria triples every 6 hours. Write an exponential model for \(t\) hours.
  \item A \$1000 investment increases by 5\% per year. Write a model for its value after \(t\) years.
  \item A new car worth \$25,000 loses 12\% of its value per year. Write the decay model.
  \item A medicine dose of 200 mg decreases by 30\% every hour. Write the decay formula.
  \item A culture of 1500 cells doubles every 8 hours. Write a model to represent this growth.
\end{enumerate}

\subsection*{Part C: Evaluating Exponential Expressions}
\begin{enumerate}
  \setcounter{enumi}{10}
  \item Evaluate \(y = 300(1.04)^5.\)
  \item Evaluate \(A = 500(0.95)^{10}.\)
  \item Find the value of \(P = 1000(1.12)^3.\)
  \item A computer’s value is \(V = 1200(0.8)^t.\) Find \(V\) after 4 years.
  \item A population grows according to \(P = 200(1.07)^t.\) Find \(P\) when \(t = 6.\)
\end{enumerate}

\subsection*{Part D: Real-World Context and Reasoning}
\begin{enumerate}
  \setcounter{enumi}{15}
  \item A bank account balance grows from \$500 to \$800 in 10 years. Estimate the growth factor \(b.\)
  \item A radioactive substance decays by 40\% each hour. What is the decay factor \(b\)?
  \item A population decreases from 10,000 to 4900 in 5 years. What is the approximate decay rate per year?
  \item A savings account doubles in value every 9 years. What is the growth factor \(b\)?
  \item A town’s population follows \(P = 20{,}000(0.98)^t.\) What does 0.98 represent?
\end{enumerate}

\subsection*{Part E: SAT-Style Applications}
(Insert appropriate graphs here: one showing growth and one showing decay over time.)

\begin{enumerate}
  \setcounter{enumi}{20}
  \item The graph of \(y = 200(1.03)^x\) passes through which key point?
  A) \((0, 0)\) \quad B) \((0, 200)\) \quad C) \((1, 200)\) \quad D) \((1, 206)\)

  \item An exponential function models a cooling process: \(T = 70 + 100(0.85)^t.\)  
  What is the starting temperature?

  \item A town’s population is 10,000 and grows 2\% per year.  
  Estimate the population after 10 years.

  \item A \$10,000 car depreciates at 15\% per year.  
  After how many years will it be worth half its original value? (Round to nearest tenth.)

  \item Two investments are modeled by  
  \(A = 1000(1.05)^t\) and \(B = 1200(1.03)^t.\)  
  After 5 years, which investment is worth more, and by how much?
\end{enumerate}

\newpage

% ============================================================
% SOLUTIONS — UNIT 7, TOPIC 1: EXPONENTIAL GROWTH AND DECAY
% ============================================================

\section*{Answer Key and Solutions: Exponential Growth and Decay}

\subsection*{Part A Solutions: Identifying Growth or Decay}
\begin{enumerate}
  \item \(y = 250(1.08)^x\)

  Factor \(b = 1.08 > 1\). This is growth at 8\% per time unit.

  \(\boxed{\text{Growth, } r = 8\%}\)

  \item \(y = 600(0.97)^x\)

  Factor \(b = 0.97 < 1\). This is decay at 3\% per time unit (keeps 97\%, loses 3\%).

  \(\boxed{\text{Decay, } r = 3\%}\)

  \item
  \[
  y = 1200(1.02)^x \quad \Rightarrow b = 1.02 > 1 \Rightarrow \text{Growth, } 2\% \text{ rate}
  \]
  \[
  y = 500(0.6)^x \quad \Rightarrow b = 0.6 < 1 \Rightarrow \text{Decay, } 40\% \text{ drop each step}
  \]

  \(\boxed{\text{First is growth, second is decay}}\)

  \item \(y = 200(1.15)^x\)

  \(b = 1.15 = 1 + r \Rightarrow r = 0.15 = 15\%.\)

  \(\boxed{15\% \text{ growth rate}}\)

  \item \(y = 400(0.9)^x\)

  \(b = 0.9 = 1 - r \Rightarrow r = 0.1 = 10\%.\)

  \(\boxed{10\% \text{ decay rate}}\)
\end{enumerate}

\subsection*{Part B Solutions: Writing and Interpreting Models}
\begin{enumerate}
  \setcounter{enumi}{5}
  \item Population 500, triples every 6 hours.

  Every 6 hours multiply by 3. Time \(t\) in hours:
  \[
  P(t) = 500 \cdot 3^{t/6}.
  \]

  \(\boxed{P(t) = 500 \cdot 3^{t/6}}\)

  \item \$1000 investment, 5\% increase per year.

  Growth factor \(1 + 0.05 = 1.05\):
  \[
  A(t) = 1000(1.05)^t.
  \]

  \(\boxed{A(t) = 1000(1.05)^t}\)

  \item Car \$25{,}000, loses 12\% per year.

  Keeps 88\% each year. Decay factor \(0.88\):
  \[
  V(t) = 25{,}000(0.88)^t.
  \]

  \(\boxed{V(t) = 25{,}000(0.88)^t}\)

  \item Medicine 200 mg, decreases 30\% per hour.

  Keeps 70\% each hour. Factor \(0.70\):
  \[
  M(t) = 200(0.70)^t.
  \]

  \(\boxed{M(t) = 200(0.70)^t}\)

  \item 1500 cells, doubles every 8 hours.

  Doubling means factor 2 each 8 hours:
  \[
  P(t) = 1500 \cdot 2^{t/8}.
  \]

  \(\boxed{P(t) = 1500 \cdot 2^{t/8}}\)
\end{enumerate}

\subsection*{Part C Solutions: Evaluating Exponential Expressions}
\begin{enumerate}
  \setcounter{enumi}{10}
  \item \(y = 300(1.04)^5\)

  \((1.04)^5 \approx 1.2167.\)

  \[
  y \approx 300 \times 1.2167 \approx 365.0.
  \]

  \(\boxed{y \approx 365.0}\)

  \item \(A = 500(0.95)^{10}\)

  \((0.95)^{10} \approx 0.5987.\)

  \[
  A \approx 500 \times 0.5987 \approx 299.4.
  \]

  \(\boxed{A \approx 299.4}\)

  \item \(P = 1000(1.12)^3\)

  \((1.12)^3 \approx 1.4049.\)

  \[
  P \approx 1000 \times 1.4049 \approx 1404.9.
  \]

  \(\boxed{P \approx 1404.9}\)

  \item \(V = 1200(0.8)^t,\; t=4\)

  \((0.8)^4 = 0.4096.\)

  \[
  V \approx 1200 \times 0.4096 \approx 491.5.
  \]

  \(\boxed{V \approx 491.5}\)

  \item \(P = 200(1.07)^t,\; t=6\)

  \((1.07)^6 \approx 1.5007.\)

  \[
  P \approx 200 \times 1.5007 \approx 300.1.
  \]

  \(\boxed{P \approx 300.1}\)
\end{enumerate}

\subsection*{Part D Solutions: Real-World Context and Reasoning}
\begin{enumerate}
  \setcounter{enumi}{15}
  \item Account goes from 500 to 800 in 10 years.

  Model: \(500 \cdot b^{10} = 800.\)

  \[
  b^{10} = \frac{800}{500} = 1.6
  \quad \Rightarrow \quad
  b \approx 1.048.
  \]

  So about \(4.8\%\) growth per year.

  \(\boxed{b \approx 1.048 \text{ (about 4.8\% growth per year)}}\)

  \item A substance loses 40\% each hour.

  Keeps 60\% each hour. So
  \[
  b = 0.60.
  \]

  \(\boxed{b = 0.60}\)

  \item Population drops from 10{,}000 to 4900 in 5 years.

  Model: \(10{,}000 \cdot b^5 = 4900.\)

  \[
  b^5 = 0.49 \quad \Rightarrow \quad b \approx 0.867.
  \]

  This means it keeps about 86.7\% each year, so decay rate is about 13.3\% per year.

  \(\boxed{\text{About } 13\% \text{ decrease per year}}\)

  \item A savings account doubles every 9 years.

  Doubling means factor 2 in 9 years:
  \[
  b = 2^{1/9} \approx 1.080.
  \]

  So about 8\% growth per year.

  \(\boxed{b \approx 1.08 \text{ (about 8\% per year)}}\)

  \item \(P = 20{,}000(0.98)^t\)

  The factor 0.98 means each year the town keeps 98\% of its population.

  That is a 2\% decrease per year.

  \(\boxed{\text{Population drops 2\% per year}}\)
\end{enumerate}

\subsection*{Part E Solutions: SAT-Style Applications}
\begin{enumerate}
  \setcounter{enumi}{20}
  \item \(y = 200(1.03)^x\)

  At \(x = 0\): \(y = 200(1.03)^0 = 200.\)

  The point \((0, 200)\) is always on an exponential \(a b^x\).

  \(\boxed{\text{Choice B } (0, 200)}\)

  \item \(T = 70 + 100(0.85)^t\)

  At \(t = 0\):
  \[
  T = 70 + 100(1) = 170.
  \]

  Starting temperature is 170.

  \(\boxed{170^\circ}\)

  \item Population 10{,}000 grows 2\% per year.

  Model:
  \[
  P(t) = 10{,}000(1.02)^t.
  \]

  After 10 years:
  \[
  P(10) = 10{,}000(1.02)^{10} \approx 10{,}000 \times 1.219 \approx 12{,}190.
  \]

  \(\boxed{\text{About } 12{,}200}\)

  \item Car \$10{,}000 depreciates 15\% per year.

  Value model:
  \[
  V = 10{,}000(0.85)^t.
  \]

  Half value is \$5{,}000:
  \[
  10{,}000(0.85)^t = 5{,}000
  \Rightarrow (0.85)^t = 0.5.
  \]

  Solve:
  \[
  t = \frac{\ln(0.5)}{\ln(0.85)} \approx 4.3 \text{ years}.
  \]

  \(\boxed{4.3 \text{ years (approx)}}\)

  \item
  \[
  A = 1000(1.05)^5, \quad
  B = 1200(1.03)^5.
  \]

  Compute:
  \[
  A \approx 1276.28,
  \quad
  B \approx 1391.13.
  \]

  \(B\) is larger.

  Difference:
  \[
  1391.13 - 1276.28 \approx 114.85.
  \]

  \(\boxed{\text{Investment B is larger by about \$115}}\)
\end{enumerate}



\end{document}
