\documentclass[12pt]{article}

\usepackage{amsmath, amssymb}
\usepackage{geometry}
\usepackage{setspace}
\usepackage{titlesec}
\usepackage{lmodern}
\usepackage{xcolor}
\usepackage{enumitem}

\geometry{margin=1in}
\setstretch{1.2}
\titleformat{\section}{\normalfont\Large\bfseries}{\thesection}{1em}{}
\titleformat{\subsection}{\normalfont\large\bfseries}{\thesubsection}{1em}{}
\pagenumbering{gobble}

\begin{document}

\begin{center}
    \LARGE \textbf{Unit 7: Exponential and General Functions} \\[6pt]
    \Large \textbf{Topic 6: Transformations — Shifts, Reflections, and Stretches}
\end{center}

\vspace{1em}

\section*{Concept Summary}

Function transformations describe how the graph of a base function changes when certain values are added, subtracted, or multiplied.  
The general form of a transformed function is:
\[
f(x) \rightarrow a\,f(b(x - h)) + k
\]

Each parameter affects the graph in a specific way:

\[
\begin{array}{c|l}
\text{Parameter} & \text{Effect on Graph} \\ \hline
h & \text{Horizontal shift: } +h \text{ right, } -h \text{ left} \\
k & \text{Vertical shift: } +k \text{ up, } -k \text{ down} \\
a & \text{Vertical stretch/shrink: } |a|>1 \text{ stretches, } |a|<1 \text{ shrinks} \\
a<0 & \text{Reflection over the } x\text{-axis} \\
b & \text{Horizontal stretch/shrink or reflection if negative}
\end{array}
\]

SAT questions often present transformations in function notation, e.g. \\
“If \(g(x) = f(x - 3) + 2\), how does the graph of \(f\) move?”

(Insert graph notes: show one original \(f(x)\), one shifted right and up, and one reflected across the x-axis.)

\section*{Core Skills}
\begin{itemize}
  \item Identify transformations from function equations.
  \item Describe shifts, reflections, and stretches verbally or graphically.
  \item Write new function equations after a given transformation.
  \item Match equations to transformed graphs.
  \item Understand the difference between horizontal and vertical changes.
\end{itemize}

\section*{Example 1: Horizontal and Vertical Shifts}

Let \(f(x) = x^2.\)

\[
g(x) = f(x - 2) + 3 = (x - 2)^2 + 3
\]

This means the parabola moves **right 2 units** and **up 3 units.**

\textbf{New vertex:} from \((0,0)\) to \((2,3)\).

(Insert graph note: original \(x^2\), shifted right and up.)

\section*{Example 2: Reflection}

\[
f(x) = x^2 \quad \Rightarrow \quad g(x) = -f(x) = -x^2
\]

The negative sign reflects the graph across the **x-axis** — opening downward.

\section*{Example 3: Vertical Stretch}

\[
f(x) = x^2, \quad g(x) = 3f(x) = 3x^2
\]

The graph is \textbf{stretched vertically} by a factor of 3 — it becomes narrower.

\section*{Example 4: Horizontal Stretch}

\[
f(x) = x^2, \quad g(x) = f\left(\frac{x}{2}\right) = \left(\frac{x}{2}\right)^2 = \frac{1}{4}x^2
\]

The graph is stretched horizontally by a factor of 2 — points move farther from the y-axis.

\section*{Example 5: Combined Transformation}

If \(f(x) = |x|\), find and describe:
\[
g(x) = -2f(x + 1) + 3 = -2|x + 1| + 3
\]

Transformations:
\begin{itemize}
  \item Shift left 1 (inside \(x + 1\))
  \item Reflect across x-axis (negative sign)
  \item Vertical stretch by 2
  \item Shift up 3
\end{itemize}

(Insert graph note: base V-shape, moved left, flipped, stretched, raised.)

\section*{Example 6: SAT Context Example}

If \(f(x)\) represents the height of a ball over time,  
then \(f(x - 2)\) represents the same motion **2 seconds later**,  
and \(f(x) + 3\) means the ball starts **3 meters higher.**

\section*{Key Takeaways}
\begin{itemize}
  \item \(f(x - h)\) shifts right, \(f(x + h)\) shifts left.
  \item \(f(x) + k\) moves up, \(f(x) - k\) moves down.
  \item Multiply by \(a\) to stretch/shrink vertically; negative \(a\) reflects.
  \item Multiply inside argument by \(b\) to stretch/shrink horizontally; negative \(b\) reflects over y-axis.
  \item Combine multiple transformations carefully, applying inside changes first.
\end{itemize}

\newpage

% ============================================================
% QUESTIONS — UNIT 7, TOPIC 6: TRANSFORMATIONS — SHIFTS, REFLECTIONS, AND STRETCHES
% ============================================================

\section*{Practice Questions: Transformations — Shifts, Reflections, and Stretches}

\subsection*{Part A: Identifying Shifts}
\begin{enumerate}
  \item If \(f(x) = x^2,\) describe the transformation to get \(g(x) = f(x - 4).\)
  \item If \(f(x) = |x|,\) describe the transformation for \(g(x) = f(x + 3).\)
  \item Given \(f(x) = x^3,\) write the equation for a function shifted up 5 units.
  \item Write the equation of a function shifted down 2 units if \(f(x) = \sqrt{x}.\)
  \item Describe how the graph of \(g(x) = f(x - 1) - 6\) compares to \(f(x).\)
\end{enumerate}

\subsection*{Part B: Reflections and Vertical Changes}
\begin{enumerate}
  \setcounter{enumi}{5}
  \item If \(f(x) = x^2,\) describe the transformation for \(g(x) = -f(x).\)
  \item If \(f(x) = |x|,\) describe \(g(x) = -f(x) + 4.\)
  \item Write the equation of the reflection of \(f(x) = x^3\) over the x-axis.
  \item Write the equation for a vertical stretch by 3 of \(f(x) = x^2.\)
  \item Write the equation for a vertical shrink by ½ of \(f(x) = |x|.\)
\end{enumerate}

\subsection*{Part C: Horizontal Stretches and Reflections}
\begin{enumerate}
  \setcounter{enumi}{10}
  \item Describe the transformation for \(g(x) = f(2x)\).
  \item Describe the transformation for \(g(x) = f\left(\frac{x}{3}\right)\).
  \item If \(f(x) = x^2,\) what does \(g(x) = f(-x)\) do to the graph?
  \item If \(f(x) = \sqrt{x},\) what is the effect of \(g(x) = f(-x)\)?
  \item Given \(f(x) = |x|,\) describe the transformation for \(g(x) = f(-x + 2)\).
\end{enumerate}

\subsection*{Part D: Combined Transformations}
\begin{enumerate}
  \setcounter{enumi}{15}
  \item Describe all transformations for \(g(x) = -f(x - 2) + 1.\)
  \item Write the equation for a function that reflects \(f(x) = x^2\) over the x-axis and shifts right 3 units.
  \item Write the equation for \(f(x) = |x|\) shifted left 2 and up 5.
  \item Describe the transformations for \(g(x) = 3f(-x + 1) - 4.\)
  \item If \(f(x) = \sqrt{x},\) write \(g(x)\) that reflects over x-axis, shifts right 2, up 1.
\end{enumerate}

\subsection*{Part E: SAT-Style Applications}
(Insert graph notes later: show original parabola and transformed versions.)

\begin{enumerate}
  \setcounter{enumi}{20}
  \item The graph of \(y = f(x)\) is shown with vertex at \((0,0)\).  
  The vertex of \(y = f(x - 5) + 2\) is where?
  \item A parabola \(y = f(x)\) opens upward.  
  What happens to its graph for \(y = -f(x) + 3\)?
  \item The graph of \(f(x)\) passes through \((2, 4)\).  
  What point will be on \(g(x) = f(x - 3) + 1\)?
  \item If \(f(x)\) models height over time, what does \(f(x - 2)\) represent physically?
  \item The graph of \(y = 2f(x + 1) - 4\) can be obtained from \(f(x)\) by what sequence of transformations?
\end{enumerate}

\newpage

% ============================================================
% SOLUTIONS — UNIT 7, TOPIC 6: TRANSFORMATIONS — SHIFTS, REFLECTIONS, AND STRETCHES
% ============================================================

\section*{Answer Key and Solutions: Transformations — Shifts, Reflections, and Stretches}

\subsection*{Part A Solutions: Identifying Shifts}
\begin{enumerate}
  \item \(f(x) = x^2,\; g(x) = f(x - 4) = (x - 4)^2\)

  \(x - 4\) means shift the graph \textbf{right 4 units}.

  \(\boxed{\text{Right 4}}\)

  \item \(f(x) = |x|,\; g(x) = f(x + 3) = |x + 3|\)

  \(x + 3\) means shift the graph \textbf{left 3 units}.

  \(\boxed{\text{Left 3}}\)

  \item \(f(x) = x^3\). Shift up 5 units:

  Add 5 outside:
  \[
  g(x) = f(x) + 5 = x^3 + 5.
  \]

  \(\boxed{g(x) = x^3 + 5}\)

  \item \(f(x) = \sqrt{x}\). Shift down 2 units:

  Subtract 2 outside:
  \[
  g(x) = f(x) - 2 = \sqrt{x} - 2.
  \]

  \(\boxed{g(x) = \sqrt{x} - 2}\)

  \item \(g(x) = f(x - 1) - 6\)

  \(x - 1\): shift \textbf{right 1}.  
  \(-6\): shift \textbf{down 6}.

  \(\boxed{\text{Right 1, down 6}}\)
\end{enumerate}

\subsection*{Part B Solutions: Reflections and Vertical Changes}
\begin{enumerate}
  \setcounter{enumi}{5}
  \item \(f(x) = x^2,\; g(x) = -f(x) = -x^2\)

  The negative sign reflects across the \textbf{x-axis}. Parabola now opens down.

  \(\boxed{\text{Reflection over x-axis}}\)

  \item \(f(x) = |x|,\; g(x) = -f(x) + 4 = -|x| + 4\)

  Steps:
  \begin{itemize}
    \item \(-|x|\): reflect over x-axis.
    \item \(+4\): shift up 4.
  \end{itemize}

  \(\boxed{\text{Reflect over x-axis, then up 4}}\)

  \item Reflection of \(f(x) = x^3\) over x-axis:

  Multiply by \(-1\):
  \[
  g(x) = -x^3.
  \]

  \(\boxed{g(x) = -x^3}\)

  \item Vertical stretch by 3 of \(f(x) = x^2\):

  Multiply output by 3:
  \[
  g(x) = 3x^2.
  \]

  \(\boxed{g(x) = 3x^2}\)

  \item Vertical shrink by \(1/2\) of \(f(x) = |x|\):

  Multiply output by \(1/2\):
  \[
  g(x) = \frac{1}{2}|x|.
  \]

  \(\boxed{g(x) = \tfrac{1}{2}|x|}\)
\end{enumerate}

\subsection*{Part C Solutions: Horizontal Stretches and Reflections}
\begin{enumerate}
  \setcounter{enumi}{10}
  \item \(g(x) = f(2x)\)

  Replacing \(x\) with \(2x\) \textbf{compresses horizontally} by factor 2. Points move closer to the y-axis.

  \(\boxed{\text{Horizontal shrink by factor 2}}\)

  \item \(g(x) = f\left(\frac{x}{3}\right)\)

  Replacing \(x\) with \(x/3\) \textbf{stretches horizontally} by factor 3. Points move farther from the y-axis.

  \(\boxed{\text{Horizontal stretch by factor 3}}\)

  \item \(f(x) = x^2,\; g(x) = f(-x) = (-x)^2 = x^2\)

  For \(x^2\), reflecting over the y-axis gives the same graph (it is symmetric). No visible change.

  \(\boxed{\text{Reflection over y-axis, but looks the same}}\)

  \item \(f(x) = \sqrt{x},\; g(x) = f(-x) = \sqrt{-x}\)

  \(\sqrt{-x}\) is only defined when \(-x \ge 0 \Rightarrow x \le 0.\)  
  This is a reflection of the right-hand square root across the \textbf{y-axis} into the negative \(x\) side.

  \(\boxed{\text{Reflect over y-axis, keeps only } x \le 0}\)

  \item \(f(x) = |x|,\; g(x) = f(-x + 2) = |-x + 2|\)

  Inside transformation: \(-x + 2 = -(x - 2)\).  
  This is a reflection across the y-axis (because of the negative on \(x\)) and then a shift right 2.

  In words:
  \[
  \text{Reflect over y-axis, then shift right 2.}
  \]

  \(\boxed{\text{Reflect over y-axis, then right 2}}\)
\end{enumerate}

\subsection*{Part D Solutions: Combined Transformations}
\begin{enumerate}
  \setcounter{enumi}{15}
  \item \(g(x) = -f(x - 2) + 1\)

  \begin{itemize}
    \item \(x - 2\): shift right 2.
    \item Negative sign in front: reflect over x-axis.
    \item \(+1\): shift up 1.
  \end{itemize}

  \(\boxed{\text{Right 2, reflect over x-axis, up 1}}\)

  \item Reflect \(f(x) = x^2\) over x-axis and shift right 3.

  Reflection over x-axis: \(-x^2.\)  
  Shift right 3: replace \(x\) with \(x - 3\).

  \[
  g(x) = -(x - 3)^2.
  \]

  \(\boxed{g(x) = -(x - 3)^2}\)

  \item \(f(x) = |x|\). Shift left 2 and up 5.

  Left 2: \(|x + 2|\).  
  Up 5: add 5.

  \[
  g(x) = |x + 2| + 5.
  \]

  \(\boxed{g(x) = |x + 2| + 5}\)

  \item \(g(x) = 3f(-x + 1) - 4\)

  Read inside-out:
  \begin{itemize}
    \item \(-x + 1 = -(x - 1)\): reflect over y-axis, then shift right 1.
    \item \(3f(\cdot)\): vertical stretch by factor 3.
    \item \(-4\): shift down 4.
  \end{itemize}

  \(\boxed{\text{Reflect over y-axis, shift right 1, vertical stretch by 3, down 4}}\)

  \item \(f(x) = \sqrt{x}\). We want: reflect over x-axis, shift right 2, up 1.

  Start with \(f(x) = \sqrt{x}\).

  Shift right 2: replace \(x\) with \(x - 2\):
  \[
  \sqrt{x - 2}.
  \]

  Reflect over x-axis: multiply by \(-1\):
  \[
  -\sqrt{x - 2}.
  \]

  Shift up 1: add 1:
  \[
  g(x) = -\sqrt{x - 2} + 1.
  \]

  \(\boxed{g(x) = -\sqrt{x - 2} + 1}\)
\end{enumerate}

\subsection*{Part E Solutions: SAT-Style Applications}
\begin{enumerate}
  \setcounter{enumi}{20}
  \item Original vertex at \((0,0)\). For \(f(x - 5) + 2\):

  \(x - 5\) shifts right 5. \(+2\) shifts up 2.

  New vertex: \(\boxed{(5, 2)}\)

  \item \(y = -f(x) + 3\)

  \(-f(x)\): reflect over x-axis.  
  \(+3\): shift up 3.

  The parabola will now open downward and be moved up 3.

  \(\boxed{\text{Reflected over x-axis, then shifted up 3}}\)

  \item If \((2,4)\) is on \(f(x)\), then for \(g(x) = f(x - 3) + 1\):

  We need \(x - 3 = 2 \Rightarrow x = 5.\)  
  Output becomes \(4 + 1 = 5.\)

  So the point moves to \((5,5)\).

  \(\boxed{(5, 5)}\)

  \item \(f(x - 2)\) means the same motion happens 2 units later in \(x\).  
  In time language: the event is delayed by 2 units of time.

  \(\boxed{\text{The same behavior, but starting 2 seconds later}}\)

  \item \(y = 2f(x + 1) - 4\)

  Step by step:
  \begin{itemize}
    \item \(x + 1\): shift left 1.
    \item Multiply by 2: vertical stretch by factor 2.
    \item \(-4\): shift down 4.
  \end{itemize}

  \(\boxed{\text{Left 1, vertical stretch by 2, down 4}}\)
\end{enumerate}



\end{document}
