\documentclass[12pt]{article}

\usepackage{amsmath, amssymb}
\usepackage{geometry}
\usepackage{setspace}
\usepackage{titlesec}
\usepackage{lmodern}
\usepackage{xcolor}
\usepackage{enumitem}

\geometry{margin=1in}
\setstretch{1.2}
\titleformat{\section}{\normalfont\Large\bfseries}{\thesection}{1em}{}
\titleformat{\subsection}{\normalfont\large\bfseries}{\thesubsection}{1em}{}
\pagenumbering{gobble}

\begin{document}

\begin{center}
    \LARGE \textbf{Unit 7: Exponential and General Functions} \\[6pt]
    \Large \textbf{Topic 4: Domain, Range, and Restrictions}
\end{center}

\vspace{1em}

\section*{Concept Summary}

Every function has a \textbf{domain} (the set of all valid inputs \(x\)) and a \textbf{range} (the set of resulting outputs \(y\)).

- The \textbf{domain} includes all \(x\)-values that make the function defined (no division by zero, no square roots of negatives for real numbers).
- The \textbf{range} includes all possible \(y\)-values the function can produce.

SAT questions often test your ability to find where a function is \emph{undefined} or identify valid intervals of input.

\[
f(x) = \frac{1}{x - 2} \quad \text{is undefined at } x = 2,
\]
so its domain is \(x \neq 2.\)

For square roots:
\[
f(x) = \sqrt{x - 3} \quad \text{requires } x - 3 \ge 0,
\]
so domain is \(x \ge 3.\)

The range depends on the resulting values of \(f(x)\):
\[
\text{If } f(x) = \sqrt{x - 3}, \text{ then } f(x) \ge 0.
\]

(Insert sample graphs: one for a rational function with a vertical asymptote, one for a square root starting at a point.)

\section*{Core Skills}
\begin{itemize}
  \item Determine domain restrictions from denominators and even roots.
  \item Identify the range based on the output behavior.
  \item Describe domain/range in set notation or inequality form.
  \item Recognize asymptotes or endpoints in graphs.
  \item Apply restrictions when combining functions.
\end{itemize}

\section*{Example 1: Rational Function Restriction}

\[
f(x) = \frac{2x + 1}{x - 3}
\]

Denominator \(x - 3 = 0\) when \(x = 3.\)  
So \(x = 3\) is not allowed.

\[
\boxed{\text{Domain: } x \neq 3}
\]

\textbf{Range:} All real \(y\) except \(y = 2\) (horizontal asymptote).

\section*{Example 2: Square Root Function}

\[
f(x) = \sqrt{x + 4}
\]

Inside the root must be nonnegative:
\[
x + 4 \ge 0 \Rightarrow x \ge -4.
\]

\[
\boxed{\text{Domain: } x \ge -4, \quad \text{Range: } y \ge 0}
\]

\section*{Example 3: Combined Function Restriction}

\[
f(x) = \frac{\sqrt{x - 2}}{x - 5}
\]

Root requires \(x - 2 \ge 0 \Rightarrow x \ge 2\).  
Denominator requires \(x \neq 5\).

\[
\boxed{\text{Domain: } x \ge 2, \; x \neq 5}
\]

\section*{Example 4: Absolute Value Function}

\[
f(x) = |x - 3|
\]

Absolute value is always defined, so:
\[
\boxed{\text{Domain: all real numbers, Range: } y \ge 0.}
\]

Graphically, it forms a “V” shape with vertex at \((3,0)\).

(Insert graph note: vertex at (3,0), symmetric V opening upward.)

\section*{Example 5: Quadratic Function}

\[
f(x) = (x - 1)^2 + 2
\]

This is always defined (domain: all real).  
The smallest value occurs when \(x = 1\):
\[
f(1) = (1 - 1)^2 + 2 = 2.
\]

\[
\boxed{\text{Domain: all real } x, \; \text{Range: } y \ge 2.}
\]

Graph has vertex at \((1, 2)\) and opens upward.

(Insert graph note: parabola with vertex (1,2), opens upward.)

\section*{Example 6: Piecewise Function}

\[
f(x) =
\begin{cases}
x + 2, & x < 0 \\
x^2, & x \ge 0
\end{cases}
\]

Domain: all real (each piece covers all \(x\)).  
Range: \(y \ge 0\) because \(x^2 \ge 0\) and \(x+2\) can go below zero when \(x < -2.\)

\[
\boxed{\text{Domain: all real, Range: } y \ge -2.}
\]

\section*{Key Takeaways}
\begin{itemize}
  \item Denominators cannot equal zero.
  \item Even roots (square roots) must have nonnegative insides.
  \item Domains tell where the function exists; ranges tell possible outputs.
  \item Graph endpoints and asymptotes reveal domain/range visually.
  \item Always check both when simplifying or combining functions.
\end{itemize}

\newpage

% ============================================================
% QUESTIONS — UNIT 7, TOPIC 4: DOMAIN, RANGE, AND RESTRICTIONS
% ============================================================

\section*{Practice Questions: Domain, Range, and Restrictions}

\subsection*{Part A: Identifying Domain Restrictions}
\begin{enumerate}
  \item Find the domain of \(f(x) = \dfrac{3x + 1}{x - 4}.\)
  \item Find the domain of \(f(x) = \dfrac{2}{x^2 - 9}.\)
  \item Determine where \(f(x) = \dfrac{1}{x(x - 5)}\) is undefined.
  \item Find the domain of \(f(x) = \dfrac{x + 2}{x^2 + x - 6}.\)
  \item For \(f(x) = \dfrac{5}{(x + 1)(x - 2)},\) state all excluded \(x\)-values.
\end{enumerate}

\subsection*{Part B: Square Roots and Even Roots}
\begin{enumerate}
  \setcounter{enumi}{5}
  \item Find the domain of \(f(x) = \sqrt{x - 3}.\)
  \item Find the domain of \(f(x) = \sqrt{5 - x}.\)
  \item Determine the domain of \(f(x) = \sqrt{2x + 4}.\)
  \item Find the domain of \(f(x) = \sqrt{x^2 - 9}.\)
  \item For \(f(x) = \sqrt{\dfrac{x - 2}{x + 1}},\) state all \(x\)-values that make it valid.
\end{enumerate}

\subsection*{Part C: Range and Output Behavior}
(Include small graph sketches later as needed.)
\begin{enumerate}
  \setcounter{enumi}{10}
  \item Determine the range of \(f(x) = x^2.\)
  \item Find the range of \(f(x) = (x - 1)^2 + 3.\)
  \item Find the range of \(f(x) = |x - 4|.\)
  \item Determine the range of \(f(x) = -x^2 + 5.\)
  \item Find the range of \(f(x) = \sqrt{x + 2}.\)
\end{enumerate}

\subsection*{Part D: Combining Restrictions}
\begin{enumerate}
  \setcounter{enumi}{15}
  \item Determine the domain of \(f(x) = \dfrac{\sqrt{x - 1}}{x - 3}.\)
  \item Determine the domain of \(f(x) = \dfrac{1}{\sqrt{x - 5}}.\)
  \item Determine the domain of \(f(x) = \sqrt{x + 2} + \dfrac{1}{x - 4}.\)
  \item For \(f(x) = \dfrac{\sqrt{x - 2}}{x^2 - 4},\) list all valid \(x\)-values.
  \item Find all \(x\) where \(f(x) = \dfrac{1}{\sqrt{x(x - 3)}}\) is defined.
\end{enumerate}

\subsection*{Part E: SAT-Style Applications}
(Insert sketches or descriptions for reference:  
– a rational function with vertical asymptote,  
– a square root starting at a point.)

\begin{enumerate}
  \setcounter{enumi}{20}
  \item A function \(f(x) = \dfrac{10}{x - 5}\) is undefined for what value of \(x?\)
  \item The graph of \(f(x) = \sqrt{x - 4}\) begins at which point?
  \item The function \(f(x) = (x - 3)^2 + 2\) has what range?
  \item The function \(f(x) = \dfrac{x + 1}{x^2 - 16}\) is undefined for which values of \(x?\)
  \item On the SAT, if a rational expression has both a numerator and denominator that contain \((x - 2)\), what must be done before identifying domain restrictions?
\end{enumerate}

\newpage

% ============================================================
% SOLUTIONS — UNIT 7, TOPIC 4: DOMAIN, RANGE, AND RESTRICTIONS
% ============================================================

\section*{Answer Key and Solutions: Domain, Range, and Restrictions}

\subsection*{Part A Solutions: Identifying Domain Restrictions}
\begin{enumerate}
  \item \(f(x) = \dfrac{3x + 1}{x - 4}\)

  Denominator cannot be 0:
  \[
  x - 4 \ne 0 \Rightarrow x \ne 4.
  \]

  \(\boxed{\text{Domain: all real } x \ne 4}\)

  \item \(f(x) = \dfrac{2}{x^2 - 9}\)

  Denominator cannot be 0:
  \[
  x^2 - 9 = 0 \Rightarrow x = \pm 3.
  \]

  \(\boxed{\text{Domain: all real } x \ne -3, 3}\)

  \item \(f(x) = \dfrac{1}{x(x - 5)}\)

  Denominator zero at \(x = 0\) or \(x = 5.\)

  \(\boxed{\text{Undefined at } x = 0 \text{ and } x = 5;\; \text{domain: } x \ne 0, 5}\)

  \item \(f(x) = \dfrac{x + 2}{x^2 + x - 6}\)

  Denominator cannot be 0. Factor:
  \[
  x^2 + x - 6 = (x + 3)(x - 2).
  \]
  So \(x \ne -3\) and \(x \ne 2.\)

  \(\boxed{\text{Domain: } x \ne -3, 2}\)

  \item \(f(x) = \dfrac{5}{(x + 1)(x - 2)}\)

  Denominator zero at \(x = -1\) or \(x = 2.\)

  \(\boxed{\text{Excluded: } x = -1,\; x = 2}\)
\end{enumerate}

\subsection*{Part B Solutions: Square Roots and Even Roots}
\begin{enumerate}
  \setcounter{enumi}{5}
  \item \(f(x) = \sqrt{x - 3}\)

  Inside the square root must be \(\ge 0\):
  \[
  x - 3 \ge 0 \Rightarrow x \ge 3.
  \]

  \(\boxed{\text{Domain: } x \ge 3}\)

  \item \(f(x) = \sqrt{5 - x}\)

  \[
  5 - x \ge 0 \Rightarrow x \le 5.
  \]

  \(\boxed{\text{Domain: } x \le 5}\)

  \item \(f(x) = \sqrt{2x + 4}\)

  \[
  2x + 4 \ge 0 \Rightarrow 2x \ge -4 \Rightarrow x \ge -2.
  \]

  \(\boxed{\text{Domain: } x \ge -2}\)

  \item \(f(x) = \sqrt{x^2 - 9}\)

  Need \(x^2 - 9 \ge 0\):
  \[
  x^2 \ge 9 \Rightarrow x \le -3 \text{ or } x \ge 3.
  \]

  \(\boxed{\text{Domain: } x \le -3 \text{ or } x \ge 3}\)

  \item \(f(x) = \sqrt{\dfrac{x - 2}{x + 1}}\)

  Two conditions:
  \begin{itemize}
    \item Denominator \(x + 1 \ne 0 \Rightarrow x \ne -1.\)
    \item The fraction under the root must be \(\ge 0.\)
  \end{itemize}

  \(\dfrac{x - 2}{x + 1} \ge 0\) is nonnegative when numerator and denominator have the same sign or numerator is 0.

  Sign chart:
  \[
  \begin{cases}
  x - 2 \ge 0 \text{ and } x + 1 > 0 \Rightarrow x \ge 2 \\
  x - 2 \le 0 \text{ and } x + 1 < 0 \Rightarrow x \le -1
  \end{cases}
  \]

  But \(x = -1\) is not allowed (division by 0).  
  Also \(x = 2\) gives 0 inside the root, which is allowed.

  \(\boxed{\text{Domain: } x \le -1 \text{ (not } -1\text{ itself)} \text{ or } x \ge 2}\)

  More precisely:
  \(\boxed{(-\infty, -1) \cup [2, \infty)}\)
\end{enumerate}

\subsection*{Part C Solutions: Range and Output Behavior}
\begin{enumerate}
  \setcounter{enumi}{10}
  \item \(f(x) = x^2\)

  A square is never negative. Minimum is 0 at \(x = 0.\)

  \(\boxed{\text{Range: } y \ge 0}\)

  \item \(f(x) = (x - 1)^2 + 3\)

  \((x - 1)^2 \ge 0\). Smallest value is 0 at \(x = 1.\)

  \[
  f(1) = 0 + 3 = 3.
  \]

  \(\boxed{\text{Range: } y \ge 3}\)

  \item \(f(x) = |x - 4|\)

  Absolute value is never negative. Smallest value is 0 at \(x = 4.\)

  \(\boxed{\text{Range: } y \ge 0}\)

  \item \(f(x) = -x^2 + 5\)

  This is an upside-down parabola. Maximum occurs at \(x = 0\):
  \[
  f(0) = 5.
  \]
  As \(|x|\) grows, \(-x^2\) goes to \(-\infty.\)

  \(\boxed{\text{Range: } y \le 5}\)

  \item \(f(x) = \sqrt{x + 2}\)

  Domain requires \(x + 2 \ge 0 \Rightarrow x \ge -2.\)

  Square roots are never negative, so:
  \[
  \sqrt{x + 2} \ge 0.
  \]

  \(\boxed{\text{Range: } y \ge 0}\)
\end{enumerate}

\subsection*{Part D Solutions: Combining Restrictions}
\begin{enumerate}
  \setcounter{enumi}{15}
  \item \(f(x) = \dfrac{\sqrt{x - 1}}{x - 3}\)

  Conditions:
  \begin{itemize}
    \item Inside the root: \(x - 1 \ge 0 \Rightarrow x \ge 1.\)
    \item Denominator: \(x - 3 \ne 0 \Rightarrow x \ne 3.\)
  \end{itemize}

  \(\boxed{\text{Domain: } x \ge 1,\; x \ne 3}\)

  \item \(f(x) = \dfrac{1}{\sqrt{x - 5}}\)

  Conditions:
  \begin{itemize}
    \item Inside the root: \(x - 5 > 0\) (must be strictly \(>\) 0, not \(\ge 0\), because the root is in the denominator and cannot be 0).
    \[
    x > 5.
    \]
  \end{itemize}

  \(\boxed{\text{Domain: } x > 5}\)

  \item \(f(x) = \sqrt{x + 2} + \dfrac{1}{x - 4}\)

  Conditions:
  \begin{itemize}
    \item \(\sqrt{x + 2}\) requires \(x + 2 \ge 0 \Rightarrow x \ge -2.\)
    \item Denominator \(x - 4 \ne 0 \Rightarrow x \ne 4.\)
  \end{itemize}

  \(\boxed{\text{Domain: } x \ge -2,\; x \ne 4}\)

  \item \(f(x) = \dfrac{\sqrt{x - 2}}{x^2 - 4}\)

  Conditions:
  \begin{itemize}
    \item \(\sqrt{x - 2}\) needs \(x - 2 \ge 0 \Rightarrow x \ge 2.\)
    \item Denominator \(x^2 - 4 = (x - 2)(x + 2) \ne 0.\)
    \[
    x \ne 2,\; x \ne -2.
    \]
  \end{itemize}

  Combine:
  \[
  x \ge 2 \text{ and } x \ne 2 \Rightarrow x > 2.
  \]
  (Note: \(x = -2\) is already not in \(x \ge 2\), so we only need \(x > 2.\))

  \(\boxed{\text{Domain: } x > 2}\)

  \item \(f(x) = \dfrac{1}{\sqrt{x(x - 3)}}\)

  Conditions:
  \begin{itemize}
    \item Inside the square root must be \(> 0\) (cannot be 0 because it is in the denominator):
    \[
    x(x - 3) > 0.
    \]
  \end{itemize}

  Solve \(x(x - 3) > 0\). This is positive when both factors are positive or both negative:
  \[
  \begin{cases}
  x > 0 \text{ and } x - 3 > 0 \Rightarrow x > 3 \\
  x < 0 \text{ and } x - 3 < 0 \Rightarrow x < 0
  \end{cases}
  \]

  So valid intervals are \(x < 0\) or \(x > 3.\)

  \(\boxed{\text{Domain: } (-\infty,0) \cup (3,\infty)}\)
\end{enumerate}

\subsection*{Part E Solutions: SAT-Style Applications}
\begin{enumerate}
  \setcounter{enumi}{20}
  \item \(f(x) = \dfrac{10}{x - 5}\)

  Denominator cannot be 0:
  \[
  x - 5 \ne 0 \Rightarrow x \ne 5.
  \]

  \(\boxed{x = 5 \text{ is not allowed}}\)

  \item \(f(x) = \sqrt{x - 4}\)

  The smallest allowed \(x\) is 4 (so that inside is 0).  
  At \(x = 4:\; f(4) = \sqrt{0} = 0.\)

  \(\boxed{\text{The graph begins at } (4,0)}\)

  \item \(f(x) = (x - 3)^2 + 2\)

  \((x - 3)^2 \ge 0\), minimum value is 0 at \(x = 3\).  
  So minimum output is 2.

  \(\boxed{\text{Range: } y \ge 2}\)

  \item \(f(x) = \dfrac{x + 1}{x^2 - 16}\)

  Denominator cannot be 0:
  \[
  x^2 - 16 = (x - 4)(x + 4) = 0
  \Rightarrow x = \pm 4.
  \]

  \(\boxed{\text{Undefined at } x = -4 \text{ and } x = 4}\)

  \item If both numerator and denominator share \((x - 2)\), you must \textbf{simplify first} before describing range or behavior, but \textbf{for domain you still exclude} the \(x\)-value that made the original denominator 0.

  Example:
  \[
  f(x) = \frac{x - 2}{(x - 2)(x + 5)} = \frac{1}{x + 5}, \quad x \ne 2.
  \]

  On SAT: you simplify for algebra, but you \emph{must still} say \(x \ne 2\).

  \(\boxed{\text{Cancel common factors, but still exclude values that made the denominator 0 originally}}\)
\end{enumerate}



\end{document}
